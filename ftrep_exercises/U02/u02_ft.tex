\documentclass[a4paper]{article}

% --- LANGUAGE ---

\usepackage[german]{babel}	% language specific quotation marks etc.

% --- DATA ---

\def\lecture{Repetitorium zur Funktionentheorie}
\def\authors{Linus Mußmächer}
\def\sheetNumber{02}
\def\sumPoints{30} 

% --- PREAMBLE ---
% === USAGE ===

% when using this preamble, setup your environment variables like this beforehand:


% \title{Stochastik 2}  %Title of exercise 
% \def\lecture{Stochastik 2}
% \def\authors{Linus Mußmächer}
% \def\sheetNumber{02}
% \def\sumPoints{30}      % maximum number of points (leave undefined)

% then use one of these commands (german or english) to print the header:

% \makeexheaderger

% and finally use subsections for your subtasks - they will be numbered as <sheetNumber><task number> by themselves

% if you have an exercise as an external .pdf, use \includetask to include it and increase the task counter


% --- OTHER ---

\usepackage{booktabs}       % professional-quality tables
\usepackage[table]{xcolor}	% color
\usepackage{pdfpages}		% to include entire pdf pages in appendix etc.
\usepackage{enumitem}		% better custom enumerations
\setlist[enumerate, 1]{label=(\roman*)}
\usepackage{etoolbox}		% toolbox for command modification

% --- FONTS & TYPESETTING ---

\usepackage[utf8]{inputenc} % allow utf-8 input
\usepackage[T1]{fontenc}    % use 8-bit T1 fonts
\usepackage{dsfont}			% font with double lines for sets
\usepackage[german,ruled,vlined,linesnumbered,commentsnumbered,algoruled]
{algorithm2e} 				%pseudo code
\usepackage{listings}		%java code
\usepackage{csquotes}

% --- URLS ---

\usepackage[colorlinks=true, linkcolor=black, citecolor=blue, urlcolor=blue]{hyperref}   	% hyperlinks
\usepackage{url}            % simple URL typesetting

% --- MATH SYMBOLS ---

\usepackage{amsmath,amssymb}% more math symbols
\usepackage{amsfonts}       % blackboard math symbols
\usepackage{latexsym}		% more math symbols
\usepackage{chngcntr}		% more math symbols
\usepackage{mathrsfs}		% math-fonts
\usepackage{mathtools}		% more math symbols
\usepackage{nchairx}		% Waldmann package for general math symbols

% --- GRAPHICS & CAPTIONS ----

\usepackage{graphicx}		% including images
\graphicspath{ {./figs/} }
\usepackage{subcaption}		% custom caption formatting
\DeclareCaptionLabelFormat{custom}{ \textbf{#1 #2}}
\captionsetup{format=hang}
\captionsetup{width=0.9\textwidth,labelformat=custom}
\usepackage{pdfpages}		% to include entire pdf pages in appendix etc.

% --- FORMAT ---

\usepackage[a4paper]{geometry} % a4 paper
\usepackage{setspace}		% spacing
\usepackage{titlesec}
\allowdisplaybreaks			% allow page breaks within math environments

% --- CUSTOM COMMANDS ---
%Logic
\newcommand{\then}{\Rightarrow}
\newcommand{\since}{\Leftarrow}
\renewcommand{\iff}{\ensuremath{\Leftrightarrow}}

%pretty epsilon
\let\oldepsilon\epsilon
\let\epsilon\varepsilon
\let\varepsilon\oldepsilon
%pretty phi
\let\oldphi\phi
\let\phi\varphi
\let\varphi\oldphi

\newcommand{\includetask}[2][pages=-]{
    \includepdf[#1]{#2}
    \addtocounter{subsection}{1}
}

% set-up for exercise specific stuff
\ifdef{\sheetNumber}{
    \setcounter{section}{\sheetNumber}
}{}

\usepackage{titling}
\newcommand{\makeexheaderger}{
    \begin{doublespace}
        \begin{center}
            \textbf{\Large{Übungsblatt \sheetNumber}}\\
            \textbf{\Large\lecture}\\
            Abgabe von: \textbf{\authors}\\
            \today
        \end{center}
        \ifdef {\sumPoints}
        {
            \hfill  \large Punkte: $\boxed{\qquad  /\; \sumPoints}$\\
        }{}
    \end{doublespace}
}

\newcommand{\makeexheadereng}{
    \begin{doublespace}
        \begin{center}
            \textbf{\Large{Exercise Sheet \sheetNumber}}\\
            \textbf{\Large\lecture}\\
            Contributors: \textbf{\authors}\\
            \today
        \end{center}
        \ifdef {\sumPoints}
        {
            \hfill  \large Points: $\boxed{\qquad  /\; \sumPoints}$\\
        }{}
    \end{doublespace}
}

\newcommand{\qmatrix}[1]{\ensuremath{\left(\begin{matrix}#1\end{matrix}\right)}}

\begin{document}

\makeexheaderger

\subsection{Ein Integral}

Wir stellen zuerst den Cosinus komplex dar:
\begin{align*}
	\int_{0}^{2 pi} (\cos t)^{2n} dt &= \frac{1}{2^{2n}} \int_{0}^{2 \pi} (\exp(it) + \exp(-it))^{2n} dt 
	\intertext{
		Dies entspricht einem Wegintegral über $f(z) = \frac{(z + \overline{z})^{2n}}{iz}$ entlang des Weges $\gamma: [0,2\pi] \to \mathds{D}, t \mapsto \exp(it)$, wie auch durch Substitution ersichtlich wird.
	}
	&= \frac{1}{i2^{2n}} \int_\gamma \frac{(z + \overline{z})^{2n}}{z} dz = \frac{1}{i2^{2n}} \int_\gamma \frac{(z + z^{-1})^{2n}}{z} dz
	\intertext{
		Die Funktion $f(z)$ ist im Einheitskreis, dem Inneren von $\gamma$, meromorph mit einer einzigen isolierten Singularität im Nullpunkt. Wir verwenden den Residuensatz.
	}
	&= \frac{1}{i2^{2n}} \cdot 2 \pi i \cdot n(0, \gamma) \cdot \res(0, f)
\end{align*}
Es gilt $n(0, \gamma) = 1$. Um das Residuum zu bestimmen, betrachten wir die Potenzreihenentwicklung von $f$ mittels des binomischen Lehrsatzes:
\begin{equation*}
	f(z) = \frac{(z + z^{-1})^{2n}}{z} = \sum_{k = 0}^{2n} \binom{2n}{k} z^{2n - k} z^{-k} z^{-1} = \sum_{k = 0}^{2n} \binom{2n}{k} z^{2n - 2k -1}
\end{equation*}
die für uns interessante Potenz $-1$ tritt hier genau für $k = n$ auf, also folgt
\begin{equation*}
	\res(0, f) = \binom{2n}{n} = \frac{(2n!)}{n! (2n-n!)} = \frac{(2n)!}{(n!)^2}
\end{equation*}
und dies zeigt
\begin{equation*}
	\int_{0}^{2 \pi} (\cos t)^{2n} = \frac{1}{i2^{2n}} \cdot 2 \pi i \cdot 1 \cdot \frac{(2n)!}{(n!)^2} = \frac{\pi}{2^{2n-1}} \cdot \frac{(2n!)}{(n!)^2}\text{,}
\end{equation*}
was zu beweisen war.

\subsection{Einige Integrale}

Wir verwenden für alle Kurvenintegrale den Residuensatz.

\begin{enumerate}
	\item Der gegebene Integrand $f(z) = \frac{\exp(iz^2) - 1}{z^2}$ ist meromorph mit einer isolierten Singularität im Nullpunkt (den im Kreisring $A_{0,2}(0)$ ist der Integrand mangels Nennernullstellen holomorph). Der Weg $\gamma$ umrundet diesen (mit Radius $2$) genau zweimal, also gilt
	\begin{equation*}
		\int_\gamma f(z) dz = 4 \pi i \res(0, f)\text{.}
	\end{equation*}
	Um das Residuum zu bestimmen, entwickeln wir die Funktion um den Nullpunkt:
	\begin{equation*}
		f(z) = \frac{\exp(iz^2) - 1}{z^2} = \frac{\exp(iz^2)}{z^2} - \frac{1}{z^2} = \sum_{n = 0}^{\infty} \frac{i^n z^{2n - 2}}{n!} - \frac{1}{z^2}\text{.}
	\end{equation*}
	Wir sehen, dass keiner der Summanden den Exponenten $-1$ hat, also gilt $res(0, f(z)) = 0$ und damit auch
	\begin{equation*}
		\int_{\gamma} \frac{\exp(iz^2) - 1}{z^2} dz = 0
	\end{equation*}
	\item Der gegebene Integrand $g(z) = \frac{\exp(z)}{(z-i)^3}$ hat lediglich bei $z = i$ eine Nennernullstelle, ist also im Kreisring $A_{0,1}(i)$ holomorph und damit im Inneren von $\eta$, dem Kreis $K_1(i)$, meromorph. $\eta$ umrundet den Punkt $i$ genau einmal in mathematisch negativer Richtung, der Residuensatz liefert also
	\begin{equation*}
		\int_\eta g(z) dz = - 2 \pi i \cdot \res (i, g)\text{.}
	\end{equation*}
	Wieder bestimmen wir die Reihenentwicklung um $i$. Da $\exp$ auf ganz $\mathds{C}$ und damit insbesondere in $K_1(i)$ holomorph ist, existiert eine in $K_1(i)$ konvergente Potenzreihe $\sum_{k =0}^{a_k} a_k (z-i)^k = \exp(z) $. Für die Koeffizienten gilt hier $a_k = \frac{\exp^{(k)}(i)}{k!}$. Dies liefert:
	\begin{equation*}
		g(z) = \sum_{k =0}^{\infty} a_k (z-i)^{k-3} 
	\end{equation*}
	und damit hat $g$ im Punkt $i$ das Residuum $a_2 = \frac{1}{2} \exp^{(2)}(i) = \frac{1}{2} \exp(i)$. Damit folgt
	\begin{equation*}
		\int_\eta g(z) dz = - \pi i e^i
	\end{equation*}
	\item Die Funktion $\frac{1}{z}$ ist auf $A_{0,2}(0)$ mangels Nennernullstellen holomorph, also ist es dort auch $\exp(1/z)$. Somit ist der Integrand $h(z) = \exp(1/z)$ auf $K_2(0)$, dem Inneren von $\gamma$, meromorph und der Residuensatz liefert mit $n(0, \gamma) = 2$
	\begin{equation*}
		\int_\gamma h(z) dz = 4 \pi i \res(0, h)\text{.}
	\end{equation*}
	Mit der Reihenentwicklung der Exponentialfunktion erhalten wir außerdem
	\begin{equation*}
		h(z) = \sum_{n = 0}^{\infty} \frac{(1/z)^n}{n!} = \sum_{n = -\infty}^{0} \frac{1}{(-n)!} z^n
	\end{equation*}
	und damit $\res(0, h) = 1$. Dies zeigt
	\begin{equation*}
		\int_\gamma h(z) dz = 4 \pi i\text{.}
	\end{equation*}
\end{enumerate}

\subsection{Sinus Hyperbolicus}

\begin{enumerate}
	\item 
	Die Funktion hat eine Nennernullstelle bei $z_1 = 0 \in S$, also $x,y = 0$.
	Weiterhin liegt eine Nennernullstelle vor, falls $\sinh(z) = \frac{1}{2} (\exp(z) - \exp(-z)) = 0$, also $\exp(z) = \exp(-z)$.
	Wir stellen um:
	\begin{equation*}
		\exp(z) = \exp(-z) \iff \exp(2z) = 1 \iff \exp(z) \in \{1, -1\}\text{.}
	\end{equation*}
	Insbesondere folgt damit $x = 0$ und $y \in k \cdot \pi i$, also liegt genau bei $z_2 = \pi i$ eine weitere Singularität vor.
	
	Wir bestimmen nun den Typ dieser Singularitäten.
	Man für $z_1 = 0$ betrachte die Reihenentwicklung des $\sinh$:
	\begin{align*}
		f(z) = \frac{1}{z \cdot \sinh(z)} = \frac{1}{z \cdot \sum_{k = 0}^{\infty} \frac{z^{2k+1}}{(2k+1)!}} = \frac{1}{z^2} \cdot \frac{1}{\sum_{k = 0}^{\infty} \frac{z^{2k}}{(2k+1)!}} = \frac{1}{z^2} \cdot \underbrace{\frac{1}{1 + \sum_{k = 1}^{\infty} \frac{z^{2k}}{(2k+1)!}}}_{\coloneq g(z)} 
	\end{align*}
	mit $g$ holomorph (da nennernullstellenfrei) in einer Umgebung von $0$ sowie $g(0) = \frac{1}{1 + 0} \neq 0$. Somit handelt es sich um einen Pol 2. Ordnung.
	Um den Typ von $z_2 = i \pi$ betrachte man unter Verwendung von $\exp(-i \pi) = \exp(i \pi) = -1$
	\begin{align*}
		\sinh(z) &= \frac{1}{2} (\exp(z) - \exp(-z)) = \frac{1}{2} (-\exp(z)\exp(-i \pi) + \exp(-z) \exp(i \pi))\\ &= - \frac{1}{2} (\exp(z - i \pi) - \exp(-(z - i \pi))) = - \sinh(z - i \pi)\text{.} 
	\end{align*}
	Dies liefert mit der Reihenentwicklung
	\begin{equation*}
		f(z) = \frac{1}{-z \sinh(z - i \pi)} = \frac{1}{z} \frac{1}{\sum_{k=0}^{\infty} \frac{(z-i \pi)^{2k+1}}{(2k + 1)!}} = -\frac{1}{z-i\pi} \underbrace{\frac{1}{z} \frac{1}{1 + \sum_{k =1}^{\infty} \frac{z^2k}{(2k + 1)!}}}_{\coloneq h(z)}
	\end{equation*}
	mit $h(z)$ holomorph (da nennernullstellenfrei) in einer Umgebung von $i \pi$ sowie $h(i \pi) = -\frac{1}{i \pi} \frac{1}{1 + 0} = \frac{i}{\pi}$.
	Somit handelt es sich um einen Pol 1. Ordnung.
	\item 
	Für $z_1 = 0$ berechnen wir 
	\begin{equation*}
		[z^2 \cdot f(z)]' = \left[\frac{z}{\sinh(z)}\right]' = \frac{\sinh(z) \cdot 1 + z \cdot \cosh(z)}{\sinh(z)^2}
	\end{equation*}
	und damit
	\begin{equation*}
		\res(0, f) = \frac{1}{(2 - 1)!} [z^2 \cdot f(z)]' |_{z = 0} = 1 \cdot 
	\end{equation*}
	Für $z_2 = i \pi$ gilt
	\begin{equation*}
		\res(i \pi, f) = \lim_{z \to z_2} (z - z_2) f(z) = \lim_{z \to z_2} g(z) = g(z_2) = \frac{i}{\pi} \text{.}
	\end{equation*}
	
\end{enumerate}




\end{document}
















