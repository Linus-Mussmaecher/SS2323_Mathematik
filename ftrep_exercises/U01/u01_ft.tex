\documentclass{article}

\usepackage[utf8]{inputenc} % allow utf-8 input
\usepackage[T1]{fontenc}    % use 8-bit T1 fonts
\usepackage[colorlinks=true, linkcolor=black, citecolor=blue, urlcolor=blue]{hyperref}       % hyperlinks
\usepackage{url}            % simple URL typesetting
\usepackage{booktabs}       % professional-quality tables
\usepackage{amsfonts}       % blackboard math symbols
\usepackage{amsmath,amssymb}% more math symbols
\usepackage{chngcntr}		% more math symbols
\usepackage{dsfont}			% font with double lines for sets
\usepackage{microtype}      % micro typography
\usepackage{graphicx}		% including images
\graphicspath{ {./figs/} }
\usepackage{setspace}		% spacing
\usepackage[german]{babel}	% german quotation marks etc.
\usepackage{pdfpages}		% to include entire pdf pages in appendix etc.
\usepackage{marvosym}		% more math symbols
\usepackage{mathtools}		% more math symbols
\usepackage{enumitem}		% better custom enumerations
\setlist[enumerate, 1]{label=\roman*)}
\usepackage{mathrsfs}
\usepackage{tikz-cd}		% drawing custom diagrams
\usepackage[a4paper]{geometry} % a4 paper
\usepackage{enumitem,setspace,graphicx}
\DeclareGraphicsExtensions{.pdf,.png,.eps,.jpg}


\def\then{\ensuremath{\Rightarrow}} % =>;
\def\since{\ensuremath{\Leftarrow}}
\def\iff{\ensuremath{\Leftrightarrow}} % genau dann wenn
\def\to{\ensuremath{\rightarrow}} % ->;
\def\Oh{\ensuremath{\mathcal{O}}} % Landau-Notation beispielsweise $\Oh(n)$
\def\N{\ensuremath{\mathbb{N}}}
\def\R{\ensuremath{\mathbb{R}}}
\def\C{\ensuremath{\mathbb{C}}}
\def\S{\ensuremath{\mathbb{S}}}
\def\id{\ensuremath{\text{id}}}
\newcommand{\angles}[1]{\left\langle #1 \right\rangle}

\newcommand{\includetask}[2][pages=-]{
	\includepdf[#1]{#2}
	\addtocounter{subsection}{1}
}

%hübscheres epsilon
\let\oldepsilon\epsilon
\let\epsilon\varepsilon
\let\varepsilon\oldepsilon
%hübscheres phi
\let\oldphi\phi
\let\phi\varphi
\let\varphi\oldphi

% ---

% --- DIESES 3 FELDER SIND AUSZUFÜLLEN ---
\def\sheetNumber{01}
\def\names{Linus Mußmächer} 
\def\sumPoints{60} 

\setcounter{section}{\sheetNumber}

\begin{document}

\begin{doublespace}
	\begin{center}
		\textbf{\Large{Übungsblatt \sheetNumber}}\\
		\textbf{\Large{Repetitorium Funktionentheorie}}\\
		Abgabe von: \textbf{\names}\\
		\today
	\end{center}
	\hfill  {\large Punkte: $\boxed{\qquad  /\; \sumPoints}$}\\
\end{doublespace}

%\vspace{3cm}

%\begin{center}
%	\includegraphics[width=0.5\textwidth]{pi.png}
%\end{center}

%\newpage

\subsection{Komplexe Identitäten}
Es ist
\begin{align*}
	|z - w|^2 & = (z-w) (\overline{z-w})                                            \\
	          & = z \overline{z} + w \overline{w} - z \overline{w} - w \overline{z} \\
	          & = |z|^2 + |w|^2 - (z \overline{w} + \overline{z \overline{w}})      \\
	          & = |z|^2 + |w|^2 - 2 \Re(z \overline{w})\text{.}
\end{align*}
Dies ist die komplexe Variante des Cosinussatzes im Dreieck, das von $0, z, w$ gebildet wird, wobei
\begin{align*}
	\Re(z \overline{w}) & = \Re(|z| \exp(i \arg(z)) \cdot \overline{ |w| \exp(i \cdot \arg(w))})                \\
	                    & = |z| \cdot |w| \cdot \Re(\exp(i \arg(z)) \cdot \overline{\exp(i \cdot \arg(w))})     \\
	                    & = |z| \cdot |w| \cdot \Re( \exp(i (\arg(z) - \arg(w)))                                \\
	                    & = |z| \cdot |w| \cdot \Re( \cos(\arg(z) - \arg(w)) + i \cdot \sin(\arg(z) - \arg(w))) \\
	                    & = |z| \cdot |w| \cdot \cos(\arg(z) - \arg(w))
\end{align*}
gilt. Falls wir also die Länge der Seiten des von $0,z,w$ gebildeten Dreiecks $[0,w], [0,z], [z,w]$ mit $a,b,c$ bezeichnen und den Winkel zwischen $[0,z]$ und $[0,w]$ mit $\gamma$, so folgt der Cosinussatz
\begin{equation*}
	c^2 = a^2 + b^2 - 2 a b \cos(\gamma)
\end{equation*}
als Verallgemeinerung des Satzes von Pythagoras für nicht-rechtwinklige Dreiecke. Weiterhin gilt
\begin{align*}
	|z - w|^2 + |z + w|^2 & = |z - w|^2 + |z - (-w)|^2                                                           \\
	                      & = |z|^2 + |w|^2 - 2 \Re(z \overline{w}) + |z|^2 + |-w|^2 - 2 \Re(z ( \overline{-w})) \\
	                      & = |z|^2 + |w|^2 - 2 \Re(z \overline{w}) + |z|^2 + |w|^2 + 2 \Re(z \overline{w})      \\
	                      & = 2 |z|^2 + 2 |w|^2	\text{.}
\end{align*}
Dies ist die Parallelogrammidentität, d.h. die Summe der Quadrate der Seiten eines Parallelogramms ist gleich der Summe der Quadrate der beiden Diagonalen. Das Parallelogramm wird hierbei von den Punkten $0, z, w, z+w$ gebildet.


\end{document}
















