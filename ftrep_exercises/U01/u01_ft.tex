\documentclass[a4paper]{article}

% --- LANGUAGE ---

\usepackage[german]{babel}	% language specific quotation marks etc.

% --- DATA ---

\def\lecture{Repetitorium zur Funktionentheorie}
\def\authors{Linus Mußmächer}
\def\sheetNumber{01}
\def\sumPoints{30} 

% --- PREAMBLE ---
% === USAGE ===

% when using this preamble, setup your environment variables like this beforehand:


% \title{Stochastik 2}  %Title of exercise 
% \def\lecture{Stochastik 2}
% \def\authors{Linus Mußmächer}
% \def\sheetNumber{02}
% \def\sumPoints{30}      % maximum number of points (leave undefined)

% then use one of these commands (german or english) to print the header:

% \makeexheaderger

% and finally use subsections for your subtasks - they will be numbered as <sheetNumber><task number> by themselves

% if you have an exercise as an external .pdf, use \includetask to include it and increase the task counter


% --- OTHER ---

\usepackage{booktabs}       % professional-quality tables
\usepackage[table]{xcolor}	% color
\usepackage{pdfpages}		% to include entire pdf pages in appendix etc.
\usepackage{enumitem}		% better custom enumerations
\setlist[enumerate, 1]{label=(\roman*)}
\usepackage{etoolbox}		% toolbox for command modification

% --- FONTS & TYPESETTING ---

\usepackage[utf8]{inputenc} % allow utf-8 input
\usepackage[T1]{fontenc}    % use 8-bit T1 fonts
\usepackage{dsfont}			% font with double lines for sets
\usepackage[german,ruled,vlined,linesnumbered,commentsnumbered,algoruled]
{algorithm2e} 				%pseudo code
\usepackage{listings}		%java code
\usepackage{csquotes}

% --- URLS ---

\usepackage[colorlinks=true, linkcolor=black, citecolor=blue, urlcolor=blue]{hyperref}   	% hyperlinks
\usepackage{url}            % simple URL typesetting

% --- MATH SYMBOLS ---

\usepackage{amsmath,amssymb}% more math symbols
\usepackage{amsfonts}       % blackboard math symbols
\usepackage{latexsym}		% more math symbols
\usepackage{chngcntr}		% more math symbols
\usepackage{mathrsfs}		% math-fonts
\usepackage{mathtools}		% more math symbols
\usepackage{nchairx}		% Waldmann package for general math symbols

% --- GRAPHICS & CAPTIONS ----

\usepackage{graphicx}		% including images
\graphicspath{ {./figs/} }
\usepackage{subcaption}		% custom caption formatting
\DeclareCaptionLabelFormat{custom}{ \textbf{#1 #2}}
\captionsetup{format=hang}
\captionsetup{width=0.9\textwidth,labelformat=custom}
\usepackage{pdfpages}		% to include entire pdf pages in appendix etc.

% --- FORMAT ---

\usepackage[a4paper]{geometry} % a4 paper
\usepackage{setspace}		% spacing
\usepackage{titlesec}
\allowdisplaybreaks			% allow page breaks within math environments

% --- CUSTOM COMMANDS ---
%Logic
\newcommand{\then}{\Rightarrow}
\newcommand{\since}{\Leftarrow}
\renewcommand{\iff}{\ensuremath{\Leftrightarrow}}

%pretty epsilon
\let\oldepsilon\epsilon
\let\epsilon\varepsilon
\let\varepsilon\oldepsilon
%pretty phi
\let\oldphi\phi
\let\phi\varphi
\let\varphi\oldphi

\newcommand{\includetask}[2][pages=-]{
    \includepdf[#1]{#2}
    \addtocounter{subsection}{1}
}

% set-up for exercise specific stuff
\ifdef{\sheetNumber}{
    \setcounter{section}{\sheetNumber}
}{}

\usepackage{titling}
\newcommand{\makeexheaderger}{
    \begin{doublespace}
        \begin{center}
            \textbf{\Large{Übungsblatt \sheetNumber}}\\
            \textbf{\Large\lecture}\\
            Abgabe von: \textbf{\authors}\\
            \today
        \end{center}
        \ifdef {\sumPoints}
        {
            \hfill  \large Punkte: $\boxed{\qquad  /\; \sumPoints}$\\
        }{}
    \end{doublespace}
}

\newcommand{\makeexheadereng}{
    \begin{doublespace}
        \begin{center}
            \textbf{\Large{Exercise Sheet \sheetNumber}}\\
            \textbf{\Large\lecture}\\
            Abgabe von: \textbf{\authors}\\
            \today
        \end{center}
        \ifdef {\sumPoints}
        {
            \hfill  \large Points: $\boxed{\qquad  /\; \sumPoints}$\\
        }{}
    \end{doublespace}
}

\begin{document}

\makeexheaderger

\subsection{Komplexe Identitäten}
Es ist
\begin{align*}
	|z - w|^2 & = (z-w) (\overline{z-w})                                            \\
	          & = z \overline{z} + w \overline{w} - z \overline{w} - w \overline{z} \\
	          & = |z|^2 + |w|^2 - (z \overline{w} + \overline{z \overline{w}})      \\
	          & = |z|^2 + |w|^2 - 2 \Re(z \overline{w})\text{.}
\end{align*}
Dies ist die komplexe Variante des Cosinussatzes im Dreieck, das von $0, z, w$ gebildet wird, wobei
\begin{align*}
	\Re(z \overline{w}) & = \Re(|z| \exp(i \arg(z)) \cdot \overline{ |w| \exp(i \cdot \arg(w))})                \\
	                    & = |z| \cdot |w| \cdot \Re(\exp(i \arg(z)) \cdot \overline{\exp(i \cdot \arg(w))})     \\
	                    & = |z| \cdot |w| \cdot \Re( \exp(i (\arg(z) - \arg(w)))                                \\
	                    & = |z| \cdot |w| \cdot \Re( \cos(\arg(z) - \arg(w)) + i \cdot \sin(\arg(z) - \arg(w))) \\
	                    & = |z| \cdot |w| \cdot \cos(\arg(z) - \arg(w))
\end{align*}
gilt. Falls wir also die Länge der Seiten des von $0,z,w$ gebildeten Dreiecks $[0,w], [0,z], [z,w]$ mit $a,b,c$ bezeichnen und den Winkel zwischen $[0,z]$ und $[0,w]$ mit $\gamma$, so folgt der Cosinussatz
\begin{equation*}
	c^2 = a^2 + b^2 - 2 a b \cos(\gamma)
\end{equation*}
als Verallgemeinerung des Satzes von Pythagoras für nicht-rechtwinklige Dreiecke. Weiterhin gilt
\begin{align*}
	|z - w|^2 + |z + w|^2 & = |z - w|^2 + |z - (-w)|^2                                                           \\
	                      & = |z|^2 + |w|^2 - 2 \Re(z \overline{w}) + |z|^2 + |-w|^2 - 2 \Re(z ( \overline{-w})) \\
	                      & = |z|^2 + |w|^2 - 2 \Re(z \overline{w}) + |z|^2 + |w|^2 + 2 \Re(z \overline{w})      \\
	                      & = 2 |z|^2 + 2 |w|^2	\text{.}
\end{align*}
Dies ist die Parallelogrammidentität, d.h. die Summe der Quadrate der Seiten eines Parallelogramms ist gleich der Summe der Quadrate der beiden Diagonalen. Das Parallelogramm wird hierbei von den Punkten $0, z, w, z+w$ gebildet.

\subsection{}

\begin{enumerate}
	\item Da $z \mapsto z$ und $z \mapsto \overline{z}$ stetig sind, ist auch die Abbildung $f: \mathds{C} \to \mathds{R}, z \mapsto \sqrt{z \cdot \overline{z}} + 1/2 (z + \overline{z}) = |z| + \Re(z)$ stetig. Dann ist das Urbild der offenen Menge $(1, \infty) \subseteq \mathds{R}$ unter $f$ offen, also $\{z \in \mathds{C} \mid f(z) > 1\}$ offen. Dann ist ihr Komplement $A = \{z \in \mathds{C} \mid f(z) \leq 1\}$ also abgeschlossen in $\mathds{C}$.
	
	$A$ ist nicht kompakt, denn für alle $a \in \mathds{R}_{<0}$ gilt $|a| - \Re(a) = 0 \leq 1$, d.h. die gesamte negative reelle Achse ist in $A$ enthalten. Diese ist aber nicht beschränkt und damit nicht kompakt, womit auch $A$ selbst nicht kompakt sein kann.
	\item Das Wurzelkriterium liefert uns
	\begin{equation*}
	\frac{1}{R} = \lim_{n \to \infty} \sqrt[n]{\left(\frac{n-1}{n}\right)^{5n^2}} = \lim_{n \to \infty} \left(\frac{n-1}{n}\right)^{5n} = \lim_{n \to \infty} \frac{1}{\left(\frac{n}{n-1}\right)^{5n}}\text{.}
	\end{equation*}
	Durch Indexverschiebung erhalten wir daraus
	\begin{equation*}
		\frac{1}{R} = \lim_{n \to \infty} \frac{1}{\left(\frac{n+1}{n}\right)^{5(n+1)}} =  \lim_{n \to \infty} \Bigg( \frac{1}{ \underbrace{\left( 1 + \frac{1}{n} \right)^n }_{\to e} } \Bigg)^5 \cdot \underbrace{\left(\frac{n}{n+1}\right)^5}_{\to 1} = \frac{1}{e^5}\text{,}
	\end{equation*}
	also den Konvergenzradius $R = e^5$.
\end{enumerate}

\end{document}
















