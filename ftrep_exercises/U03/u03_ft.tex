\documentclass[a4paper]{article}

% --- LANGUAGE ---

\usepackage[german]{babel}	% language specific quotation marks etc.

% --- DATA ---

\def\lecture{Repetitorium zur Funktionentheorie}
\def\authors{Linus Mußmächer}
\def\sheetNumber{03}
\def\sumPoints{30} 

% --- PREAMBLE ---
% === USAGE ===

% when using this preamble, setup your environment variables like this beforehand:


% \title{Stochastik 2}  %Title of exercise 
% \def\lecture{Stochastik 2}
% \def\authors{Linus Mußmächer}
% \def\sheetNumber{02}
% \def\sumPoints{30}      % maximum number of points (leave undefined)

% then use one of these commands (german or english) to print the header:

% \makeexheaderger

% and finally use subsections for your subtasks - they will be numbered as <sheetNumber><task number> by themselves

% if you have an exercise as an external .pdf, use \includetask to include it and increase the task counter


% --- OTHER ---

\usepackage{booktabs}       % professional-quality tables
\usepackage[table]{xcolor}	% color
\usepackage{pdfpages}		% to include entire pdf pages in appendix etc.
\usepackage{enumitem}		% better custom enumerations
\setlist[enumerate, 1]{label=(\roman*)}
\usepackage{etoolbox}		% toolbox for command modification

% --- FONTS & TYPESETTING ---

\usepackage[utf8]{inputenc} % allow utf-8 input
\usepackage[T1]{fontenc}    % use 8-bit T1 fonts
\usepackage{dsfont}			% font with double lines for sets
\usepackage[german,ruled,vlined,linesnumbered,commentsnumbered,algoruled]
{algorithm2e} 				%pseudo code
\usepackage{listings}		%java code
\usepackage{csquotes}

% --- URLS ---

\usepackage[colorlinks=true, linkcolor=black, citecolor=blue, urlcolor=blue]{hyperref}   	% hyperlinks
\usepackage{url}            % simple URL typesetting

% --- MATH SYMBOLS ---

\usepackage{amsmath,amssymb}% more math symbols
\usepackage{amsfonts}       % blackboard math symbols
\usepackage{latexsym}		% more math symbols
\usepackage{chngcntr}		% more math symbols
\usepackage{mathrsfs}		% math-fonts
\usepackage{mathtools}		% more math symbols
\usepackage{nchairx}		% Waldmann package for general math symbols

% --- GRAPHICS & CAPTIONS ----

\usepackage{graphicx}		% including images
\graphicspath{ {./figs/} }
\usepackage{subcaption}		% custom caption formatting
\DeclareCaptionLabelFormat{custom}{ \textbf{#1 #2}}
\captionsetup{format=hang}
\captionsetup{width=0.9\textwidth,labelformat=custom}
\usepackage{pdfpages}		% to include entire pdf pages in appendix etc.

% --- FORMAT ---

\usepackage[a4paper]{geometry} % a4 paper
\usepackage{setspace}		% spacing
\usepackage{titlesec}
\allowdisplaybreaks			% allow page breaks within math environments

% --- CUSTOM COMMANDS ---
%Logic
\newcommand{\then}{\Rightarrow}
\newcommand{\since}{\Leftarrow}
\renewcommand{\iff}{\ensuremath{\Leftrightarrow}}

%pretty epsilon
\let\oldepsilon\epsilon
\let\epsilon\varepsilon
\let\varepsilon\oldepsilon
%pretty phi
\let\oldphi\phi
\let\phi\varphi
\let\varphi\oldphi

\newcommand{\includetask}[2][pages=-]{
    \includepdf[#1]{#2}
    \addtocounter{subsection}{1}
}

% set-up for exercise specific stuff
\ifdef{\sheetNumber}{
    \setcounter{section}{\sheetNumber}
}{}

\usepackage{titling}
\newcommand{\makeexheaderger}{
    \begin{doublespace}
        \begin{center}
            \textbf{\Large{Übungsblatt \sheetNumber}}\\
            \textbf{\Large\lecture}\\
            Abgabe von: \textbf{\authors}\\
            \today
        \end{center}
        \ifdef {\sumPoints}
        {
            \hfill  \large Punkte: $\boxed{\qquad  /\; \sumPoints}$\\
        }{}
    \end{doublespace}
}

\newcommand{\makeexheadereng}{
    \begin{doublespace}
        \begin{center}
            \textbf{\Large{Exercise Sheet \sheetNumber}}\\
            \textbf{\Large\lecture}\\
            Abgabe von: \textbf{\authors}\\
            \today
        \end{center}
        \ifdef {\sumPoints}
        {
            \hfill  \large Points: $\boxed{\qquad  /\; \sumPoints}$\\
        }{}
    \end{doublespace}
}

\begin{document}

\makeexheader


\subsection{Logarithmus}

\begin{enumerate}
    \item $\overline{\mathds{D}}$ ist eine kompakte und nicht-leere Menge.
    Wir setzen $g(z) = 4z$ und berechnen für $z \in \partial \mathds{D}$:
    \begin{align*}
        |f(z) - g(z)| = |z^2 + e^z| \leq |z^2| + |e^z| \leq |z|^2 + e^{|z|} = 1 + e < 4 = |4z| = |g(z)| \leq |g(z)| + |f(z)|
    \end{align*}
    Nach dem Satz von Rouche hat somit $f(z)$ auf $\overline{\mathds{D}}$ dieselbe Anzahl an Nullstellen (gezählt nach ihrer Vielfachheit) wie $g(z) = 4z$, also genau eine (mit Vielfachheit $1$).
    Weiterhin liegt diese Nullstelle im Inneren $\mathds{D}$.
    \item Sei $z_0 \in \mathds{D}$ die eine Nullstelle von $f$.
    Dann können wir $f$ auf $\mathds{D}$ schreiben als $f(z) = (z - z_0) g(z)$ mit $g(z) \in H(\mathds{D})$ und $g(z_0) \neq 0$.
    Angenommen, $f$ besäße eine holomorphe Logarithmusfunktion $L$ auf $\mathds{D}$.
    Dann wäre $L|_{\mathds{D}\setminus\{z_0\}}$ eine holomorphe Logarithmusfunktion der (auf $\mathds{D} \setminus \{z_0\}$ nullstellenfreien und holomorphen) Funktion $f|_{\mathds{D} \setminus \{z_0\}}$.
    Somit wäre $\int_\gamma \frac{f'(z)}{f(z)} dz = 0$ für alle Wege $\gamma$ in $\mathds{D} \setminus \{z_0\}$.
    Wir berechnen dieses Integral:
    \begin{equation*}
        \int_\gamma \frac{f'(z)}{f(z)} dz = \int_\gamma \frac{(z-z_0) g'(z) + g(z)}{(z-z_0) g(z)} dz = \int_\gamma \frac{g'(z)}{g(z)} dz + \int_\gamma \frac{1}{z - z_0} dz\text{.}
    \end{equation*}
    Das erste Integral hat hier stets den Wert $0$, da $g$ und $g'$ in $\mathds{D}$ holomorph und $g$ nullstellenfrei und somit $\frac{g'}{g}$ holomorph ist.
    Das zweite Integral hat nach dem Residuensatz den Wert $n(z_0, \gamma) \cdot \res\left(z_0, \frac{1}{z-z_0}\right)$.
    Die Funktion $\frac{1}{z-z_0}$ hat in $z_0$ eine einfache Polstelle und es folgt $\res\left(z_0, \frac{1}{z-z_0}\right) = \lim_{z \to z_0} (z-z_0) \frac{1}{z -z_0} = 1 \neq 0$.
    Dies zeigt 
    \begin{equation*}
        0 = \int_\gamma \frac{f'(z)}{f(z)} dz = n(z_0, \gamma) \cdot 1 \text{.}
    \end{equation*}
    Es müsste also $n(z_0, \gamma) = 0$ für alle Wege $\gamma \in \mathds{D} \setminus \{z_0\}$ gelten, was natürlich Unsinn ist.
    Somit folgt per Widerspruch, dass $f$ in $\mathds{D} \setminus \{z_0\}$ und damit auch in $\mathds{D}$ keine holomorphe Logarithmusfunktion besitzt.
    \item Angenommen, eine solche Funktion $h \in H(\mathds{D})$ existiere.
    Dann ist $0 = f(z_0) = (w(z_0))^3$, also $w(z_0) = 0$.
    $w$ hat also in $z_0$ eine (mindestens) einfache Nullstelle.
    Daher können wir $w$ schreiben als $w(z) = (z-z_0)^k h(z)$ mit $h \in H(\mathds{D})$, $h(z_0) \neq 0$ und $k \geq 1$.
    Dann aber ist 
    \begin{equation*}
        f(z) = (w(z))^3 = (z-z_0)^{3k} (h(z))^3\text{,}
    \end{equation*}
    also hat $f$ in $z_0$ eine (mindestens) dreifache Nullstelle, ein Widerspruch.
\end{enumerate}


\subsection{Lokale Injektivität}

Für ein beliebiges $n \in \mathds{N}$ ist $f_n$ lokal injektiv auf ganz $G$, also folgt $f_n'(z) \neq 0$ für alle $z \in G$.
Für die Funktionenfolge $(f_n') \in H(G)$ gilt also $0 \notin f_n'(G)$ für alle $n$, und da $(f_n')$ nach Weierstraß ebenfalls kompakt gegen $f'$ konvergiert folgt $f' \equiv 0$ oder $0 \notin f'(G)$ nach dem Satz von Hurwitz.
In ersterem Fall folgt, da $G$ ein Gebiet und insbesondere zusammenhängend ist, dass $f$ konstant ist; in zweiterem Fall per Definition die lokale Injektivität in jedem Punkt.

\subsection{Biholomorphe Abbildungen}

\begin{enumerate}
    \item Würde eine solche biholomorphe Funktion $\phi: \mathds{C} \setminus \{2\} \to \mathds{D}$ existieren, so ließe sie sich nach dem Riemannschen Fortsetzungssatz zu einer ganzen Funktion $\tilde{\phi}: \mathds{C} \to \mathds{D}$ fortsetzen.
    Diese Funktion wäre dann aber beschränkt und somit nach Liouville konstant, hätte also insbesondere nicht Bildbereich $\mathds{D}$.
    Somit kann eine solche Funktion nicht existieren.
    \item Ja, die gesuchte Funktion ist $f: z \mapsto z^2$.
    Da $f$ ein Polynom ist, ist die Holomorphie klar.
    Jedes $z \in \mathds{C} \setminus (\infty, 0]$ lässt sich als $z = r \exp(i \phi)$ mit $r \in [0, \infty)$ und $\phi \in (-\pi, \pi)$ schreiben und hat folglich Urbild $\sqrt{r} \exp(i \phi/2)$ unter $f$.
    Wegen $\phi \in (-\pi/2, \pi/2)$ liegt dieses Urbild auch in $RH$.

    Für ein $z \in RH$ gilt außerdem $z = r \exp(i \phi)$ mit $\phi \in (- \pi/2, \pi/2)$, also $f(z) = r^2 \exp(i 2\phi)$ $\in \mathds{C} \setminus (-\infty, 0]$.
    Das Bild $f(RH)$ ist also genau die geschlitzte Ebene.

    Seien weiterhin $z_1 = r_1 \exp(i \phi_1) \in RH$ und $z_2 = r_2 \exp(i \phi_2) \in RH$, mit $r_1, r_2 \in [0, \infty)$ und $\phi_1, \phi_2 \in (-\pi/2, \pi/2)$, derart, dass $f(z_1) = f(z_2) \then r_1^2 \exp(2i \phi_1) = r_1 \exp(2i \phi_2)$.
    Dann folgt $|r_1^2 \exp(i 2 \phi_1)| = |r_2^2 \exp(2i \phi_2)| \then |r_1|^2 = |r_2|^2$, also $r_1 = r_2$ wegen $r_1, r_2 \geq 0$.
    Falls $r_1 = r_2 = 0$ so folgt bereits $z_1 = z_2 = 0$, andernfalls zeigt dies $\exp(i \cdot 2 \phi_1) = \exp(i \cdot 2 \phi_2)$, also $2 \phi_1 = 2\phi_2 \mod 2\pi$ und damit $2 \phi_1 = 2 \phi_2$, da $2 \phi_1, 2 \phi_2 \in (- \pi, -\pi)$.
    Dies zeigt $\phi_1 = \phi_2$ und damit $z_1 = z_2$, also die Injektivität von $f$.

    Somit ist $f$ bijektiv und holomorph, also eine biholomorphe Abbildung.

    \item Angenommen, eine solche Funktion existiert.
    Dann ist ihre Umkehrabbildung eine Funktion $\phi: \mathds{C} \to S$.
    Insbesondere ist $\phi$ ganz.
    Wäre $\phi$ ein Polynom, so folgt $\phi(\mathds{C}) = \mathds{C}$ aus dem Fundamentalsatz der Algebra (denn das Polynom $\tilde{p} = \phi - w$ hat für alle $w \in \mathds{C}$ eine Nullstelle $x_w$, und dann gilt $\phi(x_w) = w$, d.h. $w \in \phi(\mathds{C})$).
    Also muss $\phi$ ganz-transzendent sein.
    Nach dem Satz von Casorati-Weierstraß liegt dann aber $\phi(\mathds{C})$ dicht in $\mathds{C}$, also kann $\phi(\mathds{C})$ nicht $S$ sein.
    Eine solche Funktion kann also nicht existieren.
\end{enumerate}

\subsection{Beschränkte Ableitungen}

\begin{itemize}
    \item Sei zuerst $G = \mathds{C}$ und $z_0 \in \mathds{C}$ beliebig.
    Dann ist $f_n: \mathds{C} \to \mathds{C}, z \mapsto n(z - z_0) + z_0$ holomorph, hat Ableitung $f'(z_0) = n$ und es gilt $f(z_0) = n \cdot 0  + z_0 = z_0$.
    Dies zeigt, dass unsere Menge unbeschränkt ist.
    \item Sei nun $G \neq \mathds{C}$.
    Nach dem Riemannschen Abbildungssatz existiert eine Abbildung $g: G \to \mathds{D}$ mit $g(z_0) = 0$ und $g'(z_0) > 0$.
    Dann ist für jede Abbildung $f: G \to G$ mit $f(z_0) = z_0$ die Komposition $\tilde{f} = g \circ f \circ g^{-1}$ eine Abbildung von $\mathds{D}$ nach $\mathds{D}$ und $f(0) = g(f(z_0)) = g(z_0) = 0$.
    Für die Ableitung folgt
    \begin{align*}
        \tilde{f}'(0) &= (g \circ f \circ g^{-1})'(0)\\
        &= (g' \circ f \circ g^{-1})(0) \cdot (f \circ g^{-1})'(0)\\
        &= (g' \circ f \circ g^{-1})(0) \cdot (f' \circ g^{-1})(0) \cdot (g^{-1})'(0)\\
        &= g'(z_0) \cdot f'(z_0) \cdot (g^{-1})'(0) = g'(z_0) \cdot f'(z_0) \cdot \frac{1}{g'(g^{-1}(z_0))}\\
        &= g'(z_0) \cdot f'(z_0) \cdot \frac{1}{g'(z_0)} = f'(z_0)
    \end{align*}
    wobei im letzten Schritt $g'(z_0) \neq 0$ aus dem Riemannschen Abbildungssatz eingeht.
    Der Satz von Schwarz zeigt nun $|\tilde{f}'(0)| \leq 1$, also $|f'(z_0)| \leq 1$ für eine beliebige Funktion $f: G \to G$ mit $f(z_0) = z_0$.
    
    Dies zeigt die Beschränktheit der gegebenen Menge mit Schranke $1$. Wegen $\id: G \to G$ mit $\id'(z_0) = 1$ ist diese Schranke sogar scharf.


\end{itemize}

\end{document}