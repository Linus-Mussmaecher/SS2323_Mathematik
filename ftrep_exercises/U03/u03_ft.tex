\documentclass[a4paper]{article}

% --- LANGUAGE ---

\usepackage[german]{babel}	% language specific quotation marks etc.

% --- DATA ---

\def\lecture{Repetitorium zur Funktionentheorie}
\def\authors{Linus Mußmächer}
\def\sheetNumber{03}
\def\sumPoints{30} 

% --- PREAMBLE ---
% === USAGE ===

% when using this preamble, setup your environment variables like this beforehand:


% \title{Stochastik 2}  %Title of exercise 
% \def\lecture{Stochastik 2}
% \def\authors{Linus Mußmächer}
% \def\sheetNumber{02}
% \def\sumPoints{30}      % maximum number of points (leave undefined)

% then use one of these commands (german or english) to print the header:

% \makeexheaderger

% and finally use subsections for your subtasks - they will be numbered as <sheetNumber><task number> by themselves

% if you have an exercise as an external .pdf, use \includetask to include it and increase the task counter


% --- OTHER ---

\usepackage{booktabs}       % professional-quality tables
\usepackage[table]{xcolor}	% color
\usepackage{pdfpages}		% to include entire pdf pages in appendix etc.
\usepackage{enumitem}		% better custom enumerations
\setlist[enumerate, 1]{label=(\roman*)}
\usepackage{etoolbox}		% toolbox for command modification

% --- FONTS & TYPESETTING ---

\usepackage[utf8]{inputenc} % allow utf-8 input
\usepackage[T1]{fontenc}    % use 8-bit T1 fonts
\usepackage{dsfont}			% font with double lines for sets
\usepackage[german,ruled,vlined,linesnumbered,commentsnumbered,algoruled]
{algorithm2e} 				%pseudo code
\usepackage{listings}		%java code
\usepackage{csquotes}

% --- URLS ---

\usepackage[colorlinks=true, linkcolor=black, citecolor=blue, urlcolor=blue]{hyperref}   	% hyperlinks
\usepackage{url}            % simple URL typesetting

% --- MATH SYMBOLS ---

\usepackage{amsmath,amssymb}% more math symbols
\usepackage{amsfonts}       % blackboard math symbols
\usepackage{latexsym}		% more math symbols
\usepackage{chngcntr}		% more math symbols
\usepackage{mathrsfs}		% math-fonts
\usepackage{mathtools}		% more math symbols
\usepackage{nchairx}		% Waldmann package for general math symbols

% --- GRAPHICS & CAPTIONS ----

\usepackage{graphicx}		% including images
\graphicspath{ {./figs/} }
\usepackage{subcaption}		% custom caption formatting
\DeclareCaptionLabelFormat{custom}{ \textbf{#1 #2}}
\captionsetup{format=hang}
\captionsetup{width=0.9\textwidth,labelformat=custom}
\usepackage{pdfpages}		% to include entire pdf pages in appendix etc.

% --- FORMAT ---

\usepackage[a4paper]{geometry} % a4 paper
\usepackage{setspace}		% spacing
\usepackage{titlesec}
\allowdisplaybreaks			% allow page breaks within math environments

% --- CUSTOM COMMANDS ---
%Logic
\newcommand{\then}{\Rightarrow}
\newcommand{\since}{\Leftarrow}
\renewcommand{\iff}{\ensuremath{\Leftrightarrow}}

%pretty epsilon
\let\oldepsilon\epsilon
\let\epsilon\varepsilon
\let\varepsilon\oldepsilon
%pretty phi
\let\oldphi\phi
\let\phi\varphi
\let\varphi\oldphi

\newcommand{\includetask}[2][pages=-]{
    \includepdf[#1]{#2}
    \addtocounter{subsection}{1}
}

% set-up for exercise specific stuff
\ifdef{\sheetNumber}{
    \setcounter{section}{\sheetNumber}
}{}

\usepackage{titling}
\newcommand{\makeexheaderger}{
    \begin{doublespace}
        \begin{center}
            \textbf{\Large{Übungsblatt \sheetNumber}}\\
            \textbf{\Large\lecture}\\
            Abgabe von: \textbf{\authors}\\
            \today
        \end{center}
        \ifdef {\sumPoints}
        {
            \hfill  \large Punkte: $\boxed{\qquad  /\; \sumPoints}$\\
        }{}
    \end{doublespace}
}

\newcommand{\makeexheadereng}{
    \begin{doublespace}
        \begin{center}
            \textbf{\Large{Exercise Sheet \sheetNumber}}\\
            \textbf{\Large\lecture}\\
            Abgabe von: \textbf{\authors}\\
            \today
        \end{center}
        \ifdef {\sumPoints}
        {
            \hfill  \large Points: $\boxed{\qquad  /\; \sumPoints}$\\
        }{}
    \end{doublespace}
}

\begin{document}

\makeexheader


\subsection{Logarithmus}

\begin{enumerate}
    \item $\overline{\mathds{D}}$ ist eine kompakte und nicht-leere Menge.
    Wir setzen $g(z) = 4z$ und berechnen für $z \in \partial \mathds{D}$:
    \begin{align*}
        |f(z) - g(z)| = |z^2 + e^z| \leq |z^2| + |e^z| \leq |z|^2 + e^{|z|} = 1 + e < 4 = |4z| = |g(z)| \leq |g(z)| + |f(z)|
    \end{align*}
    Nach dem Satz von Rouche hat somit $f(z)$ auf $\overline{\mathds{D}}$ dieselbe Anzahl an Nullstellen (gezählt nach ihrer Vielfachheit) wie $g(z) = 4z$, also genau eine (mit Vielfachheit $1$).
    Weiterhin liegt diese Nullstelle im Inneren $\mathds{D}$.
    \item Sei $z_0 \in \mathds{D}$ die eine Nullstelle von $f$.
    Dann können wir $f$ auf $\mathds{D}$ schreiben als $f(z) = (z - z_0) g(z)$ mit $g(z) \in H(\mathds{D})$ und $g(z_0) \neq 0$.
    Angenommen, $f$ besäße eine holomorphe Logarithmusfunktion $L$ auf $\mathds{D}$.
    Dann wäre $L|_{\mathds{D}\setminus\{z_0\}}$ eine holomorphe Logarithmusfunktion der (auf $\mathds{D} \setminus \{z_0\}$ nullstellenfreien und holomorphen) Funktion $f|_{\mathds{D} \setminus \{z_0\}}$.
    Somit wäre $\int_\gamma \frac{f'(z)}{f(z)} dz = 0$ für alle Wege $\gamma$ in $\mathds{D} \setminus \{z_0\}$.
    Wir berechnen dieses Integral:
    \begin{equation*}
        \int_\gamma \frac{f'(z)}{f(z)} dz = \int_\gamma \frac{(z-z_0) g'(z) + g(z)}{(z-z_0) g(z)} dz = \int_\gamma \frac{g'(z)}{g(z)} dz + \int_\gamma \frac{1}{z - z_0} dz\text{.}
    \end{equation*}
    Das erste Integral hat hier stets den Wert $0$, da $g$ und $g'$ in $\mathds{D}$ holomorph und $g$ nullstellenfrei und somit $\frac{g'}{g}$ holomorph ist.
    Das zweite Integral hat nach dem Residuensatz den Wert $n(z_0, \gamma) \cdot \res\left(z_0, \frac{1}{z-z_0}\right)$.
    Die Funktion $\frac{1}{z-z_0}$ hat in $z_0$ eine einfache Polstelle und es folgt $\res\left(z_0, \frac{1}{z-z_0}\right) = \lim_{z \to z_0} (z-z_0) \frac{1}{z -z_0} = 1 \neq 0$.
    Dies zeigt 
    \begin{equation*}
        0 = \int_\gamma \frac{f'(z)}{f(z)} dz = n(z_0, \gamma) \cdot 1 \text{.}
    \end{equation*}
    Es müsste also $n(z_0, \gamma) = 0$ für alle Wege $\gamma \in \mathds{D} \setminus \{z_0\}$ gelten, was natürlich Unsinn ist.
    Somit folgt per Widerspruch, dass $f$ in $\mathds{D} \setminus \{z_0\}$ und damit auch in $\mathds{D}$ keine holomorphe Logarithmusfunktion besitzt.
    \item Angenommen, eine solche Funktion $h \in H(\mathds{D})$ existiere.
    Dann ist $0 = f(z_0) = (w(z_0))^3$, also $w(z_0) = 0$.
    $w$ hat also in $z_0$ eine (mindestens) einfache Nullstelle.
    Daher können wir $w$ schreiben als $w(z) = (z-z_0)^k h(z)$ mit $h \in H(\mathds{D})$, $h(z_0) \neq 0$ und $k \geq 1$.
    Dann aber ist 
    \begin{equation*}
        f(z) = (w(z))^3 = (z-z_0)^{3k} (h(z))^3\text{,}
    \end{equation*}
    also hat $f$ in $z_0$ eine (mindestens) dreifache Nullstelle, ein Widerspruch.
\end{enumerate}


\subsection{Lokale Injektivität}

$f_n$ ist lokal injektiv, d.h. für jeden Punkt $z$ in $G$ existiert eine offene Umgebung $U_{f_n,z} \ni z$, in der $f_n$ lokal injektiv ist.
Wähle $U_z = \bigcup_{n \in \mathds{N}} U_{f_n, z}$.
Dies ist eine abzählbare Vereinigung offener Mengen und daher immer noch offen.
Auf einer beliebigen kompakten Teilmenge $K_z \neq \{z\}$ von $U_z$ ist die Folge $(f_n)$ daher gleichmäßig konvergent und injektiv, nach dem Satz von Hurwitz ist daher auch $f$ injektiv oder konstant auf $K_z$.

\begin{enumerate}
    \item \textbf{Fall 1}: $f$ ist für alle $z$ auf $K_z$ injektiv.
    Dann ist $f$ lokal injektiv, denn $K_z^{\interior}$ ist eine Umgebung von $z$, auf der $f$ injektiv ist.
    \item \textbf{Fall 2}: Es existiert ein $z_0$ mit $f$ ist injektiv auf $K_{z_0}$ und ein $z_1$ mit $f$ ist konstant auf $K_{z_1}$.
    Da $G$ wegzusammenhängend ist, existiert ein Weg $\gamma: [0,1] \to G$ mit $\gamma(0) = z_0$ und $\gamma(1) = z_1$. 
    Die Menge $\{ K_w^\interior \mid w \in \gamma([0,1]) \}$ bildet eine offene Überdeckung der kompakten Menge $\gamma([0,1])$ (da $[0,1]$ kompakt und $\gamma$ stetig). 
    Daher finden wir eine endliche Teilüberdeckung $\{ K_{w_k}^\interior \mid k = 0, \dots, n\}$ von $\gamma([0,1])$.
    O.B.d.A. seien die $w_k$ dabei nach ihrem Urbild unter $\gamma$ geordnet, und sei $w_0 = z_0$ und $w_n = z_1$.
    Weiterhin existiert eine Teilfolge dieser, beginnend bei $K_{w_0}$ und endend bei $K_{w_n}$, Mengen derart, dass aufeinanderfolgende Mengen sich schneiden.
    O.E. sei dies bereits die Folge der $w_n$.
    Wir wissen, dass $f$ auf $K_{w_0}$ konstant ist und auf $K_{w_1}$ injektiv oder ebenfalls konstant. Da $K_{w_0}^\interior \cap K_{w_1}^\interior$ offen und nicht-leer ist, ist $f$ auf $K_{w_0}^\interior \cap K_{w_1}^\interior \subseteq K_{w_0}$ konstant und kann daher auf $K_{w_1}$ nicht injektiv sein.
    Also ist $f$ auf $K_{w_1}$ ebenfalls konstant (mit gleichem Wert wie auf $K_{w_0}$).

    So fortfahrend folgt, dass $f$ auf allen $K_{w_k}$ konstant ist, insbesondere auf $K_{w_n} = K_{z_1}$.
    Dies ist ein Widerspruch zur Injektivität von $f$ auf $K_{z_1}$ (da $K_{z_1} \neq \{z_1\}$), also tritt dieser Fall nicht ein.
    \item \textbf{Fall 3}: $f$ ist auf allen $K_z$ konstant, aber nicht mit derselben Konstante. Hier folgt der Widerspruch wie oben.
    \item \textbf{Fall 4}: $f$ ist auf allen $K_z$ konstant mit gleicher Konstante $k$. Dann ist $f$ auch insgesamt konstant, da insbesondere $f(z) = k$ für alle $z$ gilt.
\end{enumerate}




\end{document}