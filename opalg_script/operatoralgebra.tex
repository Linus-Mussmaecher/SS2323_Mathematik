\documentclass[a4paper]{article}

% --- PREAMBLE ---

\usepackage[english]{babel}	% language specific quotation marks etc.
% --- LANGUAGE ---

% Needs to be set individually!
%\usepackage[english]{babel}	% language specific quotation marks etc.

% --- OTHER ---

\usepackage{booktabs}       % professional-quality tables
\usepackage[table]{xcolor}	% color
\usepackage{pdfpages}		% to include entire pdf pages in appendix etc.
\usepackage{enumitem}		% better custom enumerations
\setlist[enumerate, 1]{label=(\roman*)}
\usepackage{etoolbox}		% toolbox for command modification

% --- FONTS & TYPESETTING ---

\usepackage[utf8]{inputenc} % allow utf-8 input
\usepackage[T1]{fontenc}    % use 8-bit T1 fonts
\usepackage{dsfont}			% font with double lines for sets
\usepackage[german,ruled,vlined,linesnumbered,commentsnumbered,algoruled]
{algorithm2e} 				%pseudo code
\usepackage{listings}		%java code
\usepackage{csquotes}

% --- URLS ---

\usepackage[colorlinks=true, linkcolor=black, citecolor=blue, urlcolor=blue]{hyperref}   	% hyperlinks
\usepackage{url}            % simple URL typesetting

% --- MATH SYMBOLS ---

\usepackage{amsmath,amssymb}% more math symbols
\usepackage{amsfonts}       % blackboard math symbols
\usepackage{latexsym}		% more math symbols
\usepackage{chngcntr}		% more math symbols
\usepackage{mathrsfs}		% math-fonts
%\usepackage{marvosym}		% more math symbols (conflicts with Waldmann)
\usepackage{mathtools}		% more math symbols
\usepackage{nchairx}		% Waldmann package for general math symbols and operators
% theorem formatting (contained in Waldmann)
%\usepackage[amsmath,thmmarks,framed,thref]{ntheorem}

% --- GRAPHICS & CAPTIONS ----

\usepackage{graphicx}		% including images
\graphicspath{ {./figs/} }
\usepackage{subcaption}		% custom caption formatting
\DeclareCaptionLabelFormat{custom}{ \textbf{#1 #2}}
\captionsetup{format=hang}
\captionsetup{width=0.9\textwidth,labelformat=custom}

% --- TIKZ ---

\usepackage{tikz}			% basic tikz for custom images
\usetikzlibrary{cd}			% custom diagrams
\usetikzlibrary{external}	% externalize images for faster compilation
\tikzexternalize[prefix=figures/]
\AtBeginEnvironment{tikzcd}{\tikzexternaldisable} %fix cd/externalize
\AtEndEnvironment{tikzcd}{\tikzexternalenable}
\usepackage{pgfplots}		% custom plotting
\usepgfplotslibrary{colormaps}
\pgfplotsset{compat=newest}	
\usetikzlibrary{patterns}	% custom patterns

% --- FORMAT ---

\usepackage[a4paper]{geometry} % a4 paper
\usepackage{setspace}		% spacing
\usepackage[nobottomtitles*]{titlesec} %prevent section titles from sometimes being on the bottom of a page
\usepackage{titlesec}
\allowdisplaybreaks			% allow page breaks within math environments

% --- CUSTOM COMMANDS ---
%Logic
\newcommand{\then}{\Rightarrow}
\newcommand{\since}{\Leftarrow}
\renewcommand{\iff}{\ensuremath{\Leftrightarrow}}

%pretty epsilon
\let\oldepsilon\epsilon
\let\epsilon\varepsilon
\let\varepsilon\oldepsilon
%pretty phi
\let\oldphi\phi
\let\phi\varphi
\let\varphi\oldphi

%matrix
\newcommand{\qmatrix}[1]{\ensuremath{\left(\begin{matrix}#1\end{matrix}\right)}}

\newcommand{\unitcircle}{\mathds{S}^1}

% --- DATA ---

\title{Introduction to Operator Algebras}
\author{Alcides Buss\\Notes by: Linus Mußmächer\\2336440}
\date{Summer 2023}

% --- DOCUMENT ---

\begin{document}

\maketitle

\newpage

\tableofcontents

\newpage

The set of all linear bounded operators $\algebra{L}(H) = \algebra{B}(H)$ on a given Banach space $H$ is a (Banach) algebra with $S \cdot T = S \circ T$.
$M \subseteq \algebra{L}$ is a Subalgebra such that $M^* \subseteq M$ where $T^*$ is the adjoint of $T$.
This is also a closed subspace with respect to the strong topology. This is equivalent to $M = M''$ (when $X \subseteq \algebra{B}(H), X' = \{ T \in \algebra{B}(H) \mid TS = ST \ \forall_{ S \in X} \}$)

\subsection*{Some topological basics}

\begin{definition}~
	\begin{itemize}
		\item Topology, Open
		\item Hausdorff, locally Hausdorff
		\item compact
	\end{itemize}
\end{definition}

\begin{definition}
	A topological space $X$ is \textbf{locally Hausdorff} if every $x \in X$ admits a compact neighborhood basis, that is for every $x \in X$ and every open set $U \ni x$ there exists an open set $V \ni x$ with $\overline{V}$ is compact.
\end{definition}

\begin{corollary}
	If a set $V$ is compact in any subset $U \subseteq X$, it is also compact in $X$.
\end{corollary}

\begin{example}[Snake with two heads]
	Consider $I = [0,1]$ with the standard topology and extend the set with an element $1^+$ such that $I \cup 1^+ \setminus 1$ is isomorphic to $I$. Then $I \cup 1^+$ is locally Hausdorff and compact, but not Hausdorff.
\end{example}

\subsection*{Some results about locally compact Hausdorff spaces}

\begin{lemma}[Uryson's Lemma]
	Let $X$ be locally compact and Hausdorff.
	For all $F \subseteq X$ closed and $K \subseteq X$ compact with $F \cap K = \emptyset$, there exists an $f: X \to [0,1]$ continuous such that $f|_K \equiv 1$ and $f|_F \equiv 0$.
\end{lemma}

\begin{theorem}[Tietze's extension theorem]
	Let $X$ be locally compact, $K \subseteq X$ compact and $f: K \to \mathds{C}$ continuous. Then there exists a continuous $\tilde f: X \to \mathds{C}$ such that $\tilde f|_K = f$.
\end{theorem}

\begin{theorem}[Alexandroff's conpactification]
	If $X$ is locally compact and Hausdorff, then $\tilde X \sqcup \{\infty\}$ is a compact Hausdorff space $\topology{O}(\tilde X) = \topology{O}(X) \cup \{K^\complement \cup \{\infty\} \mid K \text{ compact} \}$.
\end{theorem}

\begin{example}
	Compacting the real line $\mathds{R}$ yields the space $\tilde{\mathds{R}}$, which is isomorphic to the unit circle $\Pi = \mathds{S}^1$.
\end{example}

\begin{theorem}
	Conversely, if $Y$ is a compact Hausdorff space, then for all $y_0 \in Y$, $X \coloneqq Y \setminus \{y_0\}$ is locally compact (in respect to the subspace topology).

	More generally, if $Y$ is locally compact and Hausdorff, and $Z \subseteq Y$ is a difference of open and closed subsets, of $Y$ (i.e. $Z = U \setminus F$, where $U$ is open in $Y$ and $F$ is closed in $Y$), then $Z$ is locally compact.
\end{theorem}

\section{Algebras}

\begin{definition}
	An \textbf{algebra} is a (complex) vector space $\algebra{A}$ endowed with a bilinear and associative multiplication: $\algebra{A} \times \algebra{A} \to \algebra{A}, (a,b) \mapsto a \cdot b$. So
	\begin{enumerate}
		\item $(a + \alpha b) \cdot (c + \beta d) = ac + \alpha bc + \beta ad + \alpha \beta b d$
		\item  $(a \cdot b) \cdot c = a \cdot (b \cdot c)$
	\end{enumerate}
	for all $a,b,c \in \algebra{A}$ and $\alpha, \beta \in \mathds{C}$. We say that $\algebra{A}$ is
	\begin{enumerate}
		\item \textbf{commutative}, if $ab = ba$ for all $a,b \in \algebra{A}$ and
		\item \textbf{unital}, if there exists $1 = 1_\algebra{A} \in \algebra{A}$ such that $1 \cdot a = a \cdot 1 = a$ for all $a \in \algebra{A}$.
	\end{enumerate}
	~
\end{definition}

\begin{example}~
	\begin{enumerate}
		\item $\mathds{C}$, or more generally $\mathds{C}^n = \mathds{C} \oplus \dots \oplus \mathds{C}$, is an algebra.
		\item Say $X$ is any set; let $\mathds{C}^X = \{ f: X \to \mathds{C} \}$ with point wise multiplication $(f \cdot g)(x) = f(x) \cdot g(x)$.
		      These are commutative unital algebras (with $1(x) = 1 \in \mathds{C}$).
		\item Consider the polynomials $\mathds{C}[X] = \{ \sum_{i = 0}^{n} \lambda_i x^i \mid \lambda_i \in \mathds{C}, n \in \mathds{N} \}$ with the usual operations.
		      This is a commutative unital algebra.
		\item Let $X$ be a topological space and $C(X) = \{f: X \to \mathds{C} \mid f \text{ is continouus}\} \subseteq \mathds{C}^X$ the set of continuous functions on $X$.
		      This is a commutative unital (sub)algebra (of $\mathds{C}^X$).
		\item Take any vector space $A$ define a (trivial) multiplication $a \cdot b \coloneqq 0$.
		      This is a commutative Algebra (that is not unital unless $A = 0$).
		\item $M_n(\mathds{C})$ (the complex $n \times n$ matrices) with the usual multiplication are a non-commutative (unless $n=1$) unital algebra.
		\item Let $V$ be any (complex) vector space. The set of all linear operators $L(V) := \{T: V \to V \mit T \text{ linear operator}\}$ is a unital (non-commutative for $\dim V > 1$). We observe $\algebra{L}(\mathds{C}^n) \simeq M_n(\mathds{C})$.
		\item Let $S$ be a semigroup (i.e. a set with an associative operation $S \times S \to S$, e.g. $(\mathds{N}, +)$). Then $\mathds{C}[S] = \{ \sum_{s \in S} \lambda_s s \mid \lambda_s \in \mathds{C}, |\{s: \lambda_s \neq 0\}| < \infty \}$ (the finite formal sums of elements of $S$) with the following product
		      \begin{equation*}
			      \left(\sum_{s \in S'}\lambda_s s\right) \cdot \left(\sum_{t \in S} \lambda_t' t\right) := \sum_{s,t \in S} (\lambda_s \cdot \lambda'_t)(s \cdot t) \in S
		      \end{equation*}
		      Observe: As a vector space: $\mathds{C}[S] \subseteq \mathds{C}^S$.
		      In general, this is neither commutative nor unital.
	\end{enumerate}
\end{example}

\section{Normed algebras}

\begin{definition}
	An algebra $\algebra{A}$ is \textbf{normed}, if it is endowed with a (vector space) norm $\| \cdot \| \colon \algebra{A} \to [0, \infty)$ satisfying $\|a \cdot b\| \leq \|a\| \cdot \|b\|$.
	If $\algebra{A}$ is unital with unit $1_\algebra{A}$, we usually assume $\| 1_\algebra{A} \| = 1$ except for $\algebra{A} = 0$.
\end{definition}

\begin{definition}
	A \textbf{Banach algebra} is a normed algebra that is also complete (as a metric space with respect to the distance $d(a,b) := \| a - b\|$), i.e. every Cauchy sequence converges.

\end{definition}

\begin{example}
	\begin{enumerate}
		\item If $X$ is a compact space then $C(X)$ is a commutative unital Banach algebra with respect to the norm $\| f \|_\infty := \sup_{x \in X} |f(x)| < \infty$ (since $X$ is compact).
		\item If $V$ is a normed (respectively Banach) vector space, e.g. $\mathds{C}^n$ or $\ell^p(\mathds{N})$, then $\algebra{L}(V) = \{T \in L(V) \mid T \text{ is bounded/continouus} \}$ with $\|T\| := \sup_{\|v\| \leq 1} \| T(v) \| < \infty$ is a normed Banach algebra.
		\item If $X$ is a topological space, then $C_b(X) = \{f \in C(X) \mid \|f\|_\infty < \infty \}$ (bounded continuous functions) is a Banach space.
		\item Let $X$ again be a topological space. Then the set of all functions \textbf{vanishing at $\infty$},
		      \begin{align*}
			      C_0(X) & = \{ f \in C(X) \mid \forall_{\epsilon > 0} \exists_{K \subseteq X, K \text{ compact}} \forall_{x \notin K} |f(x)| < \epsilon \} \\
			             & = \{ f \in C(X) \mid \forall_{\epsilon > 0} \{x \in X \mid |f(x)| \geq \epsilon \} \text{ is compact} \}
			      \subseteq C_b(X)
			      \text{,}
		      \end{align*}
		      is also a Banach algebra.
	\end{enumerate}
\end{example}

\begin{exercise}
	Assume $X$ is locally compact and Hausdorff. Prove the following are equivalent:
	\begin{enumerate}[label=(\arabic*)]
		\item $X$ is compact.
		\item $C(X) = C_0(X)$
		\item $C_0(X)$ is unital.
		\item The unit function $1 \in C_b(X)$ belongs to $C_0(X)$.
	\end{enumerate}
\end{exercise}

\begin{proof}
	\begin{itemize}
		\item (1) $\then$ (2): Recall the definition of $C_0(X)$. If $X$ is compact, every closed subset (especially every $\{x : |f(x)| \geq \epsilon\}$) is compact, so the condition of $C_0(X)$ is trivial.
		\item (2) $\then$ (3): Since $C(X)$ is unital, $C_0(X)$ is as well.
		\item (3) $\then$ (4): Suppose $C_0$ is unital, and let $f \in C_0(X)$ be the unit. Then $f \cdot g = g$ for all $g \in C_0(X)$, i.e. $f(x) g(x) = g(x) \forall_{x \in X} \forall_{g \in C_0(X)}$. By Uryson's lemma, given any $x_0 \in X$, there exists $g \in C_0(X)$ with $g(x_0) = 1$ (by looking at $K = \{x_0\}$ and taking $F$ as the complement of any relatively compact environment of $x_0$.). Then $f(x_0) = f(x_0) g(x_0) = g(x_0) = 1$. Doing this for every $x_0 \in X$ yields $f \equiv 1$.
		\item (4) $\then$ (1): Since $1 \in C_0(X)$, for every $\epsilon > 0$ the set $\{x \mid |f(x)| \geq \epsilon\}$ is compact. Choose $\epsilon = \frac{1}{2}$. Then, $\{x \mid |f(x)| = |1| \geq \frac{1}{2}\} = X$ is compact.
	\end{itemize}
\end{proof}

\begin{exercise}
	Let $X$ be a locally compact Hausdorff space.
	Prove that $C_0(X) \cong \{f \in C(X) \mid f(\infty) = 0\}$
\end{exercise}

\section{Algebras}

\begin{definition}
	A \textbf{$^*$-algebra} is a complex algebra $\algebra{A}$ with an \textbf{involution} $^*: \algebra{A} \to \algebra{A}$ satisfying
	\begin{enumerate}
		\item $(a + \lambda b)^* = a^* + \overline{\lambda} b^*$
		\item $(a^*)^* = a$
		\item $(ab)^* = b^* a^*$
	\end{enumerate}
	for all $a,b \in \algebra{A}$ and all $\lambda \in \mathds{C}$.
\end{definition}

\begin{definition}
	A \textbf{normed $^*$-algebra} is a normed algebra $\algebra{A}$ with an involution (such that $\algebra{A}$ is a $^*$-algebra) also satisfying $\| a^*\| = \|a\|$ for all $a \in \algebra{A}$.

	A \textbf{Banach-$^*$-algebra} is a complete normed $^*$-algebra.
\end{definition}

\begin{definition}
	A $C^*$-algebra is a Banach-$^*$-algebra satisfying $\|a^* \cdot a\| = \| a \|^2$.
\end{definition}

Observation: Recall that $\|a \cdot b\| \leq \|a\| \cdot \| b \|$ in all normed algebras. Applying this to a $C^*$-algebra we get $\|a \cdot a^*\| \leq \|a^*\| \cdot \| a \|$. If $\algebra{A}$ is a $C^*$-algebra, then $\| a \| ^2 = \| a \cdot a^*\| \leq \|a^*\| \cdot \|a\|$, so $\| a\| = \|a^*\|$.

\begin{example}~
	\begin{enumerate}
		\item If $X$ is a set, then $\mathds{C}^X$ is a $^*$-algebra with $f^* = \overline{f}$ and $\algebra{C}^\infty(X)$ is a $C^*$-algebra.
		\item If $X$ is a topological space, then $C(X) \subseteq \mathds{C}^X$ is also a $^*$-subalgebra and for $\{f \in C(X) \mid \supp(f) = \closure{\{x \in X \mid |f(x)| \neq 0\}} \text{ compact} \}$ we have
		      \begin{equation*}
			      C_c(X) = \subseteq C_0(X) \subseteq C_b(X) \subseteq C(X) \subseteq C^\infty(X)
		      \end{equation*}
		      and $C^\infty$ is a $C^*$-algebra. $C_c$ is a $^*$-algebra, but not Banach in general.

		      If $X$ is compact, it follows $C_c(X) = C_0(X) = C_b(X)$.

		      Observation: If $X$ is locally compact and Hausdorff, then $\closure{C_c(X)} = C_0(X)$.
		\item Let $X$ be a measured space ($X$ is endowed with a $\sigma$-algebra). Then $B_\infty(X) = \{f \in C^\infty \mid f \text{ is measurable}\}$ is a $C^*$-algebra.
		      If $\mu$ is a measure on $X$ (e.g. $X = \mathds{R}^n$ and $\mu$ the Lebesgue measure) then $L^\infty(X,\mu)$ are the essentially bounded functions and
		      \begin{equation*}
			      L^\infty(X) = \{f: X \to \mathds{C} \mid \|f\| \coloneqq \inf \{ c \geq 0 \mid \mu(\{x \mid |f(x)| > c\}) = 0 \} \}
		      \end{equation*}
		      is also a $C^*$-algebra.

		      Observation: $L^2(X, \mu) = $ \enquote{$\mu$-separable function}, $L^\infty(X, \mu) \xrightarrow{\mu} B(L^2(X,\mu)), f \mapsto \mu_f = \{g \mapsto f \cdot g\}$
		\item A non-example: Let $\mathds{D}$ be the unit disk and $\algebra{A}(\D) = \{f \in \mathds{C}(\mathds{D}) \mid \text{ analytic in } \mathds{D}^\interior \}$

		      \textbf{Morera's Theorem} from complex analysis states that $f \in C(\mathds{D})$ is analytic if and only if $\int_{\gamma} f(z) dz = 0$ for all closed and piece wise smooth paths in $\mathds{D}^\interior$. From this, it follows that $\algebra{A}(\mathds{D})$ is closed in $C(\mathds{D})$, therefore a Banach algebra. It is also a Banach-$^*$-algebra with, but $f^* = \overline{f}$ (point wise) is not possible, as $z \mapsto \overline{z}$ is not analytic. Thus, we have to choose $f^*(z) = f(\overline{z})$.
		      But $\algebra{A}(\mathds{D})$ is not a $C^*$-algebra, as $\| f^* f\|_\infty \neq \|f\|_\infty^2$ for some $f \in \algebra{A}(\mathds{D})$.
		\item A non-commutative example: Let $H$ be a Hilbert space and $B(H) = \algebra{L}(H) = \{T: H \to H \mid T \text{bounded, continuous, linear}\}$ and $\|H\| \coloneqq \sup_{\|z\| < 1} \|T(z)\| < \infty$. This is a $C^*$-algebra where $T^*$ is the adjoint of $T$, that is $\SP{T^* z, w} = \SP{z, Tw}$ for all $z,w \in H$.

		      $C^*$-axiom: $\|T^* \cdot T\| \leq \|T\|^2$ since $\algebra{L}(H)$ is a Banach algebra, and we also have
		      \begin{align*}
			      \|T\|^2 & = \sup_{\|z\| < 1} \|T(z)\|^2 = \sup_{\|z\| < 1} \SP{Tz, Tz} = \sup_{\|z\| < 1} \SP{z, T^* T z} \\
			              & \leq \sup_{\|z\| < 1} \|z\| \| T^*T z\| \leq \sup_{\|z\| < 1}  \|z\| \| T^*T\| \leq  \| T^*T\|
		      \end{align*}
		      In particular, $M_n(\mathds{C}) \simeq \algebra{L}(\mathds{C}^n)$ is a unital $C^*$-algebra.
		\item To produce more examples, take any subset $S \subseteq \algebra{L}(H)$ and take $C^*(S) \subseteq \algebra{L}(H) = \closure{\Span\{S_i \mid S_i \in S \cup S^*, i \leq n \in \mathds{N}\}}$.
	\end{enumerate}
\end{example}

\begin{example}
	Let $s \in \algebra{L}(\ell^2(\mathds{N}))$. The shift $s$, defined by $s(e_i) = e_{i+1}$ for all $i \in \mathds{N}$ (where $\{e_i\}$ is the canonical basis of the sequence space), is an isometry, that is $s^* \cdot s = \id$.
	Since $s \cdot s^* \neq \id$, it is not surjective and not a proper isometry.
	We define
	\begin{equation*}
		T = C^*(s) = \closure{\Span\{s^n (s^*)^m \mid m,n \in \mathds{N}_0 \}} \subseteq \algebra{L}(\ell^2(\mathds{N}))
	\end{equation*}
	as the \textbf{Toeplitz algebra}.
\end{example}

\begin{example}
	Let $H$ be a Hilbert space and $S$ the set of all finite rank operators on $H$.

\end{example}

\begin{example}~
	\begin{enumerate}
		\item \textbf{Commutative}: $C_0(X)$ for a locally Hausdorff space $X$.
		\item \textbf{Non-commutative}: $\mathcal{L}(\hilbert{H}) = \mathcal{B}(\hilbert{H})$ for any Hilbert space $\hilbert{H}$ (with dimension greater $1$).
		\item \textbf{More generally}: Take any subset $S \subseteq \mathcal{L}(\hilbert{H})$ and construct $C^*(S) \subseteq \mathcal{L}(H)$ as
		      \begin{equation*}
			      \closure{\Span}\{S_1, \dots, S_n\mid S_i \in S \cap S^*\}
		      \end{equation*}
	\end{enumerate}
\end{example}

\begin{example}[Cuntz algebras]
	Take again $\hilbert{H} = \ell^2\mathds{N} = \{(\lambda_n)_{n \in \mathds{N}_0} \mid \sum_{n = 0}^{\infty} |\lambda_n|^2 < \infty\}$ where $\SP{\lambda, \lambda'} = \sum_{i \in \mathds{N}_0} \overline{\lambda_i} \lambda_i'$ and which has the orthonormal base $(e_n)_{n \in \mathds{N}}$ where $(e_n) = (\delta_{in})_{i \in \mathds{N}_0}$.

	On this algebra, define
	\begin{itemize}
		\item $S_1(e_n) = e_{2n}$.
		\item $S_2(e_n) = e_{2n + 1}$.
	\end{itemize}
	We have partitioned the natural numbers into evens and odds. This defines two (proper) isometries $S_1, S_2 \in \algebra{L}(\hilbert{H})$, that is $S_i^* S_i = \id_\hilbert{H}$, to subspaces of $\hilbert{H}$. Notice: $S_i^*S_j = 0$  for $i \neq j$ as well as $S_1 S_1^* + S_2S_2^* = \id_\hilbert{H}$.  Define $\algebra{O}_2 = C^*(S_1, S_2) = \closure{\Span}\{S_\alpha S_\beta^* \mid \alpha, \beta \text{ finite words in } \{1,2\}\}$. For example, for $\alpha = 121211$ we have $S_{\alpha} = S_1 S_2 S_1 S_2 S_1^2$. $\algebra{O}_2$ is called the \textbf{Cuntz algebra}. More generally, one can define $\algebra{O}_3, \algebra{O}_4$, ... Cuntz algebras. Joachim Cuntz proved that these are simple $C^*$-algebras with additional interesting properties we will see later.
\end{example}

\begin{example}[Rotation algebras]
	Let $\hilbert{H} = \ell^2(\mathds{Z})$ (bi-infinite sequences) with basis $(e_n)_{n \in \mathds{Z}}$  Define:
	\begin{itemize}
		\item $U(e_n) := e_{n+1}$ (bilateral shift)
		\item $V(e_n) := \lambda^n e_n$ where $\lambda\in\mathds{C}$ is some fixed number $|\lambda| = 1$.
	\end{itemize}
	This defines two \textit{unitary} operators: $U U^* = 1 = U^* U$ and $V^* V = 1 = V^*V$. If $\exp(2 \pi i \theta), \theta \in \mathds{R}$ define $A_\theta := C^*(U,V) \subseteq \algebra{L}(\ell^2 \mathds{N})$.

	There is a special relation between $U$ and $V$ where $UV = \lambda VU = \exp(2 \pi i \theta) V U$. From this relation, we can describe $A_\theta = \closure{\Span}\{\sum_{n,m \in \mathds{Z}}^{\text{finite}} a_{n,m} U^n V^m \mid a_{n,m} \in \mathds{C}\}$.

	Furthermore, if $\theta \in \mathds{R} \setminus \mathds{Q}$, $A_\theta$ is simple.
\end{example}

\begin{example}[$C^*$-algebras of groups]
	Let $G$ be a (discrete) group.
	Look at $\hilbert{H} = \ell^2(G) = \{(a_g)_{g \in G} \mid \sum_{g \in G} |a_g|^2 < \infty\}$
	(Note: This limit will only converge if there are countably (or finitely) many non-zero parts)
	with ONB $(\delta_g)_{g \in G}$ where $\delta_g(h) = \delta_{gh}$.
	Define for each $g \in G$ an operator $\lambda_g \in \algebra{L}(\ell^2 G)$ by $\lambda_g(\delta_h) = \delta_{gh}$.
	Notice that $h \mapsto gh$ is a bijection, and thus $\lambda_g$ is a unitary operator with $\lambda_g^* = \lambda_{g^{-1}}$. We can now define the \textbf{reduced $C^*$-algebra} of the group:
	\begin{equation*}
		C_R^*(G) := C_\lambda^*(G) \subseteq \algebra{L}(\ell^2 G) = C^*(\lambda_g \mid g \in G)
	\end{equation*}
	Here, we have the relation $\lambda_g \cdot \lambda_h = \lambda_{gh}$ and thus $C_R^*(G) = \{\sum a_g \lambda_g \mid a_g \in \mathds{C}\}$.

	In general, take $U: G \to \algebra{L}(H), g \mapsto U_g$ a \textbf{unitary representation of $G$} with $U_g U_h = U_{gh}$ and $U_1 = \id$ as well as $U_g^{-1} = U_{g^{-1}}$. Then $C^*_U(G) := \{\sum_{g \in G} a_g U_g \mid a_g \in \mathds{C}\} \subseteq \algebra{L}(H)$. There exists a \textbf{universal unitary representation} $C^*_\mathrm{max}(G)$, a full $C^*$-algebra of $G$.
\end{example}

\begin{remark}~
	\begin{enumerate}
		\item If $G$ is Abelian, then $C_U^*(G)$ is also abelian (commutative). In particular, $C_\lambda^*$ is abelian. Later, we will prove $C_\lambda^*(G) \simeq C(\hat G)$ where $\hat G$ is the dual of $G$, i.e. $\{X: G \to \mathds{C} \text{ characters}\}$.
		\item For many groups, like $G = \mathds{F}_n$ (the free groups) the reduced $C^*$-algebra $C^*_\lambda(G)$ is simple.
	\end{enumerate}
\end{remark}

\section{Homomorphisms of algebras}

\begin{definition}
	If $\algebra{A}, \algebra{B}$ are algebras, a \textbf{homomorphism} from $\algebra{A}$ to $\algebra{B}$ is a linear map $\phi: \algebra{A} \to \algebra{B}$ such that $\phi(ab) = \phi(a) \phi(b)$ for any $a,b \in \algebra{A}$.

	If $\algebra{A}$ and $\algebra{B}$ are $^*$-algebras, a \textbf{$^*$-homomorphism} is a homomorphism $\phi: \algebra{A} \to \algebra{B}$ such that $\phi(a^*) = \phi(a)^*$ for all $a \in \algebra{A}$.

	If $\algebra{A}, \algebra{B}$ are Banach algebras, then usually we want to have \textbf{continuous} homomorphisms. Even more, we usually ask for \textbf{contractive} homomorphisms $\phi: \algebra{A} \to \algebra{B}$, (that is $\|\phi\| \leq 1$).
\end{definition}

We will be especially interested in \textbf{characters}:

\begin{definition}
	A \textbf{character} of an algebra $\algebra{A}$ is a non-zero homomorphism $\chi: \algebra{A} \to \mathds{C}$.
\end{definition}

\begin{example}
	Take any subalgebra $\algebra{A} \subseteq \mathds{C}^X$. Take $x_0 \in X$ and set $\chi_{x_0} := \mathrm{ev}_{x_0}: \algebra{A} \to \mathds{C}, f \mapsto f(x_0)$. This is not necessarily a character, but it is for example, if $\algebra{A} = C(X)$ or $C_b(X)$ or $C_0(X)$ (if $X$ is \enquote{nice}, like Hausdorff).
\end{example}

\begin{definition}
	A  ($^*$)-isomorphism between two $(^*)$-algebras $\algebra{A}$ and $\algebra{B}$ is a bijective $(^*)$- homomorphism $
		\phi: \algebra{A} \overset{\sim}{\to} \algebra{B}$.
\end{definition}

\begin{definition}
	A \textbf{$(^*)$-ideal} of a $^*$-algebra $\algebra{A}$ is a subspace $I \subset A$ such that $I \cdot A \subseteq I$, $A \cdot I \subseteq I$ (if only one condition applies, we call this a \textbf{left ideal} or \textbf{right ideal}). For $^*$-ideals, we also want $I^* = I$. We notate this as $I \trianglelefteq \algebra{A}$.
\end{definition}

\begin{example}
	If $\phi: \algebra{A} \to \algebra{B}$ is a $(^*)$-homomorphism, then $\ker \phi \trianglelefteq \algebra{A}$.
\end{example}

\begin{example}
	If $I \trianglelefteq \algebra{A}$ for $\algebra{A}$ a $(^*)$-algebra
	\begin{equation*}
		\algebra{A} / I = \{a + I \mid a \in \algebra{A}\}
	\end{equation*}
	with $(a + I) \cdot (b + I) := ab + I$ and $(a + I)^* = a^* + I$ is a $(^*)$-algebra.
\end{example}

\begin{theorem}
	If $\algebra{A}$ is a Banach-$^*$-algebra, then $I \trianglelefteq \algebra{A}$ is a closed ideal, then the quotient $I / \algebra{A}$ is also a Banach-$^*$-algebra.
\end{theorem}

\begin{proof}
	Later.
\end{proof}

\section{Spectral theory}

\begin{notation}
	If $\algebra{A}$ is
	a unital algebra, we write
	\begin{equation*}
		\inv(\algebra{A}) = \{a \in \algebra{A} \mid a \text{ is invertible in } \algebra{A}\} = \{a \in \algebra{A} \mid \exists_{a^{-1} \in \algebra{A}} a a^{-1} = 1 = a^{-1} a \}
	\end{equation*}
	This is a group. Sometimes we also write $GL(\algebra{A})$.
\end{notation}

\begin{definition}
	Given a unital algebra $\algebra{A}$ and $a \in \algebra{A}$, we define its \textbf{spectrum} (in $\algebra{A}$) as
	\begin{equation*}
		\sigma_\algebra{A}(a) = \sigma(a) = \{\lambda \in \mathds{C} \mid \lambda \cdot 1 - a \notin \inv(\algebra{A})\}
	\end{equation*}
	and the resolvent of $a$ (in $\algebra{A}$) as
	\begin{equation*}
		\rho_\algebra{A}(a) = \rho(a) = \algebra{A} \setminus \sigma_\algebra{A}(a) = \{\lambda \in \mathds{C} \mid \lambda - a \in \inv(\algebra{A})\}
	\end{equation*}
\end{definition}

\begin{example}[Linear Algebra]
	Let $\algebra{A} = M_m(\mathds{C})$ and $a \in \algebra{A}$. Then we have
	\begin{equation*}
		\sigma(a) = \{\lambda \in \mathds{C} \mid \lambda - a \notin \inv(\algebra{A})\} = \{\lambda \in \mathds{C} \mid \det(\lambda - a) = 0\}
	\end{equation*}
	and these are the roots of the characteristic polynomial $\det(\lambda - a)$. This is exactly the usual spectrum from linear algebra.
\end{example}

\begin{example}[Functional Analysis]
	Let $\algebra{A} = \algebra{L}(\hilbert{H})$ -- where $\hilbert{H}$ is any Hilbert- or Banach space -- and $T \in \algebra{A}$. Then $\sigma_\algebra{A}(T)$ is exactly the spectrum as defined in functional analysis.

	If $S$ is the shift in $\algebra{L}(\ell^2 \mathds{N})$, then we have $\sigma(S) = \mathds{D}$.
\end{example}

\begin{example}
	Let $\algebra{A} = \mathds{C}[X]$. Here we have $\inv(\algebra{A}) = \{a_0 X^0 \mid a_0 \in \mathds{C} \setminus \{0\}\}$ the constant non-zero polynomials. If $a = \sum_{k=0}^{N} a_k x^k \in \algebra{A}$, then we have two cases:
	\begin{equation*}
		\sigma(a) = \left\{ \begin{matrix}
			\{a_0\}    & a = a_0 \text{ (constant)} \\
			\mathds{C} & \text{otherwise}
		\end{matrix}
		\right.
	\end{equation*}
\end{example}

\begin{example}
	Let $\algebra{A} = \mathds{C}(X) = \{p,q \mid p,q \in \mathds{C}[X], q \neq 0 \}$. Now we have $\inv(\algebra{A}) = \algebra{A} \setminus \{0\}$. If $a \in \algebra{A}$, then
	\begin{equation*}
		\sigma(a) = \left\{ \begin{matrix}
			\{a_0\}   & a = a_0 \text{ (constant)} \\
			\emptyset & \text{otherwise}
		\end{matrix}\right.
	\end{equation*}
\end{example}

\begin{example}
	Let $\algebra{A} = C(X)$ for any topological space $X$. Then
	\begin{equation*}
		\inv(\algebra{A}) = \{f \in C(X) \mid \forall_{x \in X} f(x) \neq 0\}
	\end{equation*}
	and
	\begin{equation*}
		\sigma(f) = \{\lambda \in \mathds{C} \mid \lambda - f \notin \inv(\algebra{A})\} = \{\lambda \in \mathds{C} \mid \exists_{x \in X} f(x) = \lambda\} = \image(f) = f(X)\text{.}
	\end{equation*}
\end{example}

\begin{example}
	Let $X$ be any topological space and consider $\algebra{A} = C_b(X)$. Then
	\begin{equation*}
		\inv(C_b(X)) = \{f \in C_b(X) \mid \exists_{\epsilon > 0} \forall_{x \in X} |f(x)| \geq \epsilon \}
	\end{equation*}
	and
	\begin{equation*}
		\sigma(f) = \{\lambda \in \mathds{C} \mid \lambda - f \notin \inv(\algebra{A}) \} = \{\lambda \in \mathds{C} \mid \exists_{(x_n)} f(x_n) \to \lambda\} = \closure{\image(f)} = \closure{f(X)} \text{.}
	\end{equation*}
	This is a compact subset of $\mathds{C}$.
\end{example}

\begin{theorem}[Algebraic spectral mapping theorem]
	Let $\algebra{A}$ be an algebra, $a \in \algebra{A}$ and $p \in \mathds{C}[X], p(X) = \sum_{k = 0}^{n} \lambda_k X^k$ and define $p(a) = \sum_{k = 0}^{n} \lambda_k a^k$. Recall that the mapping $\mathds{C}[X] \to \algebra{A}, p \mapsto p(a)$ is a unital homomorphism.

	Then $\sigma(p(a)) = p(\sigma(a))$ assuming $\sigma(a) \neq \emptyset$.
\end{theorem}

\begin{proof}
	If $p(X) = \lambda_0$ constant, this is clear (the spectrum is exactly $\lambda_0$ on both sides). Assume $p(x)$ is not constant. Fix $\mu \in \mathds{C}$ and write
	\begin{equation*}
		\mu - p(x) = \lambda_0 (x - \lambda_1) \cdots (x - \lambda_n)
	\end{equation*}
	as per the fundamental theorem of algebra (note that these are not the same $\lambda$ as before) with $\lambda_0 \neq 0$.  Then $\mu - p(a) = \lambda_0 (a - \lambda_1)\cdots(a - \lambda_n)$. Since these expressions commute, this product is invertible if and only if $(a - \lambda_i)$ is invertible for every $i$. So $\mu \in \sigma(p(a)) \iff \mu - p(a)$ is not invertible if and only if there exists an $i$ for which $\lambda_i - a$ is not invertible, so $\lambda_i \in \sigma(a)$. But the $\lambda_i$ are exactly the numbers satisfying $p(\lambda) = \mu$. Thus, $\mu$ is in $\sigma(p(a))$ if it is in the image of $\sigma(a)$ under $p$. Therefore, we conclude $\sigma(p(a)) = p(\sigma(a))$.
\end{proof}

We now focus on invertible elements in \textbf{Banach algebras}.

\begin{theorem}
	If $\algebra{A}$ is a unital Banach algebra and $a \in \algebra{A}$ with $\|a\| < 1$ then $1 - a$ is invertible and $(1-a)^{-1} = \sum_{n=0}^{\infty} a^n$.
\end{theorem}

\begin{proof}
	Observe that, since $\|a\| < 1$, we have $\sum_{n = 0}^{\infty} \|a\|^n = \frac{1}{1 - \|a\|} < \infty$. This implies the (absolute) convergence of $\sum_{n = 0}^\infty$ by the characteristic property of Banach spaces. Hence, $b \coloneqq \lim_{N \to \infty} \sum_{n = 0}^{N} a^n \in \algebra{A}$. No, if $N \in \mathds{N}$, then
	\begin{equation*}
		(1-a) \left(\sum_{n = 0}^{N} a^n\right) = \left(\sum_{n = 0}^{N} a^n\right) - \left(\sum_{n = 1}^{N + 1} a^n\right) = 1 - a^{N+1} \to 1
	\end{equation*}
	because of $\|a\| < 1$. This yields $(1-a)b = 1$.
\end{proof}

\begin{theorem}
	Let $\algebra{A}$ be a non-empty, non-zero unital Banach algebra. Then $\inv(\algebra{A})$ is an open subset of $\algebra{A}$ and the function $f: \inv(\algebra{A}) \to \algebra{A}, a \mapsto a^{-1}$ is Frechet-differentiable and in particular continuous as well as $f'(a) b = -a^{-1} b a^{-1}$.
\end{theorem}

Recall from calculus that $\frac{d}{dx} \frac{1}{x} = -\frac{1}{x^2}$. Also recall that $f: U \overset{\text{open}}{\subseteq} X \to Y$ with $X,Y$ Banach spaces is \textbf{differentiable} at $x_0 \in U$ there exists an operator $D_{x_0} = f'(x_0) \in \algebra{L}(X,Y)$ such that
\begin{equation*}
	\lim_{h \to 0} \frac{f\|(x_0 + h) - f(x_0) - D_{x_0}(h)\|}{\|h\|} = 0
\end{equation*}

\begin{proof}
	Take $a \in \inv(\algebra{A})$. If $b \in \algebra{A}$ such that $\|a - b\| < \|a^{-1}\|^{-1}$. From this, we have $\|b a^{-1} - 1\| = \| b a^{-1} - a a^{-1} \| = \|(b-a) a^{-1}\| \leq \|b - a\| \cdot \| a^{-1} \| < 1$. Per the previous theorem, $b a^{-1} \in \inv(\algebra{A})$. This implies that $b$ is also invertible. This shows that $\inv(\algebra{A})$ is open.

	Furthermore, if $\|b\| < 1$, then also ($\|-b\| < 1$). Thus, $1 + b \in \inv(\algebra{A})$ and $(1+b)^{-1} = \sum_{n = 0}^{\infty} (-1)^n b^n$. Thus,
	\begin{equation*}
		\| (1+b)^{-1} - 1 + b\| = \left\| \sum_{n = 0}^\infty (-1)^n b^n - 1 + b  \right\| \leq \left\| \sum_{n = 2}^\infty (-1)^n b^n \right\| \leq \sum_{n = 2}^{\infty} \|b^n\|  \leq \sum_{n = 2}^\infty \|b\|^n = \frac{\|b\|^2}{1 - \|b\|}
	\end{equation*}
	Now let $a \in \inf(\algebra{A})$ and $c \in \algebra{A}$ such that $\|c\| < \frac{1}{2} \|a^{-1}\|^{-1}$. Then $\|a^-1 c\| \leq \|a^{-1}\| \| c\| \leq \frac{1}{2}$. So if $b = a^{-1}$, then
	\begin{equation*}
		\|(1 + a^{-1}c)^{-1} - 1 + a^{-1} c\| = \leq \frac{\| a^{-1} c\|^2}{1 = \|a^{-1} c \|} < 2 \| a ^{-1} c \|^2
	\end{equation*}
	Now, define $U: \algebra{A} \to \algebra{A}, b \mapsto - a^{-1} b a^{-1}$. Then this is a linear odd operation with $\|U\| \leq \|a^{-1}\|^2$, and we have
	\begin{align*}
		\|(a + c)^{-1} - a^{-1} - U(c)\| & = \|(a+c)^{-1} - a^{-1} + a^{-1} c a^{-1}\|                  \\
		                                 & =  \|(1 + a^{-1}c)^{-1} a^{-1} - a^{-1} + a^{-1} c a^{-1}\|  \\
		                                 & \leq \|(1 + a^{-1}c)^{-1} -1 + a^{-1} c  \| \cdot \|a^{-1}\| \\
		                                 & \leq 2 \|a^{-1} c\|^2 \|a^{-1}\| \leq 2 \|a^{-1}\|^3 \|c\|^2
	\end{align*}
	and thus
	\begin{equation*}
		\lim_{c \to 0} \frac{\|(a + c)^{-1} - a^{-1} - U(c)\|}{\|c\|} = 0
	\end{equation*}
\end{proof}

\begin{example}
	If we choose $\algebra{A} = \mathds{C}[X]$ and the norm $\|p\| = \sup_{\lambda \in [0,1]} |p(x)|$.
	Then $(\algebra{A}, \|\cdot \|)$ is a normed (but not Banach) algebra.
	For example, we see that $\lim_{m \to 0} 1 + X/m = 1 \in \inv(\algebra{A})$, but $1 + X/m \notin \inv(\algebra{A})$ and thus $\inv(\algebra{A})$ is not open (because the complement is not closed).
\end{example}

\begin{theorem}
	If $\algebra{A}$ is a Banach algebra with unit $1$, then for all $a \in \algebra{A}$ the spectrum $\sigma(a) \subseteq \mathds{C}$ is closed and $\sigma(a) \subseteq \overline{B(0, \| a \|)} = D(0, \|a\|) \coloneq \{\lambda \in \mathds{C} \mid |\lambda| \leq \|a\| \} $. Therefore, $\sigma(a)$ is compact by the Heine-Borell theorem.
\end{theorem}

\begin{proof}
	By definition
	\begin{equation*}
		\sigma(a) = \{ \lambda \in \mathds{C} \mid \lambda - a \notin \inv(\algebra{A}) \}
	\end{equation*}
	is the inverse image of the closed subset $\algebra{A} \setminus \inv(\algebra{A}) \subseteq \algebra{A}$ by the continuous function $\lambda \mapsto \lambda - a$.
	Therefore, $\sigma(a)$ is closed.

	Now if $|\lambda| \leq \|a\|$ then $\|\lambda^{-1} a\| < 1$.
	Then $1 - \lambda^{-1} a \in \inv(\algebra{A})$.
	Multiplying by $\lambda$ yields $\lambda - a \in \inv(\algebra{A})$.
	Thus, $\{\lambda \in \mathds{C} \mid |\lambda| > \|a\|\} \subseteq \rho(a)$ and thus $\sigma(a) \subseteq D(0, \|a\|)$.
\end{proof}

\begin{lemma}
	Let $\algebra{A}$ be a unital Banach algebra and $a \in \algebra{A}$.
	Then, the map $R_a: \rho(a) \subseteq \mathds{C} \to \algebra{A}, \lambda \mapsto (a - \lambda)^{-1}$ is Frechet-differentiable.
\end{lemma}

\begin{proof}
	This follows from the following general result:

	If $g: U \overset{\text{open}}{\subseteq} X \to Y$ and $f: V \overset{\text{open}}{\subseteq} Y \to Z$ for Banach spaces $X,Y,Z$ with $g(U) \subseteq V$ are differentiable at $x_0 \in U$ or respectively $y_0 = g(x_0) \in V$, then $f \circ g$ is differentiable and $(f \circ g)' (x_0) = f'(g(x_0)) g'(x_0)$.
\end{proof}

Observation: For $R_a(\lambda) = (a - \lambda)^{-1}$ we get $R_a'(\lambda) = (a - \lambda)^{-2}$.
We have $\mathcal{L}(\mathds{C}, \algebra{A}) \simeq \algebra{A}$ by $T \mapsto T(1)$.
Recall that if $f(a) = a^{-1}$ yields $f'(a) b = - a^{-1} b a^{-1}$.

\begin{theorem}[Gelfand]
	If $\algebra{A} \neq 0$ is a unital Banach algebra and $a \in \algebra{A}$ then $\sigma(a) \neq \emptyset$.
\end{theorem}

\begin{proof}
	Suppose $\sigma(a) = \emptyset$.
	Idea: Show that $R_a: \rho(a) \subseteq \mathds{C} \to \algebra{A}, \lambda \mapsto (a - \lambda)^{-1} = \frac{1}{a - \lambda}$ is bounded and differentiable and achieve a contradiction by Liouville's theorem.

	Claim: $\| (a - \lambda)^{-1}\| < \|a\|^{-1}$ if $|\lambda| > 2 \|a\|$.
	Indeed, if $|\lambda| > 2 \|a\|$ then $\| \lambda^{-1} a \| < \frac{1}{2}$, and in particular $1 - \lambda^{-1} a \in \inv(\algebra{A})$ and
	\begin{equation*}
		\left\| (1 - \lambda^{-1} a)^{-1} - 1 \right\| = \left\| \sum_{n = 1}^{\infty} (\lambda^{-1} a)^{-1} \right\| \leq \sum_{n = 1}^{\infty} \|\lambda^{-1} a\|^n = \frac{\| \lambda^{-1} a \|}{1 - \|\lambda^{-1} a\|} \leq 2 \|\lambda^{-1} a\| < 1\text{.}
	\end{equation*}
	From here we deduce that $\|(1 - \lambda^{-1}a)^{-1}\| < 2$ and thus
	\begin{equation*}
		\|(a - \lambda)^{-1}\| < \|\lambda^{-1} (\lambda^{-1}a - 1)^{-1}\| = \frac{\|(1 - \lambda^{-1} a)^{-1}\|}{|\lambda|} < \frac{2}{\lambda} < \frac{1}{\|\lambda\|}\text{.}
	\end{equation*}

	So $R_a: \mathds{C} \to \algebra{A}$ is bounded outside $\closure{B(0, 2 \|a\|}$.
	Since $R_a$ is continuous, it is bounded on $\mathds{C} \to \algebra{A}$.
	Let $\phi \in \algebra{A}^*$ be a bounded linear functional in $\mathcal{L}(\algebra{A}, \mathds{C})$.
	Thus, $\phi$ is differentiable with $\phi'(a) = \phi$ for all $a \in \algebra{A}$.
	Then $\phi \circ R_a$ is differentiable and bounded, so it is an \enquote{integer} function.
	By Liouville's theorem, $\phi \circ R_a$ is constant.
	Therefore, $\phi \circ R_a(x) = \phi \circ R_a(y)$ for all $x,y \in \algebra{A}$.
	Especially, we have $\phi((a - \lambda)^{-1}) = \phi(a^{-1})$ for all $\phi$.
	Hahn-Banach shows $(a - \lambda)^{-1} = a^{-1}$ for all $\lambda$, proving $a - \lambda = a$ for all $a,\lambda$.
	This is a contradiction.
\end{proof}

\begin{theorem}[Gelfand-Mazur]
	If $\algebra{A}$ is a unital Banach algebra and every $a \neq 0$ admits an inverse ($\algebra{A}$ is a field), then $\algebra{A} = \mathds{C} \cdot 1$.
\end{theorem}

\begin{proof}
	By the assumption, $\inv(\algebra{A}) = \algebra{A} \setminus \{0\}$.
	By the previous theorem, if $a \in \algebra{A}$ there exists some $\lambda \in \sigma(a)$, so $a - \lambda \notin \inv(\algebra{A})$, so $a - \lambda = 0$ and thus $a = \lambda \cdot 1$.
\end{proof}

\begin{corollary}
	Let $\mathds{C}(X) = \left\{ \frac{p(x)}{q(x)} \mid p(x), q(x) \in \mathds{Q}[X] \right\}$ is a field, but it cannot be turned into a Banach algebra.
\end{corollary}

\begin{theorem}[Adjointing units - unitization of algebras]
	Let $\algebra{A}$ be any algebra.
	Consider $\tilde{ \algebra{A}} = \algebra{A} \oplus \mathds{C}$ as a vector space.
	We write elements of $\tilde{\algebra{A}}$ as $a + \lambda \cdot 1 \coloneq (a, \lambda)$. Think of $a = (a,0)$ and $\lambda = (a, \lambda)$.
	Define
	\begin{align*}
		(a + \lambda 1) (b + \lambda' 1) & = (ab + \lambda' a + \lambda b) + \lambda \cdot \lambda'\text{.}
	\end{align*}
	Ten (exercise $\tilde{\algebra{A})}$ becomes a unital algebra with $1_\algebra{A} = 1 = (0,1)$.

	Notice that $\algebra{A}$ is an ideal in $\tilde{\algebra{A}}$.

	Moreover, we get a short exact sequence
	\begin{equation*}
		0 \to \algebra{A} \hookrightarrow \tilde{\algebra{A}} \to \mathds{C} \to 0
	\end{equation*}
	so $1 + \lambda \mapsto \lambda$.

	If $\algebra{A}$ is a normed algebra, then $\tilde{\algebra{A}}$ is normed by $\|a + \lambda \cdot 1\| \coloneq \|a\| + |\lambda|$

	If $\algebra{A}$ is Banach and closed, then so is $\tilde{\algebra{A}}$.

	If $\algebra{A}$ is a $^*$-algebra, then so is $\tilde{\algebra{A}}$ with $(a + \lambda 1)^*$.

	If $\algebra{A}$ is a (Banach) normed $^*$-algebra, then so is $\tilde A$.

	If $\algebra{A}$ is a $C^*$-algebra, in general the norm given above is not a Norm on $\algebra{A}$, but $\|a + \lambda \cdot 1\| \coloneq \sup_{b \in \algebra{A}, b \in \algebra{B}, b \leq 1} \|a b + \lambda b\|$ is.
\end{theorem}

\begin{exercise}\label{exe:AplusC}
	If $\algebra{A}$ is already unital, then $\tilde A \simeq A \oplus \mathds{C}$ as algebras by $a + \lambda \cdot 1 \mapsto (a + \lambda 1_\algebra{A}, -\lambda) $.
\end{exercise}

% \begin{definition}
% 	If $\algebra{A}$ is any algebra, then $\sigma_A(a) \coloneq \sigma_{\tilde A}(a)  = \sigma_{\tilde A}(a,0)$.
% 	Thus, we can identify $A \hookrightarrow \tilde A$.
% \end{definition}

% \begin{corollary}~
% 	\begin{enumerate}
% 		\item If $A$ is any algebra, then we always have $0 \in \sigma_A(a)$ because if $a \in A$ then $a \notin \inv(\tilde{\algebra{A}})$ as $A \trianglelefteq \tilde{\algebra{A}}$.
% 		\item 

% 		Observation: If $\algebra{A}$ is unital, then $\tilde{ \algebra{A}} = \{a + \lambda 1 \mid a \in \algebra{A}, \lambda \in \mathds{C}\} \simeq \algebra{A} \oplus \mathds{C}$ as algebras via $a + \lambda 1 \mapsto (a + \lambda 1_\algebra{A}, \lambda)$.
% 		For $p = 1_\algebra{A} \in Z(\tilde{\algebra{A}})$, we have $\tilde{\algebra{A}} \simeq p \tilde{ \algebra{A}} \oplus (1-p) \tilde{\algebra{A}}$.
% 		This yields $(a + \lambda 1)(1 - 1_\algebra{A}) = a - a + \lambda ( 1 - 1_\algebra{A}) = \lambda q$ for $q \coloneq 1 - 1_\algebra{A}$.
% 		Thus, we have $\sigma_{\tilde{\algebra{A}}} = \sigma_\algebra{A}(a) \cup \{0\}$.

% 		      % \item If $\algebra{A}$ is \textbf{not} already unital and $a \in \algebra{A}$, then $\sigma_A(a) = \sigma_{\tilde A}(a)$.
% 		      % From \ref{exe:AplusC} we know that $a - \lambda \cdot  1 \in \inv(\tilde{\algebra{A}})$ if and only if $a - \lambda 1_\algebra{A} \in \inv(\algebra{A})$ and $\lambda (1 - 1_\algebra{A})  \in \inv(\mathds{C})$.
% 		      % Then $\rho_\algebra{A} = \{\lambda \in \rho_{\tilde{\algebra{A}}} \mid \lambda \neq 0\}$.
% 		      % Therefore, $\sigma_\algebra{A}(a) = \{0\} \cup \sigma_{\tilde{\algebra{A}}}(a)$.
% 	\end{enumerate}
% \end{corollary}

\begin{definition}
	Re-Definition: If $\algebra{A}$ is non-unital, then $\tilde{A} + \mathds{C} \cdot 1$ is a $(*-)$Banach algebra, and we define $\sigma_A(a) \coloneq \sigma_{\tilde{\algebra{A}}}(a)$.
\end{definition}

Observation: If $\algebra{A}$ is already unital, then for $\tilde{A} \simeq \algebra{A} \oplus \mathds{C}$ we have $\sigma_{\tilde{\algebra{A}}}(a) = \sigma_\algebra{A}(a) \cup \{ 0\}$.

\begin{remark}
	If $\algebra{A}$ is a $C^*$-algebra, then $\tilde{\algebra{A}}$ is a $C^*$-algebra.
	\begin{enumerate}
		\item If $ \algebra{A}$ is unital, then $\tilde{\algebra{A}} \simeq \algebra{A} \oplus \mathds{C}$ and $\|a + \lambda \cdot 1\| = \max\{\|a + \lambda \cdot 1\|, |\lambda|\}$.
		\item If $\algebra{A}$ is not unital, then $\|a + \lambda \cdot 1\| = \sup_{\|b\| \leq 1} \|ab + \lambda b\|$.
	\end{enumerate}
\end{remark}

\section{Spectral Radius}

\begin{definition}
	Let $\algebra{A}$ be an algebra. Given $a \in \algebra{A}$, we define:
	\begin{equation*}
		r(a) \coloneq \sup \{ |\lambda| \mid \lambda \in \sigma_\algebra{A}(a) \}
	\end{equation*}
	as the \textbf{spectral radius} of $a$ if $\emptyset \neq \sigma_\algebra{A}(a)$ is bounded (e.g. if $\algebra{A}$ is Banach).
\end{definition}

Observation: In a Banach algebra, we have $0 \leq r(a) \leq \|a\|$.

\begin{example}~
	\begin{enumerate}
		\item Let $f \in \algebra{A} = C_0(X)$ using $\sigma_A(f) = \closure{f(X)}$. Thus,
		      \begin{equation*}
			      r(f) = \sup \{ |\lambda| \mid \lambda \in \closure{f(X)} = \sup_{x \in X} |f(x)| = \|f\|_{C_0(X)}
		      \end{equation*}
		\item Let $\algebra{A} = M_2(\mathds{C})$ and $a = \qmatrix{0 & 1 \\ 0 & 0}$. Then $\sigma_\algebra{A} = \{0\}$ and $r(a) = 0$, but $\|a\| = 1 \neq 0$.
	\end{enumerate}
\end{example}

\begin{theorem}[Beurling-Gelfand]
	Let $\algebra{A}$ be a Banach algebra, then
	\begin{equation*}
		r(a) = \inf_{n \in \mathds{N}} \|a^n\|^{\frac{1}{n}} = \lim_{n \to \infty} \|a^n\|^{\frac{1}{n}}
	\end{equation*}
\end{theorem}

\begin{proof}
	We may assume $\algebra{A}$ is unital (otherwise we consider $\tilde{\algebra{A}}$). If $\lambda \in \sigma(a)$, then
	\begin{equation*}
		\lambda^n \in \sigma(a^n) \then |\lambda^n| \leq \| a^n \| \then |\lambda| \leq \|a\|^{\frac{1}{n}} \quad \forall_{n \in \mathds{N}}
	\end{equation*}
	and therefore
	\begin{equation*}
		r(a) \leq \inf_{n \in \mathds{N}} \|a^n\|^{\frac{1}{n}} \leq \liminf_{n \to \infty} \|a^n\|^{\frac{1}{n}}\text{.}
	\end{equation*}
	We prove now that $\limsup_{n \to \infty} \|a^n\|^{\frac{1}{n}} \leq r(a)$. Set $\Delta \coloneq B\left(0, \frac{1}{r(a)}\right)$. Where per convention we set $\frac{1}{r(a)} = \infty$ if $r(a) = 0$. If $\lambda \in \Delta$, then $1 - \lambda a \in \inv(\algebra{A})$ (because $|\lambda| < \frac{1}{r(a)}$ implies $|\lambda^{-1}| > r(a)$ and therefore $\lambda^{-1} \notin \sigma(a) \then \lambda^{-1} - a \in \inv{A} \then 1 - \lambda a \in \inv(A)$).

	Now fix $\phi \in \algebra{A}^*$. Then $f: \Delta \to \mathds{C}, \lambda \mapsto \phi((1 - \lambda a)^{-1})$ is analytic, so it can be written as
	\begin{equation*}
		f(x) = \sum_{n = 0}^{\infty} a_n \lambda^n, a_n = \frac{f^{(n)}(0)}{n!} \in \mathds{C}, \lambda \in \Delta\text{.}
	\end{equation*}
	On the other hand, if
	\begin{equation*}
		|\lambda| < \frac{1}{\|a\|} \leq \frac{1}{r(a)}
	\end{equation*}
	then $\|\lambda a \| < 1$, so
	\begin{equation*}
		(1 - \lambda a)^{-1} = \sum_{n = 0}^{\infty} \lambda^n a^n \then f(\lambda) = \phi((1 - \lambda)^{-1}) = \sum_{k = 0}^{\infty} \phi(a^n) \lambda^n
	\end{equation*}
	for $|\lambda| < \frac{1}{\|\lambda\|}$.

	By uniqueness of the Taylor series expansion, it follows that
	\begin{equation*}
		a_n = \phi(a^n)	 \forall_{n \in \mathds{N}}\text{.}
	\end{equation*}
	In particular, $(\phi(a^n) \lambda^n)$ converges to zero for all $\lambda \in \Delta$ and thus $(\phi(a^n) \lambda^n)$ is bounded for all $\lambda \in \Delta$.

	From the principle of uniform convergence, it follows that $(a^n \lambda^n)$ is bounded. So there exists an $M = M_\lambda$ such that
	\begin{align*}
		      & \|\lambda^n a^n\| \leq M \forall_{n \in \mathds{N}}                                                                                           \\
		\then & \| \lambda^n \|^{\frac{1}{n}} \leq \frac{M^{\frac{1}{n}}}{|\lambda|} \forall_{n \in \mathds{N}}, \forall_{\lambda \in \Delta, \lambda \neq 0} \\
		\then & \limsup_{n \to \infty} \|a^n\|^{\frac{1}{n}} \leq \frac{1}{\lambda} \forall_{\lambda \in \Delta \text{ i.e. } |\lambda| < \frac{1}{r(a)}}
	\end{align*}
	Letting $\lambda < \frac{1}{r(a)}$ yields $\limsup_{n \to \infty} \|a^n\|^{\frac{1}{n}} \leq r(a)$.
\end{proof}

\begin{example}
	Let $A = C^1([0,1]) = \{ I \in C[0,1] \mid \exists_{f'(t)} \forall_{t \in [0,1]}, t \mapsto f'(t) \text{ continuous} \}$ with $\|f\| = \|f\|_\infty + \|f'\|_\infty$.

	Then $\algebra{A}$ is unital, commutative and a Banach algebra. Consider $x \in \algebra{A}$, $x(t) = t$. We have $x^n(t) = t^n$ and
	\begin{align*}
		\|x^n\| & = \sup_{t \in [0,1]} |t^n| + \sup_{t \in [0,1]} |n t^{n-1}| = 1 + n \\
		r(x)  & = \lim_{n \to \infty} (1+n)^\frac{1}{n} = 1                         \\
		\| x \| & = 2
	\end{align*}
	Observation: $\sigma(x) = \image(x) = [0,1]$.
\end{example}

\begin{theorem}
	Let $\algebra{B} \nsubseteq \algebra{A}$ be an inclusion of unital Banach algebras with $1 = 1_\algebra{A} = 1_\algebra{B}$. Then $\sigma_\algebra{A}(b) \subseteq \sigma_\algebra{B}(b)$ for all $b \in \algebra{B}$ and the inclusion may be proper. If $\sigma_\algebra{A}(b)$ is simply connected (not holes), then $\sigma_\algebra{A}(b) = \sigma_\algebra{B}(b)$.\\
	The holes of a compact subset $K \subseteq \mathds{C}$ are the bounded connected components of $\mathds{C} \setminus K$. So saying that $K$ has no holes means that $\mathds{C} \setminus K$ is connected.
\end{theorem}

\begin{proof}
	See Murphy, 1.2.8.
\end{proof}

\begin{example}
	Let $\algebra{B} \coloneq A(\mathds{D}) = \{f \in C(\mathds{D}) \mid f \text{ analytic on } \mathds{D}^\interior\}$ and $\algebra{A} = C(\mathds{S}^1)$.
	Then we have an embedding by $\iota: \algebra{B} \hookrightarrow \algebra{A}, f \mapsto f|_{\mathds{S}^1}$.

	By the principle of maximum modules, $\iota$ is an embedding of (unital) Banach algebras.
	Consider: $f(z) = z$ for $z \in \mathds{D}$.
	(Observation: $\closure{Alg}(1,z) = A(\mathds{D})$)
	Then:
	\begin{equation*}
		\sigma_{A(\mathds{D})}(f) = f(\mathds{D}) = \mathds{D}
	\end{equation*}
	and $\sigma_{C(\mathds{S}^1)}(f|_{\mathds{S}^1}) = \mathds{S}^1$.
\end{example}

\begin{definition}[Exponentials]
	Let $\algebra{A}$ be a unital Banach algebra, given $a \in \algebra{A}$ we define
	\begin{equation*}
		e^a = \exp(a) = \sum_{n = 0}^{\infty} \frac{a^n}{n!}
	\end{equation*}
	Note $\left\| \frac{a^n}{n!} \right\| \leq \frac{\|a\|^n}{n!}$, so the series converges and $\| \exp(a) \| \leq \exp(\|a\|)$.
\end{definition}

\begin{theorem}~
	\begin{enumerate}
		\item Let $\algebra{A}$ be a unital Banach algebra. If $a \in \algebra{A}$, then $f: \mathds{R} \to \algebra{A}, t \mapsto \exp(ta)$ is the unique solution of
		      \begin{equation*}
			      \left\{ \begin{matrix}
				      f'(t) & = a f(t) \\
				      f(0)  & = 1
			      \end{matrix} \right.
		      \end{equation*}
		\item $e^a \in \inv(\algebra{A})$ and $(e^a)^{-1} = e^{-a}$.
		\item If $a,b \in \algebra{A}$ then $e^{a + b} = e^a \cdot e^b$ (here some commutativity is neccessary).
	\end{enumerate}
\end{theorem}

\begin{proof}
	See Murphy, 1.2.9.
\end{proof}


\section{Gelfand Representation for commutative Banach algebras}

\underline{Idea}: Given a commutative algebra $\algebra{A}$, we want to represent $\algebra{A}$ by a homomorphism $\phi: \algebra{A} \to C_0(X)$ for $X$ some locally compact Hausdorff space. We hope that $\phi$ is injective, or even isometric, on an isomorphism. But what is $X$, and what is $\phi$?

Notice that, if $\algebra{A} = C_0(X)$ already, then for each $x \in X$ we get a character $\mathrm{ev}_x: \algebra{A} \to \mathds{C}, f \mapsto f(x)$.

\begin{definition}
	Given an algebra $\algebra{A}$, we define
	\begin{equation*}
		\hat{\algebra{A}} = \Omega(\algebra{A}) \coloneq \{\chi: \algebra{A} \to \mathds{C} \mid \chi \text{ non-zero homomorphism}\}\text{.}
	\end{equation*}
\end{definition}

\begin{example}~
	\begin{enumerate}
		\item For $\algebra{A} = C_0(X)$ we get a map
		      \begin{equation*}
			      X \to \Omega(\algebra{A}), x \mapsto \mathrm{ev}_x
		      \end{equation*}
		      that is a bijection. After we give $\Omega(\algebra{A})$ an appropriate topology, it will also be a homomorphism.
		\item Let $\algebra{A} = M_2(\mathds{C})$ (or any $M_n(\mathds{C})$). This is a simple algebra, so non-zero homomorphisms $\chi: \algebra{X} \to \mathds{C}$ do not exist (same for any $\algebra{A}$ with dimension $>1$).

		      So in this case we have $\Omega(\algebra{A}) = \emptyset$. This can also happen in commutative algebras.
		\item Consider
		      \begin{equation*}
			      \algebra{A} = \left\{ \qmatrix{0 & \lambda \\ 0 & 0} \mid \lambda \in \mathds{C}  \right\}
		      \end{equation*}
		      Then for all $a \in \algebra{A}$ we have $a^2 = 0$, so if $\chi: \algebra{A} \to \mathds{C}$ is an homomorphism, then $\chi(a)^2 = \chi(a^2) = 0$, so $\chi(a) = 0$ for all $a \in \algebra{A}$.
		      So again, $\Omega(\algebra{A}) = \emptyset$ (and $\algebra{A}$ is commutative with $\dim \algebra{A} = 1$).
	\end{enumerate}
\end{example}

Question: Given an abstract algebra $\algebra{A}$ how do we possibly find its characters?

Idea: Assume that $\ring{I} \triangleleft \algebra{A}$ is a maximal ideal and $\algebra{A}$ is a unital Banach algebra.
Then $\algebra{A} / \ring{I} \simeq \mathds{C}$ and $\chi \in \Omega(\algebra{A})$.

\begin{theorem}
	Let $\algebra{A}$ be a unital non-zero Banach algebra.
	If $\chi \in \Omega(\algebra{A})$ then $\| \chi \| = \sup_{\|a\| = 1} |\chi(a)| = 1$ and $\ker(\chi) \triangleleft \algebra{A}$.
	So $\chi \in \algebra{A}^*$ (the topological dual of $\Omega(\algebra{A}) \subseteq D_{\algebra{A}^*}(0,1)$).

	Moreover, if $\algebra{A}$ is a unital Banach commutative algebra, then $\Omega(\algebra{A}) \ni \chi \mapsto \ker(\chi) \triangleleft \algebra{A}$ is a bijection between of characters of $\algebra{A}$ and maximal ideals of $\algebra{A}$.
\end{theorem}

\begin{proof}
	If $a \in \algebra{A}$ and $\chi$ a character, then $\chi(a) \in \sigma(\algebra{A})$, because $\chi(a - \chi(a) \cdot 1) = \chi(a) - \chi(a) \cdot \chi(1) = 0$, so $a - \chi(a) \cdot 1 \in \ker(\chi) \triangleleft \algebra{A}$ and thus $a - \chi(a) \cdot 1 \notin \inv(\algebra{A})$.

	Therefore: $|\chi(a)| \leq r(a) \leq \|a\|$. So $\|\chi\| \leq 1$. Since $\chi(1) = 1$ and $\|1\| = 1$ we have $\|\chi\| = 1$.

	Now, apply linear algebra. Then $\ker (\chi)$ is a maximal proper subspace, in particular a maximal ideal.
	And $\ker(\chi)$ is closed, because $\chi$ is continuous.
	Now assume that $\algebra{A}$ is commutative (in addition to unital and Banach).
	Then we have the mapping
	\begin{equation*}
		\phi: \Omega(\algebra{A}) \to \mathrm{MaxIdeals}(\algebra{A}), \chi \to \ker(\chi)\text{.}
	\end{equation*}
	\begin{itemize}
		\item $\phi$ is injective.
		      If $\ker(\chi_1) = \ker(\chi_2)$ for $\chi_1, \chi_2 \in \algebra{A}$, then for every $a \in \algebra{A}$ we have $a - \chi_1(a) \cdot 1 \in \ker(\chi_1) =\ker(\chi_2)$.
		      Thus, $\chi_2(a = \chi_1(a) \cdot 1) = 0$ and therefore $\chi_2(a) = \chi_1(a)$ for every $\algebra{A}$.
		\item $\phi$ is surjective.
		      Take $\ring{I} \triangleleft  \algebra{A}$ a maximal ideal.
		      Then $\ring{I} = \closure{\ring{I}}$ because $\closure{\ring{I}} \neq \algebra{A}$, otherwise $1 \in \closure{\ring{I}}$ and since $\inv(\algebra{A})$ is open in $\algebra{A}$, we get $\ring{I} \cap \inv(\algebra{A}) \neq \emptyset$.
		      But then we have an invertible element in the ideal $\ring{I}$ already, but this implies the contradiction $\ring{I} = \algebra{A}$.
		      Therefore, $\algebra{A} / \ring{I}$ is a commutative, unital Banach algebra which is simple ($\ring{I}$ is maximal).
		      \begin{quote}
			      Exercise: If $\ring{I} \triangleleft \algebra{A}$, then $\algebra{A} / \ring{I}$ is field if and only if there exists no $\ring{J} \triangleleft \algebra{A}$ such that $\ring{I} \triangleleft \ring{J}$.
		      \end{quote}
		      Thus, $\algebra{A} / \ring{I}$ is a field and $\algebra{A} / \ring{I} \simeq \mathds{C}$.
		      Then the composition
		      \begin{equation*}
			      \algebra{A} \xrightarrow{q} \algebra{A} / \ring{I} \simeq \mathds{C}
		      \end{equation*}
		      is a character with $ \ker(\chi) = \ring{I}$.
	\end{itemize}
\end{proof}

\begin{exercise}
	An application of Zorn's Lemma. Show that every ideal $I \triangleleft \algebra{A}$ in a unital algebra $\algebra{A}$ is contained in a maximal ideal.
\end{exercise}

In particular, we can apply this to $\ring{I} = 0$ in $\algebra{A} \neq 0$ (with $\algebra{A}$ is unital and commutative) and thus $\Omega(\algebra{A}) \neq \emptyset$.

\subsection*{Topology on $\Omega(\algebra{A})$}

We have for $\algebra{A}$ a Banach algebra. We can add a unit to receive $\tilde{\algebra{A}}$, which is a Banach algebra.

Observe: If $\chi \in \Omega(\algebra{A})$, then there exists a unique $\tilde{X} \in \Omega(\tilde{\algebra{A}})$ via $\tilde{X}(a + \lambda \cdot 1) = \chi(a) + \lambda$. Thus, $\|\chi\| \leq \|\tilde{X}\| = 1$ (Note that it may still be smaller than $1$. See exercises 2023-05-09).

In any case,
\begin{equation*}
	\Omega(\algebra{A}) \subseteq D_{\algebra{A}^*}(0,1) = \{ \phi \in \algebra{A}^* \mid \| \phi \| \leq 1 \}
\end{equation*}
and $\algebra{A}^*$ carries the weak $^*$-topology (the smallest topology to make all point-evaluations continuous, that is for a net $(\phi_i)  \subset \algebra{^*}$ weakly converging to $\phi \in \algebra{A}^*$ if and only if $\phi_i(a) \to \phi(a)$ for all $a \in A$).

\begin{definition}
	Given a Banach algebra $\algebra{A}$, we endow $\Omega(\algebra{A})$ with the weak $^*$-topology and call this the \textbf{Gelfand spectrum} of $\algebra{A}$.
\end{definition}

\begin{proposition}
	$\Omega(\algebra{A})$ is a locally compact Hausdorff space. If $\algebra{A}$ is unital, then $\Omega(\algebra{A})$ is compact.
\end{proposition}

\begin{proof}
	By Banach-Alaoglu-Theorem, $D_{\algebra{A}^*}(0,1)$ is compact and Hausdorff with the weak $^*$-topology.
	Let
	\begin{align*}
		S & \coloneq \{\chi: A \to \mathds{C} \mid \chi \text{ hom.} \} \\
		  & = \Omega(\algebra{A}) \cup \{0\}
	\end{align*}
	Then $S \subseteq D_{\algebra{A}^*}(0,1)$.
	So $\chi(ab) = \lim_{i \to \infty} K_i = \lim_{i \to \infty} \chi_i(a) \chi_i(b) = \chi(a) \chi(b)$ and therefore $x \in S$.
	Thus, $S$ is a compact Hausdorff space and $\Omega(\algebra{A}) = S \setminus \{0\}$ is relatively compact.

	If $\algebra{A}$ is unital, then $\Omega(\algebra{A}) \subseteq D_{\algebra{A}^*}(0,1)$ is closed. Then we have $(X_i) \subseteq \Omega(\algebra{A})$ and $X_i \to X \in \algebra{A}^*$ and thus $X \in S = \hom(\algebra{A}, \mathds{C})$.
\end{proof}

Observation: Given a Banach algebra $\algebra{A}$, we have an isomorphism
\begin{equation*}
	\Omega(\tilde{\algebra{A}}) \to \Omega(\algebra{A}) \sqcup \{ \chi_\infty\}, \phi \mapsto \left\{  \begin{matrix}
		\phi|_\algebra{A} & \phi|_{\algebra{A}} \neq 0 \\
		\chi_\infty       & \phi|_\algebra{A} = 0
	\end{matrix}\right. \text{,} \label{obs:unitization_iso}
\end{equation*}
where $\chi_\infty(a + \lambda \cdot 1) = \lambda$. Thus, $\Omega(\algebra{A}) \sqcup \{\chi_\infty\}$ is already the unitization of $\Omega(\algebra{A})$.

\begin{theorem}
	Let $\algebra{A}$ be a Banach algebra. Then for every $a \in \algebra{A}$.
	\begin{equation*}
		\{ \chi (a) \mid \chi \in \Omega(\algebra{A}) \} \subseteq \sigma(a)
	\end{equation*}
	If $\algebra{A}$ is commutative, then
	\begin{itemize}
		\item $\{ \chi(a) \mid \chi \in \Omega(\algebra{A})\} = \sigma(a)$ in case $\algebra{A}$ is unital.
		\item $\{ \chi(a) \mid \chi \in \Omega(\algebra{A})\} \cup \{0\} = \sigma_{\algebra{A}}(a)$.
	\end{itemize}
\end{theorem}

\begin{proof}~
	\begin{itemize}
		\item $\algebra{A}$ is unital and $a \in \algebra{A}$. $\chi(a - \chi(a) \cdot 1) = 0$, so $\chi(a) \in \sigma(a)$, so $\{ \chi(a) \mid x \in \Omega(a) \}  \subseteq \sigma(a)$.

		      Now if $\lambda \in \sigma(a)$, consider $\ring{I} \coloneq (a - \lambda \cdot 1) \algebra{A} \triangleleft \algebra{A}$ if $\algebra{A}$ is commutative.
		      By Zorns Lemma, we get $I \subseteq J \triangleleft \algebra{A}$ with $J = \ker(\chi)$ for some $\chi \in \Omega(\algebra{A})$. Thus we have $a - \lambda \cdot 1 \in \ring{I} \subseteq J = \ker(\chi)$ so $\chi(a) = \lambda$.

		\item $\algebra{A}$ is not unital. Consider $\tilde{\algebra{A}}$. By the first part,
		      \begin{equation*}
			      \sigma_\algebra{A}(a) = \sigma_{\tilde{\algebra{A}}}(a) \supseteq \{\chi(a) \mid \chi \in \Omega(\tilde{\algebra{A}}) \} = \{\chi(a) \mid \chi \in \Omega(\algebra{A}) \} \cup \{0\}
		      \end{equation*}
		      If $\algebra{A}$ is commutative, by the first part again:
		      \begin{equation*}
			      \sigma_\algebra{A}(a) = \sigma_{\tilde{\algebra{A}}}(a) = \{\chi(a) \mid \chi \in \Omega(\tilde{\algebra{A}})\} = \{\chi(a) \mid \chi \in \Omega(\algebra{A}) \} \cup \{0\}
		      \end{equation*}
	\end{itemize}
\end{proof}

\subsection{Gelfand-Transformation}

\begin{definition}
	Given a Banach algebra $\algebra{A}$ and $a \in \algebra{A}$, we define $\hat a: \Omega(\algebra{A}) \to \mathds{C}, \chi \mapsto \chi(a)$.
\end{definition}

Observe that $\hat a \in C(\Omega(\algebra{A}))$, because if $\chi_i \to \chi$ then we have $\hat a(\chi_i) = \chi_i(a) \to \chi(a) = \hat a(\chi)$. So we have a map $\Gamma: \algebra{A} \to C(\Omega(\algebra{A}))$. This map is called the \textbf{Gelfand transform} of $\algebra{A}$.

\begin{theorem}[Gelfand Representation]
	$\image(\Gamma) \subseteq C_0(\Omega(\algebra{A}))$ and $\Gamma: \algebra{A} \to C_0(\Omega(\algebra{A}))$ is a contractive homomorphism, i.e. $\|\Gamma(a) \| \leq r(a) \leq \|a\|$ for every Banach algebra $\algebra{A}$.
	If moreover $\algebra{A}$ is commutative, then $\|\Gamma(a)\| = r(a)$.
	Also, for all $a \in \algebra{A}$, we have
	\begin{equation*}
		\sigma(a) = \left\{ \begin{matrix}
			\image(\hat a)            & \algebra{A} \text{ unital} \\
			\image(\hat a) \cup \{0\} & \text{otherwise}
		\end{matrix}  \right. \text{.}
	\end{equation*}
\end{theorem}

\begin{proof}
	If $\algebra{A}$ is unital, then $\Omega(\algebra{A})$ is compact so $\image(\Gamma) \subseteq C(\Omega(\algebra{A})) = C_0(\Omega(\algebra{A}))$. If $\algebra{A}$ is not unital, we use observation \ref{obs:unitization_iso}. Then we have $\Omega(\tilde{\algebra{A}}) \simeq \Omega(\algebra{A}) \cup \{\chi_\infty\}$ so that
	\begin{equation*}
		C_0(\Omega(\algebra{A})) \simeq \{ f \in C(\Omega(\tilde{\algebra{A}})) \mid f(x_\infty) = 0 \}\text{.}
	\end{equation*}
	Now if $a \in \algebra{A}$, then $\hat a (\chi_\infty) = \chi_\infty(a) = 0$.

	\underline{$\Gamma$ is a homomorphism}: The linearity is obvious, as is the homomorphism property:
	\begin{equation*}
		(\Gamma(a) \Gamma(b))(\chi) = (\hat a \cdot \hat b)(\chi) = \hat a(\chi) \hat b (\chi) = \chi(a) \chi(b) = \chi(ab) = \hat{ab} (\chi) = \Gamma(ab)(\chi)\text{.}
	\end{equation*}

	\underline{$\Gamma$ is contractive}: Given $a \in \algebra{A}$, $\chi \in \Omega(\algebra{A})$, we have $\hat a (\chi) = \chi(a) \in \sigma(a)$, so $\|\hat a (\chi) \| \leq r(a)$ yielding $\|\Gamma(a)\|_\infty = \|\hat a\|_\infty \leq r(a) \leq \|a\|$. If $\algebra{A}$ is commutative, we have
	\begin{equation*}
		\sigma(a) = \left\{ \begin{matrix}
			\{\chi(a) \mid \chi \in \Omega(\algebra{A}) \}             & 1 \in \algebra{A} \\
			\{\chi(a) \mid \chi \in \Omega(\algebra{A}) \} \cup \{0 \} & \text{otherwise}
		\end{matrix} \right. = \left\{ \begin{matrix}
			\{ \hat a(\chi) \mid \chi \in \Omega(\algebra{A}) \}            & 1 \in \algebra{A} \\
			\{ \hat a(\chi) \mid \chi \in \Omega(\algebra{A}) \} \cup \{0\} & \text{otherwise}
		\end{matrix} \right.
	\end{equation*}
	and thus
	\begin{equation*}
		\|\Gamma(a)\| = \|\hat a \|_\infty = \sup_{\chi \in \Omega(\algebra{A})} | \chi(a) | = \sup_{\lambda \in \sigma(a)} |\lambda| = r(a)
	\end{equation*}
\end{proof}

As a convention, if $\Gamma(\algebra{A}) = \empty$, then $C_0(\Omega(\algebra{A})) = \{0\}$ and thus $\hat a = 0$ for all $a \in \algebra{A}$.

\begin{example}~
	\begin{enumerate}
		\item If $\algebra{A} = M_n(\mathds{C})$ with $n > 1$ or $\algebra{A}$ is any unital simple Banach algebra with $\dim \algebra{A} > 1$, then $\Omega(\algebra{A}) = \emptyset$ so $\Gamma \equiv 0$.
		\item Take the commutative subalgebra
		      \begin{equation*}
			      \algebra{A} = \left\{ \qmatrix{0 & \lambda \\ 0 & 0} \mid \lambda \in \mathds{C} \right\} \subseteq M_2(\mathds{C})
		      \end{equation*}
		      then $\algebra{A}$ is not unital, commutative, Banach and $\dim \algebra{A} = 1$. Once again, $\Omega(\algebra{A}) = \emptyset$ and thus $\Gamma \equiv 0$.
		\item Take
		      \begin{equation*}
			      \algebra{A} = \left\{ \qmatrix{\lambda & \alpha \\ 0 & \lambda} \mid \lambda, \alpha \in \mathds{C} \right\} \subseteq M_2(\mathds{C})
		      \end{equation*}
		      is a unital, commutative Banach algebra with $\dim \algebra{A} = 2$. We have
		      \begin{equation*}
			      \Omega(\algebra{A}) = \{\chi_\infty\} \qquad \chi_\infty: \algebra{A} \to \mathds{C},  \qmatrix{\lambda & \alpha \\ 0 & \lambda}  \mapsto \lambda
		      \end{equation*}
		      and thus
		      \begin{equation*}
			      \Gamma: \algebra{A} \to C_0(\Omega(\algebra{A})) = C_0(\{\chi_\infty\}) \simeq \mathds{C}, a = \qmatrix{\lambda & \alpha \\ 0 & \lambda} \mapsto \hat a \equiv \lambda
		      \end{equation*}
		      This shows that $\Gamma$ is not injective, as $\dim \algebra{A} = 2$ but $\dim \Gamma(\algebra{A}) = 1$.
	\end{enumerate}
\end{example}

\begin{definition}
	Let $\algebra{A}$ be a Banach algebra. We say that $a \in \algebra{A}$ is quasi-nilpotent if $r(a) = \lim_{n \to \infty} \|a^n\|^{\frac{1}{n}} = 0$.
	Sometimes, you will read
	\begin{equation*}
		\mathrm{Rad}(\algebra{A})  = \{a \in \algebra{A} \mid a \text{ quasi-nilpotent}\}
	\end{equation*}
	If $\mathrm{Rad}(\algebra{A}) = 0$, we say that $\algebra{A}$ is \textbf{semi-simple}.
	Notice that if $a \in \algebra{A}$ is quasi-nilpotent, then $\Gamma(a) = \hat a = 0$ because $\Gamma(a) \leq r(a) = 0$.
	If $\algebra{A}$ is commutative, then $\ker(\Gamma) = \mathrm{Rad}(\algebra{A})$.
\end{definition}

\begin{example}~
	\begin{enumerate}
		\setcounter{enumi}{3}
		\item $\algebra{A} = \ell^1(\mathds{Z}) = \{ (a_n)_{n \in \mathds{Z}} \mid \sum_{n \in \mathds{Z}} |a_n| < \infty$.
		      \begin{quote}
			      Recall from exercises, that $\Omega(\ell^1(\mathds{Z})) \simeq \mathds{D}$ with $\mathds{D} \to \Omega(\ell^1(\mathds{Z})), z \mapsto \chi_z$ defined as $\chi_z(a) = \hat a (z) = \sum_{n=0}^{\infty}a_n z^n$.

			      We define a multiplication $\delta_m \cdot \delta_n = \delta_{n + m}$. Then $\delta_0$ is the unit and $\delta_1$ is a generator of $\algebra{A} = \ell^1(\mathds{Z})$.

			      The elements $\delta_m - (\dots, 0, 1, 0, \dots )$ form a basis for $\algebra{A}$.
			      We have $a = \sum_{n \in \mathds{Z}} a_n \delta_n$ and for $\chi \in \algebra{A}^*$ it follows $\chi(a) = \sum_{n \in \mathds{Z}} a_n \chi(\delta_n)$.
		      \end{quote}

		      We now want to calculate the spectrum.
		      We have seen that $\chi(\delta_0) = \chi(1_\algebra{A}) = 1$ and $\chi(\delta_n) = \chi(\delta_1^n)  \chi(\delta_1)^n$.
		      Therefore, $\chi$ is determined by $z = \chi(\delta_1) \in \mathds{C}$.
		      We know at least that $|z| = |\chi(\delta_1)| \leq \|\delta_1\| = 1$, so $z \in \mathds{D}$.
		      Claim: $z \in \Pi = \mathds{S}^1$.
		      \begin{quote}
			      General fact: If $a \in \inv \algebra{A}$ for $\algebra{A}$ a unital Banach algebra, then $\sigma(a^{-1}) = \sigma(a)^{-1} = \{ \lambda^{-1} \mid \lambda \in \sigma(a) \}$.
		      \end{quote}
		      Observe that $\mathds{S}^1 = \inv(\algebra{A})$ with $\delta_1^{-1} = \delta_{-1}$.
		      So $\sigma(\delta) \subseteq \mathds{D}$ and $\sigma(\delta_1)^{-1} = \sigma(\delta_{-1}) \subseteq \mathds{D}$, so $\sigma(\delta_1) \subseteq \mathds{S}^1$.
		      So $z = \chi(\delta_1) \in \sigma(\delta_1) \subseteq \mathds{S}^1$.
		      Conversely, if $z \in \mathds{S}^1$, then $\chi_z: \algebra{A} \to \mathds{C}, \chi_z(a) = \sum_{n \in \mathds{Z}} a_n z^n \in \mathds{C}$ is well-defined (as the sum converges) and is a character, as
		      \begin{equation*}
			      \chi_z(\delta_n \cdot \delta_m) = \chi(\delta_{n + m}) = z^{n+m} = z^n z^m = \chi_z(\delta_n) \cdot \chi+z(\delta_m)
		      \end{equation*}
		      and checking in the basis also proves the homomorphism property for all of $\algebra{A}$. Notice that $z = \chi_z(\delta_1)$. This shows the injectivity of
		      \begin{equation*}
			      \Pi \simeq \Omega(\algebra{A}) = \Omega(\ell^1(\mathds{Z})), z \mapsto \chi_z, \chi(\delta_1) \leftarrow \chi
		      \end{equation*}
		      which is continuous and therefore a homeomorphism (isomorphism), as both spaces are compact.
		      Notice
		      \begin{equation*}
			      \sigma(\delta_1) = \{\chi(\delta_1) \mid \chi \in \Omega(\algebra{A}) \}  = \{\chi_z(\delta_1) \mid z \in \mathds{S}^1 \} = \mathds{S}^1
		      \end{equation*}
		      The Gelfand transformation is now
		      \begin{equation*}
			      \Gamma: \algebra{A} = \ell^1(\mathds{Z}) \to C(\Omega(\algebra{A})) \simeq C(\mathds{S}^1), a \mapsto  \left( \hat a: z \mapsto \sum_{n \in \mathds{Z}} a_n z^n \right)
		      \end{equation*}

		      $\Gamma$ is always a contractive algebra homomorphism, as $\|\hat{a}\|_\infty \leq \|a\|_1$.
		      $\Gamma$ is a $^*$-homomorphism where $\ell^1(\mathds{Z})$ carries the involution $a^* = \left(\sum_{n \in \mathds{Z} a_n \delta_n}\right) = \sum_{n \in \mathds{Z}} \overline{a}_n \delta_{-n}$ because of $\delta_n^* = \delta_{-n}$.
		      The involution of $C(\unitcircle)$ is complex conjugation. But on the unit circle, $\overline{z} = z^{-1}$, so we have a $^*$-homomorphism.

		      $\Gamma$ is \underline{injective}. If $f \in C(\mathds{S}^1)$, we can define its \enquote{inverse Fourier-Transform}
		      \begin{equation*}
			      \check{f}(n) = \int_{\mathds{S}^1} f(z) z^{-n} dz = \frac{1}{2 \pi} \int_0^{2 \pi} f(\exp(it)) \exp(-int) dt
		      \end{equation*}
		      This is \textbf{not} the line integral from functional analysis, as the derivative of the path is not included.
		      You can now check that $(\hat{a})^{\check{~}}(n) = a_n$.
		      $g \mapsto \int_{\mathds{S}^1} g$ is a continuous function on $C(\unitcircle)$ and we have
		      \begin{align*}
			      \hat{a}(z) & = \sum_{n \in \mathds{Z}} a_n z^n = \lim_{F \subseteq \mathds{Z} \text{ finite}} \sum_{n \in F} a_n z^n
			      = \lim_{N \to \infty} \sum_{n = -N}^N a_n z^n
		      \end{align*}
		      so
		      \begin{align*}
			      (\hat{a})^{\check{~}} (n) = \sum_{m \in \mathds{Z}} a_m (\hat{\delta_m})^{\check{~}} (n)
		      \end{align*}
		      Because of $\int_{\unitcircle} z^k = \delta_{k, 0}$, we have
		      \begin{align*}
			      \int_{\unitcircle} z^m z^n dz = \delta_{n,m}
		      \end{align*}
		      and using $\hat{\delta_m}(z) = z^m$ we can show $(\hat{\delta_m})^{\check{~}} (n) = \delta_{n,m}$ and thus
		      \begin{align*}
			      (\hat{a})^{\check{~}} (n) = \sum_{m \in \mathds{Z}} a_m (\hat{\delta_m})^{\check{~}} (n) = \sum_{m \in \mathds{Z}} a_m \delta_{m,n} = a_m
		      \end{align*}
		      This shows that we can re-gain the elements of the sequence from $\hat{a}$, so $\Gamma: (a_n) \mapsto \hat{a}$ must be injective.

		      $\Gamma$ has dense range because the polynomials are dense in $C(\unitcircle)$ because of Stone-Weierstraß theorem.

		      $\Gamma$ is \underline{not isometric}.
		      %Choose $a = a_0 \delta_0 + a_1 \delta_1 + a_2 \delta_2 \in \ell^1(\mathds{Z})$, then $\hat{a}(z) = a_0 + a_1 z + a_2 z^n$.
		      If $\Gamma$ was isometric, then $\Gamma$ were an isometric $^*$-homomorphism with dense range.
		      Since isometric homomorphisms have closed image, $\Gamma$ were surjective and thus an isometric $^*$-isomorphism $\ell^1(\mathds{Z}) = C(\unitcircle)$.
		      Then $\ell^1(\mathds{Z})$ would be a $C^*$-algebra with the $\ell^1(\mathds{Z})$-norm, and thus $\|a^* a\|_1 = \|a\|_1^2$ (with the involution as described above). Then, using the $C^*$-property of $C(\unitcircle)$ and isometry of $\Gamma$, we have
		      \begin{align*}
			      \|a^* a\|_1 = \|\Gamma(a^* a)\|_\infty = \|\Gamma(a)^* \Gamma(a) \|_\infty = \|\Gamma(a)\|_\infty^2 - \|a\|_1^2\text{.}
		      \end{align*}
		      Now we only need to find $a \in \ell^1(\mathds{Z})$ with $\|a^* a \|_1 \neq \|a\|_1^2$.
		      Choose $a = \alpha \delta_0 + \beta \delta_1 + \gamma \delta_{-1} = \alpha + \beta \delta_1 + \gamma \delta_{-1}$ (not writing $\delta_0$ as it is the unit).
		      \begin{align*}
			      a^* a = (\overline{\alpha} + \overline{\beta} \delta_{-1} + \overline{\gamma} \delta_1) (\alpha + \beta \delta_1 + \gamma \delta_{-1}) = \dots
		      \end{align*}
		      and thus
		      \begin{equation*}
			      \|a^* a\|_1 = |\alpha|^2 + |\beta|^2 + |\gamma|^2 + 2 |\overline{\alpha} \beta + \alpha \overline{\gamma} | + 2 |\gamma \beta|
		      \end{equation*}
		      while
		      \begin{equation*}
			      \|a\|_1^2 = (|\alpha| + |\beta| + |\gamma|)^2\text{.}
		      \end{equation*}
		      Now choosing $\alpha = i$ and $\beta = \gamma = 1$ yields $\| a^* a\|_1 = 5$ and $\|a\|_1^2 = 9$. This shows that $\ell^1(\mathds{Z})$ does not fulfil the $^*$-property and cannot be a $C^*$-algebra. This is a contradiction, so $\Gamma$ cannot be isometric.

		      This is also a valid counterexample for the isometry directly, because $a$ has Norm $3$, but $\Gamma(a) = (z \mapsto \frac{1}{z} + i + z = 2 \Re(z) + i)$ has maximum $2 + i$ with Norm $\sqrt{5}<3$ on the unit circle.
		      $\Gamma$ is \underline{not surjective}. This is complicated.
	\end{enumerate}
\end{example}

Recall: For $\algebra{A}$ a Banach algebra, we have a Gelfand representation
\begin{equation*}
	\Gamma: \algebra{A} \to C_0(\Omega(\algebra{A})), a \mapsto (\hat{a}: \Omega(\algebra{A}) \to \mathds{C}, \chi \mapsto \chi(a))
\end{equation*}
where $\Omega(\algebra{A}) = \{\chi: \algebra{A} \to \mathds{C} \mid \text{ non-zero hom} \} \subseteq D_{\algebra{A}^*}(0,1)$ with the weak $^*$-topology.
$\Gamma$ is a contractive homomorphism, and if $\algebra{A}$ is commutative $\|\Gamma(a)\| = r(a) \leq \|a\|$ for all $a \in \algebra{A}$.

We now want to consider commutative $C^*$-algebras.

\begin{theorem}[Gelfand]
	If $\algebra{A}$ is a commutative $C^*$-algebra, then $\Gamma: \algebra{A} \to C_0(\Omega(\algebra{A}))$ is an isometric $^*$-isomorphism.
\end{theorem}

For this proof we require a set of lemmas.

\begin{lemma}
	If $a \in \algebra{A}$, $\algebra{A}$ a $C^*$-algebra, with $a = a^*$ then $r(a) = \|a\|$.
\end{lemma}

\begin{proof}
	Use $r(a) = \lim_{n \to \infty} \|a^n\|^{\frac{1}{n}}$.
	Notice $\|a^2\| = \|a^* a \| = \|a\|^2$ and $\|a^4\| = \|(a^2)^* a^2\| = \|a^2\|^2 = \|a\|^4$ and likewise for all powers that are powers of $2$ we have $\|a^{2^n}\| = \|a\|^{2^n}$.
	So $r(a) = \lim_{n \to \infty} \|a^{2^n}\|^{\frac{1}{2^n}} = \|a\|$ is the limit of the subsequence and therefore the limit of the sequence.
\end{proof}

\begin{remark}
	In general, $\|a\| \neq r(a)$ if $a \neq a^*$ in a $C^*$-algebra, e.g.
	$a = \qmatrix{0 & 1 \\ 0 & 0} \in M_2(\mathds{C})$.
	But if $a^* a = a a^*$ ($a$ is normal), then $\|a\| = r(a)$.
\end{remark}

\begin{proof}
	Exercise.
\end{proof}

\begin{corollary}
	There exists at most one norm that makes a $^*$-algebra $\algebra{A}$ into a $C^*$-algebra.
\end{corollary}

\begin{proof}
	If $\algebra{A}$ is a $C^*$-algebra with norm $\| \cdot \|$, then for all $a \in \algebra{A}$ we have $\|a\| = \|a a^*\|^{\frac{1}{2}}$. Note that $a^*a$ is self-adjoint, so by the previous lemma we have
	\begin{equation*}
		\|a\| = \|a a^*\|^{\frac{1}{2}} = r(a^* a)^\frac{1}{2} = \sup_{\lambda \in \sigma(a^* a)} |\lambda|^{\frac{1}{2}}
	\end{equation*}
	and this only depends on the algebra structure, not its norm.
\end{proof}

\begin{corollary}
	If $\phi: \algebra{A} \to \algebra{B}$ is a $^*$-homomorphism from a Banach-$^*$-algebra $\algebra{A}$ into a $C^*$-algebra $\algebra{B}$ then $\phi$ is contractive, i.e. $\|\phi(a)\|_\algebra{B} \leq \|a\|_\algebra{A}$ for all $a \in \algebra{A}$
\end{corollary}

\begin{proof}
	Replacing $\algebra{A}, \algebra{B}$ by their unitizations $\tilde{\algebra{A}}$ and $\tilde{\algebra{B}}$ and extending $\phi$ to $\tilde{\phi}: \tilde{A} \to \tilde{B}, a + \lambda 1_\algebra{B} \mapsto \phi(a) + \lambda 1_\algebra{B}$ shows that we can just assume $\algebra{A}, \algebra{B}, \phi$ to be unital.
	
	Now, if $a \in \inv(\algebra{A})$, then $\phi(a) \in \inv(\algebra{B})$, so it follows
	\begin{equation*}
		\lambda \in \rho_\algebra{A}(a) \iff a - \lambda \in \inv(\algebra{A}) \iff \phi(a) - \lambda \in \inv(\algebra{B}) \iff \lambda \in \rho_\algebra{B}(\phi(a))
	\end{equation*}
	so $\rho_\algebra{A}(a) \subseteq \rho_\algebra{B}(\phi(a))$ and $\sigma_\algebra{A}(a) \supseteq \sigma_\algebra{B}(\phi(a))$.
	It follows for the spectral radius: $r(\phi(a)) \leq r(a)$.
	As $\algebra{B}$ is a $C^*$-algebra, this implies
	%$\|\phi(a)\|_\algebra{B} = r(\phi(a)) \leq r(a) \leq \|a\|_\algebra{A}$. %% WRONG => needs self-adjoint
	\begin{align*}
		\|\phi(a)\|_\algebra{B}^2 &= \|\phi(a)^* \phi(a)\|_\algebra{B} = \|\phi(a^*a)\|_\algebra{B} = r(\phi(a^* a)) \\ &\leq r(a^* a) \leq \|a^* a\|_\algebra{A} \leq \|a^*\|_\algebra{A} \cdot \|a\|_\algebra{A} = \|a\|_\algebra{A}^2
	\end{align*}
	and therefore $\|\phi(a)\|_\algebra{B} \leq \|a\|_\algebra{A}$.
\end{proof}

\begin{lemma}
	If $\algebra{A}$ is a $C^*$-algebra and $a \in \algebra{A}$, then 
	\begin{enumerate}
		\item If $a$ is self-adjoint, $\sigma(a) \subseteq \mathds{R}$.
		\item If $\algebra{A}$ is unital and $u \in \mathcal{U}(\algebra{A})$ is unitary (that is, $u^* u = u u^* = 1$) then $\sigma(u) \subseteq \unitcircle$.
		\item If $a \in \inv(\algebra{A})$, then $\sigma(a^{-1}) = \sigma(a)^{-1} = \{z^{-1} \mid z \in \sigma(a) \}$.
		\item $\sigma(a^*) = \overline{\sigma(a)}$.
	\end{enumerate}
\end{lemma}

\begin{proof}
	\begin{enumerate}
		\item[(iii)] If $\lambda \in \mathds{C}, \lambda \neq 0$ and $\lambda - a \notin \inv(\algebra{A})$.
		Because $\lambda - a$ is not invertible, $\lambda^{-1}(\lambda - a) = 1 - \lambda^{-1} a$ and $a^{-1}(1 - \lambda^{-1} a) = a^{-1} - \lambda^{-1}$ is also not invertible.
		So we have $\lambda^{-1} - a^{-1} \notin \inv(\algebra{A})$ and therefore $\sigma(a^{-1}) \subseteq \sigma(a)^{-1}$.
		The result follows by symmetry.
		\item[(iv)] Similarly, you can prove (iv).
		\item[(ii)] If $u \in \mathcal{U}(\algebra{A})$, then $\sigma(a) \subseteq \mathds{D} = \{z \in \mathds{C} \mid |z| \leq 1 \}$ because 
		\begin{equation*}
			\|u\| = \|u^* u\|^{\frac{1}{2}} = \|1\|^\frac{1}{2} = 1\text{.}
		\end{equation*}
		So, since $u \in \mathcal{U}(\algebra{A})$, $u^{-1}=u^* \in \mathcal{U}(\algebra{A})$ and therefore $\sigma(u)^{-1} = \sigma(u^{-1}) \subseteq \mathds{D}$.
		This implies $\|\lambda\| = 1$ for all $\lambda \in \sigma(u)$ and thus $\sigma(u) \subseteq \unitcircle$.
		\item Assume that $\algebra{A}$ is unital, otherwise work in $\tilde{\algebra{A}}$. If $a$ is self-adjoint then $u = \exp(i a) = \sum_{n=0}^{\infty} \frac{i^n a^n}{n!} \in \mathcal{U}(\algebra{A})$ because $
		\exp(i a)^* = \exp(- i a)$ and therefore $u^* u = \exp(-i a) \exp(i a) = \exp(0) = 1 = u u^*$.
		Because of (i) we know $\sigma(u) \subseteq \unitcircle$.
		Now, let $\lambda \in \sigma(u)$ and define $b = \sum_{n=1}^{\infty} \frac{i^n (a-\lambda)^n}{n!} = \exp(i(a - \lambda)) - 1$ as well as $c = \sum_{n = 1}^{\infty} \frac{i^n (a - \lambda)^{n-1}}{n!} \in \algebra{A}$ .
		Consider
		\begin{align*}
			\exp(i a) - \exp(i \lambda 1) &= (\exp(i (a - \lambda)) - 1)  \exp(i \lambda) = b \exp(i \lambda)\\
			&= \left( \sum_{n = 1}^{\infty} \frac{i^n (a - \lambda)^n}{n!} \right) \exp(i \lambda) \\ 
			&= (a - \lambda) \left( \sum_{n = 1}^{\infty} \frac{i^n (a - \lambda)^{n-1}}{n!} \right) \exp(i \lambda)\\
			&= (a - \lambda) c \exp(i \lambda)\text{.}
		\end{align*}
		Since $\lambda \in \sigma(a)$ and $c, (a - \lambda)$ commute, $\exp(i a) - \exp(i \lambda)$ is not invertible (or $a - \lambda$ would also be invertible) and we have $\exp(i \lambda) \in \sigma(u) \subseteq \unitcircle$.
		But for this to happen, we require $\lambda \in \mathds{R}$.
	\end{enumerate}
\end{proof}

\begin{corollary}
	If $\algebra{A}$ is a $C^*$-algebra and $\chi \in \Omega(\algebra{A})$, then $\chi(a^*) = \overline{\chi(a)}$ for all $a \in \algebra{A}$.
	So $\chi$ is a $^*$-homomorphism.
\end{corollary}

\begin{proof}
	If $a \in \algebra{A}$ is self-adjoint, then $\chi(a) \in \sigma(a) \subseteq \mathds{R}$ so $\overline{\chi(a)} = \chi(a) = \chi(a^*)$.

	Now, if $a \in \algebra{A}$ is any element we can write it as $a = b + i c$ where $b = \frac{a + a^*}{2}$ and $c = \frac{a - a^*}{2i}$ so that $b,c$ are self-adjoint. Now $\chi(b),\chi(c) \in \mathds{R}$ so
	\begin{equation*}
		\chi(a^*) = \chi(b - ic) = \chi(b) - i \cdot \chi(c) = \overline{\chi(b) + i \chi(c)} = \overline{\chi(b + ic)} = \overline{\chi(a)}
	\end{equation*}
\end{proof}

\begin{corollary}
	If $\algebra{A}$ is a commutative $C^*$-algebra and $\algebra{A} \neq 0$, then $\Omega(\algebra{A}) \neq \emptyset$.
\end{corollary}

\begin{proof}
	If $\algebra{A} \neq 0$ there is some self-adjoint non-zero element $a \in \algebra{A}$ so that $r(a) = \|a\| \neq 0$. But $\sigma(a) \subseteq \{ \chi(a) \mid \chi \in \Omega(\algebra{A}) \} \cup \{0\}$.
	%(with equality if $\algebra{A}$ is not unital, in the unital case $\sigma(a)$ is the first set).
	But for this to be true there must exist a character $\chi \in \Omega(\algebra{A})$, so $\Omega(\algebra{A}) \neq \emptyset$.
\end{proof}

\begin{proof}[Gelfand]~
	\begin{itemize}
		\item \textbf{$\Gamma$ is a $^*$-homomorphism}:
		Consider 
		\begin{equation*}
			\Gamma(a)^*(\chi) = \hat{a}^*(x) = \overline{\hat{a}(\chi)} = \overline{\chi(a)} = \chi(a^*) = \hat{a^*}(\chi) =  \Gamma(a^*)(\chi)
		\end{equation*}
		so $\Gamma(a)^* = \Gamma(a^*)$.
		\item \textbf{$\Gamma$ is isometric}:
		We have
		\begin{equation*}
			\|\Gamma(a)\|^2 = \|\Gamma(a)^* \Gamma(a)\| = \|\Gamma(a^* a)\| = r(a^* a) = \|a^* a\| = \|a\|^2
		\end{equation*}
		using our lemmas and the $C^*$-property.
		\item \textbf{$\Gamma$ is surjective}:
		Let $\algebra{B} \coloneq \image(\Gamma) \subseteq C_0(\algebra{A})$.
		Then $\algebra{B}$ is a $C^*$-subalgebra of $C_0(\Omega(\algebra{A}))$.
		Then
		\begin{itemize}
			\item $\algebra{B}$ does not vanish at any point, i.e. for every point $\chi \in \Omega(\algebra{A})$ there is a $b \in \algebra{B}$ with $\chi(b) \neq 0$. 
			
			As $\chi \in \Omega(\algebra{A})$ means $\chi \neq 0$, there exists an $a \in \algebra{A}$ with $\chi(a) \neq 0$.
			But we can rewrite this as $b(\chi) = \hat{a}(\chi) = \chi(a) \neq 0$ for $b = \hat{a}$.
			\item $\algebra{B}$ sperates points in $\Omega(\algebra{A})$, i.e. for every $\chi_1 \neq \chi_2$ in $\Omega(\algebra{A})$ there exists $b \in \algebra{B}$ with $b(\chi_1) \neq b(\chi_2)$.
			
			If $\chi_1 \neq \chi_2$ there exists $a \in \algebra{A}$ with $\chi_1(a) \neq \chi_2(a)$.
			Taking $b = \hat{a}$ yields the result.
		\end{itemize}
		The result $\algebra{B} = C_0(\Omega(\algebra{A}))$ follows from the Stone-Weierstraß-theorem:
		\begin{quote}
			If $X$ is a locally compact Hausdorff space and $B \subseteq C_0(X)$ is a $^*$-subalgebra satisfying
			\begin{itemize}
				\item $B$ does not vanish on any point of $X$
				\item $B$ separates points of $\algebra{A}$
			\end{itemize}
			then $B$ is dense in $C_0(X)$.
		\end{quote}
		So $\image(\Gamma)$ is dense and closed in $C_0(\Omega(\algebra{A}))$, so $\Gamma$ is surjective.
	\end{itemize}
\end{proof}

\begin{proposition}
	Conclusion: Every commutative $C^*$-algebra is (up to $^*$-isomorphism) of the form $C_0(X)$ for a locally compact Hausdorff space $X$.
	Let $\algebra{A} = C_0(X)$ for a locally compact Hausdorff space $X$. Then $\Omega(\algebra{A}) \simeq X$ with isomorphism
	\begin{equation*}
		\phi: X \to \Omega(C_0(X)), x \mapsto (\mathrm{ev}_x: C_0(X) \to \mathds{C}, f \mapsto f(x))\text{.}
	\end{equation*}
\end{proposition}

\begin{proof}~
	\begin{itemize}
		\item $\phi$ is \textbf{well-defined}, because characters are never zero.
		\item $\phi$ is \textbf{continuous}.
		Take $x_i \to x$ in $X$.
		Then, for all $f \in C_0(X)$ we have $\mathrm{ev}_{x_i}(f) \to \mathrm{ev}_x(f)$ because $f$ is continuous and therefore $f(x_i) \to f(x)$.
		This shows $\mathrm{ev}_{x_i} \to \mathrm{ev}_x$ in the weak $^*$-topology.
		\item $\phi$ is \textbf{injective}. I
		f $x_1 \neq x_2$ there exists a function $f \in C_0(X)$ that separates them, but then $\mathrm{ev}_{x_1}(f) \neq \mathrm{ev}_{x_2}(f)$, so $\mathrm{ev}_{x_1} \neq \mathrm{ev}_{x_2}$.
		\item $\phi$ is \textbf{surjective}.  Prove that every $\chi \in \Omega(\algebra{A})$ is $\chi = \mathrm{ev}_x$ for some $x \in X$.
		
		We know that the characters of $\algebra{A}$ are equivalent to the ideals in $C_0(X)$, so this is equivalent to:
		Every maximal ideal $I \triangleleft C_0(X)$ is of the form $I = C_0(X \setminus \{x_0\}) = \{ f \in C_0(X) \mid f(x_0) = 0 \}$.

		In Exercise 01-08 we have proven that every closed (2-sided) ideal $I \triangleleft C_0(X)$ has the form $I = C_0(U) \coloneq \{ f \in C_0(X) \mid f|_{X \setminus U} \equiv 0 \}$ for some open $U \subseteq X$.

		\begin{quote}
			See 01-08 for more details.

			Take any $f \in I \triangleleft C_0(X)$.
			First, prove $I^* = I$.
			Consider $f \in I$ and 
			\begin{equation*}
				f_n \coloneq \sqrt[n]{f^* f} = (\overline{f} f )^{\frac{1}{n}} = |f|^{\frac{2}{n}}\text{.}
			\end{equation*}
			We have $f_n \in I$ for all $n$, because $g \coloneq f^* f \in I$ and $t \mapsto \sqrt[n]{t}$ is a continuous function that can be uniformly approximated by polynomials on the compact sets.
			It follows that $f_n = \lim g_n$ where $g_n$ is a polynomial in $g \in I$, so $f_n \in I$.
			So $f^* f_n \in I$ for all $n$.
			Then
			\begin{align*}
				\|f^* = f_n f^*\|_\infty^2 &= \|(f^* - f_n f^*)(f^* - f_n f^*)\|_\infty = \|(f = f_n f)(f^* - f_n f^*)\|_\infty\\ &= \| f^* f - 2 f^* f f_n + f_n^2 f^* f\|_\infty \\ &\leq \|g - g \sqrt[n]{g}\| + \|g - g \sqrt[n]{g} \| \|f_n\| \to 0\text{,}
			\end{align*}
			because $|g(x) - g(x) \sqrt[n]{g(x)}| \to 0$ pointwise (as the $n$-th square root converges to the $1$ on the support and $0$ elsewhere) and $|g(x)| \leq \epsilon$ everywhere except a compact set $K$, and on that $K$ we have $\sup_{x \in K} |g(x)| |1 - \sqrt[n]{g(x)}| = |g(x_0)| |1 - \sqrt[n]{g(x_0)}| < \epsilon$ for some $n \in \mathds{N}$.
			We therefore have $f^* = \lim_{n \to \infty} f^* f_n \in I$ and thus $f^* = \lim_{n \to \infty} f_n f^*$. 
			Now let $I \triangleleft C_0(X)$ closed, so $I^* = I$ and $I$ is a $C^*$-subalgebra of $X$.

			Define $U^\complement \coloneq \{x \in X \mid f(x) = 0 \forall_{f \in I} \}$.
			This is closed (because for $x_i \to x$ in $X$, $x_i \in U^\complement$, we have $0 = f(x_i) \to f(x)$), so $U$ is open. 
			We claim $I = C_0(U)$. 
			
			If $f \in I$, $f|_{U^\complement} \equiv 0$ per Definition, so $f \in C_0(U)$. Therefore, $I$ is a closed subideal of $C_0(U)$. 

			$I$ does not vanish on $U$, because if there was an $x \in U$ with $f(x) = 0$ for all $f \in I$, we would have $x \in U^\complement$.

			$I$ separates the points of $U$. Take $x_1 \neq x_2$. We can choose $h \in C_0(X)$ with $h(x_1) = 1$ and $h(x_2) = 0$ (Uryson) as well as $g \in I$ with $g(x_1) \neq 0$, then $f = g \cdot h \in I$ separates $x_1$ from $x_2$.

			Stone-Weierstraß now proves $I = C_0(U)$.
		\end{quote}
		Notice $U \subseteq V \subseteq X$ (open) iff $C_0(U) \subseteq C_0(V) \trianglelefteq C_0(X)$ (see exercise 08-01).
		So we have a bijection between the opens of $X$ and the ideals of $C_0(X)$.
		Especially, the maximal ideals of $C_0(X)$ correspond to the maximal open sets, that is the sets of form $X \setminus \{x_0\}$ for some $x_0$, of $X$.

		Therefore, if $\chi \in \Omega(C_0(X))$ we have $\ker \chi = C_0(X\setminus\{x_0\})$, so $\chi$ maps a function to $0$ if and only if $f$ is zero on $x$.
		This proves and $\chi = \ev_x$.
		\item $\phi$ is \textbf{open}. If $X$ is compact, this is clear because $C_0(X) = C(X)$ and unital, so $\Omega(C_0(X))$ is compact and we have a bijection between two compact sets.
		In general, consider $\tilde{X}$ (the compactification) and use $\widetilde{C_0(X)} \simeq C(\tilde{X})$. So we have a homeomorphism 
		\begin{equation*}
			\tilde{X} \to \Omega(C(\tilde{x})) = \Omega(\widetilde{C_0(X)}) \simeq \Omega(C_0(X)) \sqcup \{\chi_\infty\}
		\end{equation*}
		where $\infty \mapsto \chi_\infty$, so we can restrict the homeomorphism to $X$ and are done.
	\end{itemize}
\end{proof}

\begin{theorem}[Spectral inclusion for $C^*$-algebras]
	Let $\algebra{A} \subseteq \algebra{B}$ be an inclusion of unital $C^*$-algebras with $1 = 1_\algebra{A} = 1_\algebra{B}$. Then for all $a \in \algebra{A}$ we have $\sigma_\algebra{A}(a) = \sigma_\algebra{B}(a)$, so $\inv(\algebra{A}) = \inv(\algebra{B}) \cap \algebra{A}$.
\end{theorem}

\begin{proof}
	If $a$ is self-adjoint, that is $a^* = a$, then $\sigma_\algebra{A}(a) \setminus \mathds{R}$, so $\sigma_\algebra{A}$ has no holes, i.e. the complement $\mathds{C} \subseteq \sigma_\algebra{A}(a)$ is connected in $\mathds{C}$.
	By the general result on Banach algebras $\sigma_\algebra{A}(a) = \sigma_\algebra{B}(a)$.
	In particular, this implies $a \in \inv(\algebra{A}) \iff a \in \inv(\algebra{B})$ for all self-adjoint $a \in \algebra{A}$.
	
	We now prove that this holds for all $a \in \algebra{A}$.
	Of course, $\inv(\algebra{A}) \subseteq \inv(\algebra{B}) \cap \algebra{A}$.
	Let $a \in \algebra{A}$ such that $a \in \inv(\algebra{B})$.
	Then there exists $b \in \algebra{B}$ such that $ab = ba = 1$ and $b^*a^* = a^* b^* = 1 \iff bb^* a^* a = 1 = a^*abb^*$. Therefore, $a^* a \in \inv{\algebra{B}} \cap \algebra{A} \subseteq \inv(\algebra{A})$ because $a^* a$ is self adjoint.
	So there exists $c \in albebra{A}$ with $c a^* a = 1 = a^* a c$ and thus $c a^* a b = c a^* = b$, so $b \in \algebra{A}$ as it is the product of two elements $a^*, c \in \algebra{A}$.
	This concludes the proof, as $a$ is now invertible in $\algebra{A}$.	
\end{proof}

\begin{definition}
	We say $a \in \algebra{A}$ (for $\algebra{A}$ a $C^*$-algebra) is \textbf{normal} if $a^* a = a a^*$.
	This means $C^*(a)$ (the $C^*$-subalgebra of $\algebra{A}$ generated by $a$) is commutative.
	Then $C^*(a) \simeq C_0(X)$.
\end{definition}

\begin{lemma}
	Let $a \in \algebra{A}$ ($C^*$-algebra) be a normal element.
	Assume that $1 \in \algebra{A}$ (unital).
	Then $\Omega(C^*(a,1)) \simeq \sigma(a)$ by homeomorphism $\chi \mapsto \chi(a)$.
	In general, if $\algebra{A}$ is possibly not unital, then $\Omega(C^*(a)) \simeq \sigma(a) \setminus \{0\}$.
	In particular, $\chi(a) = 0$ only if $a = 0$ but then $C^*(a)$ is just the zero space.
\end{lemma}

\begin{proof}
	It is enough to consider the unital case.

	Consider $\phi: \Omega(C^*(a,1)) \to \sigma(a), \chi \to \chi(a)$ which is well-defined because $\chi(a) \in \sigma(a)$.
	\begin{itemize}
		\item $\phi$ is \textbf{continuous}.
		If $\chi_i \to \chi$ in $\Omega(C^*(a,1))$ then this also converges point wise, so $\chi_i(a) \to \chi(a)$.
		\item $\phi$ is \textbf{injective}.
		Take $\chi_1, \chi_2 \in \Omega(C^*(a,1))$ with $\chi_1(a) = \chi_2(a)$.
		Since $\chi_1(1) = 1 = \chi_2(1)$, so the two characters coincide on the generators and are thus equal by linearity and continuity.
		\item $\phi$ is \textbf{surjective}. We know that $\sigma(a) = \{ \chi(a) \mid \chi \in \Omega(B)\}$ for all commutative unital Banach algebras $B$, in particular for $B = C^*(a,1)$.
	\end{itemize}
	Because both spaces are compact this concludes the proof.
\end{proof}

\begin{theorem}[Fundamental theorem of continuous functional calculus]
	~

	Let $\algebra{A}$ be a unital $C^*$-algebra and $a \in \algebra{A}$ normal.
	Then there exists a unique unital $^*$-homomorphism $\phi: C(\sigma(a)) \to \algebra{A}$ such that $\id_{\sigma(a)} \mapsto a$.

	In general, if $\algebra{A}$ is possibly not unital, there exists a unique $^*$-homomorphism $\phi: C_0(\sigma(a)) \to \algebra{A}$ where $C_0(\sigma(a)) \coloneq \{ f \in C(\sigma(a)) \mid f(0) = 0\}$.

	Both of these morphisms are also isometric.
\end{theorem}

Notation: If $f \in C(\sigma(a))$ we write $f(a) \coloneq \phi(a)$.
Notice: If $f$ is a polynomial in $z, \overline{z}$ then $f(a) = \phi(a)$ as usual.

\begin{proof}
	Consider $1 \in \algebra{A}$ and let $\algebra{B} = C^*(a,1) \subseteq \algebra{A}$. 
	Then $\algebra{B}$ is commutative because $a$ is normal (i.e. commutes with its adjoint).
	By Gelfand, we get an isometric $^*$-isomorphism $T: \algebra{B} \to C(\Omega(\algebra{B})), b \mapsto \hat{b}$.
	By the Lemma, $\Omega(\algebra{B}) \equiv \sigma(a), \chi \mapsto \chi(a)$.
	Via this identification (homeomorphism), we have $\hat{b}(\chi) = \chi(b)$ and $\hat{a}(\chi) = \chi(a)$.
	So $\hat{a}$ corresponds to $z \in C(\sigma(a)) \simeq C(\Omega(\algebra{B}))$.
	Therefore, considering the inverse of $T$ and identifying $\Omega(\algebra{B}) \simeq \sigma(a)$ we get an isometric 
	\begin{equation*}
		C(\sigma(a)) \simeq C(\Omega(C^*(a,1))) \simeq C^*(a,1) \simeq \algebra{A}\text{.}
	\end{equation*}
	This gives $\phi$ as defined.

	The \textbf{non-unital case}: Just consider $\tilde{\algebra{A}}$.
\end{proof}

\begin{example}
	Let $f(z) = \exp(z) = e^z = \sum_{k=0}^{\infty} \frac{z^k}{k!}$. $f$ is a continuous function on the whole plane.
	If $a \in \algebra{A}$ is normal, then $f(a) = \exp(a) = \sum_{n = 0}^{\infty} \frac{a^n}{n!}$.
	In general, $f(z) = \sum_{n = 0}^{\infty} \lambda_n z^n$ (or $f(z) = \sum_{n = 0}^{\infty} \lambda_n (z - z_0)^n$), so $f(a) = \sum_{n = 0}^{\infty} \frac{a^n}{n!}$ if $\sigma(a) \subseteq \mathrm{Domain}(f)$.
\end{example}

\begin{theorem}
	Let $\algebra{A}$ be unital $C^*$-algebra and $a \in \algebra{A}$ be normal.
	If $f \in C(\sigma(a))$, then $\sigma(f(a)) = f(\sigma(a)) = \{f(\lambda) \mid \lambda \in \sigma(a)\}$.

	Moreover, if $g \in C(\sigma(f(a)))$, then $g(f(a)) = (g \circ f)(a)$.
\end{theorem}

\begin{proof}
	Let $\algebra{B} = C^*(a,1) \subseteq \algebra{A}$.
	$\algebra{B}$ is commutative and unital.
	Then $f(a) \in \algebra{B}$ and $\sigma(f(a)) = \sigma_\algebra{B}(f(a))$. Now notice $\chi(f(a)) = f(\chi(a))$ since both maps 
	\begin{align*}
		f \mapsto \chi(f(a)) \\
		f \mapsto f(\chi(a))
	\end{align*}
	are unital $^*$-homomorphisms that coincide on $z$.
	Therefore,
	\begin{equation*}
		\sigma(f(a)) = \{ \chi(f(a)) \mid \chi \in \Omega(\algebra{B}) \} = \{ f(\chi(a)) \mid \chi \in \Omega(\algebra{B})\} = f(\sigma(a))	\text{.}
	\end{equation*}
	Now to prove $(g \circ f)(a) = g(f(a))$.
	Let $C = C^*(1, f(a)) \subseteq \algebra{B} = C^*(1,a) \subseteq \algebra{A}$.
	Let $\chi \in \Omega(\algebra{B})$. 
	Then $\chi_C \coloneq \chi|_C \in \Omega(C)$.
	So $(g \circ f)(a)$ is sensibly defined and an element of $\algebra{B}$, so we can apply a character:
	\begin{align*}
		\chi((g \circ f)(a)) &= (g \circ f)(\chi(a)) = g(f(\chi(a))) = g(\chi(f(a))) = g(\chi_C(f(a)))\\ &= \chi_C(g(f(a))) = \chi(\underbrace{g(f(a))}_{\in \algebra{B}})
	\end{align*}
	Because the Gelfand-transform is injective, this implies $(g \circ f)(a) = g(f(a))$.
\end{proof}

\begin{proposition}
	Let $\algebra{A}$ be a unital $C^*$-algebra and $u \in \mathcal{U}(\algebra{A}) = \{ u \in \algebra{A} \mid u^* u = 1 = u u^*\}$.
	If $\sigma(u) \neq \unitcircle$ there exists a self-adjoint $a \in \algebra{A}$ with $u = \exp(i a)$.
\end{proposition}

\begin{proof}
	The idea is to take $\log \approx \exp^{-1}$.
	Problem: $\exp$ is not invertible as a complex function, because it is $2 \pi i$-periodic.
	We will need to restrict it.
	Consider the principal branch of the logarithm, $\log(z) = \log |z| + i \arg(z)$.

	Given that $\sigma(a) \neq \unitcircle$, there exists an $\lambda \in \unitcircle \setminus \sigma(a)$ and therefore also an $f_\lambda \in C(\unitcircle \setminus \{\lambda\})$ (so some form of argument-mapping of $z$) such that $\exp(i f_\lambda(z)) = z$.
	This $f_\lambda$ is real-valued, continuous and analytical.
	Now use functional calculus:
	Let $a \coloneq f_\lambda(u) \in \algebra{A}$.
	Since $f_\lambda$ is real-valued, it is self-adjoint in the algebra, so $a$ is also self-adjoint.
	By the previous theorem $\exp(i a ) = \exp(i f_\lambda(u)) = (\exp \circ i f_\lambda)(u) = u$.	


\end{proof}

\subsubsection*{Multiplier Algebras}

This is another kind of unitization.
We will consider $\algebra{A} \to M(\algebra{A}) \ni \mu$ such that $\mu \cdot a \in \algebra{A} \ni a \cdot \mu$ so $\algebra{A} \trianglelefteq M(\algebra{A})$.
Remember that this was the case for the usual unitization, with Quotient $\mathds{C}$.
Here, the multiplier is usually much bigger, so the quotient is as well.
In fact, $\algebra{A} \times \mathds{C}$ is the 'smallest' unitization while $M(\algebra{A})$ is the 'largest' one.

\begin{definition}[Multiplier, see Murphy]
	Let $\algebra{A}$ be an algebra.
	A \textbf{multiplier} of $\algebra{A}$ is a pair $\mu = (L, R)$ where $L,R: \algebra{A} \to \algebra{A}$ are linear maps such that 
	\begin{enumerate}
		\item $L(ab) = L(a) \cdot b$ or $\mu (ab) = (\mu a) b$
		\item $R(ab) = a \cdot R(b)$ or $ (ab) \mu = a (b \mu)$
		\item $a \cdot L(b) = R(a) \cdot b$ or $a (\mu b) = (a \mu) b$.
	\end{enumerate}
	To simplify this, use the notation $\mu \cdot a \coloneq L(a)$ and $a \cdot \mu \coloneq R(a)$.
\end{definition}

For the space of all multipliers we write $M(A) = \{ \mu = (L,R) \mid \mu \text{ multiplier} \}$. This is a $\mathds{C}$-vector space with
\begin{equation*}
	(L_1, R_1) + (L_2, R_2) = (L_1 + L_2, R_1 + R_2) \qquad \lambda (L_1, R_1) = (\lambda L_1, \lambda R_2)
\end{equation*}
and an algebra with 
\begin{equation*}
	(L_1, R_1) \cdot (L_2, R_2) = (L_1 \cdot L_2, R_2 \cdot R_1)\text{.}
\end{equation*}
If $\algebra{A}$ is a $^*$-algebra, we further define
\begin{equation*}
	(L,R)^* = (R^*, L^*) \text{ where } L^*(a) \coloneq L(a^*)^* \text{ and } R^*(a) \coloneq R(a^*)^*
\end{equation*}
Moreover, we have a canonical ($^*$)-homomorphism $\iota: \algebra{A} \to M(\algebra{A}), a \mapsto (L_a, R_a)$ where $L_a(b) = ab $ and $R_a(b) = ba$.
Note: $\iota$ is always a ($^*$)-homomorphism but injective if and only if 
\begin{align*}
	\forall_{a \in \algebra{A}} \ & a \cdot b = 0 \ \forall_b \then a = 0 \\ 
	&b \cdot a = 0 \ \forall_b \then a = 0
\end{align*}
i.e. $\algebra{A}$ is an essential ideal of itself. This is not always true for a general algebra, consider the algebra with the $0$-product $a \cdot b = 0$, but it always holds for $C^*$-algebras or if $\algebra{A}$ is unital already.

More generally this holds if $\algebra{A}$ is a Banach algebra with an \textbf{approximate unit}, a net $e_i \subseteq \algebra{A}$ such that $e_i a \to a$ and $a \cdot e_i \to a$ for any $a \in \algebra{A}$ as well as $\|e_i\|$.
This is always the case for unital and $C^*$-algebras.

Assume $\iota$ is injective. Then $\algebra{A}$ is identified with an essential ($^*$)-ideal of $M(\algebra{A})$.

\begin{remark}[Norms on the multiplier]
	If $\algebra{A}$ is a Banach algebra with an approximate unit, we define for $\mu = (L,R) \in M(A)$ the norm 
	\begin{equation*}
		\|\mu\| \coloneq \|L\| = \|R\| < \infty\text{.}
	\end{equation*}
\end{remark}

\begin{proof}
	To show $\|L\|, \|R\| < \infty$ we use the Closed Graph Theorem. 
	Say we have $(a_n) \subseteq \algebra{A}$ with $a_n \to a$ and $L(a_n) \to b$.
	Take $c \in \algebra{A}$ and consider
	\begin{equation*}
		c\cdot L(a) = R(c) \cdot a  = \lim_{n \to \infty} R(c) \cdot a_n = \lim_{n \to \infty} c \cdot L(a_n) = c \cdot b\text{.}
	\end{equation*}
	Because of the approximate unit (or $\iota$ injective) we have $L(a) = b$.
	This shows that $L$ (and, analogously, $R$) are bounded.
	Now to prove $\|L\| = \|R\|$. Take any $a \in \algebra{A}$ and consider
	\begin{equation*}
		\|L(a)\| \overset{\text{approx. unit}}{=} \sup_{\|b\| \leq 1} \|b L(a) \| = \sup_{\|b\| \leq 1} \|R(b) a\| \leq \sup_{\|b\| \leq 1} \|R(b)\|  \|a\| \leq \|R\|  \cdot \|a\|
	\end{equation*}
	which implies $\|L\| \leq \|R\|$.
	By symmetry of the situation, we have $\|L\| = \|R\|$.
\end{proof}

With the norm above, $M(\algebra{A})$ becomes a Banach algebra.

\begin{proposition}
	If $\algebra{A}$ is a $C^*$-algebra then $M(\algebra{A})$ is too.
\end{proposition}


\begin{proof}
	Write $\mu = (L,R)$.
	We compute $\mu^* \mu = (R^*, L^*) \cdot (L, R) = (R^* L, R L^*)$.
	So $\|\mu \mu^*\| = \| R^* L \|$.
	Take $a \in \algebra{A}$ with $\|a\| \leq 1$.
	Then
	\begin{equation*}
		\|L(a)\|^2 = \|L(a) L(a)^* \| = \|L(a) L^*(a^*) \| = \| R^*(L(a)) a^* \| \leq \| R^*(L(a)) \| \leq \|R^* L\|
	\end{equation*}
	This shows $\|L\|^2 \leq \|R^* L\|$ and therefore $\|\mu\|^2 = \|L\|^2 \leq \|R^* L\| = \| \mu^* \mu\|$.
	Because $\|\mu\|^2 \geq \|\mu \mu^*\|$ is clear by submultiplicativity, the $C^*$-property follows.
\end{proof}

Compare now $\tilde{\algebra{A}}$ and $M(\algebra{A})$.
We have $\algebra{A} \trianglelefteq \tilde{A}$ and $\algebra{A} \trianglelefteq M(\algebra{A})$.
When are these ideals essential?

\begin{lemma}
	Let $\algebra{A}$ be a $C^*$-algebra or Banach algebra with approximate unit. 
	$\algebra{A} \trianglelefteq \tilde{\algebra{A}}$ if and only if $\algebra{A}$ is not unital.
\end{lemma}

\begin{proof}
	Suppose that $\algebra{A}$ is unital with $1_\algebra{A}$ as the unit.
	In this case, take $p = 1 - 1_\algebra{A} \in \tilde{\algebra{A}}$ (where $1 = (0,1)$ is the unit in $\tilde{\algebra{A}}$).
	Notice that $p \cdot \algebra{A} = 0$, but $p \neq 0$.
	So $\algebra{A}$ is not essential in $\tilde{\algebra{A}}$.

	Suppose that $\algebra{A}$ is not unital.
	To prove: For $a + \lambda \cdot 1 \in \tilde{\algebra{A}}$ and $(a + \lambda \cdot 1) \algebra{A} = 0$ we have $a = 0$, $\lambda = 0$.
	So take any $(a + \lambda \cdot 1) \cdot b = 0$ for all $b \in \algebra{A}$, that is $ab + \lambda b = 0$.
	This means  $L_a(b) = - \lambda b$, that is $L_a = - \lambda \id_\algebra{A}$.
	Notice $L: \algebra{A} \to \mathcal{L}(\algebra{A})$, a unital algbra with unit $\id_\algebra{A}$, is an injective (because $\iota$ is injective) algebra homomorphism.
	If $\lambda \neq 0$, then division by $\lambda$ implies $\id_\algebra{A} \in \image(L) \simeq \algebra{A}$.
	But then $\algebra{A}$ has a unit, a contradiction.
	So $\lambda = 0$.
	Then $a \cdot b = 0$ for every $b$, so $a = 0$ as well.
	This shows that $\algebra{A}$ is an essential ideal of $\tilde{\algebra{A}}$.
\end{proof}

\begin{remark}
	Let $\algebra{A}$ be a $C^*$-algebra or Banach algebra with approximate unit.
	Then $\algebra{A}$ is unital if and only if $M(\algebra{A}) = \algebra{A}$.
\end{remark}

\begin{proof}
	One direction is simple: $M(\algebra{A})$ is always unital, so $\algebra{A} \simeq M(\algebra{A})$ implies that $\algebra{A}$ is unital.

	Let now $\algebra{A}$ be unital and prove that every multiplier is of the form $(L_a, R_a)$. 
	Let $\mu = (L,R) \in M(\algebra{A})$ and define $a \coloneq L(1_\algebra{A})$.
	Then $L_a(b) = ab = L(1_\algebra{A}) b = L(b)$, so $L = L_a$.
	Analogously we can prove $R = R_a$.
	This shows that $\iota$ is surjective, and since it is already injective (because $\algebra{A}$ is either $C^*$ or has an approximate unit) it is an isomorphism.
\end{proof}



Say $\algebra{A}$ is a $C^*$-algebra (or a Banach algebra with an approximate unit) and not unital.
Then $\iota: \algebra{A} \to M(\algebra{A}), a \mapsto \mu_a = (L_a, R_a)$ extends to a ($^*$)-embedding 
\begin{equation*}
	\tilde{\iota}: \tilde{\algebra{A}} \to M(\algebra{A}), a + \lambda \cdot 1 \mapsto \iota(a) + \lambda \cdot \underbrace{(\id, \id)}_{= \id_{M(\algebra{A})}}\text{.}
\end{equation*}
More generally: If $\algebra{B}$ is any  $C^*$-algebra that contains $\algebra{A}$ as an essential ideal (closed), then $\algebra{B}$ embeds in the multiplier algebra via the following map:
\begin{equation*}
	\lambda: \algebra{B} \to M(\algebra{A}), b \mapsto (L_b, R_b)
\end{equation*}
where $L_b, R_b$ are the usual left and right multiplication.
We have $L_b(a), R_b(a) \in \algebra{A}$ for any $a \in \algebra{A}$ because $\algebra{A}$ is an ideal.
The above is a universal property of the multiplier algebra.
$M(\algebra{A})$ is the largest unital $C^*$-algebra that contains $\algebra{A}$ as an essential ideal.

\begin{example}
	Take $\algebra{A} = C_0(X)$ (for a locally compact Hausdorff-space, so a commutative $C^*$-algebra).
	Then $\tilde{\algebra{A}} = C(\tilde{X})$ where $\tilde{X} = X \sqcup \{\infty\}$.
	One can now show $M(\algebra{A}) \simeq C(\beta X)$ where $\beta X$ is the Stone-Cech-compactification of $X$.
	This can be proven using the universal property and the universal property of $\beta X$:
	$\beta X$ is a compact Hausdorff space such that $X \hookrightarrow \beta X$ as a dense open topological subspace and for every other compact Hausdorff space $K$ such that $X \to K$ via a continuous function $f$ there exists a unique continuous extension $beta f: \beta X \to K$.

	First: Prove that $M(\algebra{A})$ is even commutative.
	Then it is the continuous functions on some space, use the spectrum and compare the universal properties.
	For commutativity, one can show $M(C_0(X)) \simeq C_b(X)$ via the universal property.
\end{example}



\subsection{Positive Elements of $C^*$-algebras}

\begin{definition}
	Let $\algebra{A}$ be a $C^*$-algebra. We say that $a \in \algebra{A}$ is positive (and write $a \geq 0$) if $a = a^*$ and $\sigma(a) \subseteq [0, \infty)$. 

	The set of all positive elements of a given algebra we notate as $\algebra{A}_+$.
\end{definition}

\begin{example}
	Let $A = C_0(X)$ (commutative) and $f \in \algebra{A}$. Then $f = f^*$ iff $f$ is real (that is $f: X \to \mathds{R}$). Since $\sigma(f) = \closure{\image(f)}$ we see that $f \geq 0$ iff $f(x) \geq 0$ for all $x \in X$.
\end{example}

\begin{theorem}
	If $a \in \algebra{A}$ for $\algebra{A}$ a $C^*$-algebra and $a \geq 0$ then there exists a unique $b \in \algebra{A}_+$ such that $b^2 = a$.
	We sometimes notate this as $b = \sqrt{a} = a^{\frac{1}{2}}$.
\end{theorem}

\begin{proof}
	Since $a$ is positive, it is self-adjoint and therefore normal.
	Continuous functional calculus:
	\begin{equation*}
		\phi: C_0(\sigma(a)) \to \algebra{A}, f \mapsto f(a)
	\end{equation*}
	Apply this to $f(x) = \sqrt{x}$.
	Notice that $f \in C_0(\sigma(a))$ because $\sigma(a) \subseteq [0, \infty)$.
	Now simply choose $b = f(a) = \sqrt{a}$.
	Since $\phi$ is a $^*$-homomorphism, we have $b^2 = \phi(f)^2 = \phi(f^2) = \phi(id) = a$.

	Reminder: Writing '$f(a)$' does not mean to imply that $a \in \algebra{A}$ can simply be plugged into the function $f: \sigma(\algebra{A}) \to \mathds{C}$ but is simply a different way of writing $\phi(f) \in \algebra{A}$.

	\textbf{Uniqueness}: Suppose $c \in \algebra{A}_+$ such that $c^2 = a$.
	Then $c$ commutes with $c^2 = a$ and therefore $c$ commutes with $b = \sqrt{a}$ since $b = \lim_{n \to \infty} p_n(a)$ (polynomial approximation).
	Then $B \coloneq C^*(b,c) \subseteq \algebra{A}$ is a commutative $C^*$-algebra so $B \simeq C_0(X)$ for some locally compact Hausdorff space $X$.
	Since $a,b,c \in B = C_0(X)$ we have $a \simeq f, b \simeq g, c \simeq h \in C_0(X)$ with $f = g^2 = h^2$ where all these functions are positive. 
	But then $f(x) = g(x)^2 = h(x)^2$ for all $x$.
	Because $g(x), h(x) \geq 0$ for all $x$, this shows $g(x) = h(x)$ for all $x$ and therefore $g = h$ and $b = c$.
\end{proof}

\begin{remark}
	Given any self-adjoint element $a \in \algebra{A}$ ($a^* = a$) we can write it as $a^+ - a^-$ where $a^+, a^- \geq 0$ and $a^+ \cdot a^- = 0$.
	Just define $f(x) = \frac{|x| + x}{2}$ and $g(x) = \frac{|x| - x}{2}$.
	Both are positive functions with $f \cdot g = 0$. Define $a^+ = f(a)$ and $a^- = g(a)$ (once again per continuous functional calculus), transferring the neccessary properties:
	\begin{align*}
		f(a) - g(a) &= \phi(f) - \phi(g) = \phi(f - g) = \phi(\id) = a \\
		f(a) \cdot g(a) &= \phi(f) \cdot \phi(g) = \phi(f \cdot g) = \phi(0) = 0\\
		\sigma(f(a)) &= \sigma(\phi(f)) \subseteq \sigma(f) = \closure{\image(f)} = [0, \infty)
	\end{align*}
\end{remark}

\begin{remark}
	If $\algebra{A}$ is unital $C^*$-algebra and $a \in \algebra{A}$ is self-adjoint with $\|a\| \leq 1$, so $\sigma(a) \subseteq [-1, 1]$.
	Define 
	\begin{equation*}
		f(x) = x + i \sqrt{1 - x^2} \qquad g(x) = x - i \sqrt{1 - x^2}
	\end{equation*}
	This means that $f,g \in \mathcal{U}C(\sigma(a))$
	(Recall that unitaries of $\mathcal{U}(\algebra{A}) = \{u \in \algebra{A} \mid u^* u = 1 = uu^* \}$)
	and $\frac{f + g}{2} = \id_{\sigma(a)}$.
	So if we now define $u \coloneq f(a), v \coloneq g(a) \in C^*(a,1) \subseteq  \algebra{A}$ we have $\frac{u + v}{2} = a$.
	In particular $\algebra{A} = \mathrm{span}(\mathcal{U}(\algebra{A}))$.
\end{remark}

\begin{lemma}
	Let $\algebra{A}$ be a unital $C^*$-algebra, $a \in \algebra{A}$ self-adjoint and $t \in \mathds{R}_+$.
	\begin{enumerate}
		\item If $a \geq 0$ and $\|a\| \leq t$ then $\|a - t\| \leq t$.
		\item Conversely, if $\|a = t\| \leq t$ then $a \geq 0$.
	\end{enumerate}
\end{lemma}

\begin{proof}
	Replace $\algebra{A}$ by $C^*(a, 1)$ we may assume that $\algebra{A} = C(X)$ is commutative and $X$ compact.
	Let $a = f \in C_(X)$ be a self-adjoint, real function and $t \geq 0$ a real number.
	\begin{enumerate}
		\item $f \geq 0$ and $\|f\|_\infty \leq t$ and thus $f(x) - t \in [-t, 0]$ for all $x \in X$, so $\|f - t\| \leq t$.
		\item Let $f \in C(X)$ be a self-adjoint real function with $\|f - t\| \leq t$, so $|f(x) - t| \leq t$ for every $x$.
		But if $f(x) < 0$ for any $x \in X$ we have $f(x) - t < t$ and thus $|f(x) - t| > t$, a contradiction.
		So $f$ must be positive.
	\end{enumerate}
\end{proof}

\begin{corollary}
	If $\algebra{A}$ is a $C^*$-algebra, then $\algebra{A}_+$ is a closed subset (but not subspace!) of $\algebra{A}$.
\end{corollary}

\begin{proof}
	Taking unitization, we may assume that $\algebra{A}$ is unital.
	Let $(a_n) \subseteq \algebra{A_+}$ and $a_n \to a \in \algebra{A}$.
	Then $a_n^* = a_n$ for all $n \in \mathds{N}$ and therefore $a$ is also self-adjoint.
	There also exists $t \geq 0$ with $\|a_n\| \leq t$ for all $n \in \mathds{N}$ and by the Lemma $\|a_n - t \| \leq t$ and therefore $\|a - t\| \leq t$. 
	Again by the Lemma $a \geq 0$.
\end{proof}

\begin{corollary}
	If $\algebra{A}$ is a $C^*$-algebra and $a,b \in \algebra{A}_+$ then $a + b \in \algebra{A}_+$.
\end{corollary}

\begin{proof}
	Taking unitization, we may assume that $\algebra{A}$ is unital.
	Since $a,b \geq 0$ by $t = \|a\|, \|b\|$ we have $\|a - \|a\| \| \leq \|a\|$ and $\|b  - \|b\|\| \leq \|b\|$. Then
	\begin{equation*}
		\|(a + b) - (\|a\| + \|b\|) \| = \|(a - \|a\|)	+ (b - \|b\|)\|  \leq \|(a - \|a\|)\|	+ \|(b - \|b\|)\| \leq \|a\| + \|b\|
	\end{equation*}	
	and $a + b$ is positive by the lemma.
\end{proof}

\begin{theorem}
	If $\algebra{A}$ is a $C^*$-algebra and $a \in \algebra{A}$ then $a^* a \geq 0$.
\end{theorem}

\begin{proof}
	First, we prove that if $ - a^* a \geq 0$ then $a = 0$.
	For this we use the following observation $\sigma(bc) \setminus \{0\} = \sigma(cb) \setminus \{0\}$ (the two sets are equal except for the zero, which may be contained in one but not the other) because for $b,c$ in a unital algebra and $1 - bc \in \inv{\algebra{A}}$ iff $1 - cb \in \inv(\algebra{A})$ and if $d \coloneq (1 - bc)^{-1}$ then $(1 - cb)^{-1} = 1 + cdb$.

	Therefore, if $-a^* a \in \algebra{A}_+$ then also $- a^* a \in \algebra{A}_+$ (notice that $a, a^*$ are self-adjoint).
	Then write $a = b + c$ with $b,c \in \algebra{A}$ self-adjoint. Then
	\begin{equation*}
		a^* a + a a^*  = (b - ic) (b + ic) + (b + ic) (b-ic) = b^2 + c^2 + ibc - icb + b^2 + c^2 + icb - icb = 2b^2 + 2c^2 \text{.}
	\end{equation*}
	and we can write $a^* a = 2b^2 + 2c^2 - a a ^*$.
	The squares are certainly positive and we have assumed $-a a^* \geq 0$, but then $a^* a \geq 0$.
	We see that $a a^* \geq 0$ as well, so the spectrum has to be zero. 

	Now suppose that $a \in \algebra{A}$ arbitrarily. We show that $a^* a \geq 0$. Let $b \coloneq a^* a$.
	Then $b \in \algebra{A}$ is self adjoint with $b = b^+ - b^-$ where $b^+ , b^- \geq 0$.
	Let $c \coloneq a b^-$. Then
	\begin{equation*}
		- c^* c = - b^- a^* a b^- = - b^- (b^+ - b^-) b^- = (b^-)^3 \geq 0
	\end{equation*}
	and $c$ must be $0$ by our first result.
	This implies $(b^-)^3 = 0$ so $b^- = 0$.
	It follows that $b = b^+ \geq 0$.
\end{proof}

\begin{definition}
	Let $\algebra{A}$ be a self-adjoint algebra and $a,b \in \algebra{A}$. We write $a \leq b$ if $b- a \geq 0$.
	This turns $\algebra{A}$ into a poset.
	Because $A_+$ is a cone, that is $A_+ + A_+ \subseteq A_+$ and $\mathds{R}_+ \cdot \algebra{A}_+ \subseteq \algebra{A}$ as well as $A_{\text{self-adjoint}} = A_+ - A_+$ and $A_+ \cap -A_+ = \{\}$.
\end{definition}

\begin{theorem}
	Let $\algebra{A}$ be a $C^*$-algebra.
	\begin{enumerate}
		\item $A_+ = \{a^* a \mid a \in \algebra{A}\}$
		\item $a,b$ self-adjoint and $c \in \algebra{A}$. Then $a \leq b$ imples $c^* a c \leq c^* b c$.
		\item $0 \leq a \leq b$ implies $\|a\| \leq \l|b\|$
		\item If $\algebra{A}$ is unital and $a,b \geq 0$ with $a \leq b$ and $a,b \in \inv(\algebra{A})$ then $b^{-1} \leq a^{-1}$.
	\end{enumerate}
\end{theorem}

\begin{proof}~
	\begin{enumerate}
		\item It follows from the previous theorem. The fact that $a \in \algebra{A}_+$ has a square root $a = b^2 = b^* b$  with $b \geq 0$.
		\item $c^* b c - c^* a c = c^* (b - a) c$ and if we set $b - a = d^* d$ for a $d \in \algebra{A}$ we receive $c^*(b-a)c = c^* d^* d c = (dc)^* dc \geq 0$.
		\item We may assume $1 \in \algebra{A}$.
		Notice that $b \leq \|b\| \cdot 1$ (consider the commutative case).
		So wie have $a \leq b \leq \|b\| \cdot 1$ and therefore $a \leq \|b\| \cdot 1$ so $\|a\| \leq \|b\|$.
		\item Let $a,b \in \inv \algebra{A}$, $a,b \geq 0$ and $a \leq b$.
		We know that $\sigma(b^{-1}) = \sigma(b)^{-1} \subseteq \mathds{R}_+$ and thus $b^{-1} \geq 0$ and Similarly $a^{-1} \geq 0$.
		Notice that if $c \geq 1$ (in $\algebra{A}$) then $c \in \inv \algebra{A}$ (as $\sigma(c-1) \subseteq [0, \infty)$ and thus $\sigma(c) \subseteq [1, \infty))$ and $c^{-1} \leq 1$ (think once again commutative).

		Now we have $a \leq b$.
		Then $1 = a^{-\frac{1}{2}} a a^{-\frac{1}{2}} \leq a^{-\frac{1}{2}} b a^{-\frac{}{2}}$.
		Then $(a^{-\frac{1}{2}} b a^{-\frac{1}{2}}) = (a^\frac{1}{2} b^{-1} a^\frac{1}{2}) \leq 1$ by the above, so conjugation yields $b^{-1} \leq a^{-1}$.
	\end{enumerate}
\end{proof}

\subsection{Approximate units}

\begin{definition}
	Let $\algebra{A}$ be a Banach algebra. An \textbf{approximate unit} for $\algebra{A}$ is a net $(e_i)_{i \in I} \subseteq \algebra{A}$ such that $\|e_i\| \leq 1$ and $e_i a \to a, a e_i \to a$ for all $a \in \algebra{A}$. If $\algebra{A}$ is a $C^*$-algebra, then we (usually) also assume that $e_i \geq 0$ and $(e_i)$ is increasing.
\end{definition}

\begin{example}
	Let $\algebra{A} = C_0(X)$ be a commutative $C^*$-algebra ($X$ locally compact and Hausdorff).
	Then a net $(f_i)_{i \in I}$ is an approximate unit if and only if $1 \geq f_i(x) \geq f_j(x) \geq 0$ for all $x \in X$ and $j \leq i$ and $f_i g \to g$ for al $g \in C_0(X)$, that is $f_i(x) g(x) \to g(x)$ uniformly on $X$.
	This is equivalent to $f_i(x) \to 1$ uniformly on compacts.
\end{example}

\begin{example}
	Let $\algebra{A} = \mathcal{K}(H)$, the span of the compact operators on a Hilbert space $H$, and use physics notation: $|\xi \rangle \langle \eta|(\zeta) = \xi \langle \eta, \zeta\rangle$. 
	Let $(\xi_i)_{i \in I} \subseteq H$ be an orthonormal basis.
	For each $F \subseteq I$ finite we define
	\begin{equation*}
		e_F \coloneq \sum_{i \in F} | \xi_i \rangle \langle \xi_i | \in \mathcal{K}(H)
	\end{equation*}
	In particular, $0 \leq e_F \leq 1$ (because $\|e_F\| \leq 1$) and $e_F \leq e_G$ if $F \subseteq G$.
	Then $(e_F)_{F \subseteq I \text{ finite}}$, if ordered by size, is an approximate unit of for $\mathcal{K}$.

	If $H$ is separable, we could also take $e_n = \sum_{i = 1}^n | \xi_i \rangle \langle \xi_i |$.
	Just check that $e_F(\zeta) = \sum_{i \in F} \xi_i \langle  \xi_i, \zeta \rangle \to \zeta$, so $e_F \to 1$ strongly in $B(H)$ (the bounded operators).
	Then it follows $e_F a \to a$ for al $a \in \mathcal{K}(H)$ and $a e_F \to a$ likewise.
\end{example}

\begin{remark}
	If $\algebra{A}$ already has a unit $1 \in \algebra{A}$, then $(e_i) \subseteq \algebra{A}$ is an approximate unit iff $e_i \to 1$ (by the norm) and $0 \leq e_i \leq e_j \leq 1$ for $i \leq j$.

	In particular, the constant net $(1)$ is an approximate unit in any unital Banach algebra.
\end{remark}

\begin{theorem}
	Every $C^*$-algebra has an approximate unit. Moreover if $\algebra{A}$ is a $C^*$-algebra and
	\begin{equation*}
		\Lambda \coloneq \{a \in \algebra{A}_+ \mid \|a\| < 1\}
	\end{equation*}
	then $\Lambda$ is directed with the canonical order of $\algebra{A}_+ \subseteq \algebra{A}_{\text{self-adjoint}}$ and the canonical net 
	\begin{equation*}
		(e_\lambda)_{\lambda \in \Lambda} e_\lambda = \lambda
	\end{equation*}
	is an approximate unit.
\end{theorem}

\begin{proof}
	$\Lambda$ is directed. To prove: For every $a,b \in \Lambda$ there is a $c \in \Lambda$ such that $a,b \leq c$.
	Indeed, if $a \in \algebra{A}_+$, then $1 + a \geq 1$ in $\tilde{\algebra{A}} = \algebra{A} + \mathds{C} \cdot 1$.
	Here, we work in the unitization for a moment, but do not assume we have a unit in $\algebra{A}$!
	In particular, $1 + a \in \inv (\tilde{\algebra{A}})$ and $a \cdot (1 + a)^{-1} \in \algebra{A}$ as $A \trianglelefteq \tilde{A}$.

	Notice: $ a (1 + a)^{-1} = (a + 1 - 1)(1 + a)^{-1} = 1 - (1+a)^{-1}$ in the unitization.

	Claim: For $a, b \in \algebra{A}_+$ and $a \leq b$ we have $a (1 + a)^{-1} \leq b (1+b)^{-1}$.
	This should be true because $a (1 + a)^{-1} = f(a)$ where $f: [0, \infty) \to [0,1), x \mapsto \frac{x}{x+1} = x (1+x)^{-1}$ is increasing.
	$f$ is a homeomorphism with $g = f^{-1}: [0,1) \to [0, \infty)$ given by $g(x) = \frac{x}{x-1}$.

	Indeed, take $0 \leq a \leq b$ then $1 + a \leq 1 + b$ so $(1 + b)^{-1} \leq (1 + a)^{-1}$ and therefore $a(1+a)^{-1} = 1 - (1 + a)^{-1} \leq 1 - (1 + b)^{-1} = b(1+b)^{-1}$.
	Now observe that if $a \in \algebra{A}_+$ then $f(a) = a(1+a)^{-1} \in \Lambda$ because $\|f\|_{\sigma(a) \subseteq [0, \infty)}$ and thus $0 \leq f < 1$.
	So we get an increasing map $\algebra{A}_+ \to \Lambda, a \mapsto a(1+a)^{-1}$.
	Now suppose $a,b \in \Lambda$, consider $g = f^{-1}: [0,1) \to [0, \infty), x \mapsto \frac{x}{x-1}$.
	Define $a' \coloneqq g(a), b' \coloneqq g(b)$ and let $c \coloneqq (a' + b')(1 + a' + b')^{-1} = f(a' + b')$.
	Then $c \in \Lambda$ and since $a' \leq a' + b'$ we have $a = f(a') \leq f(a' + b') = c$ and likewise $b \leq c$.
	This shows that $\Lambda$ is a directed set.

	Now we have to check that $(e_\lambda)_{\lambda \in \Lambda}$ with $e_\lambda = \lambda$ is an approximate unit for $\algebra{A}$.
	Notice that $(e_\lambda)$ is increasing and $e_\lambda = \lambda \geq 0$ and $\|e_\lambda\| < 1$ for all $\lambda$.
	So we need only prove $e_\lambda \cdot a \to a \leftarrow a \cdot e_\lambda$ for every $a \in \algebra{A}$.
	But using the involution, these two are equivalent:
	\begin{equation*}
		(e_\lambda a) \to a \iff (e_\lambda a)^* \to a^* \iff a^* e_\lambda \to a \iff a^* e_\lambda \to a^*
	\end{equation*}
	It is even enough to prove $a e_\lambda \to a$ for only $a \in \Lambda$ because $\Span \Lambda = \Span(\algebra{A}_+) = \algebra{A}$.
	Let $a \in \Lambda$, in particular $a \in \algebra{A}_+$.
	Consider 'its' Gelfand representation $ \phi: C^*(a) \to C_0(X)$ and let $f = \phi(a) \in C_0(X)$.
	This function fulfils $0 \leq f(x) < 1$ for all $x \in X$ because it comes from $a \in \algebra{A}_+$.

	Let furthermore $\epsilon > 0$ and $K \coloneq \{x \in X \mid |f(x)| \geq \epsilon \} \subseteq X$ compact.
	By Uryson's Lemma, we have a $g \in C_0(X), g: X \to [0,1]$ such that $g(x) = 1$ for all $x \in K$.
	Next, choose $\delta > 0$ with $\delta < 1$ and $1 - \delta < \epsilon$.
	Then $g_\delta = \delta \cdot g \leq \delta$ and therefore
	\begin{align*}
		\|f - g_\delta \cdot f\| &= \|f - \delta g f \| = \sup_{x \in X} \|f(x)\| \cdot \| 1 - \delta g(x) \| \\ &\leq \max \{ \sup_{x \in K} \|f(x)\| \cdot \|1 - \delta g(x) \|, \sup_{x \notin K} \|f(x)\| \cdot \|1 - \delta g(x) \| \} \\ &\leq \max\{ \epsilon, 1 - \delta \}  \leq \epsilon
	\end{align*}
	Now let $b \coloneq \phi^{-1}(g_\delta) \in \algebra{A}_+$ with $\|b \| < 1$ and $\| a - ba \| < \epsilon$.

	This shows that for any $a \in \Lambda$ we can find $\lambda_0 = b \in \Lambda$ such that $\|a - e_{\lambda_0} a \| < \epsilon$.
	If now $\lambda \in \Lambda, \lambda \geq \lambda_0$ we have $e_{\lambda_0} \leq e_\lambda$, so $1 - e_\lambda \leq 1 - e_{\lambda_0}$ (in $\tilde{\algebra{A}}$) and therefore $a (1 - e_\lambda ) a \leq a (1 - e_{\lambda_0}) a$ (*) (by conjugation property and because $a$ is self-adjoint).
	But then
	\begin{align*}
		\|a - e_\lambda a\|^2 \|(1 - e_\lambda a)\|^2 &= \|\overbrace{\underbrace{(1 - e_\lambda)^{\frac{1}{2}} \cdot (1-e_\lambda)^{\frac{1}{2}}}_{\in \tilde{\algebra{A}}} a}^{\in \algebra{A} \trianglelefteq \tilde{\algebra{A}}} \|
		\leq \|(1 - e_\lambda)^{\frac{1}{2}} a \|^2 \\
		&\overset{(*)}{\leq} \| a (1 - e_\lambda) a \| \leq \| a (1 - e_{\lambda_0}) a \| \overset{\|a\| \leq 1}{\leq} \|(1 - e_{\lambda_0}) a \|\\
		&= \|a - e_{\lambda_0}\| < \epsilon
	\end{align*}
	so $e_\lambda a \to a$.
\end{proof}

\begin{definition}
	In general, $C^*$-algebras do not admit a sequential approximate unit.

	We say that a $C^*$-algebra $\algebra{A}$ is $\sigma$-unital if there exists such a sequential approximate unit $(e_n)_{n \in \mathds{N}}$.
\end{definition}

\begin{example}
	$\algebra{A} = C_0(X)$ is $\sigma$-unital if and only if $X$ is $\sigma$-compact: $X = \bigcup_{n = 1}^\infty K_n$ where $K_n \subseteq X$ are compact spaces.
\end{example}

\section{Ideals in $C^*$-algebras}

\begin{theorem}
	Let $\algebra{A}$ be a $C^*$-algebra and $L \subseteq \algebra{A}$ a left closed ideal. 
	Then there exists a net $(u_\lambda)_{\lambda \in \Lambda} \subseteq A_{+, 1} \cap L$ (that is, elements with $0 \leq u_\lambda$ and $\|u_\lambda\| \leq 1$) such that $a = \lim_\lambda a u_\lambda$ for all $a \in L$.
\end{theorem}

\begin{proof}
	Set $B = L \cap L^*$.
	This is clearly a $C^*$-subalgebra.
	There is now an approximate unit $(u_\lambda) \subseteq B_{+,1} \subseteq A_{+,1}$ for $B$.
	Let $a \in L$.
	Then $a^* a \in L \cap L^* \in B$ and we have $\lim_\lambda a^* a u_\lambda = a^* a = \lim_\lambda u_\lambda a$.
	It follows that
	\begin{align*}
		\lim_\lambda \|a - a u_\lambda\|^2 &= \lim_\lambda \|(a - a u_\lambda)^* (a - a u_\lambda)\| = \lim_\lambda \|a^* a - a^* a u_\lambda - u_\lambda a^* a - u_\lambda a^* a u_\lambda\|\\
		&\leq \lim_\lambda \| a^* a - a^* a u_\lambda \| + \lim_\lambda \|u_\lambda\| \cdot \| a^* a - a^* a u_\lambda\| = 0
	\end{align*}
\end{proof}

Let $L \subseteq \algebra{A}$ be a closed left ideal and $(u_\lambda) \subseteq B = L \cap L^* \subseteq \algebra{A}$. Then $\lim_\lambda a u_\lambda = a$ for all $a \in L$.
As a consequence:

\begin{theorem}
	Every closed two-sided ideal $I \trianglelefteq \algebra{A}$ of a $C^*$-algebra satiesfies $I^* = I$, so it is a $^*$-ideal and in particular a $C^*$-algebra.
\end{theorem}

\begin{proof}
	By the lemma above, we find a net $(u_\lambda) \subseteq I$, $u_\lambda \geq 0$, such that $a = \lim_\lambda a u_\lambda$
	Then $a^* = \lim_\lambda u_\lambda a^* \in I$ (because $u_\lambda \in I$).
\end{proof}

\begin{corollary}
	Let $I \trianglelefteq \algebra{A}$ be a closed two-sided ideal of a $C^*$-algebra $\algebra{A}$.
	Then for all $a \in \algebra{A}$, $\|a + I \| = \lim_\lambda \|a - u_\lambda a\| = \lim_\lambda \|a - a u_\lambda\|$ where $(u_\lambda)$ is an approximate unit for $I$.
\end{corollary}

\begin{proof}
	Let $\epsilon > 0$ and take $b \in I$ such that $\|a + b\| \leq \|a + I \| + \frac{\epsilon}{2}$.
	Recall that $\|a + I\| = \mathrm{dist}(a, I) = \inf_{b \in I} \|a + b\|$.

	Since $\lim_\lambda u_\lambda b = b$.
	Then there exists $\lambda_0$ such that $\|b - u_\lambda b\| < \frac{\epsilon}{2}$ for all $\lambda \geq \lambda_0$. 
	Then 
	\begin{align*}
		\|a - u_\lambda a\|
		&\leq \| (1 - u_\lambda) (a + b) \| + \|b - u_\lambda b\| \\
		&\leq \|a + b\| + \|b - u_\lambda b\| \\
		&< \|a + I\| + \frac{\epsilon}{2} + \frac{\epsilon}{2} = \|a + I\| + \epsilon
	\end{align*}
	On the other hand, $\|a - u_\lambda a\| \geq \|a + I\|$ for all $\lambda$ and $\|a + I\| = \lim_\lambda \|a + u_\lambda a\| = \inf_\lambda \|a - u_\lambda a\|$.
	This shows the existence of the limit and therefore that the norm equals the distance.
\end{proof}

\begin{theorem}
	If $I \trianglelefteq \algebra{A}$ is a closed $^*$-ideal in a $C^*$-algebra $\algebra{A}$, then $\algebra{A}/I$ is itself a $C^*$-algebra.
\end{theorem}

\begin{proof}
	We already know that $\algebra{A}/I$ is a Banach $^*$-algebra. We only need to show that $\|a + I\| = \| (a + I)^* (a + I)\|$.

	Let $(u_\lambda) \subseteq I$ be an approximate unit and take $b \in I$. Then
	\begin{align*}
		\|a + I\|^2 &= \lim_{\lambda} \|a - a u_\lambda\|_A^2  \overset{*}{=} \lim_\lambda \|(1 - u_\lambda) a^* a (1 - u_\lambda)\|\\
		&\leq \sup_\lambda  \|(1 - u_\lambda) (a^* a + b) (1 - u_\lambda) \| + \lim_\lambda \|(1 - u_\lambda) b (1 - u_\lambda)\| \\
		&\leq \|a^* a + b\|
	\end{align*}
	\begin{quote}
		Where $*$ is because we can use the $C^*$-property of $\algebra{A}$ and $(1 - u_\lambda)$ is self-adjoint.
		The last inequality follows because the latter limit tends to $0$.
	\end{quote}
	Since $b$ was arbitrary, we get
	\begin{equation*}
		\|a + I \|^2 \leq \inf_{b \in I} \|a^* a + b\|_\algebra{A} = \|a^* a + I \| = \|(a + I)^* (a + I) \|
	\end{equation*}
\end{proof}

\begin{theorem}
	If $\phi: \algebra{A} \to \algebra{B}$ (where $\algebra{A}, \algebra{B}$ are $C^*$ algebras) is an injective $^*$-homomorphism, then $\phi$ is isometric, i.e. $\|\phi(a)\| = \|a\|$ for all $a \in \algebra{A}$.
\end{theorem}

\begin{proof}
	It suffices to show that $\|\phi(a)\|^2 = \|a\|^2$ or $\|\phi(a^* a)\| = \|a^* a\|$.

	Replacing $\algebra{A}$ by the $C^*$-algebra $C^*(a^* a)$ and $B$ by $C^*(\phi(a^* a)) \subseteq B$ (with $a^*a, \phi(a^* a) = \phi(a)^* \phi(a) \geq 0$) we may assume that $\algebra{A}, \algebra{B}$ are commutative.
	Also by adding units and extending $\phi$ to the unitization $\tilde{\phi}: \tilde{\algebra{A}} \to \tilde{\algebra{B}}$ we may assume that $\algebra{A}, \algebra{B}, \phi$ are unital.
	Now given $\chi \in \Omega(\algebra{B})$ notice that $\chi \circ \phi \in \Omega(\algebra{A})$.
	So we get a map $\phi_*: \Omega(\algebra{B}) \to \Omega(\algebra{A}), \chi \mapsto \chi \circ \phi$. 
	This is clearly continuous. 
	Since $\Omega(\algebra{B})$ is compact, $K \coloneq \phi_*(\Omega(\algebra{B}))$ is compact (in particular closed).
	By Uryson's Lemma, we find some continuous function $f \in C(\Omega(\algebra{A}))$ such that $f|_K \equiv 0$ and $f \neq 0$ (if we assume $K \neq \Omega(\algebra{A})$).
	By Gelfand-Representation we find ($\algebra{A} \simeq C(\Omega(\algebra{A}))$) and $a \in \algebra{A}$ such that $\hat{a} = f$.
	Then for each $\chi \in \Omega(\algebra{B})$,
	\begin{equation*}
		\chi(\phi(a)) = \hat{a}(\chi \circ \phi) = \underbrace{\hat{a}}_{f} (\underbrace{\phi_*(\chi)}_{\in K}) = 0 \then \phi(a) = 0
	\end{equation*} 
	and if $f \neq 0$, then $a \neq 0$.
	But we have $\phi(a) = 0$ for all $a$, a contradiction. 
	Therefore, $\phi_*$ is surjective.
	Now
	\begin{align*}
		\|a\|_\algebra{A} = \|\hat{a}\|_\infty = \sup_{\chi \in \Omega(\algebra{A})} | \chi(a) | = \sup_{\chi \in \Omega(\algebra{B})} |(\chi \circ \phi) (a) | = \| \widehat{\phi(a)} \|_\infty = \| \phi(a) \|_\algebra{B}
	\end{align*}
\end{proof}

\begin{corollary}
	If $\phi: \algebra{A} \to \algebra{B}$ is any $^*$-homomorphism ($\algebra{A}, \algebra{B}$ $C^*$-algebras) then $\phi(\algebra{A})$ is closed, hence a $C^*$-subalgebra of $\algebra{B}$.
\end{corollary}

\begin{proof}
	Consider $\psi: \algebra{A} /_{\ker \phi} \to  \algebra{B}, a + \ker \phi \mapsto \phi(a)$.
	Then $\psi$ is a well-defined $^*$-homomorphism and $\psi$ is injective and therefore isometric.
	This shows that $\psi(\algebra{A}/_{\ker \phi}) = \phi(\algebra{A})$ is closed.
\end{proof}

\begin{remark}
	For some other related consequences, see Murphy's book.
	\begin{enumerate}
		\item If $\algebra{A} \subseteq \algebra{B}$ are $C^*$-algebras and $I \trianglelefteq \algebra{B}$ is a closed 2-sided ideal then $\algebra{A} + I$ is a $C^*$-subalgebra of $\algebra{B}$. 
		In particular, the sum of ideals in $C^*$-algebras are ideals: For any $I,J \trianglelefteq \algebra{A}$ have that $I + J \trianglelefteq \algebra{A}$ as well.
		\item If $I, J \trianglelefteq \algebra{A}$ then $I \cdot J = I \cap J$.
		The product here is defined as the linear span of products ($I \cdot J = \closure{\Span}\{i \cdot j \mid i \in I, j \in J\}$) but is actually just the products.
	\end{enumerate}
\end{remark}


\begin{proof}[Ideas]~
	\begin{enumerate}
		\item To prove that $\algebra{A} + I$ is closed, check that $(A + I)/I$ is Banach by identifying it with
		\begin{equation*}
			(A + I) / I \simeq \algebra{A}/(\algebra{A} \cap I), a + I \leftarrow a + A \cap I
		\end{equation*}
		Can also build arbitrary familys of ideals and the sum will be an ideal, also the intersection and product of ideals exist.
		\item $I \cdot J \subseteq I \cap J$ is clear.
		To prove the converse, use the approximate unit.
		$I \cap J$ is clearly a $C^*$-algebra, take an approximate unit $(u_\lambda) \subseteq I \cap J$ and $x \in I \cap J$.
		Then $x = \lim_{\lambda} x u_\lambda$ where $x u_\lambda$ is in $I \cdot J$ at all times.
	\end{enumerate}
\end{proof}

\section{Gelfand-Neymark representation}

We know for commutative $\algebra{A}$ that $\algebra{A} = C_0(\Omega(\algebra{A}))$.
But if $\algebra{A}$ is not commutative, $\Omega(\algebra{A}) - \emptyset$ and this is useless.
So we want to look at non-homomorphism functionals (the elements of the spectrum are homomorphism functionals) and hope that this is not empty. Hence we want to study positive linear functionals.

\begin{definition}
	Let $\algebra{A}, \algebra{B}$ $C^*$-algebras. A linear map $\phi: \algebra{A} \to \algebra{B}$ is called \textbf{positive} if $\phi(\algebra{A}_+) = \algebra{B}_+$.
	We write $\phi \geq 0$ for this.
\end{definition}


\begin{remark}
	Let $\algebra{A}, \algebra{B}$ $C^*$-algebras and $\phi \geq 0$.
	\begin{enumerate}
		\item $\phi \geq 0$ implies that $\phi(\algebra{A}_{sa}) = \algebra{B}_{sa}$ (self-adjoint to self-adjoint).
		This follows because for any $a \in \algebra{A}_{sa}$, we have $a = a^+ = a^-$ and $\phi(a) = \phi(a^+) - \phi(a^-) \in B_{sa}$.
		\item $a_1 \leq a_2$ in $\algebra{A}$ yields $\phi(a_1) \leq \phi(a_2)$. 
		This is because every $^*$-homomorphism is primitive because $\phi: \algebra{A} \to \algebra{B}$ a $^*$-homomorphism and $a \geq 0$ in $\algebra{A}$ imply $a = x^* x$ for some $x \in \algebra{A}$ andthus $\phi(a) = \phi(x)^* \phi(x) \geq 0$.
	\end{enumerate}
\end{remark}


\begin{example}
	Let $\phi: M_m(\mathds{C}) \to M_m(\mathds{C}), a \mapsto a^T$ is positive but not a homomorphism.

	For this, consider $(a^*)^T = (a^T)^*$ and therefore $(a^* a)^T = (a^T) (a^T)^* \geq 0$, but not $(a^* a)^T \neq (a^T)^* (a^T)$.
\end{example}
	
\begin{example}
	$\algebra{A} = C_0(X)$. If $B(X)$ are the Borell-subsetes of $X$ $\mu: B(X) \to [0, \infty]$ is a positive bounded measure, then
	\begin{equation*}
		\phi_\mu: C_0(X) \to \mathds{C}, f \mapsto \int_X f(x) d \mu(x)
	\end{equation*}
	is clearly positive, linear but (usually) not a homomorphism.
	If $\mu$ is a Dirac-measure this is a homomorphism and a character.
\end{example}

\section{Positive linear maps and functionals}



\begin{definition}
	Let $\algebra{A}, \algebra{B}$ be $C^*$-algebras, a linear map $\phi: \algebra{A} \to \algebra{B}$ is called positive if $\phi(\algebra{A}_+) \subseteq \algebra{B}_+$, that is $a \geq 0 \then \phi(a) \geq 0$.
	We write this as $\phi \geq 0$.
\end{definition}


\begin{remark}
	Observe that $\phi \geq 0$ implies $\phi(a^*) = \phi(a)^*$ for all $a \in \algebra{A}$ and $\phi(\algebra{A}_{sa}) \subseteq \algebra{B}_{sa}$.	

	Also, $\phi$ respects inequality.
\end{remark}

\begin{proof}
	Just write $a \in \algebra{A}_{sa}$ as $a = a_+ - a_-$ with $a_+, a_- \in \algebra{A}_+$.
\end{proof}

\begin{example}
	\begin{enumerate}
		\item Let $\algebra{A} = M_n(\mathds{C})$ the usual trace $\mathrm{tr}: M_n(\mathds{C}) \to \mathds{C}, A \mapsto \sum_{i=1}^n a_{ii}$ is a positive linear functional
		In general a \textbf{trace} in a $C^*$-algebra is any positive linear map $\phi: \algebra{A} \to \mathds{C}$ with $\phi(ab) = \phi(ba)$.
	\end{enumerate}
\end{example}

\begin{proposition}
	If $\phi: \algebra{A} \to \algebra{B}$ is a positive linear map, then $\phi$ is bounded (i.e. continuous).
\end{proposition}

\begin{proof}
	Let $M = \sup_{a \in \algebra{A}_+} \|\phi(a)\|$. 
	If we had $M = \infty$ there exists $(a_n) \in \algebra{A}_{+,1}$ where $\|\phi(a_n)\| \geq 2^n$ for all $n$. 
	Define $a \coloneq \sum_{n=1}^{\infty} \frac{a_n}{2^n} \in \algebra{A}_{+,1}$.
	Since $\phi \geq 0$ and $\sum_{n = 1}^{N} \frac{a_n}{2^n} \leq a$, we have $\sum_{n=1}^{N} \frac{\phi(a_n)}{2^n} \leq \phi(a)$.
	Notice that $\phi(a_n) \geq 2^n$ in $\tilde{\algebra{B}}$ because whenever $b \in \algebra{B}_+$ and $\|b\| \geq c \geq 0$ so $b \geq c \cdot 1$.
	So in conclusion $\phi(a) \geq \sum_{n = 1}^{N} \frac{\phi(a_n)}{2^n} \geq N \cdot 1$ (in $\tilde{\algebra{B}}$), implying $\|\phi(a)\| \geq N$ for all $N \in \mathds{N}$, a contradiction.

	Now given any $a \in \algebra{A}$ write it as $a = b + ic$ where $b,c \in \algebra{A}_{sa}$ where $b = \frac{a + a^*}{2}$ and $c = \frac{a - a^*}{2i}$.
	If $\|a\| \leq 1$ then $\|b\|, \|c\| \leq 1$ and $b = b_+ - b_-$, $c = c_+ - c_-$ so $b_+ = \frac{b + |b|}{2}$, $b_- = \frac{b = |b|}{2}$, $c_+ = \frac{c + |c|}{2i}$ and $c_- = \frac{c - |c|}{2i}$ where $|b| = \sqrt{b b^*}$ so $\|b_+\|^2, \|b_-\| \leq 1$. Then
	\begin{equation*}
		\|\phi(a) \| = \|\phi(b) + i \phi(c)\| = \|\phi(b_+) + \phi(b_-) + i \phi(c_+) + i \phi(c_-)\| \leq 4 M
	\end{equation*} 
\end{proof}

We concentrate from now on positive linear functionals $\phi: \algebra{A} \to \mathds{C}$. 
The main point is the following observation:

\begin{remark}
	If $\phi: \algebra{A} \to \mathds{C}$ is a positive linear functional, then $\langle a, b \rangle_{\phi} \coloneq \phi(a^* b)$ is a semi-inner product on the vector space (fulfilling all requirements of an inner product except for $\langle a, a \rangle_\phi = 0 \then a = 0$).
	So Cauchy-Schwarz-inequality holds: $|\langle a, b \rangle_\phi| \leq \|a\|_\phi \cdot \|b\|_\phi$ where $\|a\|_\phi \coloneq \langle a, a \rangle_\phi^\frac{1}{2} = \phi(a^* a)^\frac{1}{2}$ is the semi-norm implied by $\langle \cdot, \cdot \rangle_\phi$.
	Therefore, $|\phi(a^* b)|^2 \leq \phi(a^* a) \cdot \phi(b^* b)$ for all $a, b \in \algebra{A}$.
\end{remark}

\begin{proposition}
	Let $\algebra{A}$ be a $C^*$-algebra and $\phi \in \algebra{A}_+^* = \{ \phi: \algebra{A} \to \mathds{C} \mid  \text{ positive linear } \}$.
	Then $|\phi(a)|^2 \leq \|\phi\| \phi(a^* a)$ for all $a \in \algebra{A}$.
\end{proposition}

\begin{proof}
	Let $(e_\lambda) \subseteq \algebra{A}_{+, 1}$ be an approximate unit.
	Using CS, we get
	\begin{equation*}
		|\phi(e _\lambda a) |^2 \leq \phi(e_\lambda^2) \cdot \phi(a^* a) \leq \|\phi\| \phi(a^* a)
	\end{equation*}
	and taking the limit yields the statement.
\end{proof}

\begin{theorem}
	Let $\phi \in \algebra{A}^* = \{ \phi: \algebra{A} \to \mathds{C} \mid \text{ bounded linear } \}$.
	Then the following are equivalent
	\begin{enumerate}
		\item $\phi \geq 0$
		\item For each approximate unit $(e_\lambda) \subseteq \algebra{A}_{+, 1}$ we have $\|\phi\| = \lim_\lambda \phi(e_\lambda) = \sup_\lambda \phi(e_\lambda)$.
		\item For some approximate unit $(e_\lambda) \subseteq \algebra{A}_{+, 1}$ we have $\|\phi\| = \lim_\lambda \phi(e_\lambda) = \sup_\lambda \phi(e_\lambda)$.
	\end{enumerate}
\end{theorem}

\begin{proof}~
	\begin{enumerate}
		\item $\then$ (ii): B\ the previous proposition, $|\phi(a)|^2 \leq \|\phi\| \phi(a^* a)$.
		Applying this for $a = e_\lambda$, we get $|\phi(e_\lambda)|^2 \leq \|\phi\| \phi(e_\lambda)^2$
		Notice $e_\lambda^2 = e_\lambda^\frac{1}{2} e_\lambda e_\lambda^\frac{1}{2} \leq e_\lambda$.
		Since $\phi$ preserves inequality, we have $|\phi(e_\lambda)|^2 \leq \|\phi\| \phi(e_\lambda)$, so $\phi(e_\lambda) \leq \|\phi\|$ and therefore $\limsup_{\lambda} \phi(e_\lambda) \leq \sup_\lambda \phi(e_\lambda) \leq \|\phi\|$.
		We apply CS again: $|\phi(e_\lambda a)|^2 \leq \phi(e_\lambda)^2 \phi(a^* a) \leq \phi(e_\lambda) \phi(a^* a)$ and hence $|\phi(a)|^2 = \liminf_\lambda |\phi(e_\lambda a)|^ \leq \liminf_\lambda \phi(e_\lambda)  \|a\|^2 \|\phi\|$, as$\phi(a^* a) \leq \|a\|^2 \|\phi\|$.

		Now taking $\sup$ over $\|a\| \leq 1$ yields
		\begin{equation*}
			\|\phi\|^2 \leq \liminf_{\lambda} \phi(e_\lambda) \|\phi\| \then \|\phi\| \leq \liminf_\lambda \phi(e_\lambda)
		\end{equation*}
		\item $\then$ (iii): This is clear, as some linear morhpisms always exist.
		\item $\then$ (i): Let $a \in \algebra{A}_{sa}$ and $\|a\| \leq 1$. 
		Write $\phi(a) = \alpha + i \beta$ with $\alpha, \beta \in \mathds{R}$. 
		We prove that $\beta = 0$, that is $\phi(a) \in \mathds{R}$. 
		We may assume $\beta \leq 0$ (or just take $-a$ instead).
		Let $n \in \mathds{N}$. Then 
		\begin{equation*}
			\| a - i n e_\lambda \|^2 = \| (a + i n e_\lambda) (a - i n e_\lambda)\|= \|a^2 n^2 e_\lambda^2 - 2n(a e_\lambda - e_\lambda a) \leq 1 + n^2 + n \| a e_\lambda - e_\lambda a \| 
		\end{equation*}THen we h
		and we have
		\begin{equation*}
			\|\phi(a - i n e_\lambda)\|^2 \leq \|a - i n e_\lambda^2 \leq 1 + n^2 + n \underbrace{\|a e_\lambda - e_\lambda a\|}_{\to 0}
		\end{equation*}
		Taking $\lambda \to \infty$, we get $\phi(e_\lambda) \leq 1 + n^2$.
		Using $\phi(a) = \alpha + i \beta$ and we get 
		\begin{equation*}
			\|\alpha + i \beta - in\|^2 \leq 1 + n^2 \then \alpha^2 + \beta^2 - 2 n \beta + i n^2 \leq 1 + n^2 \then - 2 n\beta \leq 1- \alpha^2 - \beta^2
		\end{equation*}.
		Because $\beta \leq 0$, we have to take $\beta = 0$.

		Now to prove $\phi \geq 0$: Take $a \in \algebra{A}_+$ with $\|a\| \leq 1$. Then $e_\lambda - a \in \algebra{A}_{sa}$ and 
		\begin{equation*}
			-1 \leq -a \leq e_\lambda \leq e_\lambda \leq 
		\end{equation*}
		So $\|e_\lambda\| \leq 1$.
		\begin{equation*}
			\underbrace{\phi(e_\lambda - a)}_{\in \mathds{R}} \leq |\phi(e_\lambda)| \leq 1
		\end{equation*}
		Letting $\lambda \to \infty$, then $1 - \phi(a)
		 \leq 1$ so $\phi(a) \geq 0$.k
	\end{enumerate}
\end{proof}


\begin{corollary}
	If $\algebra{A}$ is unital and $\phi \in \algebra{A}^+$ then $\phi \geq 0 \iff \phi(1) = \| \phi \|$.
\end{corollary}

\begin{corollary}

	If $\algebra{A}$ is a unital $C^*$-algebra and $\phi \in \algebra{A}^*$, then $\phi \geq 0 \iff \phi(1) = \|\phi\|$.
\end{corollary}

\begin{definition}
	A \textbf{state} on a $C^*$-algebra $\algebra{A}$ is a positive linear functional $\phi \in \algebra{A}_+^*$ with $\|\phi\| = 1$.

	We denote the set of all states by $S(\algebra{A})$.
\end{definition}

\begin{example}
	If $\algebra{A} = B(H)$ or $\algebra{A} = K(H)$ (bounded/compact operators on a hilbert space $\algebra{A}$) or $\algebra{A}$ a subalgebra of any of these sets with non-degenerate $e_\lambda \to 1$.
	Let $\zeta, \eta \in H$ and define $\phi_{\zeta, \eta}(a) \coloneq \langle \zeta, a \eta \rangle$.
	Then $\phi_{\zeta, \eta} \in \algebra{A}^*$ with $\|\phi_{\zeta, \eta}\| \leq \|\zeta\| \cdot \|\eta\|$.
	If $(e_\lambda) \subseteq \algebra{A}_{+, 1}$ is an approximate unit, then, using $e_\lambda \to 1$ (strictly) shows $\phi_{\zeta,\eta} (e_\lambda) \to \langle \zeta, \eta \rangle$.
	If $\zeta = \eta$, then $\phi_\zeta \coloneq \phi_{\zeta, \zeta}$ is positive and so $\phi_\zeta(a^* a) = \langle a \zeta, a \zeta \rangle = \|a \zeta \|^2 \geq 0$.
	By the previous theorem, $\|\phi_\zeta\|  = \lim_\lambda \phi_\zeta(e_\lambda) = \|\zeta\|^2$.
	So $\phi_\zeta$ is a state if and only if $\|\zeta\| = 1$.

	Note that there are states that are not of this form at all!
	The ones presented here are the so-called \textbf{pure states}.
\end{example}

\begin{theorem}
	If $\algebra{A}$ is a $C^*$-algebra and $a \in \algebra{A}$ is normal with $\algebra{A} \neq 0$ there exists a state $\phi \in S(\algebra{A})$ with $|\phi(a)| = \|a\|$ 
\end{theorem}

\begin{proof}
	We may assume $a \neq 0$ (we would only need to prove that any state exists, but this follows from the construction).
	Let $\algebra{B} = C^*(a, 1) \subseteq \tilde{\algebra{A}}$.
	$\algebra{B}$ is abelian, $\hat{a} \in C(X)$ and $X = \Omega(\algebra{B})$ (compact).
	Then there exists a $\chi \in \Omega(\algebra{B}) = X$ (compact) such that $| \hat{a}(\chi)| = |\chi(a)| = \|\hat{a}\|_\infty = \|a\|$.
	By Hahn-Banach, extend $\chi: \algebra{B} \to \mathds{C}$ to $\psi \in (\tilde{\algebra{A}})^*$ with $\|\psi\| = \|\phi\| = 1$.
	So $|\psi(a)| = |\chi(a)| = \|a\|$ and also $|\psi(1)| = |\chi(1)| = 1$.
	By the corollary, $\psi \geq 0$ and $\psi \in S(\algebra{A})$.
	Taking $\phi \coloneq \psi|_\algebra{A} \in \algebra{A}_+^*$ shows $\|\phi\| \leq \|\psi\| = 1$ and $|\phi(a)| = |\psi(a)| = \|a\|$, so $\|\phi\| \geq 1$, so $\|\phi\| = 1$ and $\phi$ is also a state.
\end{proof}

\begin{theorem}[Extension of positive linear functionals]
	Let $\algebra{A} \subseteq \algebra{B}$ be an inclusion of $C^*$-algebras and $\phi \in \algebra{A}_+^*$. Then, there exists $\tilde{\phi} \in \algebra{B}_+^*$ with $\tilde{\phi}|_\algebra{A} = \phi$ and $\|\tilde{\phi}\| = \|\phi\|$.
\end{theorem} 

\begin{proof}
	First consider the case $\algebra{B} = \tilde{\algebra{A}}$.
	In this case, define $\tilde{\phi}: \tilde{\algebra{A}} \to \mathds{C}, a + \lambda \cdot 1 \mapsto \phi(a) + \lambda \|\phi\|$.
	Of course, $\tilde{\phi}$ is linear and $\tilde{\phi}|_\algebra{A} = \phi$. 
	To prove that $\tilde{\phi}$ is bounded, let $(e_i) \subseteq \algebra{A}$ be an approximate unit. Then
	\begin{align*}
		|\tilde{\phi}(a + \lambda \cdot 1) &= |\phi(a) + \lambda \|\phi\| | = |\lim_i \phi(a e_i) + \lambda \lim_i \phi(e_i)| = \lim_i |\phi(a e_i + \lambda e_i)|\\ 
		&= \lim_i |\phi((a + \lambda 1) e_i)| \leq \|\phi\| \| a + \lambda 1 \| \|e_i\| \leq \|\phi\| \|a + \lambda 1\|
	\end{align*}
	because $\phi$ is bounded.
	So $\tilde{\phi}$ is also bounded and $\|\tilde{\phi}\| \leq \|\phi\|$.
	But $\tilde{\phi}(1) = \|\phi\|$, so $\|\tilde{\phi}\| = \|\phi\|$ and $\tilde{\phi}$ is therefore also positive.

	Now the general case: Passing to the unitizations, we have an embedding $\tilde{\algebra{A}} \subseteq \tilde{\algebra{B}}$ and may assume that both $\algebra{A}, \algebra{B}$ are unital with the same unit.
	By the unital case above, $\phi$ extends to $\tilde{\algebra{A}}$ and then also to $\algebra{A}$ by Hahn-Banach. 
	So there exists $\tilde{\phi} \in \algebra{B}^*$ with $\tilde{\phi}|$.
	Since $\phi \geq 0$, we know that $\tilde{\phi(1)} = \phi(1) = \|\phi\| = \|\tilde{\phi}|$, so $\tilde{\phi} \geq 0$.
\end{proof}

\begin{remark}~
	\begin{enumerate}
		\item In certain cases the extension $\phi$ to $\tilde{\phi}$ is unique.
		This is true if $ \algebra{A} \trianglelefteq \algebra{B}$ ore more generally if $\algebra{A} \subseteq \algebra{B}$ is a hereditary $C^*$-subalgebra (see Murphy: $\algebra{ABA} = \algebra{B}$ or $\algebra{A} = L \cap L^*$ for some left-handed ideal $L$).
		In this case, $\tilde{\phi}(b) = \lim \phi(u_\lambda a u_\lambda)$ where $(u_\lambda) \subseteq \algebra{A}$ where $(u_\lambda)$ is an approximate unit.
		\item Say $\phi \in \algebra{A}^*$ is self-adjoint. If $\phi^* = \phi$ where $\phi^*(a) = \overline{\phi(a^*)}$ (involution on $\algebra{A}^*$).
		We can write $\phi \in \algebra{A}^*$ as $\phi = \Re(\phi) + i \Im(\phi)$ where $\Re(\phi) = \frac{\phi + \phi^*}{2}$ and $\Im(\phi) = \frac{\phi - \phi^*}{2i}$ are self-adjoint, contained in $\algebra{A}^*_{sa}$.
		Observe that $\algebra{A}_{sa}^* = (\algebra{A}_{sa})'$, the topological dual of $\algebra{A}_{sa}$ as an $\mathds{R}$-vecotr Banach space.
		\item Any $\phi \in \algebra{A}_{sa}^*$ can be uniquely written as $\phi = \phi_+ - \phi_-$ where $\phi_+, \phi_- \in \algebra{A}_+^*$ and $\|\phi\| = \|\phi_+\| + \|\phi_-\|$.
	\end{enumerate}
\end{remark}

\section{The Gelfand-Naimark-Theorem}

\begin{definition}
	Let $\algebra{A}$ be a $C^*$-algebra.
	A \textbf{representation} of $\algebra{A}$ is a $^*$-homomorphism $\pi: \algebra{A} \to \mathcal{L}(H)$ for some Hilbert space $H$.

	
	We say that $\pi$ is
	\begin{enumerate}
		\item \textbf{faithful} if $\pi$ is injective (and therefore isometric).
		\item \textbf{non-degenerate} if $\Span \pi(\algebra{A}) H = H$.
		\item \textbf{irreducible} if for all closed subspaces $K \subseteq H$ with $\pi(\algebra{A})K \subseteq K $ ($K$ is $\pi$-invariant) we have $K = 0$ or $K = H$.
	\end{enumerate}
\end{definition}

\begin{remark}
	The exercises show that $\pi$ non-degenerate is equivalent to $\pi(e_\lambda) \to 1$ (strongly) for an approximate unit $(e_\lambda) \subseteq \algebra{A}$
\end{remark}

We want to show that there is always a faithful homomorphism.

\begin{definition}
	Let $\pi: \algebra{A} \to \mathcal{L}(H)$, $\rho: \algebra{A} \to \mathcal{L}(K)$ two representations. We say that $\pi, \rho$ are (unitarily) equivalent if there exists a surjective isometry $u: H \to K$ such that $u^* \rho(a) u = \pi(a)$, i.e. $\rho = \mathrm{Ad}_{u^*} \pi$.
\end{definition}

\begin{definition}[Spectrum]
	We define
	\begin{equation*}
		\hat{\algebra{A}} = \{[\pi] \mid \pi: \algebra{A} \to \mathcal{L}(H), \pi \neq 0\}
	\end{equation*}
	Also define $\mathrm{Prim}(\algebra{A}) = \{ \ker(\pi) \mid [\pi]\in \hat{\algebra{A}}\}$ and $\hat{\algebra{A}} \to \mathrm{Prim}(\algebra{A}), [\pi] \mapsto \ker \pi$ (primitive ideals). 
	Let $\chi \in \Omega(\algebra{A})$ be a character $\chi: \algebra{A} \to \mathds{C} = \mathcal{L}(\mathds{C})$.
	Then $[\chi] \in \hat{\algebra{A}}$ and $\ker \chi \in \mathrm{Prim}(\algebra{A})$.

	%Observe that if $\algebra{A} = C_0(X)$, then ???
\end{definition}

How do we get representations of $\algebra{A}$?

\subsection*{Gelfand-Naimark-Siegal-Construction (GNS)}

\begin{theorem}
Let $\phi \in \algebra{A}_+^*$ be any positive linear functional. 
We know that $\langle a, b \rangle_\phi \coloneq \phi(a^* b)$ defines a semi-inner-product and $\|a\|_\phi = \phi(a^* a)^\frac{1}{2}$ is a semi-norm.

Let $N_\phi \coloneq \{a \in \algebra{A} \mid \|a\|_\phi = 0 \}$. 
\end{theorem}

\begin{remark}
Notice: $N_\phi \subseteq \algebra{A}$ is a closed left ideal.
\end{remark}

\begin{proof}
	From Cauchy-Schwarz: 
	\begin{align*}
		|\phi(a^* b)|^* \leq \phi(a^* a) \phi(b^* b)
	\end{align*}
	and therefore
	\begin{equation*}
		N_\phi = \{ b \in \algebra{A} \mid \phi(ab) = 0 \}
	\end{equation*}
	Let $H_\phi^\circ \coloneq \algebra{A} / N_\phi$ the quotient vector space.
	Then $\langle, \dot, \dot \rangle_\phi$ factors through an inner product of $H_\phi^\circ$ and 
	\begin{align*}
		\langle a + N_\phi, b + N_\phi \rangle = \langle a, b \rangle  =\phi(a^* b)
	\end{align*}
	By completion we get a Hilbert space $H_\phi= \closure{H^\circ_\phi}^{\langle \cdot, \cdot, \rangle}$.

	Now we define (with $L$ the linear operators)
	\begin{equation*}
		\pi_\phi^\circ: \algebra{A} \to L(H_\phi^\circ)
	\end{equation*}
	and thus
	\begin{equation*}
		\pi_\phi^\circ(a) (b + N_\phi)\coloneq ab + N_\phi
	\end{equation*}
	meaning that $\pi_\phi^\circ (a) \cdot \pi_\phi^\circ(b) = \pi_\phi^\circ (ab)$ and $\pi_\phi^\circ(a^*) = (\pi_\phi^\circ(a))^*$. Then
	\begin{align}
		\phi(b^* a c) = \langle \phi_\phi^\circ (a^*) (b + N_\phi), c + N_\phi \rangle = \langle b + N_\phi, \pi_\phi(a)(c + N_\phi) \rangle\text{.}
	\end{align}
	We claim now that $\pi_\phi^\circ$ is bounded for $\|\cdot\|_\phi$ and therefore show that $\pi_\phi(a)$ extends to  $\pi_\phi(a) \in \mathcal{L}(H_\phi)$. 

	Take 
	\begin{align*}
		\|\pi_\phi^\circ(a)(b + N_\phi)\|_\phi^2 = \|ab + N_\phi\|_\phi^2 = \phi((ab^* ab)) = \phi(b^* a^* a b) \leq \|a\|^2 \phi(b^* b) \leq \|a\|^2 \|b + N_\phi|
	\end{align*}
	Therefore we get a representation: The GNS-Representation associated to $\phi$.
	\begin{equation*}
		\pi_\phi: \algebra{A} \to \mathcal{L}(H_\phi), a \mapsto \pi_\phi(a) = [b + N_\phi \mapsto ab + N_\phi] 
	\end{equation*}
 \end{proof}

 If $(\pi_i)_{i \in I}$ is a family of representations $\pi_i: \algebra{A} \to H$.
 We define the direct sum $\bigoplus_{i \in I} \pi_i: A \to \mathcal{L}(\bigoplus_{i \in I} H_i), a \mapsto (\pi_i(a))_{i \in I}$ where $(\pi_i(a))_{i \in I}: \zeta \mapsto (\pi_i(a)\zeta)$.


\begin{theorem}[Gelfand-Naimar-Representation]
	Let $\algebra{A}$ be a $C^*$-algebra and define $\pi_U \coloneq \bigoplus_{\phi \in S(\algebra{A})} \pi_\phi: \algebra{A} \to \mathcal{L}(H_U)$ with $H_U = \oplus_{\phi \in S(\algebra{A})} H_\phi$ for $H_\phi = \closure{\algebra{A}}/N_\phi$ with the semi-inner product $\langle \cdot, \cdot \rangle_\phi$ and $\pi_\phi(a)(b + N_\phi) = ab + N_\phi$.
	Then $(\pi_U, H_U)$ is \textbf{faithful}.
\end{theorem}

\begin{proof}
	Suppose $0 \neq a \in \algebra{A}$, $pi_U(a) = 0$ and $pi_U(a) = 0$.
	Then there exists $\phi \in S(\algebra{A})$ such that $\phi(a^* a) = \|a^* 
	a\| = \|a\|^2$.
	We know $\langle a, a \rangle_\phi = \|a\|_\phi$.
	Then $\pi_U(a) = 0$, so $\pi_\phi(a) = 0$, so $\pi_\phi(a^*a) = 0$ and therefore $\pi_\phi(a)(b + N_\phi) = ab + N_\phi = 0$.
	This shows 
	\begin{equation*}
	i	0 = \langle \pi_\phi(a) (b + N_\phi), \pi_\phi(a)(b + N_\phi) \rangle = \phi(b^* a^* a b)
	\end{equation*}
	for all $b \in \algebra{A}$, so $b = e_\lambda$ (for $\lambda \to \infty$).
	But then $\phi(a^* a) = 0$ and thus $a = 0$.
\end{proof}

Observe that $(\pi_U, H_U)$ is called the universal representation of $\algebra{A}$.
This is always non-degenerate.
Indeed, each $(\pi_\phi, H_\phi)$  is non-degenerate.
Moreover, these are \textbf{cyclic representations}:

\begin{definition}
	A representation $\rho: \algebra{A} \to L(H)$ is \textbf{cyclic} if there is a $\zeta \in H$, $\|\zeta\| = 1$ such that $\closure{\rho(\algebra{A}) \zeta} = H$.
	$\zeta$ is called a cyclic vector for $(\rho, H)$.
\end{definition}

Observe: Every non-degenerate representation is a sum of cyclic representations (proof via Zorn's Lemma omitted).

\begin{proposition}
	Every GNS-representation $(\pi_U, H_U)$ is cyclic.
\end{proposition}

\begin{proof}
	If $\algebra{A}$ is unital, then $\zeta \phi \coloneq 1 + N_\phi \in H_\phi$ is a cyclic vector for $\pi_\phi$. 
	Then $\pi_\phi(a)(\zeta \phi) = a + N_\phi$ and thus $\pi_\phi(\algebra{A}) \zeta_\phi = \algebra{A}/N_\phi \subseteq H_\phi$ (dense).
	Therefore $\zeta_\phi$ is cyclic and
	\begin{equation*}
		\|\zeta \phi\|^2 = \langle 1 + N_\phi, 1 + N_\phi\rangle = \phi(1) = =\|\phi\| = 1
	\end{equation*}
	so $\phi \in S(\algebra{A})$. 
	Moreover: $\langle \zeta_\phi, \pi_\phi(a) \zeta_\phi\rangle = \langle 1 + N_\phi, 1 + N_\phi\rangle = \phi(a)$.

	Let us now look at the general case.
	Consider the linear map $\phi_0: \algebra{A}/N_\phi \to \mathds{C}, a + N_\phi \to \phi(a)$.
	This is well-defined and bounded:
	\begin{equation*}
		\|\phi(a)\|^2 \leq \|\phi\| \phi(a^* a) = \phi(a^*a)
	\end{equation*}
	as $\phi$ is a state (and thus $\|\phi\| = 1$).
	So $\|\phi\| \leq 1$.
	So $\phi_0$ extends to a bounded linear factorial map on $\tilde{\phi}_0 H_\phi: \to \mathds{C}$.
	By Riesz-Representation thoerem, we have a $\zeta_\phi \in_\phi$ such that $\tilde{\phi}_0 (\eta) = \langle \zeta_\phi, \eta \rangle$ and $\|\zeta_\phi\| = \|\phi_0\| = 1$.
	In particular $\phi(a) = \phi_0(a + N_\phi) = \langle \zeta_\phi, a + N_\phi \rangle$.
	Now for $a,b \in \algebra{A}$ we have
	\begin{equation*}
		\langle \pi_\phi , b + N_\phi \rangle =  \langle \zeta_\phi, \pi_\phi(a^*) (b + N \phi) \rangle = \langle \zeta_\phi, a^* b \rangle = \phi(a^*b) = \langle a + N_\phi, b + N_\phi\rangle
	\end{equation*}
	Therefor $\pi_\phi(a) \zeta_\phi = a + N_\phi$ (*) as well as $\closure{\pi_\pi(\algebra{A}) \zeta_\phi} = H_\phi$ and
	$\phi(a) = \langle \zeta_U, \pi_\phi(a) \zeta_\phi \rangle$.

	If $(e_\lambda) \subseteq \algebra{A}$ is an approximate unit so $\pi_\phi(a^\lambda) \to 1$ strong as $a \to \infty$.
	Then $\|\phi\| \leftarrow \phi(e_\lambda) = \langle \zeta_\phi, \pi_\phi(e_\lambda) \zeta_\phi \rangle \to \|\zeta_\phi\|^2$, so l$\|\zeta_\phi\| = 1$ and it is a cyclic representation. 

	Also, from (*) we know $\zeta_\phi = \lim_\lambda \pi_\phi(e_\lambda) \zeta_\phi = \lim_\lambda e_\lambda + N_\phi$.

	So the GNS-construction gives a triple $(\pi_\phi, H_\phi, \zeta_\phi)$ satisfying our conditions.
\end{proof}

Conversely, if $(\pi, H, \zeta)$ is a cyclic representation of $\algebra{A}$, then $\phi(a) \coloneq \langle \zeta, \pi(a) \zeta \rangle$ defines a style $\phi \in S(\algebra{A})$.

\begin{remark}[irreducible representations and pure states]
	Notice: $\Omega(\algebra{A}) \subseteq S(\algebra{A}) \subseteq \algebra{A}_1^*$.
	In particular, we can endow this with the weak $^*$-topology.

	This is closed and therefore compact: Take $\phi_i \in S(\algebra{A})$ with $\pi_i \to \phi \in \algebra{A}^*_1$ ($\|\phi\| leq 1$). Then 
	\begin{equation*}
		1 \xleftarrow{\lambda}\phi_i(e_\lambda) \xrightarrow{i} \phi(e_\lambda)
	\end{equation*}
	with $\|\phi\| = 1 = \lim_\lambda \phi(e_\lambda)$.

	Moreover, $S(\algebra{A})$ is convex, so for $\phi_1, \dots, \phi_n \in S(\algebra{A})$ and $t_1, \dots, t_n \in \mathds{R}_+$ with $\sum_{i = 1}^{n} t_i = 1$ we have $\sum_{i = 1}^{n} t_i \phi_i \in S(\algebra{A})$.

	Recall the Kreim-Milman-Theorem: If $K$ is a compact convex subset of $\algebra{A}_1^*$, then $K = \closure{\mathrm{conv}(\mathrm{Ext}(K))}$ where $\mathrm{conv}$ is the convex hull and $\mathrm{Ext}$ are the extremal points, that is all points in $K$ that cannot be reached as linear combinations of other points (e.g. the corners of a closed triangle). 
	In particular, any compact convex set must have extremal points (unless it is empty).

	We will apply this to the states $K = S(\algebra{A})$.
\end{remark}

\begin{definition}
	Call $PS(\algebra{A}) \coloneq \mathrm{Ext}(S(\algebra{A}))$ the \textbf{pure states} of $\algebra{A}$.
\end{definition}

\begin{theorem}
	A state $\phi \in S(\algebra{A})$ is pure if and only if $\pi_\phi \algebra{A} \to \mathcal{B}(H_\phi)$ is irreducible if and only iff $\pi_\phi(\algebra{A})' \coloneq \{ T \in \mathcal{B}(H_\phi) \mid T \pi_\phi(a) = \pi_\phi(a) T \} = \mathds{C} \cdot 1$ by Schur's lemma.
\end{theorem}

\begin{proof}
	See Murphy.
\end{proof}


\begin{example}
	Let $\algebra{A} = C_0(X)$. Take $\phi \in C_0(X)^* \simeq $ Complex bounded Radon measure of $X$.
		If $\mu: \mathrm{Borells} (X) \to \mathds{C}, E \to \mu(E)$ has $\mu = \Re{\mu} + i \Im{\mu}$.
		$\Re{\mu} = \Re(\mu)_+ - \Re(\mu)_-$ is a complex (Radon) measure, then the associated $\phi = \phi_\mu \in C_0(X)^*$ is $\phi_\mu(f) = \int_X f(x) d\mu(x)$.
		
		Moreover, $\phi_\mu \geq 0 \iff \mu \geq 0$, so $C_0(X)^*_+$ consists of the positive Radon measures on $X$.

		Note: Characters correspond to Dirac measures: $\mu_{x_0}(E) = 1$ if $x_0 \in E$ and $0$ otherwise.	The real measures correspond to the self-adjoint elements and the states correspond to those measures with $\mu(X) = 1$, that is the probability (positive Radon) measures on $X$.
\end{example}

\begin{remark}
	Look at the GNS construction for $\phi = \phi_\mu$. Define
	\begin{equation*}
		\langle f, g\rangle_\phi = \phi(f^* \cdot g) = \phi(\overline{f} \cdot g) = \int_X \overline{f(x)} g(x) d\mu(x)
	\end{equation*}
	Then 
	\begin{equation*}
		N_\phi = \left\{ f \in C_0(X) \mid \phi(\overline{f} f) = \int_X |f(x)|^2 d\mu(x) = 0 \right\} \trianglelefteq C_0(X)
	\end{equation*}
	Indeed, $N_\phi$ corresponds to the support of $\mu$: $\mathrm{supp}(\mu) = \{x \in X \mid \forall_{U \subseteq X \text{ open}} x \in U \then \mu(U) > 0 \}$ (this is always closed). Now we want to show for $U = \mathrm{supp}(\mu)^\complement$:
	\begin{equation*}
		N_\phi = C_0(U) = \{ f \in C_0(X) \mid f|_{U^\complement} \equiv 0 \}
	\end{equation*}
	''$\supseteq$'': If $f \in C_0(U)$, $f|_{\mathrm{supp}(\mu)} \equiv 0$ then $\int_X |f(x)|^2 d \mu(x) = 0$. TODO

	
	Then $H_\phi = L^2(X, \mu) = \closure{C_0(X)}$ (with closure in respect to $\langle \cdot, \cdot \rangle_{2, \mu}$) and $\pi_\phi(f)(\zeta + N_\phi) = f \cdot \zeta + N_\phi$ (where the added class $N_\phi$ represents that the functions are equal $\mu$-almost everywhere).
	These correspond to $M_f(\zeta) = f \cdot \zeta$.
\end{remark}

\section{Inverse Semigroups}

Our main results so far:
\begin{itemize}
	\item Every commutative $C^*$-algebra $\algebra{A}$ is $\algebra{A} \simeq C_0(X)$ where $X$ is a locally compact Hausdorff space.
	\item Every $C^*$-algebra can be embedded into $\mathcal{B}(H)$ for some Hilbert space $H$.
\end{itemize}
How to now model $C^*$-algebras in general?
We look for general constructions of $C^*$-algebras and hope that many $C^*$-algebras in practice are 'part' of this construction.
We are going to look at the class of $C^*$-algebras associated to (inverse) semigroups and groupoids.
These include, in particular, groups.
One of the motivating examples:
\begin{example}
	Recall that the Cuntz-$C^*$-algebra is the (universal, unital) $C^*$-algebra  $\mathcal{O}_n$  generated by $n$ isometries $S_1, \dots, S_n \in \mathcal{O}_n$ satisfying the relations $S_i^* S_j = \delta_{ij} \cdot 1$ and $S_1 S_1^* + \dots + S_n S_n^* = 1$.
	Then we can look at the set 
	\begin{equation*}
		S \coloneq \{ S_\alpha S_\beta^* \mid \alpha, \beta \text{ finite words in } \{1, 2, \dots, n\} \} \cup \{0\} \subseteq \mathcal{O}_n
	\end{equation*}
	where for $\alpha = \alpha_1 \alpha_2 \cdots \alpha_k$ with $\alpha_i \in \{1, \dots, n\}$ we have $S_\alpha = S_{\alpha_1} \cdots S_{\alpha_k}$ and we convention that for the empty word $\epsilon$ we have $S_\epsilon = 1$.

	Note that $\mathcal{O}_n = C^*(S)$ and each non-zero element $s \in S$ is an isometry, and every element is a partial isometry, that is $s s^* s = s$.
	Also, $S$ is closed under multiplication (of $\mathcal{O}_n$).
	So, this means that $S$ is a sub semigroup of the multiplicative semigroup of $\mathcal{O}_n$.
	So $S$ is a $^*$-semigroup of partial isometries.
	We now consdier only $C^*$-algebras that thusly arise.
\end{example}

\begin{definition}
	An \textbf{inverse semigroup} is a semigroup $S$ (it is endowed with an associative multiplication $S \times S \to S$) which is also a $^*$-semigroup (it is endowed with an involution $^*: S \to S$ sometimes also called a 'pseudo-inverse') satisfying:
	\begin{itemize}
		\item $(s^*)^*=s$ (involution)
		\item $(st)^* = t^* s^*$ (antimultiplicative)
		\item $s s^* s = s$
		\item The elements of $E(S) = \{s^* s \mid s \in S \}$ should commute.
		These elements are called \textbf{idempotents} and $E(S) = \{e \in S \mid e = e^2 \}$.
	\end{itemize}
\end{definition}

\begin{remark}~
	\begin{itemize}
		\item 
		Where the first two properties makes it a $^*$-semigroup and the last makes it a $^*$-semigroup of partial isometries.
		\item $E(S)$ is a commutative (inverse) subsemigroup of $S$ and $e^*=e$ for all $e \in E(S)$.
		\item Given $s \in S$, $t = s^* \in S$ is the unique element of $S$ satisfying $sts = s$ and $tst = t$.
	\end{itemize}
\end{remark}

\begin{example}~
	\begin{enumerate}
		\item Groups are always inverse semigroups with exactly $1$ idempotent, that is $E(S) = \{e\}$.
		Furthermore, we have $E(S) = \{s^* s \mid s \in S\} = \{s s^* \mid s \in S \}$.

		If $s^*s e ss^* $ for all $s$ then $s s^* s = s = ss^*s$.
		\item \textbf{Commutative Inverse}:
		Semigroups that are exactly the () semilattices, that is partially order set $(E, \leq)$ for which every $e,f \in E$s has $e \wedge f = \in \{e, f\} = e \cdot f = f \cdot e$.

		\item If $S$ is an inverse semigroup which is commutative, then $S = E(S) = $ is the set of idempotents and this is a semilattice with $e \leq f \iff e \cdot f = e$.
		\item Let $X$ be any set (with the natural order) and $\mathcal{X} $ the powerset of $X$, then $A \cdot B = A \cap B$ is a commutative ISG.
		\item Consider $(\mathds{N}_0, +)$.
		This can be viewed multiplicatively or additively. 
		We will look at the addition.
		Certainly, this is a semigroup.
		Can it be an inverse semigroup?
		No, because we would need an element $n \in \mathds{N}$ with $n + m + n =0$ but that would imply $n = 0$.
		For $(\mathds{N}, \cdot)$, we have the same problem.

		Lets look at $\closure{(\mathds{N}, \min)}$ with multiplication $n \cdot m = \min (n,m)$ in compliance with the lattice.
		This is commutative and therefore an Inverse Semigroup.
		\item Take now $M_n(\mathds{C})$ with basis $(e_{ij})^n$, then $e_{ij} e_{kl} = \delta_{j,k} e_{i,l}$ and $e_{ij}^* = e_{ji}$.
		Then $e_{ij}: \mathds{C} \to \mathds{C}, e_j \to e_i$ and partial isometries $e_{ij} e_{ij}^* e_{ij} = e_{ij}$.
		So $S = \{ e_{ij} \mid i,j = 1, \dots, n \} \cup \{0\} \subseteq M_n(\mathds{C})$ is a $^*$-semigroup of partial isometries, so it is an inverse semigroup.
		The $C^*$-algebra of this inverse semigroup is $M_n(\mathds{C})$.

		\item Let $X$ be any set. Then
		\begin{equation*}
			I(X) = \{ f \mid f \text{ is partial bijection between subsets of } X \}
		\end{equation*}
		i.e. $f: U \to V$ is a bijection where $U,V \subseteq X$ (note that these may be any set, even the empty set and need not be open, as $X$ does not even have a topology).
		We must still find a suitable product.

		Take $f: U \to V$, $g: U' \to V'$. Take $\tilde{U} = f^{-1}(U' \cap V)$ and define $f \cdot g: \tilde{U} \to g(V \cap U'), x \mapsto f(g(x))$.

		So $I(X)$ is an inverse semigroup with $f^* = f^{-1}$ and $f(f^{-1}f) = \id_{D(f)} \cdot f = f$.

		Additionally, we get $E(I(X)) = \{ \id_U \mid U \subseteq X \} = 2^X$.
 	\end{enumerate}
\end{example}

\begin{example}[About $I(X)$]
	Take $X = \{1,2\}$. Then 
	\begin{equation*}
		I(X) = \{ 0 = \emptyset, \id_{\{1\}}, \id_{\{2\}}, \id_X = 1, \{1\} \to \{2\}, \{2\} \to \{1\}, (\{i1\} \to \{2\}, \{2\} \to \{1\}) = (12)  \}
	\end{equation*}
	whereas $\mathrm{Bij}(X) = S_2 = \{\id_X, (12)\}$.

	One can also consider $I(\algebra{A}) $
	\begin{itemize}
		\item $\supseteq \mathrm{Aut}(\algebra{A}) = \{f: \algebra{A} \to \algebra{A} : f ~^*\text{-automorphism} \}$
		\item $\supseteq \mathrm{pAut}(\algebra{A}) = \{ f: I \to J \mid I,J \trianglelefteq \algebra{A}, f ~^*\text{-automorphism} \}$
	\end{itemize}
\end{example}

\begin{theorem}[Vagner-Preston-Theorem]
	Every inverse semigroup $S$ can be embedded (as an inverse sub semigroup) into $I(X)$ for some $X$.

	This is somewhat of a generalization of Caley's theorem.
\end{theorem}

\begin{proof}[Idea]
	Take $s \in S$ and $\closure{X} = S$ defines a partial bijection.
	$f_s(x) = sx$. 
	Take $D_s = \{ x \in X \mid s^*s x = x \} \subseteq X$, so $f_s: D_s \to R_s, f^{-1}_s = f_{s^*}$ where $R_s = \{x \mid s s^* x = x\} = D_{s^*}$ is the partial inverse..
\end{proof}

\begin{definition}
	Let $S$ be an inverse semigroup (that is, $S$ is a semigroup and for all $s \in S$ we have $s^* \in S$ and $ss^*s = s$).
	Then $C^*(S)$ is the \underline{universal} $C^*$-algebra generated by (a 'copy' of) $S$ as a $^*$-semigroup.

	More precisely: $C^*(S)$ is a $C^*$-algebra endowed with a $^*$-homomorphism $\iota: S \to C^*(S)$ such that for every other $C^*$-algebra $\algebra{B}$ with a $^*$-homomorphism $\pi: S \to \algebra{B}$ there exists a unique $^*$-homomorphism $\tilde{\pi}: C^*(S) \to \algebra{B}$ such that $\tilde{\pi} \circ \iota = \pi$.
\end{definition}

\begin{remark}~
	\begin{enumerate}
		\item We are going to prove that the $C^*$-algebra $C^*(S)$ exists.
		\item An inverse semigroup $S$ might have a unit $1 \in S$ (i.e. $1s = s = s1$).
		If this is the case, $C^*(S)$ and $\iota: S \to C^*(S)$ will be unital, and in the universal property we may assume $\algebra{B}$ and $\pi$ to be unital.		
		
		Also, you can always formally add such a unit (and only this unit) to any inverse semigroup.
		Therefore, we will most of the time only consider such unital semigroups.
		\item An inverse semigroup $S$ might have a zero $0$ (i.e. $0s = 0 = s0$).
		If this is the case, we would like that $0 \in S$ ''is'' also $0 \in C^*(S)$, that is the embedding $\iota$ is zero-preserving: $\iota(0) = 0$.
		This is not automatic, but we can change the definition and force this to be true.
		Formally, we define another $C^*$-algebra $C^*_0(S)$ in a similar way by asking $\iota, \algebra{B}, \pi$ to be zero-preserving.
		\item We will proof that $C^*_0(S)$ exists.
		It is actually $C^*_0(S) = C^*(S)/\langle \iota(0) \rangle$.
	\end{enumerate}
\end{remark}

\begin{example}
	Let $S = \{s\}$ be a single-element semigroup (with $s = s^* = s^2$).
	In this case $s = 0 = 1$.
	Then $C^*(S)$ is the universal $C^*$-algebra generated by a projection.
	We claim $C^*(S) = \mathds{C}$.
	Indeed, $p = 1 \in \mathds{C}$ is a projection with $\mathds{C} = C^*(1) =\mathscr{C} \cdot 1$.
	So this means we have $\iota: S \to \mathds{C}, s \mapsto 1$.
	If $\algebra{B}$ is any algebra with $^*$-homomorphism $\pi: S \to \algebra{B}$, this just means that $p = \pi(s) \in \algebra{B}$ is a projection. 
	Then $\tilde{\pi}: \mathds{C} \to \algebra{B}, \lambda \mapsto \lambda \cdot p$.

	If, however, we treat $s \in S$ as the zero, then $C^*_0* = 0$.
\end{example}

\begin{example}
	Set $S = \{p,q\}$ the inverse semigroup with two elments $qq = q$, and $pp = qq = qp = p$.
	$C^*(S)$ is the universal $C^*$-algebra generated. So there are two projections $P,Q$ mit $P= P$ ($P \leq Q$). Then the $C^*$-algebra is Commutative!i.
	Claim: $C^*(S) \simeq C\mathds{c}^2 \to \mathds{C}$.
	This is indeed that case as $P(1,0) = Q(1,1)$.
	So we have $C_0^*(S) \simeq \mathds{C}$.
\end{example}

\begin{example}
	Let $S = \{1, g\}$ and $g = g^* = g^{-1}$ (and $g^2 = 1$).
	This is a full group.
	Then $C^*(S)$ is the universal unital $C^*$-algebra generated by a self-adjoint unit, so $C^*(S) = C^*_{univ} (1, u)$ for osme self-adjoint with $u^2 =1$ and $u^*=u$.
	Then $C^*_0(S) = \mathds{C}^2$.

	Take $\iota: S \to \mathds{C} \oplus \mathds{C}$ where $1 \mapsto (1,1)$ and $g \mapsto u = (\alpha, \beta)$ with $\alpha, \beta \in \mathds{R}$ and $\alpha^2 = 1 = \beta^2$.
	Then $\mathds{C} \oplus \mathds{C}$
\end{example}

\begin{proof}[Existence of $C^*(S)$]
	First, consider the $^*$-algebra of $S$.
	Take
	\begin{equation*}
		\mathds{C}[S] = \left\{ \sum_{s \in S}^\mathrm{fin} a_s S_s \mid a_s \in \mathds{C}\right\}
	\end{equation*}
	then $\delta_s \cdot \delta_t = \delta(st)$ and $\delta_s^* = \delta_{s^*} $.

	The idea is now to complete this to a $C^*$-algebra. To get $C^*(S)$, take $C^*(S) = \overline{C[S]}^{\|\cdot\|}$.
	For $C^*$ to be 'universal', it must be the largest and its norm must be the largest $\|\cdot\|$ $C^*$-algebra norm.
	As a $^*$-homomorphism between $C^*$-algebras is automatically contractive, we can define for $a \in \mathds{C}[S]$.
	Then $\|a\|_{\max} = \sup \{ p(a) \mid p: \mathds{C}[S] \to [0, \infty), C^*\text{seminorm} \}$.
	This set is non-empty, but it could be unbounded.
	We prove that, in the current case of a semigroup construction, this is not the case, and the supremum thus itself defines a $C^*$-seminorm.
	Write $a = \sum_{s \in S}^\mathrm{fin} a_s \delta_s = \sum_{i = 1}^{m} a_{s_i}$.
	Take $p \in \mathds{C}[S] \to [0, \infty)$ a $C^*$-seminorm.
	Idea: 
	\begin{equation*}
		p(a) \leq \sum_{i=1}^n |a_{s_i}| p(\delta_{s_i})
	\end{equation*}
	if $s \in S$, $p(\delta_s)^2 = p(\delta_s^* \delta_1) =\dots$.
	
	\begin{enquote}
		Let $\algebra{A}$ be a $C^*$-algebra, $p: \algebra{A} \to [0, \infty )$ with a $C^*$-seminorm and $a \in \algebra{A}$ a partial isometry.
		Then $p(a) < 1$.

		Proof: Omitted.
	\end{enquote}
	Define $N_p = \{a \in \algebra{A} \mid p(a) = 0\} \trianglelefteq \algebra{A}$. 
	Then 
	\begin{equation*}
		\algebra{A}/N_p \xrightarrow{\|\cdot \|_p} [0, \infty), \|a + N_p\|_p \coloneq p(a)
	\end{equation*}
	is a $C^*$-norm and $C_p^*(\algebra{A}) = \overline{\algebra{A} / N_p}^{\|\cdot\|_p}$.
	Then $\pi: \algebra{A} \xrightarrow{q} \algebra{A}$ is a $^*$-homomorphism.
	Furthermore, $p(a)^2 = \|a + N_p\|^2 = \|\pi(a)\|^2 = \|\pi(a^*a)\| \leq 1$.
	Then $\pi(a)$ is a partial isometry of a $C^*$-algebra.

	So $\|\cdot\|_{\max}$ defined by the supremum is a $C^*$-seminorm.
	As in the lemma, define $C^*(S) = \overline{\mathds{C}[S] / N_{\|\cdot\|}}^{\|\cdot\|}$.
	Then $C^*(S)$ is a $C^*$-algebra and we have a $^*$-homomorphism $q: \mathds{C}[S] \to C^*(S)$ as a quotient map (in completion plus embedding).
	Concatenating this with $s \mapsto \delta_s$ yields the final $^*$-homomorphism.

	\underline{Universal property}: Take $\algebra{B}$ any $C^*$-algebra with $^*$-homomorphism $\pi: S \to \algebra{B}$.
	This induces a $^*$-homomorphism $\rho: \mathds{C}[S] \to \algebra{B}$ by $\rho(\sum a_s \delta_s) = \sum a_s \pi(s)$ and then $\|a\|_\rho \coloneq \|\rho(a)\|_\algebra{B}$ defines a $C^*$-seminorm.
	So $\rho(a) = \|a\|_\rho \leq \|a\|_{\max}$.
	This means that $\rho$ is contractive and continuous (for the max norm).
	In particular, $\rho$ vanishes on $N_{\|\cdot\|_max}$ and therefore induces a $^*$-homomorphism $\tilde{\pi}: C^*(S) \to \algebra{B}$.

	This also shows the existence of $C_0^*(S)$.	
\end{proof}


\end{document}































































