\documentclass[a4paper]{article}

% --- PREAMBLE ---

\usepackage[english]{babel}	% language specific quotation marks etc.
% --- LANGUAGE ---

% Needs to be set individually!
%\usepackage[english]{babel}	% language specific quotation marks etc.

% --- OTHER ---

\usepackage{booktabs}       % professional-quality tables
\usepackage[table]{xcolor}	% color
\usepackage{pdfpages}		% to include entire pdf pages in appendix etc.
\usepackage{enumitem}		% better custom enumerations
\setlist[enumerate, 1]{label=(\roman*)}
\usepackage{etoolbox}		% toolbox for command modification

% --- FONTS & TYPESETTING ---

\usepackage[utf8]{inputenc} % allow utf-8 input
\usepackage[T1]{fontenc}    % use 8-bit T1 fonts
\usepackage{dsfont}			% font with double lines for sets
\usepackage[german,ruled,vlined,linesnumbered,commentsnumbered,algoruled]
{algorithm2e} 				%pseudo code
\usepackage{listings}		%java code
\usepackage{csquotes}

% --- URLS ---

\usepackage[colorlinks=true, linkcolor=black, citecolor=blue, urlcolor=blue]{hyperref}   	% hyperlinks
\usepackage{url}            % simple URL typesetting

% --- MATH SYMBOLS ---

\usepackage{amsmath,amssymb}% more math symbols
\usepackage{amsfonts}       % blackboard math symbols
\usepackage{latexsym}		% more math symbols
\usepackage{chngcntr}		% more math symbols
\usepackage{mathrsfs}		% math-fonts
%\usepackage{marvosym}		% more math symbols (conflicts with Waldmann)
\usepackage{mathtools}		% more math symbols
\usepackage{nchairx}		% Waldmann package for general math symbols and operators
% theorem formatting (contained in Waldmann)
%\usepackage[amsmath,thmmarks,framed,thref]{ntheorem}

% --- GRAPHICS & CAPTIONS ----

\usepackage{graphicx}		% including images
\graphicspath{ {./figs/} }
\usepackage{subcaption}		% custom caption formatting
\DeclareCaptionLabelFormat{custom}{ \textbf{#1 #2}}
\captionsetup{format=hang}
\captionsetup{width=0.9\textwidth,labelformat=custom}

% --- TIKZ ---

\usepackage{tikz}			% basic tikz for custom images
\usetikzlibrary{cd}			% custom diagrams
\usetikzlibrary{external}	% externalize images for faster compilation
\tikzexternalize[prefix=figures/]
\AtBeginEnvironment{tikzcd}{\tikzexternaldisable} %fix cd/externalize
\AtEndEnvironment{tikzcd}{\tikzexternalenable}
\usepackage{pgfplots}		% custom plotting
\usepgfplotslibrary{colormaps}
\pgfplotsset{compat=newest}	
\usetikzlibrary{patterns}	% custom patterns

% --- FORMAT ---

\usepackage[a4paper]{geometry} % a4 paper
\usepackage{setspace}		% spacing
\usepackage[nobottomtitles*]{titlesec} %prevent section titles from sometimes being on the bottom of a page
\usepackage{titlesec}
\allowdisplaybreaks			% allow page breaks within math environments

% --- CUSTOM COMMANDS ---
%Logic
\newcommand{\then}{\Rightarrow}
\newcommand{\since}{\Leftarrow}
\renewcommand{\iff}{\ensuremath{\Leftrightarrow}}

%pretty epsilon
\let\oldepsilon\epsilon
\let\epsilon\varepsilon
\let\varepsilon\oldepsilon
%pretty phi
\let\oldphi\phi
\let\phi\varphi
\let\varphi\oldphi

%matrix
\newcommand{\qmatrix}[1]{\ensuremath{\left(\begin{matrix}#1\end{matrix}\right)}}

% --- DATA ---

\title{Introduction to Operator Algebras}
\author{Alcides Buss\\Notes by: Linus Mußmächer\\2336440}
\date{Summer 2023}

% --- DOCUMENT ---

\begin{document}

\maketitle

\newpage

\tableofcontents

\newpage

The set of all linear bounded operators $\algebra{L}(H) = \algebra{B}(H)$ on a given Banach space $H$ is a (Banach) algebra with $S \cdot T = S \circ T$.
$M \subseteq \algebra{L}$ is a Subalgebra such that $M^* \subseteq M$ where $T^*$ is the adjoint of $T$.
This is also a closed subspace with respect to the strong topology. This is equivalent to $M = M''$ (when $X \subseteq \algebra{B}(H), X' = \{ T \in \algebra{B}(H) \mid TS = ST \ \forall_{ S \in X} \}$)

\subsection*{Some topological basics}

\begin{definition}~
	\begin{itemize}
		\item Topology, Open
		\item Hausdorff, locally Hausdorff
		\item compact
	\end{itemize}
\end{definition}

\begin{definition}
	A topological space $X$ is \textbf{locally Hausdorff} if every $x \in X$ admits a compact neighborhood basis, that is for every $x \in X$ and every open set $U \ni x$ there exists an open set $V \ni x$ with $\overline{V}$ is compact.
\end{definition}

\begin{corollary}
	If a set $V$ is compact in any subset $U \subseteq X$, it is also compact in $X$.
\end{corollary}

\begin{example}[Snake with two heads]
	Consider $I = [0,1]$ with the standard topology and extend the set with an element $1^+$ such that $I \cup 1^+ \setminus 1$ is isomorphic to $I$. Then $I \cup 1^+$ is locally Hausdorff and compact, but not Hausdorff.
\end{example}

\subsection*{Some results about locally compact Hausdorff spaces}

\begin{lemma}[Uryson's Lemma]
	Let $X$ be locally compact and Hausdorff.
	For all $F \subseteq X$ closed and $K \subseteq X$ compact with $F \cap K = \emptyset$, there exists an $f: X \to [0,1]$ continuous such that $f|_K \equiv 1$ and $f|_F \equiv 0$.
\end{lemma}

\begin{theorem}[Tietze's extension theorem]
	Let $X$ be locally compact, $K \subseteq X$ compact and $f: K \to \mathds{C}$ continuous. Then there exists a continuous $\tilde f: X \to \mathds{C}$ such that $\tilde f|_K = f$.
\end{theorem}

\begin{theorem}[Alexandroff's conpactification]
	If $X$ is locally compact and Hausdorff, then $\tilde X \sqcup \{\infty\}$ is a compact Hausdorff space $\topology{O}(\tilde X) = \topology{O}(X) \cup \{K^\complement \cup \{\infty\} \mid K \text{ compact} \}$.
\end{theorem}

\begin{example}
	Compacting the real line $\mathds{R}$ yields the space $\tilde{\mathds{R}}$, which is isomorphic to the unit circle $\Pi = \mathds{S}^1$.
\end{example}

\begin{theorem}
	Conversely, if $Y$ is a compact Hausdorff space, then for all $y_0 \in Y$, $X \coloneqq Y \setminus \{y_0\}$ is locally compact (in respect to the subspace topology).

	More generally, if $Y$ is locally compact and Hausdorff, and $Z \subseteq Y$ is a difference of open and closed subsets, of $Y$ (i.e. $Z = U \setminus F$, where $U$ is open in $Y$ and $F$ is closed in $Y$), then $Z$ is locally compact.
\end{theorem}

\section{Algebras}

\begin{definition}
	An \textbf{algebra} is a (complex) vector space $\algebra{A}$ endowed with a bilinear and associative multiplication: $\algebra{A} \times \algebra{A} \to \algebra{A}, (a,b) \mapsto a \cdot b$. So
	\begin{enumerate}
		\item $(a + \alpha b) \cdot (c + \beta d) = ac + \alpha bc + \beta ad + \alpha \beta b d$
		\item  $(a \cdot b) \cdot c = a \cdot (b \cdot c)$
	\end{enumerate}
	for all $a,b,c \in \algebra{A}$ and $\alpha, \beta \in \mathds{C}$. We say that $\algebra{A}$ is
	\begin{enumerate}
		\item \textbf{commutative}, if $ab = ba$ for all $a,b \in \algebra{A}$ and
		\item \textbf{unital}, if there exists $1 = 1_\algebra{A} \in \algebra{A}$ such that $1 \cdot a = a \cdot 1 = a$ for all $a \in \algebra{A}$.
	\end{enumerate}
	~
\end{definition}

\begin{example}~
	\begin{enumerate}
		\item $\mathds{C}$, or more generally $\mathds{C}^n = \mathds{C} \oplus \dots \oplus \mathds{C}$, is an algebra.
		\item Say $X$ is any set; let $\mathds{C}^X = \{ f: X \to \mathds{C} \}$ with point wise multiplication $(f \cdot g)(x) = f(x) \cdot g(x)$.
		      These are commutative unital algebras (with $1(x) = 1 \in \mathds{C}$).
		\item Consider the polynomials $\mathds{C}[X] = \{ \sum_{i = 0}^{n} \lambda_i x^i \mid \lambda_i \in \mathds{C}, n \in \mathds{N} \}$ with the usual operations.
		      This is a commutative unital algebra.
		\item Let $X$ be a topological space and $C(X) = \{f: X \to \mathds{C} \mid f \text{ is continouus}\} \subseteq \mathds{C}^X$ the set of continuous functions on $X$.
		      This is a commutative unital (sub)algebra (of $\mathds{C}^X$).
		\item Take any vector space $A$ define a (trivial) multiplication $a \cdot b \coloneqq 0$.
		      This is a commutative Algebra (that is not unital unless $A = 0$).
		\item $M_n(\mathds{C})$ (the complex $n \times n$ matrices) with the usual multiplication are a non-commutative (unless $n=1$) unital algebra.
		\item Let $V$ be any (complex) vector space. The set of all linear operators $L(V) := \{T: V \to V \mit T \text{ linear operator}\}$ is a unital (non-commutative for $\dim V > 1$). We observe $\algebra{L}(\mathds{C}^n) \simeq M_n(\mathds{C})$.
		\item Let $S$ be a semigroup (i.e. a set with an associative operation $S \times S \to S$, e.g. $(\mathds{N}, +)$). Then $\mathds{C}[S] = \{ \sum_{s \in S} \lambda_s s \mid \lambda_s \in \mathds{C}, |\{s: \lambda_s \neq 0\}| < \infty \}$ (the finite formal sums of elements of $S$) with the following product
		      \begin{equation*}
			      \left(\sum_{s \in S'}\lambda_s s\right) \cdot \left(\sum_{t \in S} \lambda_t' t\right) := \sum_{s,t \in S} (\lambda_s \cdot \lambda'_t)(s \cdot t) \in S
		      \end{equation*}
		      Observe: As a vector space: $\mathds{C}[S] \subseteq \mathds{C}^S$.
		      In general, this is neither commutative nor unital.
	\end{enumerate}
\end{example}

\section{Normed algebras}

\begin{definition}
	An algebra $\algebra{A}$ is \textbf{normed}, if it is endowed with a (vector space) norm $\| \cdot \| \colon \algebra{A} \to [0, \infty)$ satisfying $\|a \cdot b\| \leq \|a\| \cdot \|b\|$.
	If $\algebra{A}$ is unital with unit $1_\algebra{A}$, we usually assume $\| 1_\algebra{A} \| = 1$ except for $\algebra{A} = 0$.
\end{definition}

\begin{definition}
	A \textbf{Banach algebra} is a normed algebra that is also complete (as a metric space with respect to the distance $d(a,b) := \| a - b\|$), i.e. every Cauchy sequence converges.

\end{definition}

\begin{example}
	\begin{enumerate}
		\item If $X$ is a compact space then $C(X)$ is a commutative unital Banach algebra with respect to the norm $\| f \|_\infty := \sup_{x \in X} |f(x)| < \infty$ (since $X$ is compact).
		\item If $V$ is a normed (respectively Banach) vector space, e.g. $\mathds{C}^n$ or $\ell^p(\mathds{N})$, then $\algebra{L}(V) = \{T \in L(V) \mid T \text{ is bounded/continouus} \}$ with $\|T\| := \sup_{\|v\| \leq 1} \| T(v) \| < \infty$ is a normed Banach algebra.
		\item If $X$ is a topological space, then $C_b(X) = \{f \in C(X) \mid \|f\|_\infty < \infty \}$ (bounded continuous functions) is a Banach space.
		\item Let $X$ again be a topological space. Then the set of all functions \textbf{vanishing at $\infty$},
		      \begin{align*}
			      C_0(X) & = \{ f \in C(X) \mid \forall_{\epsilon > 0} \exists_{K \subseteq X, K \text{ compact}} \forall_{x \notin K} |f(x)| < \epsilon \} \\
			             & = \{ f \in C(X) \mid \forall_{\epsilon > 0} \{x \in X \mid |f(x)| \geq \epsilon \} \text{ is compact} \}
			      \subseteq C_b(X)
			      \text{,}
		      \end{align*}
		      is also a Banach algebra.
	\end{enumerate}
\end{example}

\begin{exercise}
	Assume $X$ is locally compact and Hausdorff. Prove the following are equivalent:
	\begin{enumerate}[label=(\arabic*)]
		\item $X$ is compact.
		\item $C(X) = C_0(X)$
		\item $C_0(X)$ is unital.
		\item The unit function $1 \in C_b(X)$ belongs to $C_0(X)$.
	\end{enumerate}
\end{exercise}

\begin{proof}
	\begin{itemize}
		\item (1) $\then$ (2): Recall the definition of $C_0(X)$. If $X$ is compact, every closed subset (especially every $\{x : |f(x)| \geq \epsilon\}$) is compact, so the condition of $C_0(X)$ is trivial.
		\item (2) $\then$ (3): Since $C(X)$ is unital, $C_0(X)$ is as well.
		\item (3) $\then$ (4): Suppose $C_0$ is unital, and let $f \in C_0(X)$ be the unit. Then $f \cdot g = g$ for all $g \in C_0(X)$, i.e. $f(x) g(x) = g(x) \forall_{x \in X} \forall_{g \in C_0(X)}$. By Uryson's lemma, given any $x_0 \in X$, there exists $g \in C_0(X)$ with $g(x_0) = 1$ (by looking at $K = \{x_0\}$ and taking $F$ as the complement of any relatively compact environment of $x_0$.). Then $f(x_0) = f(x_0) g(x_0) = g(x_0) = 1$. Doing this for every $x_0 \in X$ yields $f \equiv 1$.
		\item (4) $\then$ (1): Since $1 \in C_0(X)$, for every $\epsilon > 0$ the set $\{x \mid |f(x)| \geq \epsilon\}$ is compact. Choose $\epsilon = \frac{1}{2}$. Then, $\{x \mid |f(x)| = |1| \geq \frac{1}{2}\} = X$ is compact.
	\end{itemize}
\end{proof}

\begin{exercise}
	Let $X$ be a locally compact Hausdorff space.
	Prove that $C_0(X) \cong \{f \in C(X) \mid f(\infty) = 0\}$
\end{exercise}

\section{Algebras}

\begin{definition}
	A \textbf{$^*$-algebra} is a complex algebra $\algebra{A}$ with an \textbf{involution} $^*: \algebra{A} \to \algebra{A}$ satisfying
	\begin{enumerate}
		\item $(a + \lambda b)^* = a^* + \overline{\lambda} b^*$
		\item $(a^*)^* = a$
		\item $(ab)^* = b^* a^*$
	\end{enumerate}
	for all $a,b \in \algebra{A}$ and all $\lambda \in \mathds{C}$.
\end{definition}

\begin{definition}
	A \textbf{normed $^*$-algebra} is a normed algebra $\algebra{A}$ with an involution (such that $\algebra{A}$ is a $^*$-algebra) also satisfying $\| a^*\| = \|a\|$ for all $a \in \algebra{A}$.

	A \textbf{Banach-$^*$-algebra} is a complete normed $^*$-algebra.
\end{definition}

\begin{definition}
	A $C^*$-algebra is a Banach-$^*$-algebra satisfying $\|a^* \cdot a\| = \| a \|^2$.
\end{definition}

Observation: Recall that $\|a \cdot b\| \leq \|a\| \cdot \| b \|$ in all normed algebras. Applying this to a $C^*$-algebra we get $\|a \cdot a^*\| \leq \|a^*\| \cdot \| a \|$. If $\algebra{A}$ is a $C^*$-algebra, then $\| a \| ^2 = \| a \cdot a^*\| \leq \|a^*\| \cdot \|a\|$, so $\| a\| = \|a^*\|$.

\begin{example}~
	\begin{enumerate}
		\item If $X$ is a set, then $\mathds{C}^X$ is a $^*$-algebra with $f^* = \overline{f}$ and $\algebra{C}^\infty(X)$ is a $C^*$-algebra.
		\item If $X$ is a topological space, then $C(X) \subseteq \mathds{C}^X$ is also a $^*$-subalgebra and for $\{f \in C(X) \mid \supp(f) = \closure{\{x \in X \mid |f(x)| \neq 0\}} \text{ compact} \}$ we have
		      \begin{equation*}
			      C_c(X) = \subseteq C_0(X) \subseteq C_b(X) \subseteq C(X) \subseteq C^\infty(X)
		      \end{equation*}
		      and $C^\infty$ is a $C^*$-algebra. $C_c$ is a $^*$-algebra, but not Banach in general.

		      If $X$ is compact, it follows $C_c(X) = C_0(X) = C_b(X)$.

		      Observation: If $X$ is locally compact and Hausdorff, then $\closure{C_c(X)} = C_0(X)$.
		\item Let $X$ be a measured space ($X$ is endowed with a $\sigma$-algebra). Then $B_\infty(X) = \{f \in C^\infty \mid f \text{ is measurable}\}$ is a $C^*$-algebra.
		      If $\mu$ is a measure on $X$ (e.g. $X = \mathds{R}^n$ and $\mu$ the Lebesgue measure) then $L^\infty(X,\mu)$ are the essentially bounded functions and
		      \begin{equation*}
			      L^\infty(X) = \{f: X \to \mathds{C} \mid \|f\| \coloneqq \inf \{ c \geq 0 \mid \mu(\{x \mid |f(x)| > c\}) = 0 \} \}
		      \end{equation*}
		      is also a $C^*$-algebra.

		      Observation: $L^2(X, \mu) = $ \enquote{$\mu$-separable function}, $L^\infty(X, \mu) \xrightarrow{\mu} B(L^2(X,\mu)), f \mapsto \mu_f = \{g \mapsto f \cdot g\}$
		\item A non-example: Let $\mathds{D}$ be the unit disk and $\algebra{A}(\D) = \{f \in \mathds{C}(\mathds{D}) \mid \text{ analytic in } \mathds{D}^\interior \}$

		      \textbf{Morera's Theorem} from complex analysis states that $f \in C(\mathds{D})$ is analytic if and only if $\int_{\gamma} f(z) dz = 0$ for all closed and piece wise smooth paths in $\mathds{D}^\interior$. From this, it follows that $\algebra{A}(\mathds{D})$ is closed in $C(\mathds{D})$, therefore a Banach algebra. It is also a Banach-$^*$-algebra with, but $f^* = \overline{f}$ (point wise) is not possible, as $z \mapsto \overline{z}$ is not analytic. Thus, we have to choose $f^*(z) = f(\overline{z})$.
		      But $\algebra{A}(\mathds{D})$ is not a $C^*$-algebra, as $\| f^* f\|_\infty \neq \|f\|_\infty^2$ for some $f \in \algebra{A}(\mathds{D})$.
		\item A non-commutative example: Let $H$ be a Hilbert space and $B(H) = \algebra{L}(H) = \{T: H \to H \mid T \text{bounded, continuous, linear}\}$ and $\|H\| \coloneqq \sup_{\|z\| < 1} \|T(z)\| < \infty$. This is a $C^*$-algebra where $T^*$ is the adjoint of $T$, that is $\SP{T^* z, w} = \SP{z, Tw}$ for all $z,w \in H$.

		      $C^*$-axiom: $\|T^* \cdot T\| \leq \|T\|^2$ since $\algebra{L}(H)$ is a Banach algebra, and we also have
		      \begin{align*}
			      \|T\|^2 & = \sup_{\|z\| < 1} \|T(z)\|^2 = \sup_{\|z\| < 1} \SP{Tz, Tz} = \sup_{\|z\| < 1} \SP{z, T^* T z} \\
			              & \leq \sup_{\|z\| < 1} \|z\| \| T^*T z\| \leq \sup_{\|z\| < 1}  \|z\| \| T^*T\| \leq  \| T^*T\|
		      \end{align*}
		      In particular, $M_n(\mathds{C}) \simeq \algebra{L}(\mathds{C}^n)$ is a unital $C^*$-algebra.
		\item To produce more examples, take any subset $S \subseteq \algebra{L}(H)$ and take $C^*(S) \subseteq \algebra{L}(H) = \closure{\Span\{S_i \mid S_i \in S \cup S^*, i \leq n \in \mathds{N}\}}$.
	\end{enumerate}
\end{example}

\begin{example}
	Let $s \in \algebra{L}(\ell^2(\mathds{N}))$. The shift $s$, defined by $s(e_i) = e_{i+1}$ for all $i \in \mathds{N}$ (where $\{e_i\}$ is the canonical basis of the sequence space), is an isometry, that is $s^* \cdot s = \id$.
	Since $s \cdot s^* \neq \id$, it is not surjective and not a proper isometry.
	We define
	\begin{equation*}
		T = C^*(s) = \closure{\Span\{s^n (s^*)^m \mid m,n \in \mathds{N}_0 \}} \subseteq \algebra{L}(\ell^2(\mathds{N}))
	\end{equation*}
	as the \textbf{Toeplitz algebra}.
\end{example}

\begin{example}
	Let $H$ be a Hilbert space and $S$ the set of all finite rank operators on $H$.

\end{example}

\begin{example}~
	\begin{enumerate}
		\item \textbf{Commutative}: $C_0(X)$ for a locally Hausdorff space $X$.
		\item \textbf{Non-commutative}: $\mathcal{L}(\hilbert{H}) = \mathcal{B}(\hilbert{H})$ for any Hilbert space $\hilbert{H}$ (with dimension greater $1$).
		\item \textbf{More generally}: Take any subset $S \subseteq \mathcal{L}(\hilbert{H})$ and construct $C^*(S) \subseteq \mathcal{L}(H)$ as
		      \begin{equation*}
			      \closure{\Span}\{S_1, \dots, S_n\mid S_i \in S \cap S^*\}
		      \end{equation*}
	\end{enumerate}
\end{example}

\begin{example}[Cuntz algebras]
	Take again $\hilbert{H} = \ell^2\mathds{N} = \{(\lambda_n)_{n \in \mathds{N}_0} \mid \sum_{n = 0}^{\infty} |\lambda_n|^2 < \infty\}$ where $\SP{\lambda, \lambda'} = \sum_{i \in \mathds{N}_0} \overline{\lambda_i} \lambda_i'$ and which has the orthonormal base $(e_n)_{n \in \mathds{N}}$ where $(e_n) = (\delta_{in})_{i \in \mathds{N}_0}$.

	On this algebra, define
	\begin{itemize}
		\item $S_1(e_n) = e_{2n}$.
		\item $S_2(e_n) = e_{2n + 1}$.
	\end{itemize}
	We have partitioned the natural numbers into evens and odds. This defines two (proper) isometries $S_1, S_2 \in \algebra{L}(\hilbert{H})$, that is $S_i^* S_i = \id_\hilbert{H}$, to subspaces of $\hilbert{H}$. Notice: $S_i^*S_j = 0$  for $i \neq j$ as well as $S_1 S_1^* + S_2S_2^* = \id_\hilbert{H}$.  Define $\algebra{O}_2 = C^*(S_1, S_2) = \closure{\Span}\{S_\alpha S_\beta^* \mid \alpha, \beta \text{ finite words in } \{1,2\}\}$. For example, for $\alpha = 121211$ we have $S_{\alpha} = S_1 S_2 S_1 S_2 S_1^2$. $\algebra{O}_2$ is called the \textbf{Cuntz algebra}. More generally, one can define $\algebra{O}_3, \algebra{O}_4$, ... Cuntz algebras. Joachim Cuntz proved that these are simple $C^*$-algebras with additional interesting properties we will see later.
\end{example}

\begin{example}[Rotation algebras]
	Let $\hilbert{H} = \ell^2(\mathds{Z})$ (bi-infinite sequences) with basis $(e_n)_{n \in \mathds{Z}}$  Define:
	\begin{itemize}
		\item $U(e_n) := e_{n+1}$ (bilateral shift)
		\item $V(e_n) := \lambda^n e_n$ where $\lambda\in\mathds{C}$ is some fixed number $|\lambda| = 1$.
	\end{itemize}
	This defines two \textit{unitary} operators: $U U^* = 1 = U^* U$ and $V^* V = 1 = V^*V$. If $\exp(2 \pi i \theta), \theta \in \mathds{R}$ define $A_\theta := C^*(U,V) \subseteq \algebra{L}(\ell^2 \mathds{N})$.

	There is a special relation between $U$ and $V$ where $UV = \lambda VU = \exp(2 \pi i \theta) V U$. From this relation, we can describe $A_\theta = \closure{\Span}\{\sum_{n,m \in \mathds{Z}}^{\text{finite}} a_{n,m} U^n V^m \mid a_{n,m} \in \mathds{C}\}$.

	Furthermore, if $\theta \in \mathds{R} \setminus \mathds{Q}$, $A_\theta$ is simple.
\end{example}

\begin{example}[$C^*$-algebras of groups]
	Let $G$ be a (discrete) group.
	Look at $\hilbert{H} = \ell^2(G) = \{(a_g)_{g \in G} \mid \sum_{g \in G} |a_g|^2 < \infty\}$
	(Note: This limit will only converge if there are countably (or finitely) many non-zero parts)
	with ONB $(\delta_g)_{g \in G}$ where $\delta_g(h) = \delta_{gh}$.
	Define for each $g \in G$ an operator $\lambda_g \in \algebra{L}(\ell^2 G)$ by $\lambda_g(\delta_h) = \delta_{gh}$.
	Notice that $h \mapsto gh$ is a bijection, and thus $\lambda_g$ is a unitary operator with $\lambda_g^* = \lambda_{g^{-1}}$. We can now define the \textbf{reduced $C^*$-algebra} of the group:
	\begin{equation*}
		C_R^*(G) := C_\lambda^*(G) \subseteq \algebra{L}(\ell^2 G) = C^*(\lambda_g \mid g \in G)
	\end{equation*}
	Here, we have the relation $\lambda_g \cdot \lambda_h = \lambda_{gh}$ and thus $C_R^*(G) = \{\sum a_g \lambda_g \mid a_g \in \mathds{C}\}$.

	In general, take $U: G \to \algebra{L}(H), g \mapsto U_g$ a \textbf{unitary representation of $G$} with $U_g U_h = U_{gh}$ and $U_1 = \id$ as well as $U_g^{-1} = U_{g^{-1}}$. Then $C^*_U(G) := \{\sum_{g \in G} a_g U_g \mid a_g \in \mathds{C}\} \subseteq \algebra{L}(H)$. There exists a \textbf{universal unitary representation} $C^*_\mathrm{max}(G)$, a full $C^*$-algebra of $G$.
\end{example}

\begin{remark}~
	\begin{enumerate}
		\item If $G$ is Abelian, then $C_U^*(G)$ is also abelian (commutative). In particular, $C_\lambda^*$ is abelian. Later, we will prove $C_\lambda^*(G) \simeq C(\hat G)$ where $\hat G$ is the dual of $G$, i.e. $\{X: G \to \mathds{C} \text{ characters}\}$.
		\item For many groups, like $G = \mathds{F}_n$ (the free groups) the reduced $C^*$-algebra $C^*_\lambda(G)$ is simple.
	\end{enumerate}
\end{remark}

\section{Homomorphisms of algebras}

\begin{definition}
	If $\algebra{A}, \algebra{B}$ are algebras, a \textbf{homomorphism} from $\algebra{A}$ to $\algebra{B}$ is a linear map $\phi: \algebra{A} \to \algebra{B}$ such that $\phi(ab) = \phi(a) \phi(b)$ for any $a,b \in \algebra{A}$.

	If $\algebra{A}$ and $\algebra{B}$ are $^*$-algebras, a \textbf{$^*$-homomorphism} is a homomorphism $\phi: \algebra{A} \to \algebra{B}$ such that $\phi(a^*) = \phi(a)^*$ for all $a \in \algebra{A}$.

	If $\algebra{A}, \algebra{B}$ are Banach algebras, then usually we want to have \textbf{continuous} homomorphisms. Even more, we usually ask for \textbf{contractive} homomorphisms $\phi: \algebra{A} \to \algebra{B}$, (that is $\|\phi\| \leq 1$).
\end{definition}

We will be especially interested in \textbf{characters}:

\begin{definition}
	A \textbf{character} of an algebra $\algebra{A}$ is a non-zero homomorphism $\chi: \algebra{A} \to \mathds{C}$.
\end{definition}

\begin{example}
	Take any subalgebra $\algebra{A} \subseteq \mathds{C}^X$. Take $x_0 \in X$ and set $\chi_{x_0} := \mathrm{ev}_{x_0}: \algebra{A} \to \mathds{C}, f \mapsto f(x_0)$. This is not necessarily a character, but it is for example, if $\algebra{A} = C(X)$ or $C_b(X)$ or $C_0(X)$ (if $X$ is \enquote{nice}, like Hausdorff).
\end{example}

\begin{definition}
	A  ($^*$)-isomorphism between two $(^*)$-algebras $\algebra{A}$ and $\algebra{B}$ is a bijective $(^*)$- homomorphism $
		\phi: \algebra{A} \overset{\sim}{\to} \algebra{B}$.
\end{definition}

\begin{definition}
	A \textbf{$(^*)$-ideal} of a $^*$-algebra $\algebra{A}$ is a subspace $I \subset A$ such that $I \cdot A \subseteq I$, $A \cdot I \subseteq I$ (if only one condition applies, we call this a \textbf{left ideal} or \textbf{right ideal}). For $^*$-ideals, we also want $I^* = I$. We notate this as $I \trianglelefteq \algebra{A}$.
\end{definition}

\begin{example}
	If $\phi: \algebra{A} \to \algebra{B}$ is a $(^*)$-homomorphism, then $\ker \phi \trianglelefteq \algebra{A}$.
\end{example}

\begin{example}
	If $I \trianglelefteq \algebra{A}$ for $\algebra{A}$ a $(^*)$-algebra
	\begin{equation*}
		\algebra{A} / I = \{a + I \mid a \in \algebra{A}\}
	\end{equation*}
	with $(a + I) \cdot (b + I) := ab + I$ and $(a + I)^* = a^* + I$ is a $(^*)$-algebra.
\end{example}

\begin{theorem}
	If $\algebra{A}$ is a Banach-$^*$-algebra, then $I \trianglelefteq \algebra{A}$ is a closed ideal, then the quotient $I / \algebra{A}$ is also a Banach-$^*$-algebra.
\end{theorem}

\begin{proof}
	Later.
\end{proof}

\section{Spectral theory}

\begin{notation}
	If $\algebra{A}$ is
	a unital algebra, we write
	\begin{equation*}
		\inv(\algebra{A}) = \{a \in \algebra{A} \mid a \text{ is invertible in } \algebra{A}\} = \{a \in \algebra{A} \mid \exists_{a^{-1} \in \algebra{A}} a a^{-1} = 1 = a^{-1} a \}
	\end{equation*}
	This is a group. Sometimes we also write $GL(\algebra{A})$.
\end{notation}

\begin{definition}
	Given a unital algebra $\algebra{A}$ and $a \in \algebra{A}$, we define its \textbf{spectrum} (in $\algebra{A}$) as
	\begin{equation*}
		\sigma_\algebra{A}(a) = \sigma(a) = \{\lambda \in \mathds{C} \mid \lambda \cdot 1 - a \notin \inv(\algebra{A})\}
	\end{equation*}
	and the resolvent of $a$ (in $\algebra{A}$) as
	\begin{equation*}
		\rho_\algebra{A}(a) = \rho(a) = \algebra{A} \setminus \sigma_\algebra{A}(a) = \{\lambda \in \mathds{C} \mid \lambda - a \in \inv(\algebra{A})\}
	\end{equation*}
\end{definition}

\begin{example}[Linear Algebra]
	Let $\algebra{A} = M_m(\mathds{C})$ and $a \in \algebra{A}$. Then we have
	\begin{equation*}
		\sigma(a) = \{\lambda \in \mathds{C} \mid \lambda - a \notin \inv(\algebra{A})\} = \{\lambda \in \mathds{C} \mid \det(\lambda - a) = 0\}
	\end{equation*}
	and these are the roots of the characteristic polynomial $\det(\lambda - a)$. This is exactly the usual spectrum from linear algebra.
\end{example}

\begin{example}[Functional Analysis]
	Let $\algebra{A} = \algebra{L}(\hilbert{H})$ -- where $\hilbert{H}$ is any Hilbert- or Banach space -- and $T \in \algebra{A}$. Then $\sigma_\algebra{A}(T)$ is exactly the spectrum as defined in functional analysis.

	If $S$ is the shift in $\algebra{L}(\ell^2 \mathds{N})$, then we have $\sigma(S) = \mathds{D}$.
\end{example}

\begin{example}
	Let $\algebra{A} = \mathds{C}[X]$. Here we have $\inv(\algebra{A}) = \{a_0 X^0 \mid a_0 \in \mathds{C} \setminus \{0\}\}$ the constant non-zero polynomials. If $a = \sum_{k=0}^{N} a_k x^k \in \algebra{A}$, then we have two cases:
	\begin{equation*}
		\sigma(a) = \left\{ \begin{matrix}
			\{a_0\}    & a = a_0 \text{ (constant)} \\
			\mathds{C} & \text{otherwise}
		\end{matrix}
		\right.
	\end{equation*}
\end{example}

\begin{example}
	Let $\algebra{A} = \mathds{C}(X) = \{p,q \mid p,q \in \mathds{C}[X], q \neq 0 \}$. Now we have $\inv(\algebra{A}) = \algebra{A} \setminus \{0\}$. If $a \in \algebra{A}$, then
	\begin{equation*}
		\sigma(a) = \left\{ \begin{matrix}
			\{a_0\}   & a = a_0 \text{ (constant)} \\
			\emptyset & \text{otherwise}
		\end{matrix}\right.
	\end{equation*}
\end{example}

\begin{example}
	Let $\algebra{A} = C(X)$ for any topological space $X$. Then
	\begin{equation*}
		\inv(\algebra{A}) = \{f \in C(X) \mid \forall_{x \in X} f(x) \neq 0\}
	\end{equation*}
	and
	\begin{equation*}
		\sigma(f) = \{\lambda \in \mathds{C} \mid \lambda - f \notin \inv(\algebra{A})\} = \{\lambda \in \mathds{C} \mid \exists_{x \in X} f(x) = \lambda\} = \image(f) = f(X)\text{.}
	\end{equation*}
\end{example}

\begin{example}
	Let $X$ be any topological space and consider $\algebra{A} = C_b(X)$. Then
	\begin{equation*}
		\inv(C_b(X)) = \{f \in C_b(X) \mid \exists_{\epsilon > 0} \forall_{x \in X} |f(x)| \geq \epsilon \}
	\end{equation*}
	and
	\begin{equation*}
		\sigma(f) = \{\lambda \in \mathds{C} \mid \lambda - f \notin \inv(\algebra{A}) \} = \{\lambda \in \mathds{C} \mid \exists_{(x_n)} f(x_n) \to \lambda\} = \closure{\image(f)} = \closure{f(X)} \text{.}
	\end{equation*}
	This is a compact subset of $\mathds{C}$.
\end{example}

\begin{theorem}[Algebraic spectral mapping theorem]
	Let $\algebra{A}$ be an algebra, $a \in \algebra{A}$ and $p \in \mathds{C}[X], p(X) = \sum_{k = 0}^{n} \lambda_k X^k$ and define $p(a) = \sum_{k = 0}^{n} \lambda_k a^k$. Recall that the mapping $\mathds{C}[X] \to \algebra{A}, p \mapsto p(a)$ is a unital homomorphism.

	Then $\sigma(p(a)) = p(\sigma(a))$ assuming $\sigma(a) \neq \emptyset$.
\end{theorem}

\begin{proof}
	If $p(X) = \lambda_0$ constant, this is clear (the spectrum is exactly $\lambda_0$ on both sides). Assume $p(x)$ is not constant. Fix $\mu \in \mathds{C}$ and write
	\begin{equation*}
		\mu - p(x) = \lambda_0 (x - \lambda_1) \cdots (x - \lambda_n)
	\end{equation*}
	as per the fundamental theorem of algebra (note that these are not the same $\lambda$ as before) with $\lambda_0 \neq 0$.  Then $\mu - p(a) = \lambda_0 (a - \lambda_1)\cdots(a - \lambda_n)$. Since these expressions commute, this product is invertible if and only if $(a - \lambda_i)$ is invertible for every $i$. So $\mu \in \sigma(p(a)) \iff \mu - p(a)$ is not invertible if and only if there exists an $i$ for which $\lambda_i - a$ is not invertible, so $\lambda_i \in \sigma(a)$. But the $\lambda_i$ are exactly the numbers satisfying $p(\lambda) = \mu$. Thus, $\mu$ is in $\sigma(p(a))$ if it is in the image of $\sigma(a)$ under $p$. Therefore, we conclude $\sigma(p(a)) = p(\sigma(a))$.
\end{proof}

We now focus on invertible elements in \textbf{Banach algebras}.

\begin{theorem}
	If $\algebra{A}$ is a unital Banach algebra and $a \in \algebra{A}$ with $\|a\| < 1$ then $1 - a$ is invertible and $(1-a)^{-1} = \sum_{n=0}^{\infty} a^n$.
\end{theorem}

\begin{proof}
	Observe that, since $\|a\| < 1$, we have $\sum_{n = 0}^{\infty} \|a\|^n = \frac{1}{1 - \|a\|} < \infty$. This implies the (absolute) convergence of $\sum_{n = 0}^\infty$ by the characteristic property of Banach spaces. Hence, $b \coloneqq \lim_{N \to \infty} \sum_{n = 0}^{N} a^n \in \algebra{A}$. No, if $N \in \mathds{N}$, then
	\begin{equation*}
		(1-a) \left(\sum_{n = 0}^{N} a^n\right) = \left(\sum_{n = 0}^{N} a^n\right) - \left(\sum_{n = 1}^{N + 1} a^n\right) = 1 - a^{N+1} \to 1
	\end{equation*}
	because of $\|a\| < 1$. This yields $(1-a)b = 1$.
\end{proof}

\begin{theorem}
	Let $\algebra{A}$ be a non-empty, non-zero unital Banach algebra. Then $\inv(\algebra{A})$ is an open subset of $\algebra{A}$ and the function $f: \inv(\algebra{A}) \to \algebra{A}, a \mapsto a^{-1}$ is Frechet-differentiable and in particular continuous as well as $f'(a) b = -a^{-1} b a^{-1}$.
\end{theorem}

Recall from calculus that $\frac{d}{dx} \frac{1}{x} = -\frac{1}{x^2}$. Also recall that $f: U \overset{\text{open}}{\subseteq} X \to Y$ with $X,Y$ Banach spaces is \textbf{differentiable} at $x_0 \in U$ there exists an operator $D_{x_0} = f'(x_0) \in \algebra{L}(X,Y)$ such that
\begin{equation*}
	\lim_{h \to 0} \frac{f\|(x_0 + h) - f(x_0) - D_{x_0}(h)\|}{\|h\|} = 0
\end{equation*}

\begin{proof}
	Take $a \in \inv(\algebra{A})$. If $b \in \algebra{A}$ such that $\|a - b\| < \|a^{-1}\|^{-1}$. From this, we have $\|b a^{-1} - 1\| = \| b a^{-1} - a a^{-1} \| = \|(b-a) a^{-1}\| \leq \|b - a\| \cdot \| a^{-1} \| < 1$. Per the previous theorem, $b a^{-1} \in \inv(\algebra{A})$. This implies that $b$ is also invertible. This shows that $\inv(\algebra{A})$ is open.

	Furthermore, if $\|b\| < 1$, then also ($\|-b\| < 1$). Thus, $1 + b \in \inv(\algebra{A})$ and $(1+b)^{-1} = \sum_{n = 0}^{\infty} (-1)^n b^n$. Thus,
	\begin{equation*}
		\| (1+b)^{-1} - 1 + b\| = \left\| \sum_{n = 0}^\infty (-1)^n b^n - 1 + b  \right\| \leq \left\| \sum_{n = 2}^\infty (-1)^n b^n \right\| \leq \sum_{n = 2}^{\infty} \|b^n\|  \leq \sum_{n = 2}^\infty \|b\|^n = \frac{\|b\|^2}{1 - \|b\|}
	\end{equation*}
	Now let $a \in \inf(\algebra{A})$ and $c \in \algebra{A}$ such that $\|c\| < \frac{1}{2} \|a^{-1}\|^{-1}$. Then $\|a^-1 c\| \leq \|a^{-1}\| \| c\| \leq \frac{1}{2}$. So if $b = a^{-1}$, then
	\begin{equation*}
		\|(1 + a^{-1}c)^{-1} - 1 + a^{-1} c\| = \leq \frac{\| a^{-1} c\|^2}{1 = \|a^{-1} c \|} < 2 \| a ^{-1} c \|^2
	\end{equation*}
	Now, define $U: \algebra{A} \to \algebra{A}, b \mapsto - a^{-1} b a^{-1}$. Then this is a linear odd operation with $\|U\| \leq \|a^{-1}\|^2$, and we have
	\begin{align*}
		\|(a + c)^{-1} - a^{-1} - U(c)\| & = \|(a+c)^{-1} - a^{-1} + a^{-1} c a^{-1}\|                  \\
		                                 & =  \|(1 + a^{-1}c)^{-1} a^{-1} - a^{-1} + a^{-1} c a^{-1}\|  \\
		                                 & \leq \|(1 + a^{-1}c)^{-1} -1 + a^{-1} c  \| \cdot \|a^{-1}\| \\
		                                 & \leq 2 \|a^{-1} c\|^2 \|a^{-1}\| \leq 2 \|a^{-1}\|^3 \|c\|^2
	\end{align*}
	and thus
	\begin{equation*}
		\lim_{c \to 0} \frac{\|(a + c)^{-1} - a^{-1} - U(c)\|}{\|c\|} = 0
	\end{equation*}
\end{proof}

\begin{example}
	If we choose $\algebra{A} = \mathds{C}[X]$ and the norm $\|p\| = \sup_{\lambda \in [0,1]} |p(x)|$.
	Then $(\algebra{A}, \|\cdot \|)$ is a normed (but not Banach) algebra.
	For example, we see that $\lim_{m \to 0} 1 + X/m = 1 \in \inv(\algebra{A})$, but $1 + X/m \notin \inv(\algebra{A})$ and thus $\inv(\algebra{A})$ is not open (because the complement is not closed).
\end{example}

\begin{theorem}
	If $\algebra{A}$ is a Banach algebra with unit $1$, then for all $a \in \algebra{A}$ the spectrum $\sigma(a) \subseteq \mathds{C}$ is closed and $\sigma(a) \subseteq \overline{B(0, \| a \|)} = D(0, \|a\|) \coloneq \{\lambda \in \mathds{C} \mid |\lambda| \leq \|a\| \} $. Therefore, $\sigma(a)$ is compact by the Heine-Borell theorem.
\end{theorem}

\begin{proof}
	By definition
	\begin{equation*}
		\sigma(a) = \{ \lambda \in \mathds{C} \mid \lambda - a \notin \inv(\algebra{A}) \}
	\end{equation*}
	is the inverse image of the closed subset $\algebra{A} \setminus \inv(\algebra{A}) \subseteq \algebra{A}$ by the continuous function $\lambda \mapsto \lambda - a$.
	Therefore, $\sigma(a)$ is closed.

	Now if $|\lambda| \leq \|a\|$ then $\|\lambda^{-1} a\| < 1$.
	Then $1 - \lambda^{-1} a \in \inv(\algebra{A})$.
	Multiplying by $\lambda$ yields $\lambda - a \in \inv(\algebra{A})$.
	Thus, $\{\lambda \in \mathds{C} \mid |\lambda| > \|a\|\} \subseteq \rho(a)$ and thus $\sigma(a) \subseteq D(0, \|a\|)$.
\end{proof}

\begin{lemma}
	Let $\algebra{A}$ be a unital Banach algebra and $a \in \algebra{A}$.
	Then, the map $R_a: \rho(a) \subseteq \mathds{C} \to \algebra{A}, \lambda \mapsto (a - \lambda)^{-1}$ is Frechet-differentiable.
\end{lemma}

\begin{proof}
	This follows from the following general result:

	If $g: U \overset{\text{open}}{\subseteq} X \to Y$ and $f: V \overset{\text{open}}{\subseteq} Y \to Z$ for Banach spaces $X,Y,Z$ with $g(U) \subseteq V$ are differentiable at $x_0 \in U$ or respectively $y_0 = g(x_0) \in V$, then $f \circ g$ is differentiable and $(f \circ g)' (x_0) = f'(g(x_0)) g'(x_0)$.
\end{proof}

Observation: For $R_a(\lambda) = (a - \lambda)^{-1}$ we get $R_a'(\lambda) = (a - \lambda)^{-2}$.
We have $\mathcal{L}(\mathds{C}, \algebra{A}) \simeq \algebra{A}$ by $T \mapsto T(1)$.
Recall that if $f(a) = a^{-1}$ yields $f'(a) b = - a^{-1} b a^{-1}$.

\begin{theorem}[Gelfand]
	If $\algebra{A} \neq 0$ is a unital Banach algebra and $a \in \algebra{A}$ then $\sigma(a) \neq \emptyset$.
\end{theorem}

\begin{proof}
	Suppose $\sigma(a) = \emptyset$.
	Idea: Show that $R_a: \rho(a) \subseteq \mathds{C} \to \algebra{A}, \lambda \mapsto (a - \lambda)^{-1} = \frac{1}{a - \lambda}$ is bounded and differentiable and achieve a contradiction by Liouville's theorem.

	Claim: $\| (a - \lambda)^{-1}\| < \|a\|^{-1}$ if $|\lambda| > 2 \|a\|$.
	Indeed, if $|\lambda| > 2 \|a\|$ then $\| \lambda^{-1} a \| < \frac{1}{2}$, and in particular $1 - \lambda^{-1} a \in \inv(\algebra{A})$ and
	\begin{equation*}
		\left\| (1 - \lambda^{-1} a)^{-1} - 1 \right\| = \left\| \sum_{n = 1}^{\infty} (\lambda^{-1} a)^{-1} \right\| \leq \sum_{n = 1}^{\infty} \|\lambda^{-1} a\|^n = \frac{\| \lambda^{-1} a \|}{1 - \|\lambda^{-1} a\|} \leq 2 \|\lambda^{-1} a\| < 1\text{.}
	\end{equation*}
	From here we deduce that $\|(1 - \lambda^{-1}a)^{-1}\| < 2$ and thus
	\begin{equation*}
		\|(a - \lambda)^{-1}\| < \|\lambda^{-1} (\lambda^{-1}a - 1)^{-1}\| = \frac{\|(1 - \lambda^{-1} a)^{-1}\|}{|\lambda|} < \frac{2}{\lambda} < \frac{1}{\|\lambda\|}\text{.}
	\end{equation*}

	So $R_a: \mathds{C} \to \algebra{A}$ is bounded outside $\closure{B(0, 2 \|a\|}$.
	Since $R_a$ is continuous, it is bounded on $\mathds{C} \to \algebra{A}$.
	Let $\phi \in \algebra{A}^*$ be a bounded linear functional in $\mathcal{L}(\algebra{A}, \mathds{C})$.
	Thus, $\phi$ is differentiable with $\phi'(a) = \phi$ for all $a \in \algebra{A}$.
	Then $\phi \circ R_a$ is differentiable and bounded, so it is an \enquote{integer} function.
	By Liouville's theorem, $\phi \circ R_a$ is constant.
	Therefore, $\phi \circ R_a(x) = \phi \circ R_a(y)$ for all $x,y \in \algebra{A}$.
	Especially, we have $\phi((a - \lambda)^{-1}) = \phi(a^{-1})$ for all $\phi$.
	Hahn-Banach shows $(a - \lambda)^{-1} = a^{-1}$ for all $\lambda$, proving $a - \lambda = a$ for all $a,\lambda$.
	This is a contradiction.
\end{proof}

\begin{theorem}[Gelfand-Mazur]
	If $\algebra{A}$ is a unital Banach algebra and every $a \neq 0$ admits an inverse ($\algebra{A}$ is a field), then $\algebra{A} = \mathds{C} \cdot 1$.
\end{theorem}

\begin{proof}
	By the assumption, $\inv(\algebra{A}) = \algebra{A} \setminus \{0\}$.
	By the previous theorem, if $a \in \algebra{A}$ there exists some $\lambda \in \sigma(a)$, so $a - \lambda \notin \inv(\algebra{A})$, so $a - \lambda = 0$ and thus $a = \lambda \cdot 1$.
\end{proof}

\begin{corollary}
	Let $\mathds{C}(X) = \left\{ \frac{p(x)}{q(x)} \mid p(x), q(x) \in \mathds{Q}[X] \right\}$ is a field, but it cannot be turned into a Banach algebra.
\end{corollary}

\begin{theorem}[Adjointing units - unitization of algebras]
	Let $\algebra{A}$ be any algebra.
	Consider $\tilde{ \algebra{A}} = \algebra{A} \oplus \mathds{C}$ as a vector space.
	We write elements of $\tilde{\algebra{A}}$ as $a + \lambda \cdot 1 \coloneq (a, \lambda)$. Think of $a = (a,0)$ and $\lambda = (a, \lambda)$.
	Define
	\begin{align*}
		(a + \lambda 1) (b + \lambda' 1) & = (ab + \lambda' a + \lambda b) + \lambda \cdot \lambda'\text{.}
	\end{align*}
	Ten (exercise $\tilde{\algebra{A})}$ becomes a unital algebra with $1_\algebra{A} = 1 = (0,1)$.

	Notice that $\algebra{A}$ is an ideal in $\tilde{\algebra{A}}$.

	Moreover, we get a short exact sequence
	\begin{equation*}
		0 \to \algebra{A} \hookrightarrow \tilde{\algebra{A}} \to \mathds{C} \to 0
	\end{equation*}
	so $1 + \lambda \mapsto \lambda$.

	If $\algebra{A}$ is a normed algebra, then $\tilde{\algebra{A}}$ is normed by $\|a + \lambda \cdot 1\| \coloneq \|a\| + |\lambda|$

	If $\algebra{A}$ is Banach and closed, then so is $\tilde{\algebra{A}}$.

	If $\algebra{A}$ is a $^*$-algebra, then so is $\tilde{\algebra{A}}$ with $(a + \lambda 1)^*$.

	If $\algebra{A}$ is a (Banach) normed $^*$-algebra, then so is $\tilde A$.

	If $\algebra{A}$ is a $C^*$-algebra, in general the norm given above is not a Norm on $\algebra{A}$, but $\|a + \lambda \cdot 1\| \coloneq \sup_{b \in \algebra{A}, b \in \algebra{B}, b \leq 1} \|a b + \lambda b\|$ is.
\end{theorem}

\begin{exercise}\label{exe:AplusC}
	If $\algebra{A}$ is already unital, then $\tilde A \simeq A \oplus \mathds{C}$ as algebras by $a + \lambda \cdot 1 \mapsto (a + \lambda 1_\algebra{A}, -\lambda) $.
\end{exercise}

% \begin{definition}
% 	If $\algebra{A}$ is any algebra, then $\sigma_A(a) \coloneq \sigma_{\tilde A}(a)  = \sigma_{\tilde A}(a,0)$.
% 	Thus, we can identify $A \hookrightarrow \tilde A$.
% \end{definition}

% \begin{corollary}~
% 	\begin{enumerate}
% 		\item If $A$ is any algebra, then we always have $0 \in \sigma_A(a)$ because if $a \in A$ then $a \notin \inv(\tilde{\algebra{A}})$ as $A \trianglelefteq \tilde{\algebra{A}}$.
% 		\item 

% 		Observation: If $\algebra{A}$ is unital, then $\tilde{ \algebra{A}} = \{a + \lambda 1 \mid a \in \algebra{A}, \lambda \in \mathds{C}\} \simeq \algebra{A} \oplus \mathds{C}$ as algebras via $a + \lambda 1 \mapsto (a + \lambda 1_\algebra{A}, \lambda)$.
% 		For $p = 1_\algebra{A} \in Z(\tilde{\algebra{A}})$, we have $\tilde{\algebra{A}} \simeq p \tilde{ \algebra{A}} \oplus (1-p) \tilde{\algebra{A}}$.
% 		This yields $(a + \lambda 1)(1 - 1_\algebra{A}) = a - a + \lambda ( 1 - 1_\algebra{A}) = \lambda q$ for $q \coloneq 1 - 1_\algebra{A}$.
% 		Thus, we have $\sigma_{\tilde{\algebra{A}}} = \sigma_\algebra{A}(a) \cup \{0\}$.
		
% 		      % \item If $\algebra{A}$ is \textbf{not} already unital and $a \in \algebra{A}$, then $\sigma_A(a) = \sigma_{\tilde A}(a)$.
% 		      % From \ref{exe:AplusC} we know that $a - \lambda \cdot  1 \in \inv(\tilde{\algebra{A}})$ if and only if $a - \lambda 1_\algebra{A} \in \inv(\algebra{A})$ and $\lambda (1 - 1_\algebra{A})  \in \inv(\mathds{C})$.
% 		      % Then $\rho_\algebra{A} = \{\lambda \in \rho_{\tilde{\algebra{A}}} \mid \lambda \neq 0\}$.
% 		      % Therefore, $\sigma_\algebra{A}(a) = \{0\} \cup \sigma_{\tilde{\algebra{A}}}(a)$.
% 	\end{enumerate}
% \end{corollary}

\begin{definition}
	Re-Definition: If $\algebra{A}$ is non-unital, then $\tilde{A} + \mathds{C} \cdot 1$ is a $(*-)$Banach algebra, and we define $\sigma_A(a) \coloneq \sigma_{\tilde{\algebra{A}}}(a)$.
\end{definition}

Observation: If $\algebra{A}$ is already unital, then for $\tilde{A} \simeq \algebra{A} \oplus \mathds{C}$ we have $\sigma_{\tilde{\algebra{A}}}(a) = \sigma_\algebra{A}(a) \cup \{ 0\}$.

\begin{remark}
	If $\algebra{A}$ is a $C^*$-algebra, then $\tilde{\algebra{A}}$ is a $C^*$-algebra.
	\begin{enumerate}
		\item If $ \algebra{A}$ is unital, then $\tilde{\algebra{A}} \simeq \algebra{A} \oplus \mathds{C}$ and $\|a + \lambda \cdot 1\| = \max\{\|a + \lambda \cdot 1\|, |\lambda|\}$.
		\item If $\algebra{A}$ is not unital, then $\|a + \lambda \cdot 1\| = \sup_{\|b\| \leq 1} \|ab + \lambda b\|$.
	\end{enumerate}
\end{remark}

\section{Spectral Radius}

\begin{definition}
	Let $\algebra{A}$ be an algebra. Given $a \in \algebra{A}$, we define:
	\begin{equation*}
		\pi(a) \coloneq \sup \{ |\lambda| \mid \lambda \in \sigma_\algebra{A}(a) \}
	\end{equation*}
	as the \textbf{spectral radius} of $a$ if $\emptyset \neq \sigma_\algebra{A}(a)$ is bounded (e.g. if $\algebra{A}$ is Banach).
\end{definition}

Observation: In a Banach algebra, we have $0 \leq \pi(a) \leq \|a\|$.

\begin{example}~
	\begin{enumerate}
		\item Let $f \in \algebra{A} = C_0(X)$ using $\sigma_A(f) = \closure{f(X)}$. Thus,
		\begin{equation*}
			\pi(f) = \sup \{ |\lambda| \mid \lambda \in \closure{f(X)} = \sup_{x \in X} |f(x)| = \|f\|_{C_0(X)}
 		\end{equation*}
		\item Let $\algebra{A} = M_2(\mathds{C})$ and $a = \qmatrix{0 & 1 \\ 0 & 0}$. Then $\sigma_\algebra{A} = \{0\}$ and $\pi(a) = 0$, but $\|a\| = 1 \neq 0$.
	\end{enumerate}
\end{example}

\begin{theorem}[Beurling-Gelfand]
	Let $\algebra{A}$ be a Banach algebra, then
	\begin{equation*}
		\pi(a) = \inf_{n \in \mathds{N}} \|a^n\|^{\frac{1}{n}} = \lim_{n \to \infty} \|a^n\|^{\frac{1}{n}}
	\end{equation*}
\end{theorem}

\begin{proof}
	We may assume $\algebra{A}$ is unital (otherwise we consider $\tilde{\algebra{A}}$). If $\lambda \in \sigma(a)$, then
	\begin{equation*}
		\lambda^n \in \sigma(a^n) \then |\lambda^n| \leq \| a^n \| \then |\lambda| \leq \|a\|^{\frac{1}{n}} \quad \forall_{n \in \mathds{N}}
	\end{equation*}
	and therefore
	\begin{equation*}
		\pi(a) \leq \inf_{n \in \mathds{N}} \|a^n\|^{\frac{1}{n}} \leq \liminf_{n \to \infty} \|a^n\|^{\frac{1}{n}}\text{.}
	\end{equation*}
	We prove now that $\limsup_{n \to \infty} \|a^n\|^{\frac{1}{n}} \leq \pi(a)$. Set $\Delta \coloneq B\left(0, \frac{1}{\pi(a)}\right)$. Where per convention we set $\frac{1}{\pi(a)} = \infty$ if $\pi(a) = 0$. If $\lambda \in \Delta$, then $1 - \lambda a \in \inv(\algebra{A})$ (because $|\lambda| < \frac{1}{\pi(a)}$ implies $|\lambda^{-1}| > \pi(a)$ and therefore $\lambda^{-1} \notin \sigma(a) \then \lambda^{-1} - a \in \inv{A} \then 1 - \lambda a \in \inv(A)$).

	Now fix $\phi \in \algebra{A}^*$. Then $f: \Delta \to \mathds{C}, \lambda \mapsto \phi((1 - \lambda a)^{-1})$ is analytic, so it can be written as
	\begin{equation*}
		f(x) = \sum_{n = 0}^{\infty} a_n \lambda^n, a_n = \frac{f^{(n)}(0)}{n!} \in \mathds{C}, \lambda \in \Delta\text{.}
	\end{equation*}
	On the other hand, if 
	\begin{equation*}
		|\lambda| < \frac{1}{\|a\|} \leq \frac{1}{\pi(a)}
	\end{equation*}
	then $\|\lambda a \| < 1$, so
	\begin{equation*}
		(1 - \lambda a)^{-1} = \sum_{n = 0}^{\infty} \lambda^n a^n \then f(\lambda) = \phi((1 - \lambda)^{-1}) = \sum_{k = 0}^{\infty} \phi(a^n) \lambda^n
	\end{equation*}
	for $|\lambda| < \frac{1}{\|\lambda\|}$.

	By uniqueness of the Taylor series expansion, it follows that
	\begin{equation*}
		a_n = \phi(a^n)	 \forall_{n \in \mathds{N}}\text{.}
	\end{equation*}
	In particular, $(\phi(a^n) \lambda^n)$ converges to zero for all $\lambda \in \Delta$ and thus $(\phi(a^n) \lambda^n)$ is bounded for all $\lambda \in \Delta$.

	From the principle of uniform convergence, it follows that $(a^n \lambda^n)$ is bounded. So there exists an $M = M_\lambda$ such that
	\begin{align*}
		& \|\lambda^n a^n\| \leq M \forall_{n \in \mathds{N}}\\
		\then & \| \lambda^n \|^{\frac{1}{n}} \leq \frac{M^{\frac{1}{n}}}{|\lambda|} \forall_{n \in \mathds{N}}, \forall_{\lambda \in \Delta, \lambda \neq 0}\\
		\then & \limsup_{n \to \infty} \|a^n\|^{\frac{1}{n}} \leq \frac{1}{\lambda} \forall_{\lambda \in \Delta \text{ i.e. } |\lambda| < \frac{1}{\pi(a)}}
	\end{align*}
	Letting $\lambda < \frac{1}{\pi(a)}$ yields $\limsup_{n \to \infty} \|a^n\|^{\frac{1}{n}} \leq \pi(a)$.
\end{proof}

\begin{example}
	Let $A = C^1([0,1]) = \{ I \in C[0,1] \mid \exists_{f'(t)} \forall_{t \in [0,1]}, t \mapsto f'(t) \text{ continuous} \}$ with $\|f\| = \|f\|_\infty + \|f'\|_\infty$. 

	Then $\algebra{A}$ is unital, commutative and a Banach algebra. Consider $x \in \algebra{A}$, $x(t) = t$. We have $x^n(t) = t^n$ and
	\begin{align*}
		\|x^n\| &= \sup_{t \in [0,1]} |t^n| + \sup_{t \in [0,1]} |n t^{n-1}| = 1 + n\\
		\pi(x) &= \lim_{n \to \infty} (1+n)^\frac{1}{n} = 1\\
		\| x \| &= 2
	\end{align*}
	Observation: $\sigma(x) = \image(x) = [0,1]$.
\end{example}

\begin{theorem}
	Let $\algebra{B} \nsubseteq \algebra{A}$ be an inclusion of unital Banach algebras with $1 = 1_\algebra{A} = 1_\algebra{B}$. Then $\sigma_\algebra{A}(b) \subseteq \sigma_\algebra{B}(b)$ for all $b \in \algebra{B}$ and the inclusion may be proper. If $\sigma_\algebra{A}(b)$ is simply connected (not holes), then $\sigma_\algebra{A}(b) = \sigma_\algebra{B}(b)$.\\
	The holes of a compact subset $K \subseteq \mathds{C}$ are the bounded connected components of $\mathds{C} \setminus K$. So saying that $K$ has no holes means that $\mathds{C} \setminus K$ is connected.
\end{theorem}

\begin{proof}
	See Murphy, 1.2.8.
\end{proof}

\begin{example}
	Let $\algebra{B} \coloneq A(\mathds{D}) = \{f \in C(\mathds{D}) \mid f \text{ analytic on } \mathds{D}^\interior\}$ and $\algebra{A} = C(\mathds{S}^1)$.
	Then we have an embedding by $\iota: \algebra{B} \hookrightarrow \algebra{A}, f \mapsto f|_{\mathds{S}^1}$.

	By the principle of maximum modules, $\iota$ is an embedding of (unital) Banach algebras.
	Consider: $f(z) = z$ for $z \in \mathds{D}$.
	(Observation: $\closure{Alg}(1,z) = A(\mathds{D})$)
	Then:
	\begin{equation*}
		\sigma_{A(\mathds{D})}(f) = f(\mathds{D}) = \mathds{D}
	\end{equation*}
	and $\sigma_{C(\mathds{S}^1)}(f|_{\mathds{S}^1}) = \mathds{S}^1$.
\end{example}

\begin{definition}[Exponentials]
	Let $\algebra{A}$ be a unital Banach algebra, given $a \in \algebra{A}$ we define
	\begin{equation*}
		e^a = \exp(a) = \sum_{n = 0}^{\infty} \frac{a^n}{n!}
	\end{equation*}
	Note $\left\| \frac{a^n}{n!} \right\| \leq \frac{\|a\|^n}{n!}$, so the series converges and $\| \exp(a) \| \leq \exp(\|a\|)$.
\end{definition}

\begin{theorem}~
	\begin{enumerate}
		\item Let $\algebra{A}$ be a unital Banach algebra. If $a \in \algebra{A}$, then $f: \mathds{R} \to \algebra{A}, t \mapsto \exp(ta)$ is the unique solution of 
		\begin{equation*}
			\left\{ \begin{matrix}
				f'(t) &= a f(t)\\
				f(0) &= 1
			\end{matrix} \right.
		\end{equation*}
		\item $e^a \in \inv(\algebra{A})$ and $(e^a)^{-1} = e^{-a}$.
		\item If $a,b \in \algebra{A}$ then $e^{a + b} = e^a \cdot e^b$ (here some commutativity is neccessary).
	\end{enumerate}
\end{theorem}

\begin{proof}
	See Murphy, 1.2.9.
\end{proof}


\section{Gelfand Representation for commutative Banach algebras}

\underline{Idea}: Given a commutative algebra $\algebra{A}$, we want to represent $\algebra{A}$ by a homomorphism $\phi: \algebra{A} \to C_0(X)$ for $X$ some locally compact Hausdorff space. We hope that $\phi$ is injective, or even isometric, on an isomorphism. But what is $X$, and what is $\phi$?

Notice that, if $\algebra{A} = C_0(X)$ already, then for each $x \in X$ we get a character $\mathrm{ev}_x: \algebra{A} \to \mathds{C}, f \mapsto f(x)$.

\begin{definition}
	Given an algebra $\algebra{A}$, we define 
	\begin{equation*}
		\hat{\algebra{A}} = \Omega(\algebra{A}) \coloneq \{\chi: \algebra{A} \to \mathds{C} \mid \chi \text{ non-zero homomorphism}\}\text{.}
	\end{equation*}
\end{definition}

\begin{example}~
	\begin{enumerate}
		\item For $\algebra{A} = C_0(X)$ we get a map
		\begin{equation*}
			X \to \Omega(\algebra{A}), x \mapsto \mathrm{ev}_x
		\end{equation*}
		that is a bijection. After we give $\Omega(\algebra{A})$ an appropriate topology, it will also be a homomorphism.
		\item Let $\algebra{A} = M_2(\mathds{C})$ (or any $M_n(\mathds{C})$). This is a simple algebra, so non-zero homomorphisms $\chi: \algebra{X} \to \mathds{C}$ do not exist (same for any $\algebra{A}$ with dimension $>1$).
		
		So in this case we have $\Omega(\algebra{A}) = \emptyset$. This can also happen in commutative algebras. 
		\item Consider
		\begin{equation*}
			\algebra{A} = \left\{ \qmatrix{0 & \lambda \\ 0 & 0} \mid \lambda \in \mathds{C}  \right\}
		\end{equation*}
		Then for all $a \in \algebra{A}$ we have $a^2 = 0$, so if $\chi: \algebra{A} \to \mathds{C}$ is an homomorphism, then $\chi(a)^2 = \chi(a^2) = 0$, so $\chi(a) = 0$ for all $a \in \algebra{A}$.
		So again, $\Omega(\algebra{A}) = \emptyset$ (and $\algebra{A}$ is commutative with $\dim \algebra{A} = 1$).
	\end{enumerate}
\end{example}

Question: Given an abstract algebra $\algebra{A}$ how do we possibly find its characters? 

Idea: Assume that $\ring{I} \triangleleft \algebra{A}$ is a maximal ideal and $\algebra{A}$ is a unital Banach algebra.
Then $\algebra{A} / \ring{I} \simeq \mathds{C}$ and $\chi \in \Omega(\algebra{A})$. 

\begin{theorem}
	Let $\algebra{A}$ be a unital non-zero Banach algebra.
	If $\chi \in \Omega(\algebra{A})$ then $\| \chi \| = \sup_{\|a\| = 1} |\chi(a)| = 1$ and $\ker(\chi) \triangleleft \algebra{A}$.
	So $\chi \in \algebra{A}^*$ (the topological dual of $\Omega(\algebra{A}) \subseteq D_{\algebra{A}^*}(0,1)$).

	Moreover, if $\algebra{A}$ is a unital Banach commutative algebra, then $\Omega(\algebra{A}) \ni \chi \mapsto \ker(\chi) \triangleleft \algebra{A}$ is a bijection between of characters of $\algebra{A}$ and maximal ideals of $\algebra{A}$.
\end{theorem}

\begin{proof}
	If $a \in \algebra{A}$ and $\chi$ a character, then $\chi(a) \in \sigma(\algebra{A})$, because $\chi(a - \chi(a) \cdot 1) = \chi(a) - \chi(a) \cdot \chi(1) = 0$, so $a - \chi(a) \cdot 1 \in \ker(\chi) \triangleleft \algebra{A}$ and thus $a - \chi(a) \cdot 1 \notin \inv(\algebra{A})$.

	Therefore: $|\chi(a)| \leq \pi(a) \leq \|a\|$. So $\|\chi\| \leq 1$. Since $\chi(1) = 1$ and $\|1\| = 1$ we have $\|\chi\| = 1$.

	Now, apply linear algebra. Then $\ker (\chi)$ is a maximal proper subspace, in particular a maximal ideal.
	And $\ker(\chi)$ is closed, because $\chi$ is continuous.
	Now assume that $\algebra{A}$ is commutative (in addition to unital and Banach).
	Then we have the mapping 
	\begin{equation*}
		\phi: \Omega(\algebra{A}) \to \mathrm{MaxIdeals}(\algebra{A}), \chi \to \ker(\chi)\text{.}
	\end{equation*}
	\begin{itemize}
		\item $\phi$ is injective.
		If $\ker(\chi_1) = \ker(\chi_2)$ for $\chi_1, \chi_2 \in \algebra{A}$, then for every $a \in \algebra{A}$ we have $a - \chi_1(a) \cdot 1 \in \ker(\chi_1) =\ker(\chi_2)$.
		Thus, $\chi_2(a = \chi_1(a) \cdot 1) = 0$ and therefore $\chi_2(a) = \chi_1(a)$ for every $\algebra{A}$. 
		\item $\phi$ is surjective.
		Take $\ring{I} \triangleleft  \algebra{A}$ a maximal ideal.
		Then $\ring{I} = \closure{\ring{I}}$ because $\closure{\ring{I}} \neq \algebra{A}$, otherwise $1 \in \closure{\ring{I}}$ and since $\inv(\algebra{A})$ is open in $\algebra{A}$, we get $\ring{I} \cap \inv(\algebra{A}) \neq \emptyset$.
		But then we have an invertible element in the ideal $\ring{I}$ already, but this implies the contradiction $\ring{I} = \algebra{A}$. 
		Therefore, $\algebra{A} / \ring{I}$ is a commutative, unital Banach algebra which is simple ($\ring{I}$ is maximal).
		\begin{quote}
			Exercise: If $\ring{I} \triangleleft \algebra{A}$, then $\algebra{A} / \ring{I}$ is field if and only if there exists no $\ring{J} \triangleleft \algebra{A}$ such that $\ring{I} \triangleleft \ring{J}$.
		\end{quote}
		Thus, $\algebra{A} / \ring{I}$ is a field and $\algebra{A} / \ring{I} \simeq \mathds{C}$.
		Then the composition
		\begin{equation*}
			\algebra{A} \xrightarrow{q} \algebra{A} / \ring{I} \simeq \mathds{C}
		\end{equation*}
		is a character with $ \ker(\chi) = \ring{I}$.
	\end{itemize}
\end{proof}

\begin{exercise}
	An application of Zorn's Lemma. Show that every ideal $I \triangleleft \algebra{A}$ in a unital algebra $\algebra{A}$ is contained in a maximal ideal.
\end{exercise}

In particular, we can apply this to $\ring{I} = 0$ in $\algebra{A} \neq 0$ (with $\algebra{A}$ is unital and commutative) and thus $\Omega(\algebra{A}) \neq \emptyset$.

\subsection*{Topology on $\Omega(\algebra{A})$}

We have for $\algebra{A}$ a Banach algebra. We can add a unit to receive $\tilde{\algebra{A}}$, which is a Banach algebra.

Observe: If $\chi \in \Omega(\algebra{A})$, then there exists a unique $\tilde{X} \in \Omega(\tilde{\algebra{A}})$ via $\tilde{X}(a + \lambda \cdot 1) = \chi(a) + \lambda$. Thus, $\|\chi\| \leq \|\tilde{X}\| = 1$ (Note that it may still be smaller than $1$. See exercises 2023-05-09).

In any case, 
\begin{equation*}
	\Omega(\algebra{A}) = D_{\algebra{A}^*}(0,1) = \phi\{\phi \in \algebra{A}^* = \{ \phi \in \algebra{A}^* \mid \| \phi \| \leq 1 \}
\end{equation*}
and $\algebra{A}^*$ corries the weak $^*$-topology. $\phi_i \to \phi$ in $\algebra{^*}$. 

\begin{definition}
	Given a Banach algebra $\algebra{A}$, we endow $\Omega(\algebra{A})$ with the weak $^*$-topology and call thus this the \textbf{Gelfand spectrum} of $\algebra{A}$.
\end{definition}

\begin{proposition}
	$\Omega(\algebra{A})$ is a locally compact Hausdorff space. If $\algebra{A}$ is unital, then $\Omega(\algebra{(A)})$ is compact.
\end{proposition}

\begin{proof}
	By Banach-Alaoglu-Theorem, $D_{\algebra{A}^*}(0,1)$ is compact and Hausdorff with the weak $^*$-topology.
	Let
	\begin{align*}
		S &\coloneq \{\chi: A \to \mathds{C} \mid \chi \text{ hom.} \}\\
		&= \Omega(\algebra{A}) \cup \{0\}
	\end{align*}
	Then $S \subseteq D_{\algebra{A}^*}(0,1)$.
	So $\chi(ab) = \lim_{i \to \infty} K_i = \lim_{i \to \infty} \chi_i(a) \chi_i(b) = \chi(a) \chi(b)$ and therefore $x \in S$.
	Thus, $S$ is a compact Hausdorff space and $\Omega(\algebra{A}) = S \setminus \{0\}$ is relatively compact.

	If $\algebra{A}$ is unital, then $\Omega(\algebra{A}) \subseteq D_{\algebra{A}^*}(0,1)$ is closed. Then we have $(X_i) \subseteq \Omega(\algebra{A})$ and $X_i \to X \in \algebra{A}^*$ and thus $X \in S = \hom(\algebra{A}, \mathds{C})$.
\end{proof}

Observation: Given a Banach algebra $\algebra{A}$, we have a homeomorphism 
\begin{equation*}
	\Omega(\tilde{\algebra{A}}) \to \Omega(\algebra{A}) \sqcup \{ \chi_\infty\}, \phi \mapsto \left\{  \begin{matrix}
		\phi|_\algebra{A} & \phi|_{\algebra{A}} \neq 0\\
		\chi_\infty & \phi|_\algebra{A} = 0
	\end{matrix}\right. \text{,}
\end{equation*}
where $\chi_\infty(a + \lambda \cdot 1) = \lambda$. Thus, $\Omega(\algebra{A}) \sqcup \{\chi_\infty\}$ is already the unitization of $\Omega(\algebra{A})$. 

\begin{theorem}
	Let $\algebra{A}$ be a Banach algebra. Then for every $a \in \algebra{A}$.
	\begin{equation*}
		\{ \chi (a) \mid \chi \in \Omega(\algebra{A}) \} \subseteq \sigma(a)
	\end{equation*}
	If $\algebra{A}$ is commutative, then
	\begin{itemize}
		\item $\{ \chi(a) \mid \chi \in \Omega(\algebra{A})\} = \sigma(a)$ in case $\algebra{A}$ is unital.
		\item $\{ \chi(a) \mid \chi \in \Omega(\algebra{A})\} \cup \{0\} = \sigma_{\algebra{A}}(a)$.
	\end{itemize}
\end{theorem}

\begin{proof}~
	\begin{itemize}
		\item $\algebra{A}$ is unital and $a \in \algebra{A}$. $\chi(a - \chi(a) \cdot 1) = 0$, so $\chi(a) \in \sigma(a)$, so $\{ \chi(a) \mid x \in \Omega(a) \}  \subseteq \sigma(a)$.
		
		Now if $\lambda \in \sigma(a)$, consider $\ring{I} \coloneq (a - \lambda \cdot 1) \algebra{A} \triangleleft \algebra{A}$ if $\algebra{A}$ is commutative.
		By Zorns Lemma, we get $I \subseteq J \triangleleft \algebra{A}$ with $J = \ker(\chi)$ for some $\chi \in \Omega(\algebra{A})$. Thus we have $a - \lambda \cdot 1 \in \ring{I} \subseteq J = \ker(\chi)$ so $\chi(a) = \lambda$.

		\item $\algebra{A}$ is not unital. Consider $\tilde{\algebra{A}}$. By the first part,
		\begin{equation*}
			\sigma_\algebra{A}(a) = \sigma_{\tilde{\algebra{A}}}(a) \supseteq \{\chi(a) \mid \chi \in \Omega(\tilde{\algebra{A}}) \} = \{\chi(a) \mid \chi \in \Omega(\algebra{A}) \} \cup \{0\}
		\end{equation*}
		If $\algebra{A}$ is commutative, by the first part again: 
		\begin{equation*}
			\sigma_\algebra{A}(a) = \sigma_{\tilde{\algebra{A}}}(a) = \{\chi(a) \mid \chi \in \Omega(\tilde{\algebra{A}})\} = \{\chi(a) \mid \chi \in \Omega(\algebra{A}) \} \cup \{0\}
		\end{equation*}
	\end{itemize}
\end{proof}

















\end{document}




























































