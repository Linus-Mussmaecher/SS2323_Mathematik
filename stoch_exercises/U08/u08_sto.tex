\documentclass[a4paper]{article}

% --- DATA ---

\def\lecture{Stochastik 2}
\def\authors{Linus Mußmächer}
\def\sheetNumber{08}
%\def\sumPoints{30} 

% --- PREAMBLE ---

\usepackage[german]{babel}	% language specific quotation marks etc.
% === USAGE ===

% when using this preamble, setup your environment variables like this beforehand:


% \title{Stochastik 2}  %Title of exercise 
% \def\lecture{Stochastik 2}
% \def\authors{Linus Mußmächer}
% \def\sheetNumber{02}
% \def\sumPoints{30}      % maximum number of points (leave undefined)

% then use one of these commands (german or english) to print the header:

% \makeexheaderger

% and finally use subsections for your subtasks - they will be numbered as <sheetNumber><task number> by themselves

% if you have an exercise as an external .pdf, use \includetask to include it and increase the task counter


% --- OTHER ---

\usepackage{booktabs}       % professional-quality tables
\usepackage[table]{xcolor}	% color
\usepackage{pdfpages}		% to include entire pdf pages in appendix etc.
\usepackage{enumitem}		% better custom enumerations
\setlist[enumerate, 1]{label=(\roman*)}
\usepackage{etoolbox}		% toolbox for command modification

% --- FONTS & TYPESETTING ---

\usepackage[utf8]{inputenc} % allow utf-8 input
\usepackage[T1]{fontenc}    % use 8-bit T1 fonts
\usepackage{dsfont}			% font with double lines for sets
\usepackage[german,ruled,vlined,linesnumbered,commentsnumbered,algoruled]
{algorithm2e} 				%pseudo code
\usepackage{listings}		%java code
\usepackage{csquotes}

% --- URLS ---

\usepackage[colorlinks=true, linkcolor=black, citecolor=blue, urlcolor=blue]{hyperref}   	% hyperlinks
\usepackage{url}            % simple URL typesetting

% --- MATH SYMBOLS ---

\usepackage{amsmath,amssymb}% more math symbols
\usepackage{amsfonts}       % blackboard math symbols
\usepackage{latexsym}		% more math symbols
\usepackage{chngcntr}		% more math symbols
\usepackage{mathrsfs}		% math-fonts
\usepackage{mathtools}		% more math symbols
\usepackage{nchairx}		% Waldmann package for general math symbols

% --- GRAPHICS & CAPTIONS ----

\usepackage{graphicx}		% including images
\graphicspath{ {./figs/} }
\usepackage{subcaption}		% custom caption formatting
\DeclareCaptionLabelFormat{custom}{ \textbf{#1 #2}}
\captionsetup{format=hang}
\captionsetup{width=0.9\textwidth,labelformat=custom}
\usepackage{pdfpages}		% to include entire pdf pages in appendix etc.

% --- FORMAT ---

\usepackage[a4paper]{geometry} % a4 paper
\usepackage{setspace}		% spacing
\usepackage{titlesec}
\allowdisplaybreaks			% allow page breaks within math environments

% --- CUSTOM COMMANDS ---
%Logic
\newcommand{\then}{\Rightarrow}
\newcommand{\since}{\Leftarrow}
\renewcommand{\iff}{\ensuremath{\Leftrightarrow}}

%pretty epsilon
\let\oldepsilon\epsilon
\let\epsilon\varepsilon
\let\varepsilon\oldepsilon
%pretty phi
\let\oldphi\phi
\let\phi\varphi
\let\varphi\oldphi

\newcommand{\includetask}[2][pages=-]{
    \includepdf[#1]{#2}
    \addtocounter{subsection}{1}
}

% set-up for exercise specific stuff
\ifdef{\sheetNumber}{
    \setcounter{section}{\sheetNumber}
}{}

\usepackage{titling}
\newcommand{\makeexheaderger}{
    \begin{doublespace}
        \begin{center}
            \textbf{\Large{Übungsblatt \sheetNumber}}\\
            \textbf{\Large\lecture}\\
            Abgabe von: \textbf{\authors}\\
            \today
        \end{center}
        \ifdef {\sumPoints}
        {
            \hfill  \large Punkte: $\boxed{\qquad  /\; \sumPoints}$\\
        }{}
    \end{doublespace}
}

\newcommand{\makeexheadereng}{
    \begin{doublespace}
        \begin{center}
            \textbf{\Large{Exercise Sheet \sheetNumber}}\\
            \textbf{\Large\lecture}\\
            Abgabe von: \textbf{\authors}\\
            \today
        \end{center}
        \ifdef {\sumPoints}
        {
            \hfill  \large Points: $\boxed{\qquad  /\; \sumPoints}$\\
        }{}
    \end{doublespace}
}

% --- DOCUMENT ---

\begin{document}

\makeexheader

\subsection{Zentralübung}

$(X_n)$ konvergiert $\mathds{P}$ stochastisch gegen $X = 0$.
Sei dazu $\epsilon > 0$ (und o.B.d.A. $< 1$) beliebig.
Dann existiert ein $N \in \mathds{N}$ mit $\frac{1}{N} < \epsilon$ und es gilt $|X_n - X| \geq \epsilon \iff X_n = n$ für alle $n \geq N$.
Somit folgt $\lim_{n \to \infty} \mathds{P}(|X_n - X| \geq \epsilon) = \lim_{n \to \infty} \mathds{P}(X_n = n) = \lim_{n \to \infty} \frac{1}{n} = 0$.

$(X_n)$ konvergiert nicht in $L_p$, denn wäre sie $L_p$-konvergent gegen eine Grenzvariable $\tilde{X}$, dann wäre $(X_n)$ auch stochastisch konvergent gegen $\tilde{X}$ und aufgrund der Eindeutigkeit des Grenzwertes folgt $X = \tilde{X}$.
Wir zeigen daher, dass $(X_n)$ nicht in $L_p$ gegen $X$ konvergieren kann. Es ist $X = 0$, also $|X_n - X|^p = X_n^p$. Dann gilt
\begin{equation*}
    \mathds{E}[|X_n - X|^p] = \mathds{E}[X_n^p] = (1 - \frac{1}{n}) \cdot 0^p + \frac{1}{n} \cdot n^p = n^{p-1}
\end{equation*}
wobei $n^{p-1} = 1 \to 1$ für $p =1$ und $n^{p-1} \to \infty$ für $p > 1$ gilt.
Die $L_p$-Konvergenz ist also für kein $p$ gegeben.

Die fast sichere Konvergenz kann nicht entschieden werden.
Wie oben muss $(X_n)$, falls $\mathds{P}$-fast sicher konvergent, gegen $X$ konvergieren.
 Sind die $X_n$ unabhängig verteilt, so gilt für beliebiges $\epsilon > 0$ (und o.B.d.A. $< 1$)
\begin{equation*}
    \mathds{P}(\bigcup_{k = n}^\infty \{|X_k - X| \geq \epsilon\}) =  \mathds{P}(\bigcup_{k = n}^\infty \{ X_k = k\} ) = \sum{k = n}^\infty \mathds{P}(\{ X_k = k\} ) = \sum_{k = n}^{\infty} \frac{1}{k}
\end{equation*}
und diese Summe kann für $k \to \infty$ nicht gegen $0$ konvergieren, da dann die harmonische Reihe beschränkt wäre.
Also gilt für unabhängige $X_n$, dass $X_n$ nicht fast sicher konvergiert.

Für abhängige $X_n$ lassen sich allerdings fast sicher konvergente Beispiele formulieren. 
Wir wollen dazu $X_n = 0 \then X_{n+1} =0$ festlegen und im Fall $X_n = n$ verlangen, dass $X_{n+1} = n+1$ mit bedingter Wahrscheinlichkeit $\frac{n}{n+1}$ und $X_{n+1} = 0$ mit bedingter Wahrscheinlichkeit $\frac{1}{n+1}$.
Dann erfüllt die Folge $(X_n)$ alle Forderungen und es gilt $X_{n+1} \neq 0 \then X_n \neq 0$ für alle $n$, also $\bigcup_{k = n}^\infty \{ X_k = k\} = \bigcup_{k = n}^\infty \{ X_k \neq 0\} \subseteq \{X_n \neq 0\} = \{X_n = n\}$ und somit
\begin{equation*}
    \mathds{P}(\bigcup_{k = n}^\infty \{|X_k - X| \geq \epsilon\}) =  \mathds{P}(\bigcup_{k = n}^\infty \{ X_k = k\} ) \leq \mathds{P}(\{X_n = n\}) = \frac{1}{n} \to 0
\end{equation*}
und die $X_n$ sind fast sicher konvergent.

\subsection{}

$X_n \xrightarrow{\mathds{P}} X$: Für alle $\epsilon, \delta > 0$ existiert ein $n_0 \in \mathds{N}$, sodass für alle $n \geq n_0$ gilt $\mathds{P}(|X_n - X| \geq \epsilon ) < \delta$. Insbesondere für $\epsilon = \delta$!

\subsection{}

\subsection{}

(b) Keine Lebesguedichte, da nur Punktmasse




\end{document}























