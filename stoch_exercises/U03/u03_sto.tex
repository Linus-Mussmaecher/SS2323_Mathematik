\documentclass[a4paper]{article}

% --- DATA ---

\def\lecture{Stochastik 2}
\def\authors{Linus Mußmächer}
\def\sheetNumber{03}
%\def\sumPoints{30} 

% --- PREAMBLE ---

\usepackage[german]{babel}	% language specific quotation marks etc.
% === USAGE ===

% when using this preamble, setup your environment variables like this beforehand:


% \title{Stochastik 2}  %Title of exercise 
% \def\lecture{Stochastik 2}
% \def\authors{Linus Mußmächer}
% \def\sheetNumber{02}
% \def\sumPoints{30}      % maximum number of points (leave undefined)

% then use one of these commands (german or english) to print the header:

% \makeexheaderger

% and finally use subsections for your subtasks - they will be numbered as <sheetNumber><task number> by themselves

% if you have an exercise as an external .pdf, use \includetask to include it and increase the task counter


% --- OTHER ---

\usepackage{booktabs}       % professional-quality tables
\usepackage[table]{xcolor}	% color
\usepackage{pdfpages}		% to include entire pdf pages in appendix etc.
\usepackage{enumitem}		% better custom enumerations
\setlist[enumerate, 1]{label=(\roman*)}
\usepackage{etoolbox}		% toolbox for command modification

% --- FONTS & TYPESETTING ---

\usepackage[utf8]{inputenc} % allow utf-8 input
\usepackage[T1]{fontenc}    % use 8-bit T1 fonts
\usepackage{dsfont}			% font with double lines for sets
\usepackage[german,ruled,vlined,linesnumbered,commentsnumbered,algoruled]
{algorithm2e} 				%pseudo code
\usepackage{listings}		%java code
\usepackage{csquotes}

% --- URLS ---

\usepackage[colorlinks=true, linkcolor=black, citecolor=blue, urlcolor=blue]{hyperref}   	% hyperlinks
\usepackage{url}            % simple URL typesetting

% --- MATH SYMBOLS ---

\usepackage{amsmath,amssymb}% more math symbols
\usepackage{amsfonts}       % blackboard math symbols
\usepackage{latexsym}		% more math symbols
\usepackage{chngcntr}		% more math symbols
\usepackage{mathrsfs}		% math-fonts
\usepackage{mathtools}		% more math symbols
\usepackage{nchairx}		% Waldmann package for general math symbols

% --- GRAPHICS & CAPTIONS ----

\usepackage{graphicx}		% including images
\graphicspath{ {./figs/} }
\usepackage{subcaption}		% custom caption formatting
\DeclareCaptionLabelFormat{custom}{ \textbf{#1 #2}}
\captionsetup{format=hang}
\captionsetup{width=0.9\textwidth,labelformat=custom}
\usepackage{pdfpages}		% to include entire pdf pages in appendix etc.

% --- FORMAT ---

\usepackage[a4paper]{geometry} % a4 paper
\usepackage{setspace}		% spacing
\usepackage{titlesec}
\allowdisplaybreaks			% allow page breaks within math environments

% --- CUSTOM COMMANDS ---
%Logic
\newcommand{\then}{\Rightarrow}
\newcommand{\since}{\Leftarrow}
\renewcommand{\iff}{\ensuremath{\Leftrightarrow}}

%pretty epsilon
\let\oldepsilon\epsilon
\let\epsilon\varepsilon
\let\varepsilon\oldepsilon
%pretty phi
\let\oldphi\phi
\let\phi\varphi
\let\varphi\oldphi

\newcommand{\includetask}[2][pages=-]{
    \includepdf[#1]{#2}
    \addtocounter{subsection}{1}
}

% set-up for exercise specific stuff
\ifdef{\sheetNumber}{
    \setcounter{section}{\sheetNumber}
}{}

\usepackage{titling}
\newcommand{\makeexheaderger}{
    \begin{doublespace}
        \begin{center}
            \textbf{\Large{Übungsblatt \sheetNumber}}\\
            \textbf{\Large\lecture}\\
            Abgabe von: \textbf{\authors}\\
            \today
        \end{center}
        \ifdef {\sumPoints}
        {
            \hfill  \large Punkte: $\boxed{\qquad  /\; \sumPoints}$\\
        }{}
    \end{doublespace}
}

\newcommand{\makeexheadereng}{
    \begin{doublespace}
        \begin{center}
            \textbf{\Large{Exercise Sheet \sheetNumber}}\\
            \textbf{\Large\lecture}\\
            Contributors: \textbf{\authors}\\
            \today
        \end{center}
        \ifdef {\sumPoints}
        {
            \hfill  \large Points: $\boxed{\qquad  /\; \sumPoints}$\\
        }{}
    \end{doublespace}
}

\newcommand{\qmatrix}[1]{\ensuremath{\left(\begin{matrix}#1\end{matrix}\right)}}

% --- DOCUMENT ---

\begin{document}

\makeexheaderger

\subsection{Zentralübung}

\begin{enumerate}
    \item Sei $X_0$ eine Zufallsvariable mit charakteristischer Funktion $\phi$, d.h. $\mathds{E}[\exp(itX)] = \phi(t)$. Sei nun $\{X_k \mid k \in \mathds{N} \}$ eine (abzählbare) Menge an paarweise unabhängigen, zu $X$ gleichverteilten Zufallsvariablen. Weiterhin sei $N$ eine Poisson-verteilte Zufallsvariable mit Parameter $\lambda$, d.h. $\mathds{P}(N = k) = \exp(\lambda) \cdot \frac{\lambda^k}{k!}$ und $\phi_N(t) = \exp(\lambda(\exp(it) - 1))$. $N$ sei weiterhin von allen $X_k$ unabhängig. Dann betrachten wir eine Zufallsvariable mit der folgenden Definition:
    \begin{equation*}
        X = \sum_{k = 1}^{\infty} \mathds{1}(N \geq k) X_k = \sum_{k = 1}^{N} X_k \text{.}
    \end{equation*}
    Dann ist 
    \begin{equation*}
        \mathds{E}[X] = \sum_{n = 0}^{\infty} \left( \mathds{P}(N = n) \sum_{k = 1}^{n} E[X_k] \right)
    \end{equation*}
    sowie 
    \begin{align*}
        \phi_X(t) = \mathds{E}[\exp(i t X)] &\overset{(*)}{=} \sum_{n = 0}^{\infty} \left( \mathds{P}(N = n) \mathds{E} \left[ \exp \left( i t \cdot \sum_{k = 1}^{n} X_i \right) \right]  \right)
        = \sum_{n = 0}^{\infty} \left( \exp(- \lambda) \frac{\lambda^n}{n!} \phi(t)^n  \right)\\
        &= \exp(- \lambda) \sum_{n = 0}^{\infty} \left( \frac{(\lambda \cdot \phi(t))^n}{n!}   \right)
        = \exp(- \lambda) \exp(\lambda \phi(t))\\
        &= \exp(\lambda(\phi(t) - 1))\text{,}
    \end{align*}
    wobei wir in $(*)$ dasselbe Prinzip wie in (ii) anwenden. Damit ist $e^{\lambda(\phi - 1)}$ wieder eine charakteristische Funktion.
    \item Wir berechnen
    \begin{align*}
        \mathds{E}[\exp(it Z)]     
        &= \mathds{P}(Y = 0) \cdot \mathds{E}[\exp(itX_1)] + \mathds{P}(Y = 1) \cdot \mathds{E}[\exp(itX_2)]\\
        &= \alpha \phi_1(t) + (1 - \alpha) \phi_2(t)
    \end{align*}
\end{enumerate}



\end{document}