\documentclass[a4paper]{article}

% --- DATA ---

\def\lecture{Stochastik 2}
\def\authors{Linus Mußmächer}
\def\sheetNumber{06}
%\def\sumPoints{30} 

% --- PREAMBLE ---

\usepackage[german]{babel}	% language specific quotation marks etc.
% === USAGE ===

% when using this preamble, setup your environment variables like this beforehand:


% \title{Stochastik 2}  %Title of exercise 
% \def\lecture{Stochastik 2}
% \def\authors{Linus Mußmächer}
% \def\sheetNumber{02}
% \def\sumPoints{30}      % maximum number of points (leave undefined)

% then use one of these commands (german or english) to print the header:

% \makeexheaderger

% and finally use subsections for your subtasks - they will be numbered as <sheetNumber><task number> by themselves

% if you have an exercise as an external .pdf, use \includetask to include it and increase the task counter


% --- OTHER ---

\usepackage{booktabs}       % professional-quality tables
\usepackage[table]{xcolor}	% color
\usepackage{pdfpages}		% to include entire pdf pages in appendix etc.
\usepackage{enumitem}		% better custom enumerations
\setlist[enumerate, 1]{label=(\roman*)}
\usepackage{etoolbox}		% toolbox for command modification

% --- FONTS & TYPESETTING ---

\usepackage[utf8]{inputenc} % allow utf-8 input
\usepackage[T1]{fontenc}    % use 8-bit T1 fonts
\usepackage{dsfont}			% font with double lines for sets
\usepackage[german,ruled,vlined,linesnumbered,commentsnumbered,algoruled]
{algorithm2e} 				%pseudo code
\usepackage{listings}		%java code
\usepackage{csquotes}

% --- URLS ---

\usepackage[colorlinks=true, linkcolor=black, citecolor=blue, urlcolor=blue]{hyperref}   	% hyperlinks
\usepackage{url}            % simple URL typesetting

% --- MATH SYMBOLS ---

\usepackage{amsmath,amssymb}% more math symbols
\usepackage{amsfonts}       % blackboard math symbols
\usepackage{latexsym}		% more math symbols
\usepackage{chngcntr}		% more math symbols
\usepackage{mathrsfs}		% math-fonts
\usepackage{mathtools}		% more math symbols
\usepackage{nchairx}		% Waldmann package for general math symbols

% --- GRAPHICS & CAPTIONS ----

\usepackage{graphicx}		% including images
\graphicspath{ {./figs/} }
\usepackage{subcaption}		% custom caption formatting
\DeclareCaptionLabelFormat{custom}{ \textbf{#1 #2}}
\captionsetup{format=hang}
\captionsetup{width=0.9\textwidth,labelformat=custom}
\usepackage{pdfpages}		% to include entire pdf pages in appendix etc.

% --- FORMAT ---

\usepackage[a4paper]{geometry} % a4 paper
\usepackage{setspace}		% spacing
\usepackage{titlesec}
\allowdisplaybreaks			% allow page breaks within math environments

% --- CUSTOM COMMANDS ---
%Logic
\newcommand{\then}{\Rightarrow}
\newcommand{\since}{\Leftarrow}
\renewcommand{\iff}{\ensuremath{\Leftrightarrow}}

%pretty epsilon
\let\oldepsilon\epsilon
\let\epsilon\varepsilon
\let\varepsilon\oldepsilon
%pretty phi
\let\oldphi\phi
\let\phi\varphi
\let\varphi\oldphi

\newcommand{\includetask}[2][pages=-]{
    \includepdf[#1]{#2}
    \addtocounter{subsection}{1}
}

% set-up for exercise specific stuff
\ifdef{\sheetNumber}{
    \setcounter{section}{\sheetNumber}
}{}

\usepackage{titling}
\newcommand{\makeexheaderger}{
    \begin{doublespace}
        \begin{center}
            \textbf{\Large{Übungsblatt \sheetNumber}}\\
            \textbf{\Large\lecture}\\
            Abgabe von: \textbf{\authors}\\
            \today
        \end{center}
        \ifdef {\sumPoints}
        {
            \hfill  \large Punkte: $\boxed{\qquad  /\; \sumPoints}$\\
        }{}
    \end{doublespace}
}

\newcommand{\makeexheadereng}{
    \begin{doublespace}
        \begin{center}
            \textbf{\Large{Exercise Sheet \sheetNumber}}\\
            \textbf{\Large\lecture}\\
            Abgabe von: \textbf{\authors}\\
            \today
        \end{center}
        \ifdef {\sumPoints}
        {
            \hfill  \large Points: $\boxed{\qquad  /\; \sumPoints}$\\
        }{}
    \end{doublespace}
}

% --- DOCUMENT ---

\begin{document}

\makeexheader

\subsection{Zentralübung}

\begin{enumerate}[label=(\alph*)]
    \item Es ist 
    \begin{equation*}
        |Y_n| = \left|Y_n - \frac{1}{n} + \frac{1}{n}\right| \leq \left| Y_n - \frac{1}{n} \right| + \left| \frac{1}{n} \right| = |Y_n - \mathds{E}[Y_n]| + \frac{1}{n}\text{.}
    \end{equation*}
    \item Es sei $\epsilon > 0$ beliebig aber fest gewählt.
    Aufgrund von $\frac{1}{n} \to 0$ existiert ein $N \in \mathds{N}$ mit $\frac{1}{n} < \frac{\epsilon}{2}$ für alle $n \geq N$.
    Wegen $|Y_n - \mathds{E}[Y_n]| + \frac{1}{n} \geq |Y_n| = |X_n - X|$ gilt $|X_n - X| \geq \epsilon \then |Y_n - \mathds{E}[Y_n]| + \frac{1}{n} \geq \epsilon$, also 
    \begin{align*}
        P(|X_n - X| \geq \epsilon) &\leq P(|Y_n - \mathds{E}[Y_n]| + \frac{1}{n} \geq \epsilon - \frac{1}{n}) = P(|Y_n - \mathds{E}[Y_n]| \geq \frac{\epsilon}{2}) \\
        &\leq Var(Y_n) \cdot \left(\frac{2}{\epsilon}\right)^2 = \frac{(\epsilon \sigma)^2}{4n} \to 0
    \end{align*}
    da $\epsilon, \sigma$ fest und $\frac{1}{n} \to 0$.
    Somit ist $X_n$ stochastisch konvergent gegen $X$.
\end{enumerate}

\subsection{}

\subsection{}

Setze $\epsilon = 1$. Dann existiert ein $n \in \mathds{N}$ mit
\begin{equation*}
    \sup_{Z \in \mathcal{F}} \int_{\{|Z| \geq n\}} |Z| d\mathds{P} < 1  \then 
    \int_{\{|Z| \geq n\}} |Z| d\mathds{P} < 1 \forall_{Z \in \mathcal{F}}\text{.}
\end{equation*}
Dies zeigt für beliebiges $Z \in \mathcal{F}$:
\begin{equation*}
    \mathds{E}[|Z|] = \int_\Omega |Z| d\mathds{P} = \int_{\{|Z| \geq n\}} |Z| d\mathds{P} + \int_{\{|Z| < n\}} |Z| d\mathds{P} < 1 + n \cdot \mathds{P}(\{|Z| < n\}) \leq 1 + n \cdot  \text{.}
\end{equation*}
Die Aussage folgt nun durch die Festlegung $K \coloneq n+2$.

Die umgekehrte Aussage ist im Allgemeinen nicht richtig. Es sei $\mathds{P}$ das übliche Maß auf $[0,1]$ und $X_n$ definiert wie folgt:
\begin{equation*}
    X_n(t) = \left\{ \begin{matrix}
        n & t \in [0, 1/n]\\ 0 & (1/n, 1]
    \end{matrix} \right.\text{.}
\end{equation*}
Diese Zufallsvariablen haben alle den festen Erwartungswert $n \cdot \frac{1}{n} + 0 \cdot (1 - \frac{1}{n}) = 1$ (und diese Erwartungswerte sind damit natürlich beschränkt), aber es gilt 
\begin{equation*}
    \int_{\{|Z|\geq n\}} |Z_n| d\mathds{P} = \int_0^\frac{1}{n} n d \mathds{P} = n \cdot \frac{1}{n} = 1
\end{equation*}
und damit auch $\sup_{Z \in \mathcal{F}} \int_{\{|Z|\geq n\}} |Z_n| d\mathds{P} = 1$, womit natürlich auch der Limes $1 \neq 0$ wird.


\subsection{}

\begin{enumerate}[label=(\alph*)]
    \item Wir zeigen die drei Metrik-Eigenschaften:
    \begin{enumerate}[label=(\roman*)]
        \item $\since$: Angenommen, es ist $X = Y$ $\mathds{P}$ fast-sicher, d.h. $X$ und $Y$ unterscheiden sich nur auf einer Nullmenge $N \subseteq \mathds{R}$. Dann ist
        \begin{align*}
            d(X,Y) &= \mathds{E}\left[\frac{|X-Y|}{1 + |X-Y|}\right] = \int_\mathds{R} \frac{|X-Y|}{1 + |X-Y|} d\mathds{P}\\
            &= \int_{\mathds{R} \setminus N} \frac{|X-Y|}{1 + |X-Y|} d\mathds{P} = \int_{\mathds{R} \setminus N} \frac{0}{1 + 0} d\mathds{P} = 0
        \end{align*}
        $\then$: Angenommen, es ist $X \neq Y$ $\mathds{P}$ fast-sicher.
        Dann existiert ein $\epsilon > 0$ mit $\mathds{P}(|X - Y| \geq \epsilon) = \lambda > 0$, d.h. es existiert eine Menge $K \subseteq \mathds{R}$ mit Maß $\lambda > 0$ und $|X-Y| \geq \epsilon$ auf $K$.
        Folglich gilt auf $K$ auch $\frac{|X-Y|}{1+ |X-Y|} \geq \frac{\epsilon}{1 + 0} = \epsilon$ und demnach
        \begin{align*}
            d(X,Y) &= \mathds{E}\left[ \frac{|X-Y|}{1+|X-Y|} \right] = \int_\mathds{R} \frac{|X-Y|}{1+|X-Y|} d\mathds{P} \\
            &\geq \int_K \frac{|X-Y|}{1+|X-Y|} d\mathds{P} \geq \epsilon \cdot \lambda > 0
        \end{align*}
        und somit $d(X,Y) \neq 0$.
        \item Die Symmetrie folgt direkt aus der Symmetrie von $|X-Y|$.
        \item Wir zeigen zuerst $\frac{|x-y|}{1 + |x-y|} + \frac{|y-z|}{1 + |y-z|} \geq \frac{|x-z|}{1+|x-z|}$ für reelle Zahlen $x,y,z \in \mathds{R}$ (man betrachte entsprechende Aufgaben aus der Analysis):
        \begin{align*}
            \frac{|x-y|}{1 + |x-y|} + \frac{|y-z|}{1 + |y-z|}  &\geq \frac{|x-y|}{1 + |x-y| + |y-z|} + \frac{|y-z|}{1 + |y-z| + |x-y|}\\
            &= \frac{|x-y| + |y-z|}{1 + |x-y| + |y-z|}\\
            &= 1 - \frac{1}{1 +|x-y| + |y-z| } \geq 1 - \frac{1}{1+|x-z|}\\
            &= \frac{|x-z|}{1 + |x-z|}\text{.}
        \end{align*}
        Somit gilt eine entsprechende Beziehung auch für reelle Zufallsvariablen und (aufgrund der Monotonie des Erwartungswertes) auch für die Erwartungswerte und damit für $d$.
    \end{enumerate}
    Somit ist $d$ eine Metrik auf dem Raum der reellen Zufallsvariablen.
    \item $\then$: Sei $\epsilon > 0$ beliebig.
    Wegen $d(X_n, X) \to 0$ existiert dann ein $N \in \mathds{N}$ mit $d(X_n, X) < \epsilon^2 > 0$ für alle $n \geq N$, also $\mathds{E}\left[ |X_n-X| \right] < \frac{\epsilon^2}{1 + \epsilon}$.
    Wegen $\mathds{E}[Z] \geq \epsilon \cdot P(Z \geq \epsilon)$ muss dann auch $\mathds{P}\left(\frac{|X_n-X|}{1 + |X_n - X|}\right) < \epsilon$ sein.
    Außerdem gilt
    \begin{equation*}
        |X_n - X| = \frac{|X_n - X|}{1} \leq \frac{|X_n-X|}{1+|X_n-X|}
    \end{equation*}
    und somit $\mathds{P}(|X_n - X| \geq \epsilon) \leq \mathds{P}\left(\frac{|X_n-X|}{1 + |X_n - X|}\geq \epsilon \right) < \epsilon$.
    
    $\since$: Aus Zeitgründen dem Korrektor zum Nachrechnen überlassen.

\end{enumerate}



\end{document}























