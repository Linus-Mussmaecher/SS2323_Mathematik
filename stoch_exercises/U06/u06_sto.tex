\documentclass[a4paper]{article}

% --- DATA ---

\def\lecture{Stochastik 2}
\def\authors{Linus Mußmächer}
\def\sheetNumber{06}
%\def\sumPoints{30} 

% --- PREAMBLE ---

\usepackage[german]{babel}	% language specific quotation marks etc.
% === USAGE ===

% when using this preamble, setup your environment variables like this beforehand:


% \title{Stochastik 2}  %Title of exercise 
% \def\lecture{Stochastik 2}
% \def\authors{Linus Mußmächer}
% \def\sheetNumber{02}
% \def\sumPoints{30}      % maximum number of points (leave undefined)

% then use one of these commands (german or english) to print the header:

% \makeexheaderger

% and finally use subsections for your subtasks - they will be numbered as <sheetNumber><task number> by themselves

% if you have an exercise as an external .pdf, use \includetask to include it and increase the task counter


% --- OTHER ---

\usepackage{booktabs}       % professional-quality tables
\usepackage[table]{xcolor}	% color
\usepackage{pdfpages}		% to include entire pdf pages in appendix etc.
\usepackage{enumitem}		% better custom enumerations
\setlist[enumerate, 1]{label=(\roman*)}
\usepackage{etoolbox}		% toolbox for command modification

% --- FONTS & TYPESETTING ---

\usepackage[utf8]{inputenc} % allow utf-8 input
\usepackage[T1]{fontenc}    % use 8-bit T1 fonts
\usepackage{dsfont}			% font with double lines for sets
\usepackage[german,ruled,vlined,linesnumbered,commentsnumbered,algoruled]
{algorithm2e} 				%pseudo code
\usepackage{listings}		%java code
\usepackage{csquotes}

% --- URLS ---

\usepackage[colorlinks=true, linkcolor=black, citecolor=blue, urlcolor=blue]{hyperref}   	% hyperlinks
\usepackage{url}            % simple URL typesetting

% --- MATH SYMBOLS ---

\usepackage{amsmath,amssymb}% more math symbols
\usepackage{amsfonts}       % blackboard math symbols
\usepackage{latexsym}		% more math symbols
\usepackage{chngcntr}		% more math symbols
\usepackage{mathrsfs}		% math-fonts
\usepackage{mathtools}		% more math symbols
\usepackage{nchairx}		% Waldmann package for general math symbols

% --- GRAPHICS & CAPTIONS ----

\usepackage{graphicx}		% including images
\graphicspath{ {./figs/} }
\usepackage{subcaption}		% custom caption formatting
\DeclareCaptionLabelFormat{custom}{ \textbf{#1 #2}}
\captionsetup{format=hang}
\captionsetup{width=0.9\textwidth,labelformat=custom}
\usepackage{pdfpages}		% to include entire pdf pages in appendix etc.

% --- FORMAT ---

\usepackage[a4paper]{geometry} % a4 paper
\usepackage{setspace}		% spacing
\usepackage{titlesec}
\allowdisplaybreaks			% allow page breaks within math environments

% --- CUSTOM COMMANDS ---
%Logic
\newcommand{\then}{\Rightarrow}
\newcommand{\since}{\Leftarrow}
\renewcommand{\iff}{\ensuremath{\Leftrightarrow}}

%pretty epsilon
\let\oldepsilon\epsilon
\let\epsilon\varepsilon
\let\varepsilon\oldepsilon
%pretty phi
\let\oldphi\phi
\let\phi\varphi
\let\varphi\oldphi

\newcommand{\includetask}[2][pages=-]{
    \includepdf[#1]{#2}
    \addtocounter{subsection}{1}
}

% set-up for exercise specific stuff
\ifdef{\sheetNumber}{
    \setcounter{section}{\sheetNumber}
}{}

\usepackage{titling}
\newcommand{\makeexheaderger}{
    \begin{doublespace}
        \begin{center}
            \textbf{\Large{Übungsblatt \sheetNumber}}\\
            \textbf{\Large\lecture}\\
            Abgabe von: \textbf{\authors}\\
            \today
        \end{center}
        \ifdef {\sumPoints}
        {
            \hfill  \large Punkte: $\boxed{\qquad  /\; \sumPoints}$\\
        }{}
    \end{doublespace}
}

\newcommand{\makeexheadereng}{
    \begin{doublespace}
        \begin{center}
            \textbf{\Large{Exercise Sheet \sheetNumber}}\\
            \textbf{\Large\lecture}\\
            Abgabe von: \textbf{\authors}\\
            \today
        \end{center}
        \ifdef {\sumPoints}
        {
            \hfill  \large Points: $\boxed{\qquad  /\; \sumPoints}$\\
        }{}
    \end{doublespace}
}

% --- DOCUMENT ---

\begin{document}

\makeexheaderger

\subsection{Zentralübung}

\begin{enumerate}
    \item Es ist $p_{X|X}(x|x) = 1$, also $\mathds{E}[X|X=x] = x$ und damit $g: X(\Omega) \to \mathds{R}, x \mapsto x \cdot 1_{X(\Omega)}$. Es folgt $\mathds{E}[X|X](\omega) = g(X(\omega)) = X(\omega)$ und damit $\mathds{E}[X|X] = X$.
    \item Es ist $p_{X|Y}(x|y) = \mathds{P}(X = x)$ aufgrund der Unabhängigkeit. Somit ist $\mathds{E}[X|Y=y] = \mathds{E}[X]$ sowie $g: Y(\Omega) \to \mathds{R}, y \mapsto \mathds{E}[X]$ konstant. Es folgt $\mathds{E}[X|Y](\omega) = g(Y(\omega)) = \mathds{E}[X]$ und damit $\mathds{E}[X|Y] = \mathds{E}[X]$.
    \item Aus (i) folgt $\mathds{E}[X + Y | X + Y] = X + Y$. Weiterhin ist, da $X$ und $Y$ gleichverteilt sind, $\mathds{E}[X | X + Y] = \mathds{E}[Y | X + Y]$. Zusammen mit der Linearität des (bedingten) Erwartungswertes liefert dies
    \begin{align*}
        &\mathds{E}[X + Y | X + Y] = \mathds{E}[X| X + Y] + \mathds{E}[ Y | X + Y] = 2 \cdot \mathds{E}[X| X + Y] \\ \iff \ &\mathds{E}[X| X + Y] = \frac{1}{2} \mathds{E}[X + Y | X + Y] = \frac{1}{2} (X + Y)
    \end{align*}
\end{enumerate}

\subsection{}

\subsection{}

\begin{enumerate}
    \item Wir setzen $\Omega = \{0,1\}^3 $ sowie 
    \begin{align}
        Z: \Omega \to \mathds{R} & (x,y,z) \mapsto x + y + z\\
        G: \Omega \to \mathds{R} & (x,y,z) \mapsto x + \frac{y+z}{2}
    \end{align}
    d.h. die erste Komponente von $\omega \in \Omega$ beschreibt, ob die 1-Euro-Münze auf Zahl gefallen ist und die zweite und dritte Komponente betrachtet analog die 50-Cent-Münzen. Da alle acht Elementarereignisse in $\Omega$ gleich wahrscheinlich sind, gilt für die Verteilungen:
    \begin{align*}
        P(Z = 0) = \frac{1}{8} & \qquad A_0 = \{(0,0,0)\} \\ 
        P(Z = 1) = \frac{3}{8} & \qquad A_1 = \{(1,0,0), (0,1,0), (0,0,1) \}\\
        P(Z = 2) = \frac{3}{8} & \qquad A_2 = \{(0,1,1), (1,0,1), (1,1,0) \}\\
        P(Z = 3) = \frac{1}{8} & \qquad A_3 = \{(1,1,1)\}
    \end{align*}
    sowie
    \begin{align*}
        P(G = 0) = \frac{1}{8} & \qquad B_0 = \{(0,0,0)\} \\ 
        P(G = 0.5) = \frac{2}{8} & \qquad B_{0.5} = \{(0,1,0), (0,0,1) \}\\
        P(G = 1) = \frac{2}{8} & \qquad B_{1} = \{(1,0,0), (0,1,1)\}\\
        P(G = 1.5) = \frac{2}{8} & \qquad B_{1.5} = \{(1,1,0), (1,0,1)\}\\
        P(G = 2) = \frac{1}{8} & \qquad B_2 = \{(1,1,1)\}
    \end{align*}
    und für den Erwartungswert folgt
    \begin{equation*}
        \mathds{E}[Z] = \frac{1}{8} \cdot 0 + \frac{3}{8} \cdot 1 + \frac{3}{8} \cdot 2 + \frac{1}{8} \cdot 3 = \frac{12}{8} = \frac{3}{2} \text{.}
    \end{equation*}
    \item $Z = 2$ wird genau für die Elementarereignisse in der Menge $A_2$ realisiert. Wir betrachten also jeweils die Schnittmenge der $B$-Mengen mit $A_2$:
    \begin{align*}
        P(G = 0 | Z = 2) = \frac{0}{3} & \qquad B_0 \cap A_2= \{\} \\ 
        P(G = 0.5 | Z = 2) = \frac{0}{3} & \qquad B_{0.5} \cap A_2= \{\}\\
        P(G = 1 | Z = 2) = \frac{1}{3} & \qquad B_{1} \cap A_2= \{(0,1,1)\}\\
        P(G = 1.5 | Z = 2) = \frac{2}{3} & \qquad B_{1.5} \cap A_2= \{(1,1,0), (1,0,1)\}\\
        P(G = 2 | Z = 2) = \frac{0}{3} & \qquad B_2 \cap A_2= \{\}
    \end{align*}
    Der zugehörige Erwartungswert ist damit $\mathds{E}[G | Z = 2] = 1 \cdot \frac{1}{3} + \frac{3}{2} \cdot \frac{2}{3} = \frac{4}{3}$.
    \item Es ist $A_0 = \{(0,0,0)\}$. Dieses Ereignis ist nur in $B_{0}$ enthalten, womit $\mathds{E}[G|Z=0] = 0$ folgt.
    Ebenfalls gilt $A_3 = \{(1,1,1)\}$. Auch hier liegt nur ein einziges Ereignis vor, dass nur in $B_{2}$ enthalten ist. Somit folgt $\mathds{E}[G | Z = 3] = 2$.
    Analog zur (ii) lässt sich $\mathds{E}[G|Z=1] = \frac{2}{3}$ nachrechnen. Damit gilt für alle möglichen Realisierungen von $Z$ tatsächlich $\mathds{E}[G|Z = z] = \frac{2}{3} z$, also $\mathds{E}[G|Z] = \frac{2}{3}Z$.
    \item Die Tower-Rule besagt nun 
    \begin{equation*}
        \mathds{E}[G] = \mathds{E}[\mathds{E}[G|Z]] = \mathds{E}[\frac{2}{3}Z] = \frac{2}{3} \mathds{E}[Z] = \frac{2}{3} \cdot \frac{3}{2} = 1\text{.}
    \end{equation*}
\end{enumerate}

\subsection{}

\newcommand{\var}{\mathds{V}\mathrm{ar}}

\begin{enumerate}
    \item Wir setzen im Folgenden $S \coloneq X_1 + \cdots X_N = \sum_{i = 1}^n X_i \cdot \mathds{1}_{\{1, \dots, N\}}(i)$. Dann gilt:
    \begin{align*}
        \mathds{E}[S \mid N = k]
        &= \mathds{E} \left[ \sum_{i = 1}^n X_i \cdot \mathds{1}_{\{1, \dots, N\}}(i) \mid N = k  \right]& \text{Erwartungswert linear}\\
        &= \sum_{i = 1}^n \mathds{E}[X_i \cdot \mathds{1}_{\{1, \dots, N\}}(i) \mid N = k] &\text{$X_i$ von $N$ unabhängig} \\
        &= \sum_{i = 1}^{n} \mathds{E}[X_i \cdot 1_{\{1, \dots, k\}(i)}] = \sum_{i = 1}^k \mathds{E}[X_i] = k \cdot \mathds{E}[X_1]\text{.}
    \end{align*}
    Somit folgt $\mathds{E}[S \mid N = k] = k \cdot \mathds{E}[X_1]$, also $\mathds{E}[S \mid N] = N \cdot \mathds{E}[X_1]$.
    Dies zeigt mit der Tower-Rule
    \begin{equation*}
        \mathds{E}[S] = \mathds{E}[\mathds{E}[S|N]] = \mathds{E}[N \cdot \mathds{E}[X_1]] = \mathds{E}[N] \cdot \mathds{E}[X_1]\text{.}
    \end{equation*}
    Analog betrachten wir
    \begin{align*}
        \var(S|N = k) 
        &= \var\left(\sum_{i = 1}^n X_i \cdot \mathds{1}_{\{1, \dots, N\}}(i) \mid N = k\right) &\text{$X_i$ unabhängig}\\
        &= \sum_{i = 1}^n \var(X_i \cdot \mathds{1}_{\{1, \dots, N\}}(i) | N = k) &\text{$X_i$ von $N$ unabhängig} \\
        &= \sum_{i = 1}^n \var(X_i \cdot \mathds{1}_{\{1, \dots, k\}}(i))\\
        &= \sum_{i = 1}^{k} \var(X_i) & \text{$X_i$ gleichverteilt}\\
        &= k \cdot \var(X_1)
    \end{align*}
    und dies zeigt $\var(S|N) = N \cdot \var(X_1)$. Damit folgt:
    \begin{align*}
        \var(S) &= \mathds{E}[\var(S|N)] + \var(\mathds{E}[S|N]) \\
        &= \mathds{E}[N \cdot \var(X_1)] + \var(N \cdot \mathds{E}[X_1]) \\
        &= \mathds{E}[N] \cdot \var(X_1) + \var(N) \cdot \mathds{E}[X_1]^2\text{.}
    \end{align*}
    \item Da $N$ zum Parameter $\lambda$ Poisson-verteilt ist, gilt $\mathds{E}[N] = \var(N) = \lambda$.
    Für $X_1$ folgt weiterhin (durch Rechnung):
    \begin{equation*}
        \mathds{E}[X_1] = 0.1 \cdot 20 + 0.05 \cdot 30 + 0.1 \cdot 40 + 0.15 \cdot 50 + 0.6 \cdot 60 = 2 + 1.5 + 4 + 7.5 + 36 = 51
    \end{equation*}
    sowie
    \begin{equation*}
        \var(X_1) = 0.1 \cdot 31^2 + 0.05 + 21^2 + 0.1 \cdot 11^2 + 0.15 \cdot 1^2 + 0.6 \cdot 9^2 = 179\text{.}
    \end{equation*}
    Mit (i) folgt $\mathds{E}[Z] = 51\lambda$ sowie $\var(Z) = \lambda \cdot 179 + \lambda \cdot 51^2 = 2780\lambda$.
\end{enumerate}


\end{document}























