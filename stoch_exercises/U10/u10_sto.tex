\documentclass[a4paper]{article}

% --- DATA ---

\def\lecture{Stochastik 2}
\def\authors{Linus Mußmächer}
\def\sheetNumber{10}
%\def\sumPoints{30} 

% --- PREAMBLE ---

\usepackage[german]{babel}	% language specific quotation marks etc.
% === USAGE ===

% when using this preamble, setup your environment variables like this beforehand:


% \title{Stochastik 2}  %Title of exercise 
% \def\lecture{Stochastik 2}
% \def\authors{Linus Mußmächer}
% \def\sheetNumber{02}
% \def\sumPoints{30}      % maximum number of points (leave undefined)

% then use one of these commands (german or english) to print the header:

% \makeexheaderger

% and finally use subsections for your subtasks - they will be numbered as <sheetNumber><task number> by themselves

% if you have an exercise as an external .pdf, use \includetask to include it and increase the task counter


% --- OTHER ---

\usepackage{booktabs}       % professional-quality tables
\usepackage[table]{xcolor}	% color
\usepackage{pdfpages}		% to include entire pdf pages in appendix etc.
\usepackage{enumitem}		% better custom enumerations
\setlist[enumerate, 1]{label=(\roman*)}
\usepackage{etoolbox}		% toolbox for command modification

% --- FONTS & TYPESETTING ---

\usepackage[utf8]{inputenc} % allow utf-8 input
\usepackage[T1]{fontenc}    % use 8-bit T1 fonts
\usepackage{dsfont}			% font with double lines for sets
\usepackage[german,ruled,vlined,linesnumbered,commentsnumbered,algoruled]
{algorithm2e} 				%pseudo code
\usepackage{listings}		%java code
\usepackage{csquotes}

% --- URLS ---

\usepackage[colorlinks=true, linkcolor=black, citecolor=blue, urlcolor=blue]{hyperref}   	% hyperlinks
\usepackage{url}            % simple URL typesetting

% --- MATH SYMBOLS ---

\usepackage{amsmath,amssymb}% more math symbols
\usepackage{amsfonts}       % blackboard math symbols
\usepackage{latexsym}		% more math symbols
\usepackage{chngcntr}		% more math symbols
\usepackage{mathrsfs}		% math-fonts
\usepackage{mathtools}		% more math symbols
\usepackage{nchairx}		% Waldmann package for general math symbols

% --- GRAPHICS & CAPTIONS ----

\usepackage{graphicx}		% including images
\graphicspath{ {./figs/} }
\usepackage{subcaption}		% custom caption formatting
\DeclareCaptionLabelFormat{custom}{ \textbf{#1 #2}}
\captionsetup{format=hang}
\captionsetup{width=0.9\textwidth,labelformat=custom}
\usepackage{pdfpages}		% to include entire pdf pages in appendix etc.

% --- FORMAT ---

\usepackage[a4paper]{geometry} % a4 paper
\usepackage{setspace}		% spacing
\usepackage{titlesec}
\allowdisplaybreaks			% allow page breaks within math environments

% --- CUSTOM COMMANDS ---
%Logic
\newcommand{\then}{\Rightarrow}
\newcommand{\since}{\Leftarrow}
\renewcommand{\iff}{\ensuremath{\Leftrightarrow}}

%pretty epsilon
\let\oldepsilon\epsilon
\let\epsilon\varepsilon
\let\varepsilon\oldepsilon
%pretty phi
\let\oldphi\phi
\let\phi\varphi
\let\varphi\oldphi

\newcommand{\includetask}[2][pages=-]{
    \includepdf[#1]{#2}
    \addtocounter{subsection}{1}
}

% set-up for exercise specific stuff
\ifdef{\sheetNumber}{
    \setcounter{section}{\sheetNumber}
}{}

\usepackage{titling}
\newcommand{\makeexheaderger}{
    \begin{doublespace}
        \begin{center}
            \textbf{\Large{Übungsblatt \sheetNumber}}\\
            \textbf{\Large\lecture}\\
            Abgabe von: \textbf{\authors}\\
            \today
        \end{center}
        \ifdef {\sumPoints}
        {
            \hfill  \large Punkte: $\boxed{\qquad  /\; \sumPoints}$\\
        }{}
    \end{doublespace}
}

\newcommand{\makeexheadereng}{
    \begin{doublespace}
        \begin{center}
            \textbf{\Large{Exercise Sheet \sheetNumber}}\\
            \textbf{\Large\lecture}\\
            Contributors: \textbf{\authors}\\
            \today
        \end{center}
        \ifdef {\sumPoints}
        {
            \hfill  \large Points: $\boxed{\qquad  /\; \sumPoints}$\\
        }{}
    \end{doublespace}
}

\newcommand{\qmatrix}[1]{\ensuremath{\left(\begin{matrix}#1\end{matrix}\right)}}

% --- DOCUMENT ---

\begin{document}

\makeexheader

\subsection{Zentralübung}

Die Parameter der geometrischen Verteilung sind $\theta \in \Theta = (0,1)$, der Stichprobenraum ist $\mathcal{X} = \mathds{N}^n$.
Aus der Definition der geometrischen Verteilung bestimmen wir die Likelihood-Funktion und direkt daraus folgend die Log-Likelihood-Funktion:
\begin{align*}
    L: \Theta \times \mathcal{X} \to [0,1], & \ (\theta, x) \mapsto \prod_{k = 1}^{n} (1-\theta)^{x_k-1} \theta \\ 
    \mathcal{L}: \Theta \times \mathcal{X} \to [0,1], & \ (\theta, x) \mapsto \log\left(\prod_{k = 1}^{n} (1-\theta)^{x_k-1} \theta\right) = n \cdot \log \theta + \log (1-\theta) \cdot \sum_{k = 1}^{n}(x_k-1) 
\end{align*}
Wir bestimmen ihr Maximum durch die Nullstellen der Ableitung:
\begin{align*}
    &\frac{\partial}{\partial \theta} \mathcal{L}(\theta, x) = \frac{n}{\theta} - \frac{1}{1 - \theta} \sum_{k = 1}^{n}(x_k-1) = \frac{n}{\theta} - \frac{1}{1 - \theta} \left(-n + \sum_{k = 1}^{n} x_k  \right) = 0\\
    \iff \ & n \cdot (1 - \theta) = \theta \sum_{k = 1}^{n}x_k - n \theta \iff n = \theta \sum_{k = 1}^{n}x_k \iff \theta = \frac{n}{\sum_{k = 1}^{n}x_k} = \frac{1}{\overline{x}}
\end{align*}
wobei $\overline{x}$ den Durschnitt von $x_1, \dots, x_n$ bezeichne. Der (in diesem Fall aufgrund des an einer eindeutigen Stelle angenommen Maximums eindeutige) ML-Schätzer ist daher
\begin{equation*}
    \hat{\theta}^{ML}: \mathcal{X} \to \overline{\Theta}, x \mapsto \frac{1}{\overline{x}}\text{.}
\end{equation*}

\subsection{}

\begin{enumerate}
    \item Es ist 
    \begin{equation*}
        \mathds{E}_\theta[X] = \frac{2}{3} \theta \cdot 0 + \frac{1}{3} \theta \cdot 1 + \frac{2}{3} ( 1- \theta) \cdot 2 + \frac{1}{3} (1-\theta) \cdot 3 = 0 + \frac{1}{3} \theta + \frac{4}{3} - \frac{4}{3} \theta + 1 - \theta = \frac{7}{3} - 2 \theta\text{.}
    \end{equation*}
    \item Uns ist der Erwartungswert (in Abhängigkeit von $\theta$) bereits bekannt.
    Wir können nun für eine Menge an Zufallsvariablen $X_1, \dots, X_n$ die unabhängig identisch wie $X$ verteilt sind, den Momentenschätzer dieser Variablen unter Rückgriff auf den Durchschnitt $\overline{X} = \frac{1}{n}\sum_{k = 1}^{n} X_n$ berechnen als
    \begin{align*}
        \overline{X} = \mathds{E}_\theta[X] = \frac{7}{3} - 2 \theta \iff \theta = \frac{7 - 3 \overline{X}}{6}\text{.}
    \end{align*}
    Insbesondere fällt auf, dass dieser für extreme Durchschnitte (z.B. $\overline{X} = 3$) auch Werte außerhalb von $\Theta = [0,1]$ annehmen kann.
    \item Der Durchschnitt der Beobachtungen ist $\overline{X} = \frac{1}{10} \cdot 15 = \frac{3}{2}$. Damit ist der Wert des Momentenschätzers gleich $\frac{7 - \frac{9}{2}}{6} = \frac{5}{12}$.
    
    Um den Wert des ML-Schätzers für diese Beobachtungen zu bestimmen, müssen wir nicht die gesamte Funktion $\hat{\theta}^{ML}$ bestimmen, sondern lediglich die Wahrscheinlichkeit genau dieses Ereignisses bestimmen:
    \begin{equation*}
        \mathds{P}(3,0,2,1,3,2,1,0,2,1) = \left(\frac{2}{3}\theta\right)^2 \cdot \left(\frac{1}{3} \theta\right)^3 \cdot \left(\frac{2}{3}(1-\theta)\right)^3 \cdot \left(\frac{1}{3}(1-\theta)\right)^2 = \frac{32 \theta^5 (1-\theta^5)}{59049}\text{.}
    \end{equation*}
    Diese Funktion wird am Maximum von $\theta^5 (1-\theta)^5$ maximiert, also bei $\theta = \frac{1}{2}$.
    Damit ist dies der Wert des Maximum-Likelihood-Schätzers.

    Dies ist auch insofern plausibel, dass jedes Teilereignis $0,1$ genauso oft auftritt wie das 'symmetrische' Ereignis $2,3$ und insofern die Wahrscheinlichkeiten (vermutlich) gleich hoch sein sollten.
    Der Momentenschätzer, der nur den Durchschnitt 'sieht', kann dies allerdings nicht erkennen und tendiert daher zu niedrigerem $\theta$.
\end{enumerate} 

\subsection{}

Wir bestimmen zuerst die Likelihood-Funktion der gemeinsamen Verteilung, indem wir die einzelnen Dichten als Funktionen in $y_k, \mu$ betrachten und multiplizieren:
\begin{equation*}
    L((y_1, \dots, y_n), \mu) = \frac{1}{(\sqrt{2 \pi \sigma^2})^n} \exp\left( - \sum_{k=1}^n \frac{(\log(y_k) - \mu)^2}{2 \sigma^2} \right) \prod_{k = 1}^n \frac{1}{y_k}
\end{equation*}
Um unsere Rechnungen zu vereinfachen, betrachten wir die Log-Likelihood-Funktion
\begin{equation*}
    \mathcal{L}((y_1, \dots, y_n), \mu) = -\frac{n}{2} \cdot \log\left(2 \pi \sigma^2 \right) - \sum_{k=1}^n \frac{(\log(y_k) - \mu)^2}{2 \sigma^2} - \sum_{k = 1}^{n} \log(y_k)
\end{equation*}
und leiten für festes, aber beliebiges $y$ nach $\mu$ ab
\begin{equation*}
    \frac{\partial}{\partial \mu} \mathcal{L}((y_1, \dots, y_n), \mu) = \frac{1}{\sigma^2} \sum_{k=1}^n (\mu - \log y_k)\text{.}
\end{equation*}
Damit folgt
\begin{equation*}
    \frac{\partial}{\partial \mu} \mathcal{L}((y_1, \dots, y_n), \mu) = 0 \iff n \mu = \sum_{k = 1}^{n} \log y_k \iff \mu = \frac{1}{n} \sum_{k = 1}^{n} \log y_k
\end{equation*}
und dies ist somit das $\mu$, das für festes $y$ die Log-Likelihood-Funktion und damit auch die Likelihood-Funktion maximiert.
Der ML-Schätzer ist somit
\begin{equation*}
    \hat{\theta}^{ML}: \mathds{R}_+^n \to \mathds{R}, (y_1, \dots, y_n) \mapsto \frac{1}{n} \sum_{k=1}^{n} \log y_k
\end{equation*}

\subsection{}

\begin{enumerate}
    \item Wir wollen das Maximum von 
    \begin{equation*}
        f(x, \nu) = \nu x_0^\nu x^{-(\nu + 1)} \mathds{1}_{x \geq x_0} = \nu (x_0 x^{-1})^\nu x^{-1} \mathds{1}_{x \geq x_0} 
    \end{equation*}
    für festes $x$, aber variables $\nu$ bestimmen. Sei dazu $x \geq x_0$ beliebig, aber fest.
    Wir leiten nach $\nu$ ab:
    \begin{equation*}
        \frac{\partial}{\partial \nu} f(x, \nu) = 1 \cdot (x_0 x^{-1})^\nu x^{-1} + \nu \ln(x x_0^{-1}) (x_0 x^{-1})^\nu x^{-1}
    \end{equation*}
    und damit 
    \begin{align*}
        &\frac{\partial}{\partial \nu} f(x, \nu) = 0 \\
        \iff \ & (x_0 x^{-1})^\nu x^{-1} = - \nu ln(x_0 x^{-1}) (x_0 x^{-1})^\nu x^{-1} \\
        \iff \ & 1 = - \nu \ln (x_0 x^{-1}) \\
        \iff \ & \nu = \frac{1}{\ln(x / x_0)}
    \end{align*}
    und damit, da $\nu \in (1, \infty)$:
    \begin{equation*}
        \hat{\theta}^{ML} (x): \mathcal{X} \to [1, \infty), x \mapsto \max \left(1, \frac{1}{\ln(x) - \ln(x_0)}\right)
    \end{equation*}
    \item Für $x = (x_1, \dots, x_n) \in \mathds{R}^n$ und $\nu \in \mathds{R}$ ist die Likelihood-Funktion
    \begin{equation*}
        f(x, \nu) = \left\{ \begin{matrix}
            1 & \forall_{1 \leq k \leq n} \ x_k \in [\nu, \nu+1] \\
            0 & \exists_{1 \leq k \leq n} \ x_k \notin [\nu, \nu+1]
        \end{matrix} \right.
    \end{equation*}
    Setze $x_{(0)} = \min\{x_k \mid 1 \leq k \leq n\}$ und $x_{(n)} = \max\{x_k \mid 1 \leq k \leq n\}$.

    Falls  $x_{(n)} - x_{(0)} > 1$, so existiert kein $\nu$ für das $f(x, \nu) = 1$ ist.
    Da wir aber annehmen, dass unsere Datenpunkte wie angegeben verteilt sind, kann dies nicht auftreten. 

    Betrachten wir nun die Funktion
    \begin{equation*}
        \hat{\theta}^{ML}: \{ (x_1, \dots, x_k) \in \mathds{R}^n \mid x_{(n)} - x_{(0)} \leq 1 \} \to \mathds{R}, (x_1, \dots, x_k) \mapsto x_{(0)} \text{.}
    \end{equation*}
    Sicher gilt $x_{(0)} \in [x_{(0)}, x_{(0)} + 1]$ und ebenso $x_{(n)} \in [x_{(0)}, x_{(0)} + 1]$, da $\max \{x_k\} - x_{(0)} \leq 1$.
    Damit gilt $x_k \in [x_{(0)}, x_{(0)} + 1]$ für alle $k$, also ist $f(x, \hat{\theta}^ML(x)) = 1$, was dem Supremum von $f$ entspricht (da $f$ ja nur zwei Werte annimmt).

    Betrachten wir aber analog $\hat{\Theta}^{ML_2}(x) = x_{(n)} - 1$, so gilt $x_{(0)} \in [x_{(n)} - 1, x_{(n)}]$ aufgrund der Abstandbedingung und $x_{(n)} \in [x_{(n)} - 1, x_{(n)}]$.
    Wieder folgt daher $f(x, \hat{\Theta}^{ML_2}(x)) = 1$ für alle $x \in \mathcal{X}$ und auch $\hat{\Theta}^{ML_2}(x)$ ist ein ML-Schätzer.

    Da insbesondere $x$ mit $x_{(0)} - x_{(n)} < 1$ existieren, sind diese beiden Schätzer außerdem unterschiedlich und ''der'' ML-Schätzer somit nicht eindeutig festgelegt. Man sollte also stets von ''einem'', nicht ''dem'', ML-Schätzer sprechen.
\end{enumerate}


\end{document}