\documentclass[a4paper]{article}

% --- DATA ---

\def\lecture{Stochastik 2}
\def\authors{Linus Mußmächer}
\def\sheetNumber{02}
%\def\sumPoints{30} 

% --- PREAMBLE ---

\usepackage[german]{babel}	% language specific quotation marks etc.
% === USAGE ===

% when using this preamble, setup your environment variables like this beforehand:


% \title{Stochastik 2}  %Title of exercise 
% \def\lecture{Stochastik 2}
% \def\authors{Linus Mußmächer}
% \def\sheetNumber{02}
% \def\sumPoints{30}      % maximum number of points (leave undefined)

% then use one of these commands (german or english) to print the header:

% \makeexheaderger

% and finally use subsections for your subtasks - they will be numbered as <sheetNumber><task number> by themselves

% if you have an exercise as an external .pdf, use \includetask to include it and increase the task counter


% --- OTHER ---

\usepackage{booktabs}       % professional-quality tables
\usepackage[table]{xcolor}	% color
\usepackage{pdfpages}		% to include entire pdf pages in appendix etc.
\usepackage{enumitem}		% better custom enumerations
\setlist[enumerate, 1]{label=(\roman*)}
\usepackage{etoolbox}		% toolbox for command modification

% --- FONTS & TYPESETTING ---

\usepackage[utf8]{inputenc} % allow utf-8 input
\usepackage[T1]{fontenc}    % use 8-bit T1 fonts
\usepackage{dsfont}			% font with double lines for sets
\usepackage[german,ruled,vlined,linesnumbered,commentsnumbered,algoruled]
{algorithm2e} 				%pseudo code
\usepackage{listings}		%java code
\usepackage{csquotes}

% --- URLS ---

\usepackage[colorlinks=true, linkcolor=black, citecolor=blue, urlcolor=blue]{hyperref}   	% hyperlinks
\usepackage{url}            % simple URL typesetting

% --- MATH SYMBOLS ---

\usepackage{amsmath,amssymb}% more math symbols
\usepackage{amsfonts}       % blackboard math symbols
\usepackage{latexsym}		% more math symbols
\usepackage{chngcntr}		% more math symbols
\usepackage{mathrsfs}		% math-fonts
\usepackage{mathtools}		% more math symbols
\usepackage{nchairx}		% Waldmann package for general math symbols

% --- GRAPHICS & CAPTIONS ----

\usepackage{graphicx}		% including images
\graphicspath{ {./figs/} }
\usepackage{subcaption}		% custom caption formatting
\DeclareCaptionLabelFormat{custom}{ \textbf{#1 #2}}
\captionsetup{format=hang}
\captionsetup{width=0.9\textwidth,labelformat=custom}
\usepackage{pdfpages}		% to include entire pdf pages in appendix etc.

% --- FORMAT ---

\usepackage[a4paper]{geometry} % a4 paper
\usepackage{setspace}		% spacing
\usepackage{titlesec}
\allowdisplaybreaks			% allow page breaks within math environments

% --- CUSTOM COMMANDS ---
%Logic
\newcommand{\then}{\Rightarrow}
\newcommand{\since}{\Leftarrow}
\renewcommand{\iff}{\ensuremath{\Leftrightarrow}}

%pretty epsilon
\let\oldepsilon\epsilon
\let\epsilon\varepsilon
\let\varepsilon\oldepsilon
%pretty phi
\let\oldphi\phi
\let\phi\varphi
\let\varphi\oldphi

\newcommand{\includetask}[2][pages=-]{
    \includepdf[#1]{#2}
    \addtocounter{subsection}{1}
}

% set-up for exercise specific stuff
\ifdef{\sheetNumber}{
    \setcounter{section}{\sheetNumber}
}{}

\usepackage{titling}
\newcommand{\makeexheaderger}{
    \begin{doublespace}
        \begin{center}
            \textbf{\Large{Übungsblatt \sheetNumber}}\\
            \textbf{\Large\lecture}\\
            Abgabe von: \textbf{\authors}\\
            \today
        \end{center}
        \ifdef {\sumPoints}
        {
            \hfill  \large Punkte: $\boxed{\qquad  /\; \sumPoints}$\\
        }{}
    \end{doublespace}
}

\newcommand{\makeexheadereng}{
    \begin{doublespace}
        \begin{center}
            \textbf{\Large{Exercise Sheet \sheetNumber}}\\
            \textbf{\Large\lecture}\\
            Abgabe von: \textbf{\authors}\\
            \today
        \end{center}
        \ifdef {\sumPoints}
        {
            \hfill  \large Points: $\boxed{\qquad  /\; \sumPoints}$\\
        }{}
    \end{doublespace}
}

% --- DOCUMENT ---

\begin{document}

\makeexheaderger

Würde 2.1/2.3/2.4 auch vorrechnen.

\subsection{Zentralübung}

Es sei $f$ die Dichte von $X$. Dann ist aufgrund der Symmetrie von $X$ gerade, d.h. $f(x) = f(-x)$ für alle $x \in \mathds{R}$. Es folgt für die charakteristische Funktion von $X$:
\begin{align*}
	\phi_X(t) = \mathds{E}[\exp(i \langle t, X \rangle)] & = \int_{-\infty}^{\infty} \exp(i\langle t, x \rangle) f(x) dx                                                                                     \\
	                                                     & = \int_{-\infty}^{0} \exp(i\langle t, x \rangle) f(x) dx + \int_{0}^{\infty} \exp(i\langle t, x \rangle) f(x) dx                                  \\
	                                                     & = \int_{0}^{\infty} \exp(-i\langle t, x \rangle) f(-x) dx + \int_{0}^{\infty} \exp(i\langle t, x \rangle) f(x) dx                                 \\
	                                                     & = \int_{0}^{\infty} (\overline{\exp(i\langle t, x \rangle)} + \exp(i\langle t, x \rangle)) f(x) dx
	\\
	                                                     & = \int_{0}^{\infty} \underbrace{2 \Re(\exp(i\langle t, x \rangle))}_{\in \mathds{R}} \underbrace{f(x)}_{\in \mathds{R}} dx \in \mathds{R}\text{.}
\end{align*}
Dies zeigt $\phi_X(t) \in \mathds{R}$ für alle $t \in \mathds{R}$, also ist $\phi_X$ rein reellwertig.

\addtocounter{subsection}{1}

\subsection{}

\begin{enumerate}
	\item Wir berechnen zuerst die Verteilung von $X_{(1)}$. Hierzu beachten wir, dass das Minimum von $X_1, \dots, X_n$ genau dann größer oder gleich $t \in \mathds{R}$ ist, wenn jeder der Werte $X_1, \dots, X_n$ größer oder gleich $t$ ist.

	      \begin{align*}
		      F_{(1)}(t)              & = P(X_{(1)} \leq t) = 1 - P(X_{(1) \geq t})                                                                                                                                                              \\
		      \text{(Unabhängigkeit)} & = 1- P(X_1 \geq t \wedge X_2 \geq t \wedge \dots \wedge X_n \geq t)                                                                                                                                      \\
		                              & = 1 - \prod_{i=1}^{n} P(X_i \geq t) = 1- \prod_{i=1}^{n} 1 - P(X_i \leq t)                                                                                                                             & \\
		      \text{(3.47)}           & = 1 - \left(\prod_{i = 1}^{n} 1 - \left( 1 - \exp(- \lambda_i t) \mathds{1}_{[0, \infty)}(t)\right) \right)                                                                                              \\
		                              & = \left( 1 - \prod_{i = 1}^{n} \exp(- \lambda_i t) \right) \mathds{1}_{[0, \infty)}(t) = \left( 1 - \exp\left( -t \cdot \sum_{i =1}^{n} \lambda_i  \right) \right) \mathds{1}_{[0, \infty)}(t)\text{.}
	      \end{align*}
	      Dies entspricht einer Exponentialverteilung mit Parameter $\sum_{i=1}^{n} \lambda_i$. Diese hat bekanntermaßen den Erwartungswert $\mathds{E}[X_{(1)}] = \left(\sum_{i=1}^{n} \lambda_i\right)^{-1}$.
	\item

	      Wir betrachten zuerst nur den Fall für zwei Variablen $X_1, X_2$. Dann ist $X_1 + X_2 = \max(X_1, X_2) + \min(X_1, X_2)$ und die Linearität des Erwartungswertes liefert
	      \begin{align*}
		      \mathds{E}[X_{(2)}] & = \mathds{E}[\max(X_1, X_2)] = \mathds{E}[X_1 + X_2 - \min(X_1, X_2)] \\ &= \mathds{E}[X_1] + \mathds{E}[X_2] - \mathds{E}[\min(X_1, X_2)]
		      = \frac{1}{\lambda} + \frac{1}{\lambda} - \frac{1}{\lambda + \lambda} = \frac{1}{\lambda}(1 + \frac{1}{2})
	      \end{align*}
	      unter Verwendung von (a).
	      Für allgemeines $X_{(n)}$ betrachten wir zuerst wieder die Verteilung und nutzen, dass das Maximum von $X_1, \dots, X_n$ genau dann kleiner oder gleich $t \in \mathds{R}$ ist, wenn alle Einzelvariablen kleiner oder gleich $t$ sind.
	      \begin{align*}
		      F_{(n)}(t)              & = P(X_{(n)} \leq t ) = P(X_1 \leq t \wedge \dots X_n \leq t)           \\
		      \text{(Unabhängigkeit)} & = \prod_{i = 1}^{n} P(X_i \leq t)                                       \overset{\text{(3.47)}  }{=} \prod_{i = 1}^{n} (1 - \exp(-\lambda t)) \mathds{1}_{[0, \infty)}(t) \\
		                              & = (1 - \exp(-\lambda t))^n \mathds{1}_{[0, \infty)}(t)
	      \end{align*}
	      Bezeichnen wir die Dichte dieser Verteilung mit $f_{(n)}$, so erhalten wir unseren Erwartungswert als ein Integral, das wir mithilfe des Satzes von Fubini-Tonelli umformen:
	      \begin{align*}
		      \mathds{E}[X_{(n)}] & = \int_{-\infty}^{\infty} t \cdot f_{(n)} (t) dt = \int_{0}^{\infty} t \cdot f_{(n)}(t) dt             \\
		                          & = \int_{0}^{\infty} \int_{x}^{\infty} f_{(n)} (t) dt dx = \int_{0}^{\infty} \mathds{P}(X_{(n)} \geq x) \\
		                          & = \int_{0}^{\infty} 1 - \mathds{P}(X_{(n)} \leq x) = \int_{0}^{\infty} 1 - F_{(n)}(t) dt\text{.}
	      \end{align*}
	      Dieses Integral lässt sich vielleicht direkt berechnen, aber es scheint kompliziert. Stattdessen wollen wir unter Verwendung von $\mathds{E}[X_{(2)}]$ (Erwartungswert des Maximums der ersten zwei Zufallsvariablen) den Wert $\mathds{E}[X_{(n)}]$ induktiv berechnen, indem wir das folgende Integral verwenden:
	      \begin{align*}
		      \mathds{E}[X_{(k)} - X_{(k-1)}] & = \int_{0}^{\infty} F_{(k-1)}(t) - F_{(k)}(t) dt                               \\
		                                      & = \int_{0}^{\infty} (1 - \exp(-\lambda t))^{k-1} - (1 - \exp(-\lambda t))^k dt \\
		                                      & = \int_{0}^{\infty} (1 - \exp(-\lambda t))^{k-1} \exp(-\lambda t) dt
	      \end{align*}
	      Substitution mit $x(t) = 1 - \exp(-\lambda t)$ und $\frac{d}{dt}x(t) = \lambda \exp(- \lambda t)$ liefert dann
	      \begin{align*}
		      \mathds{E}[X_{(k)} - X_{(k-1)}] & = \int_{0}^{1} \frac{1}{\lambda} x^{k-1} dx = \frac{1}{\lambda} \left[ \frac{x^k}{k} \right]_0^1 = \frac{1}{\lambda} \frac{1}{k}\text{.}
	      \end{align*}
	      Nun folgt unsere Aussage per Induktion. Den Induktionsanfang für $n = 2$ haben wir bereits eingangs gezeigt, und falls die Aussage für beliebiges aber festes $n-1 \in \mathds{N}$ bereits gezeigt ist, so erhalten wir für $n$:
	      \begin{equation*}
		      \mathds{E}[X_{(n)}] = \mathds{E}[X_{(n-1)}] + \mathds{E}[X_{(n)} - X_{(n-1)}] = \frac{1}{\lambda} \sum_{k =1 }^{n-1} \frac{1}{k} + \frac{1}{\lambda} \frac{1}{n} = \frac{1}{\lambda} \sum_{k =1 }^{n} \frac{1}{k}\text{.}
	      \end{equation*}
	      Dies zeigt die Aussage.
\end{enumerate}

\subsection{}

\begin{enumerate}
	\item Wir wissen bereits aus Beispiel 1.13, dass $\phi_{\delta_x}(t) = \exp(i t x)$ für das Dirac-Maß auf $\mathds{R}$. Unter Verwendung von $\cos(t) = \frac{1}{2} (\exp(-it) + \exp(-it))$ können wir daher folgern, dass $\cos$ die charakteristische Funktion der (diskreten) Zufallsvariable $X$ ist, die nach $\mu_X = \frac{1}{2}(\delta_{-1} + \delta_{1})$ verteilt ist und damit $P(X = -1) = P(X=1) = 1/2$ und $P(X = x) = 0$ für $x \notin \{-1,1\}$ erfüllt. Zum Beweis rechnen wir diese Behauptung noch nach:
	      \begin{equation*}
		      \phi_X(t) = \mathds{E}[\exp(itX)] = \frac{1}{2} \exp(it\cdot 1) + \frac{1}{2} \exp(it \cdot (-1)) = \frac{1}{2} (\exp(it) + \exp(-it))  = \cos(t)\text{.}
	      \end{equation*}
	      Weiterhin können wir $Y = \sum_{i = 1}^{n} X_i$ definieren, wobei $X_i$ voneinander unabhängige zu $X$ gleichverteilte Zufallsvariablen seien. Satz 1.15 zeigt dann
	      \begin{equation*}
		      \phi_Y(t) = \prod_{i=1}^{n} \phi_{X_i}(t) = \phi_X(t)^n = \cos(t)^n\text{.}
	      \end{equation*}
	      Somit ist die Aussage falsch.
	\item Sei $f_Z$ die Dichte von $Z$, dann ist $f_Z(t) = \mathds{E}[\exp(itZ)] = \int_{-\infty}^{\infty} \exp(itx) f_Z(x) dx$. Definieren wir $\tilde Z$ als Zufallsvariable mit der Dichte $f_{\tilde Z} = \frac{1}{2} (f_Z(t) + f_Z(-t))$, so erhalten wir
	      \begin{align*}
		      \phi_{\tilde Z}(t) & = \mathds{E[\exp(it\tilde Zi)]} = \int_{-\infty}^{\infty} \exp(itx) \frac{1}{2} (f_Z(x) + f_Z(-x)) dx                           \\
		                         & = \frac{1}{2} \left( \int_{-\infty}^{\infty} \exp(itx)  f_Z(x) dx + \int_{-\infty}^{\infty} \exp(i(-t)(-x))  f_Z(-x) dx \right) \\
		                         & = \frac{1}{2} \left( \phi_Z(t) + \phi_Z(-t) \right) = \Re(\phi_Z(t))\text{.}
	      \end{align*}
	      Somit ist die Aussage korrekt.
\end{enumerate}

\end{document}
















