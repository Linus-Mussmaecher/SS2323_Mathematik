\documentclass[a4paper]{article}

% --- DATA ---

\def\lecture{Stochastik 2}
\def\authors{Linus Mußmächer}
\def\sheetNumber{11}
%\def\sumPoints{30} 

% --- PREAMBLE ---

\usepackage[german]{babel}	% language specific quotation marks etc.
% === USAGE ===

% when using this preamble, setup your environment variables like this beforehand:


% \title{Stochastik 2}  %Title of exercise 
% \def\lecture{Stochastik 2}
% \def\authors{Linus Mußmächer}
% \def\sheetNumber{02}
% \def\sumPoints{30}      % maximum number of points (leave undefined)

% then use one of these commands (german or english) to print the header:

% \makeexheaderger

% and finally use subsections for your subtasks - they will be numbered as <sheetNumber><task number> by themselves

% if you have an exercise as an external .pdf, use \includetask to include it and increase the task counter


% --- OTHER ---

\usepackage{booktabs}       % professional-quality tables
\usepackage[table]{xcolor}	% color
\usepackage{pdfpages}		% to include entire pdf pages in appendix etc.
\usepackage{enumitem}		% better custom enumerations
\setlist[enumerate, 1]{label=(\roman*)}
\usepackage{etoolbox}		% toolbox for command modification

% --- FONTS & TYPESETTING ---

\usepackage[utf8]{inputenc} % allow utf-8 input
\usepackage[T1]{fontenc}    % use 8-bit T1 fonts
\usepackage{dsfont}			% font with double lines for sets
\usepackage[german,ruled,vlined,linesnumbered,commentsnumbered,algoruled]
{algorithm2e} 				%pseudo code
\usepackage{listings}		%java code
\usepackage{csquotes}

% --- URLS ---

\usepackage[colorlinks=true, linkcolor=black, citecolor=blue, urlcolor=blue]{hyperref}   	% hyperlinks
\usepackage{url}            % simple URL typesetting

% --- MATH SYMBOLS ---

\usepackage{amsmath,amssymb}% more math symbols
\usepackage{amsfonts}       % blackboard math symbols
\usepackage{latexsym}		% more math symbols
\usepackage{chngcntr}		% more math symbols
\usepackage{mathrsfs}		% math-fonts
\usepackage{mathtools}		% more math symbols
\usepackage{nchairx}		% Waldmann package for general math symbols

% --- GRAPHICS & CAPTIONS ----

\usepackage{graphicx}		% including images
\graphicspath{ {./figs/} }
\usepackage{subcaption}		% custom caption formatting
\DeclareCaptionLabelFormat{custom}{ \textbf{#1 #2}}
\captionsetup{format=hang}
\captionsetup{width=0.9\textwidth,labelformat=custom}
\usepackage{pdfpages}		% to include entire pdf pages in appendix etc.

% --- FORMAT ---

\usepackage[a4paper]{geometry} % a4 paper
\usepackage{setspace}		% spacing
\usepackage{titlesec}
\allowdisplaybreaks			% allow page breaks within math environments

% --- CUSTOM COMMANDS ---
%Logic
\newcommand{\then}{\Rightarrow}
\newcommand{\since}{\Leftarrow}
\renewcommand{\iff}{\ensuremath{\Leftrightarrow}}

%pretty epsilon
\let\oldepsilon\epsilon
\let\epsilon\varepsilon
\let\varepsilon\oldepsilon
%pretty phi
\let\oldphi\phi
\let\phi\varphi
\let\varphi\oldphi

\newcommand{\includetask}[2][pages=-]{
    \includepdf[#1]{#2}
    \addtocounter{subsection}{1}
}

% set-up for exercise specific stuff
\ifdef{\sheetNumber}{
    \setcounter{section}{\sheetNumber}
}{}

\usepackage{titling}
\newcommand{\makeexheaderger}{
    \begin{doublespace}
        \begin{center}
            \textbf{\Large{Übungsblatt \sheetNumber}}\\
            \textbf{\Large\lecture}\\
            Abgabe von: \textbf{\authors}\\
            \today
        \end{center}
        \ifdef {\sumPoints}
        {
            \hfill  \large Punkte: $\boxed{\qquad  /\; \sumPoints}$\\
        }{}
    \end{doublespace}
}

\newcommand{\makeexheadereng}{
    \begin{doublespace}
        \begin{center}
            \textbf{\Large{Exercise Sheet \sheetNumber}}\\
            \textbf{\Large\lecture}\\
            Contributors: \textbf{\authors}\\
            \today
        \end{center}
        \ifdef {\sumPoints}
        {
            \hfill  \large Points: $\boxed{\qquad  /\; \sumPoints}$\\
        }{}
    \end{doublespace}
}

\newcommand{\qmatrix}[1]{\ensuremath{\left(\begin{matrix}#1\end{matrix}\right)}}

% --- DOCUMENT ---

\begin{document}

\makeexheader

\subsection{Zentralübung}

\begin{enumerate}
    \item Die gemeinsame Dichte der $n$ Zufallsvariablen $X_1, \dots, X_n$ ist
    \begin{equation*}
        f_\theta(x_1, \dots, x_n) =  \prod_{i = 1}^n \frac{1}{\theta} x_i^{\frac{1 - \theta}{\theta}} \mathds{1}_{(0,1)}(x_i)
    \end{equation*}
    Für $(x_1, \dots, x_n) \in (0,1)^n = \mathcal{X}$ ist dann die Likelihood-Funktion
    \begin{equation*}
        f(\theta, x_1, \dots, x_n) = \frac{1}{\theta^n} \left(\prod_{i = 1}^n x_i\right)^{\frac{1-\theta}{\theta}}
    \end{equation*}
    und ihre Ableitung
    \begin{equation*}
        \frac{\partial}{\partial \theta} f(\theta, x_1, \dots, x_n) = - \left(\prod_{i = 1}^n x_i\right)^{\frac{1-\theta}{\theta}} \frac{1}{\theta^{n+2}} \left( n \theta + \sum_{i = 1}^{n} \log(x_i) \right)
    \end{equation*}
    Diese hat die Nullstelle $n \theta = - \sum_{i = 1}^{n} \log(x_i)$, also ist der ML-Schätzer gleich
    \begin{equation*}
        \hat{\theta}^{ML}(x_1, \dots, x_n) = \frac{1}{n} \sum_{i = 1}^{n} \log(1/x_i)
    \end{equation*}
    \item Der Erwartungswert von $X_1$ beträgt
    \begin{equation*}
        \int_0^1 x \cdot \frac{1}{\theta} x^{\frac{1 - \theta}{\theta}} dx = \frac{1}{\theta} \int_{0}^{1}  x^{\frac{1}{\theta}} dx = \frac{1}{\theta} \left[ \frac{1+\theta}{\theta} x^{\frac{\theta}{1+\theta}} \right]_0^1 = \frac{1}{1 + \theta} 
    \end{equation*}
    und die Gleichsetzung $\frac{1}{1 + \theta} = \overline{x} = \frac{1}{n} \sum_{i =1}^{n} x_i$ liefert den Schätzer 
    \begin{equation*}
        \hat{\theta}(x_1, \dots, x_n) = \frac{n}{\sum_{i = 1}^{n} x_i} - 1
    \end{equation*}
    \item Der ML-Schätzer liefert hier
    \begin{equation*}
        \hat{\theta}^{ML}(0.1, 0.22, 0.54, 0.36) = \frac{1}{4} \log\left(\frac{1}{0.1 \cdot 0.22 \cdot 0.54 \cdot 0.36}\right) \approx 1.36
    \end{equation*}
    und aus dem Momentenschätzer erhält man
    \begin{equation*}
        \hat{\theta}(0.1, 0.22, 0.54, 0.36) = \frac{4}{0.1 + 0.22 + 0.54 + 0.36} -1 = \frac{4}{1.22} - 1 \approx 2.27
    \end{equation*}
\end{enumerate}



\end{document}