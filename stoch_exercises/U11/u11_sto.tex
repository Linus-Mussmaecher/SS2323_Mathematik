\documentclass[a4paper]{article}

% --- DATA ---

\def\lecture{Stochastik 2}
\def\authors{Linus Mußmächer}
\def\sheetNumber{11}
%\def\sumPoints{30} 

% --- PREAMBLE ---

\usepackage[german]{babel}	% language specific quotation marks etc.
% === USAGE ===

% when using this preamble, setup your environment variables like this beforehand:


% \title{Stochastik 2}  %Title of exercise 
% \def\lecture{Stochastik 2}
% \def\authors{Linus Mußmächer}
% \def\sheetNumber{02}
% \def\sumPoints{30}      % maximum number of points (leave undefined)

% then use one of these commands (german or english) to print the header:

% \makeexheaderger

% and finally use subsections for your subtasks - they will be numbered as <sheetNumber><task number> by themselves

% if you have an exercise as an external .pdf, use \includetask to include it and increase the task counter


% --- OTHER ---

\usepackage{booktabs}       % professional-quality tables
\usepackage[table]{xcolor}	% color
\usepackage{pdfpages}		% to include entire pdf pages in appendix etc.
\usepackage{enumitem}		% better custom enumerations
\setlist[enumerate, 1]{label=(\roman*)}
\usepackage{etoolbox}		% toolbox for command modification

% --- FONTS & TYPESETTING ---

\usepackage[utf8]{inputenc} % allow utf-8 input
\usepackage[T1]{fontenc}    % use 8-bit T1 fonts
\usepackage{dsfont}			% font with double lines for sets
\usepackage[german,ruled,vlined,linesnumbered,commentsnumbered,algoruled]
{algorithm2e} 				%pseudo code
\usepackage{listings}		%java code
\usepackage{csquotes}

% --- URLS ---

\usepackage[colorlinks=true, linkcolor=black, citecolor=blue, urlcolor=blue]{hyperref}   	% hyperlinks
\usepackage{url}            % simple URL typesetting

% --- MATH SYMBOLS ---

\usepackage{amsmath,amssymb}% more math symbols
\usepackage{amsfonts}       % blackboard math symbols
\usepackage{latexsym}		% more math symbols
\usepackage{chngcntr}		% more math symbols
\usepackage{mathrsfs}		% math-fonts
\usepackage{mathtools}		% more math symbols
\usepackage{nchairx}		% Waldmann package for general math symbols

% --- GRAPHICS & CAPTIONS ----

\usepackage{graphicx}		% including images
\graphicspath{ {./figs/} }
\usepackage{subcaption}		% custom caption formatting
\DeclareCaptionLabelFormat{custom}{ \textbf{#1 #2}}
\captionsetup{format=hang}
\captionsetup{width=0.9\textwidth,labelformat=custom}
\usepackage{pdfpages}		% to include entire pdf pages in appendix etc.

% --- FORMAT ---

\usepackage[a4paper]{geometry} % a4 paper
\usepackage{setspace}		% spacing
\usepackage{titlesec}
\allowdisplaybreaks			% allow page breaks within math environments

% --- CUSTOM COMMANDS ---
%Logic
\newcommand{\then}{\Rightarrow}
\newcommand{\since}{\Leftarrow}
\renewcommand{\iff}{\ensuremath{\Leftrightarrow}}

%pretty epsilon
\let\oldepsilon\epsilon
\let\epsilon\varepsilon
\let\varepsilon\oldepsilon
%pretty phi
\let\oldphi\phi
\let\phi\varphi
\let\varphi\oldphi

\newcommand{\includetask}[2][pages=-]{
    \includepdf[#1]{#2}
    \addtocounter{subsection}{1}
}

% set-up for exercise specific stuff
\ifdef{\sheetNumber}{
    \setcounter{section}{\sheetNumber}
}{}

\usepackage{titling}
\newcommand{\makeexheaderger}{
    \begin{doublespace}
        \begin{center}
            \textbf{\Large{Übungsblatt \sheetNumber}}\\
            \textbf{\Large\lecture}\\
            Abgabe von: \textbf{\authors}\\
            \today
        \end{center}
        \ifdef {\sumPoints}
        {
            \hfill  \large Punkte: $\boxed{\qquad  /\; \sumPoints}$\\
        }{}
    \end{doublespace}
}

\newcommand{\makeexheadereng}{
    \begin{doublespace}
        \begin{center}
            \textbf{\Large{Exercise Sheet \sheetNumber}}\\
            \textbf{\Large\lecture}\\
            Abgabe von: \textbf{\authors}\\
            \today
        \end{center}
        \ifdef {\sumPoints}
        {
            \hfill  \large Points: $\boxed{\qquad  /\; \sumPoints}$\\
        }{}
    \end{doublespace}
}

% --- DOCUMENT ---

\begin{document}

\makeexheader

\subsection{Zentralübung}

\begin{enumerate}
    \item Die gemeinsame Dichte der $n$ Zufallsvariablen $X_1, \dots, X_n$ ist
    \begin{equation*}
        f_\theta(x_1, \dots, x_n) =  \prod_{i = 1}^n \frac{1}{\theta} x_i^{\frac{1 - \theta}{\theta}} \mathds{1}_{(0,1)}(x_i)
    \end{equation*}
    Für $(x_1, \dots, x_n) \in (0,1)^n = \mathcal{X}$ ist dann die Likelihood-Funktion
    \begin{equation*}
        f(\theta, x_1, \dots, x_n) = \frac{1}{\theta^n} \left(\prod_{i = 1}^n x_i\right)^{\frac{1-\theta}{\theta}}
    \end{equation*}
    und ihre Ableitung
    \begin{equation*}
        \frac{\partial}{\partial \theta} f(\theta, x_1, \dots, x_n) = - \left(\prod_{i = 1}^n x_i\right)^{\frac{1-\theta}{\theta}} \frac{1}{\theta^{n+2}} \left( n \theta + \sum_{i = 1}^{n} \log(x_i) \right)
    \end{equation*}
    Diese hat die Nullstelle $n \theta = - \sum_{i = 1}^{n} \log(x_i)$, also ist der ML-Schätzer gleich
    \begin{equation*}
        \hat{\theta}^{ML}(x_1, \dots, x_n) = \frac{1}{n} \sum_{i = 1}^{n} \log(1/x_i)
    \end{equation*}
    \item Der Erwartungswert von $X_1$ beträgt
    \begin{equation*}
        \int_0^1 x \cdot \frac{1}{\theta} x^{\frac{1 - \theta}{\theta}} dx = \frac{1}{\theta} \int_{0}^{1}  x^{\frac{1}{\theta}} dx = \frac{1}{\theta} \left[ \frac{1+\theta}{\theta} x^{\frac{\theta}{1+\theta}} \right]_0^1 = \frac{1}{1 + \theta} 
    \end{equation*}
    und die Gleichsetzung $\frac{1}{1 + \theta} = \overline{x} = \frac{1}{n} \sum_{i =1}^{n} x_i$ liefert den Schätzer 
    \begin{equation*}
        \hat{\theta}(x_1, \dots, x_n) = \frac{n}{\sum_{i = 1}^{n} x_i} - 1
    \end{equation*}
    \item Der ML-Schätzer liefert hier
    \begin{equation*}
        \hat{\theta}^{ML}(0.1, 0.22, 0.54, 0.36) = \frac{1}{4} \log\left(\frac{1}{0.1 \cdot 0.22 \cdot 0.54 \cdot 0.36}\right) \approx 1.36
    \end{equation*}
    und aus dem Momentenschätzer erhält man
    \begin{equation*}
        \hat{\theta}(0.1, 0.22, 0.54, 0.36) = \frac{4}{0.1 + 0.22 + 0.54 + 0.36} -1 = \frac{4}{1.22} - 1 \approx 2.27
    \end{equation*}
\end{enumerate}

\subsection{}

Wir setzen das empirische erste und zweite Moment mit dem ersten und zweiten Moment der Log-Normalverteilung gleich (man beachte, dass $n$ in der Aufgabenstellung doppelt verwendet wurde, als Anzahl der $Y_i$ und zur Indizierung der Momente. Wir übernehmen nur die erste Verwendung):
\begin{align*}
    \hat{\mu}_1 = \frac{1}{n} \sum_{i = 1}^{n} x_i &= \exp\left(\mu + \frac{\sigma^2}{2}\right) \iff \mu + \frac{1}{2} \sigma^2 = \log\left( \frac{1}{n} \sum_{i = 1}^{n} x_i \right) \tag{I}\\
    \hat{\mu}_2 = \frac{1}{n} \sum_{i = 1}^{n} x_i^2 &= \exp\left( 2 \mu + 2 \sigma^2 \right) \iff 2\mu + 2 \sigma^2 = \log\left( \frac{1}{n} \sum_{i=1}^{n} x_i^2 \right) \tag{II}
\end{align*}
$(II) - 2 (I)$ liefert 
\begin{align*}
    \sigma^2 &= \log\left( \frac{1}{n} \sum_{i=1}^{n} x_i^2 \right) - \log\left(\left( \frac{1}{n} \sum_{i = 1}^{n} x_i \right)^2 \right) = \log \left( \frac{n \sum_{i=1}^{n} x_i^2}{\left(\sum_{i=1}^{n}x_i\right)^2}\right)\\
    &= \log(\hat{\mu}_2) - \log(\hat{\mu}_1^2) = \log\left(\frac{\hat{\mu}_2}{\hat{\mu}_1^2}\right)  
\end{align*}
und damit einen Schätzer für $\sigma^2$; und aus $2 (I) - 4 (II)$ erhalten wir 
\begin{align*}
    \mu &= \log\left( \left( \frac{1}{n} \sum_{i = 1}^{n} x_i \right)^2 \right) - \log\left(\left( \frac{1}{n} \sum_{i=1}^{n} x_i^2 \right)^4\right) = 2 \log \left( \frac{n \sum_{i = 1}^{n} x_i}{\left(\sum_{i=1}^{n} x_i^2 \right)^2} \right)\\
    &= \log(\hat{\mu}_1^2) - \log(\hat{\mu}_2^4) = \log\left(\frac{\hat{\mu}_1^2}{\hat{\mu}_2^4}\right) = 2 \log\left(\frac{\hat{\mu}_1}{\hat{\mu}_2^2}\right)
\end{align*}

\subsection{}

\begin{enumerate}
    \item Wir bestimmen die Ableitung durch die Produktregel:
    \begin{align*}
        \frac{\partial}{\partial \lambda} F_x(\lambda) &= e^{- \lambda} \sum_{k = 0}^{x} \frac{k \cdot \lambda^{k-1}}{k!} + (-1) \exp(- \lambda) \sum_{k = 0}^{x} \frac{\lambda^{k-1}}{k!} = e^{- \lambda} \sum_{k = 0}^{x-1} \frac{\lambda^{k}}{k!} - \exp(- \lambda) \sum_{k = 0}^{x} \frac{\lambda^{k-1}}{k!}\\
        &= - \underbrace{e^{-\lambda}}_{> 0} \underbrace{\frac{\lambda^{x}}{x!}}_{\geq 0} < 0
    \end{align*}
    und somit ist $F_x(\lambda)$ für alle $x \in \mathds{N}_0$ streng monoton steigend auf $(0, \infty)$.
    Weiterhin halten wir fest, dass $F_x(\lambda) = \mathds{P}_\lambda\{ X \leq x \}$ für eine zum Parameter $\lambda$ Poisson-verteilte Zufallsvariable $X$ gilt.
    \item Wir wollen Satz 4.16 verwenden. Dann gilt:
    \begin{align*}
        n_r &= \min \left( n_0 \in \mathds{N_0} \mid \sup_{\lambda \leq \lambda_0} \mathds{P}_{\lambda}(\left\{n \in \mathds{N}_0 \mid n \geq n_0 \right\} ) \leq \alpha  \right) \\
        &=\min \left( n_0 \in \mathds{N_0} \mid \sup_{\lambda \leq \lambda_0} 1 - \mathds{P}_{\lambda}(\left\{n \in \mathds{N}_0 \mid n \leq n_0+1 \right\} ) \leq \alpha  \right) \\
        &=\min \left( n_0 \in \mathds{N_0} \mid \sup_{\lambda \leq \lambda_0} 1 - F_{n_0 + 1}(\lambda) \leq \alpha  \right) \quad \text{| mit } F_{n_0 + 1} \text{ s.m.f.}\\
        &=\min \left( n_0 \in \mathds{N_0} \mid 1 - F_{n_0 + 1}(\lambda_0) \leq \alpha  \right) =\min \left( n_0 \in \mathds{N_0} \mid F_{n_0 + 1}(\lambda_0) \geq 0.95  \right)\text{.}
    \end{align*}
    Da $F_{n_0 + 1}(\lambda_0)$ in $n_0$ streng monoton steigend ist, müssen wir lediglich eine Zahl $r \in \mathds{R}$ mit $F_{r + 1}(\lambda_0) = 0.95$ bestimmen und aufrunden, um $n_r = \lceil r \rceil$ zu erhalten.
    Dies ist allerdings nur numerisch möglich. 

    Für dieses $n_r$ setzen wir dann $T = \mathds{1}_{[n_r, \infty]}$ als unseren Test.
    Nach Satz 4.16 hat er Niveau $\alpha$.
    \item Für $\lambda_0 = 10$ bestimmen wir numerisch $F_{r + 1}(10) = 0.95 \iff r \approx 13.95$, also wählen wir $n_r = 14$ und erhalten einen Verwerfungsbereich von $[0, 13]$.
    
    Für $\lambda_0 = 1$ erhalten wir analog $F_{r+1}(1) = 0.95 \iff x \approx 1.35$, also $n_r = 2$ und damit den Verwerfungsbereich $\{0,1\}$.
\end{enumerate}

\subsection{}

\begin{enumerate}
    \item Wir setzen $\mathcal{X} = \{0,1\}$ (Antwort falsch, Antwort richtig) und betrachten $\mathcal{X}^4$ mit der Statistik $S: \mathcal{X}^4 \to \mathds{R}, (x_1, x_2, x_3, x_4) \mapsto \sum_{k = 1}^{4} x_k$.
    Diese ist dann binomialverteilt zu einem Parameter $p \in [0,1] = \Theta$, d.h. $S \sim \mathrm{Bin}_{4, p}$.

    Als Nullhypothese könnten wir nun $H_0: p \in \Theta_0 = [0, 1/4]$, d.h. der Kandidat ist unqualifiziert und rät nur.
    Kurioserweise hat diese Versicherung die interessante Definition von ''qualifiziert'' als ''besser als reines Raten''.
    Dies ist nicht anspruchsvoll, und ich möchte mich bitte anderswo versichern lassen.

    In diesem Beispiel wäre dann die Gegenhypothese $K: p \in (1/4, 1]$ und ein Test wäre eine Abbildung $T_4: \mathcal{X}^4 \to \{0,1\}$.
    Ein solcher Test ließe sich beispielsweise nach Wahl eines Signifikanzniveaus mitttels Satz 4.16 konstruieren. Möchte man ein Signifikanzniveaus $\alpha = 0$ sicherstellen, d.h. unter keinen Umständen einen unqualifizierten Bewerber einstellen, so bietet sich die Wahl von $T_4(x_1, x_2, x_3, x_4) = 1 \iff x_k = 1 \ \forall_{k = 1, \dots, 4}$ an.
    \item Beim Fehler erster Art wird die Nullhypothese verworfen (d.h. der Kandidat als qualifiziert eingestuft und eingestellt), obwohl sie eigentlich erfüllt ist (der Kandidat rät nur). 
    Im Fall, dass der Kandidat wirklich nur rät ist die Wahrscheinlichkeit hierfür laut Bernoulli
    \begin{equation*}
        \mathds{P}(S \geq 3) = p_{4, 1/4}(3) + p_{4, 1/4}(4) = \binom{4}{3} \frac{1}{4^3} \frac{3}{4} + \frac{1}{4^4} = \frac{13}{255} \approx 0.051\text{.}
    \end{equation*}
    Falls der Kandidat so wenig Ahnung hat, dass er des Öfteren falsche Antworten für richtig hält, dann ist die Wahrscheinlichkeit noch kleiner.
    \item Wie wir in (b) gesehen haben, dass beim Verlangen dreier richtiger Antworten immer noch eine Chance $> 5\%$ besteht, dass ein nur-ratender Kandidat akzeptiert wird, muss ein Test zum Signifikanzniveau $0.05$ schärfer sein, d.h. wir müssen $T$ wie in (a) wählen als $T(x_1, x_2, x_3, x_4) = 1 \iff x_k = 1 \ \forall_{k = 1, \dots, 4}$.
    \item Dies entspricht der Wahrscheinlichkeit dafür, dass der Statistiker keine, eine oder zwei Antworten korrekt beantwortet bei $p = 0.75$. Bernoulli:
    \begin{equation*}
        \mathds{P}(S < 3) = p_{4, 3/4}(0) + p_{4, 3/4}(1) + p_{4, 3/4}(2) = \frac{1}{4}^4 + 4 \cdot \frac{3}{4} \cdot \frac{1}{4}^3 + \binom{4}{2} \frac{3}{4}^2 \frac{1}{4}^2 = \frac{31}{256} \approx 0.121\text{.}
    \end{equation*}
    Etwas mehr Vorbereitung kann also nicht schaden.
\end{enumerate}



\end{document}