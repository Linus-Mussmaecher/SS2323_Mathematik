\documentclass[a4paper]{article}

% --- LANGUAGE ---

\usepackage[german]{babel}	% language specific quotation marks etc.

% --- DATA ---

\def\lecture{Repetitorium zu Differenzialgleichungen}
\def\authors{Linus Mußmächer}
\def\sheetNumber{01}
\def\sumPoints{30} 

% --- PREAMBLE ---
% === USAGE ===

% when using this preamble, setup your environment variables like this beforehand:


% \title{Stochastik 2}  %Title of exercise 
% \def\lecture{Stochastik 2}
% \def\authors{Linus Mußmächer}
% \def\sheetNumber{02}
% \def\sumPoints{30}      % maximum number of points (leave undefined)

% then use one of these commands (german or english) to print the header:

% \makeexheaderger

% and finally use subsections for your subtasks - they will be numbered as <sheetNumber><task number> by themselves

% if you have an exercise as an external .pdf, use \includetask to include it and increase the task counter


% --- OTHER ---

\usepackage{booktabs}       % professional-quality tables
\usepackage[table]{xcolor}	% color
\usepackage{pdfpages}		% to include entire pdf pages in appendix etc.
\usepackage{enumitem}		% better custom enumerations
\setlist[enumerate, 1]{label=(\roman*)}
\usepackage{etoolbox}		% toolbox for command modification

% --- FONTS & TYPESETTING ---

\usepackage[utf8]{inputenc} % allow utf-8 input
\usepackage[T1]{fontenc}    % use 8-bit T1 fonts
\usepackage{dsfont}			% font with double lines for sets
\usepackage[german,ruled,vlined,linesnumbered,commentsnumbered,algoruled]
{algorithm2e} 				%pseudo code
\usepackage{listings}		%java code
\usepackage{csquotes}

% --- URLS ---

\usepackage[colorlinks=true, linkcolor=black, citecolor=blue, urlcolor=blue]{hyperref}   	% hyperlinks
\usepackage{url}            % simple URL typesetting

% --- MATH SYMBOLS ---

\usepackage{amsmath,amssymb}% more math symbols
\usepackage{amsfonts}       % blackboard math symbols
\usepackage{latexsym}		% more math symbols
\usepackage{chngcntr}		% more math symbols
\usepackage{mathrsfs}		% math-fonts
\usepackage{mathtools}		% more math symbols
\usepackage{nchairx}		% Waldmann package for general math symbols

% --- GRAPHICS & CAPTIONS ----

\usepackage{graphicx}		% including images
\graphicspath{ {./figs/} }
\usepackage{subcaption}		% custom caption formatting
\DeclareCaptionLabelFormat{custom}{ \textbf{#1 #2}}
\captionsetup{format=hang}
\captionsetup{width=0.9\textwidth,labelformat=custom}
\usepackage{pdfpages}		% to include entire pdf pages in appendix etc.

% --- FORMAT ---

\usepackage[a4paper]{geometry} % a4 paper
\usepackage{setspace}		% spacing
\usepackage{titlesec}
\allowdisplaybreaks			% allow page breaks within math environments

% --- CUSTOM COMMANDS ---
%Logic
\newcommand{\then}{\Rightarrow}
\newcommand{\since}{\Leftarrow}
\renewcommand{\iff}{\ensuremath{\Leftrightarrow}}

%pretty epsilon
\let\oldepsilon\epsilon
\let\epsilon\varepsilon
\let\varepsilon\oldepsilon
%pretty phi
\let\oldphi\phi
\let\phi\varphi
\let\varphi\oldphi

\newcommand{\includetask}[2][pages=-]{
    \includepdf[#1]{#2}
    \addtocounter{subsection}{1}
}

% set-up for exercise specific stuff
\ifdef{\sheetNumber}{
    \setcounter{section}{\sheetNumber}
}{}

\usepackage{titling}
\newcommand{\makeexheaderger}{
    \begin{doublespace}
        \begin{center}
            \textbf{\Large{Übungsblatt \sheetNumber}}\\
            \textbf{\Large\lecture}\\
            Abgabe von: \textbf{\authors}\\
            \today
        \end{center}
        \ifdef {\sumPoints}
        {
            \hfill  \large Punkte: $\boxed{\qquad  /\; \sumPoints}$\\
        }{}
    \end{doublespace}
}

\newcommand{\makeexheadereng}{
    \begin{doublespace}
        \begin{center}
            \textbf{\Large{Exercise Sheet \sheetNumber}}\\
            \textbf{\Large\lecture}\\
            Contributors: \textbf{\authors}\\
            \today
        \end{center}
        \ifdef {\sumPoints}
        {
            \hfill  \large Points: $\boxed{\qquad  /\; \sumPoints}$\\
        }{}
    \end{doublespace}
}

\newcommand{\qmatrix}[1]{\ensuremath{\left(\begin{matrix}#1\end{matrix}\right)}}

\begin{document}

\makeexheader

\subsection{Trennung der Variablen}

Es sind $g: (-\infty, 0) \to \mathds{R}, x \mapsto \frac{1}{x}$ sowie $h: (-\infty, 0) \to \mathds{R}, y \mapsto \frac{y^2+1}{2y}$ stetige Funktionen auf offenen, nicht-leeren reellen Intervallen. Damit handelt es sich bei
\begin{equation*}
	y'(x) = \frac{y^2 + 1}{2xy} = g(x) \cdot h(y) \qquad y(x_0) = y_0
\end{equation*}
um eine Differenzialgleichung mit getrennten Variablen, die demnach eine eindeutige Lösung besitzt. Um diese zu bestimmen, berechnen wir zuerst die Integrale
\begin{align*}
	\int_{y_0}^{y(x)} \frac{2s}{s^2 + 1} ds & = \left[ \ln(|s^2 + 1|) \right]_{y_0}^{y(x)}   & \text{da } \frac{d}{ds} s^2 + 1 = 2s \\
	                                        & = \ln(|y(x)^2 + 1|) - \ln(|y_0^2 + 1|)                                                \\
	                                        & = \ln\left(\frac{y(x)^2 + 1}{y_0^2 + 1}\right)
\end{align*}
sowie
\begin{align*}
	\int_{x_0}^{x} \frac{1}{\tau} d\tau & = \ln(|x|) - \ln(|x_0|) \ln\left(\frac{x}{x_0}\right)
\end{align*}
wobei wir die Beträge weglassen können, da $y(x)^2 + 1$ sowie $y_0^2 + 1$ sicher positiv und $x,x_0$ beide negativ sind. Gleichsetzen und Auflösen nach $y(x)$ liefert:
\begin{align*}
	       & \ln\left(\frac{y(x)^2 + 1}{y_0^2 + 1}\right) = \ln\left(\frac{x}{x_0}\right) \\
	\iff\  & \frac{y(x)^2 + 1}{y_0^2 + 1} = \frac{x}{x_0}                                 \\
	\iff\  & y(x) = - \sqrt{\frac{x}{x_0} (y_0^2 + 1) - 1}\text{,}
\end{align*}
wobei wir aufgrund von $y_0 < 0$ den negativen Ast wählen. Diese Lösung existiert, solange das Argument der Wurzel nicht-negativ bleibt, also falls
\begin{equation*}
	\frac{x}{x_0} (y_0^2 + 1) \geq 1 \iff x \leq \frac{x_0}{y_0^2 + 1}\text{.}
\end{equation*}
Das maximale Existenzintervall ist also $\left(-\infty, \frac{x_0}{y_0^2 + 1}\right]$ und
\begin{equation*}
	y: \left(-\infty, \frac{x_0}{y_0^2 + 1}\right] \to \mathds{R}, x \mapsto  -\sqrt{\frac{x}{x_0} (y_0^2 + 1) - 1}
\end{equation*}
ist die maximale eindeutige Lösung des Anfangswertproblems.

\subsection{Beweisen und Widerlegen}

\begin{enumerate}
	\item Diese Aussage ist falsch, nach Picard-Lindelöf existiert zwar eine (eindeutige) maximale Lösung, diese muss aber nicht notwendigerweise auf ganz $(-1, 1)$ definiert sein. Betrachte dazu das folgende Anfangswertproblem:
	      \begin{equation*}
		      x' = x^2 \qquad x(0) = x_0
	      \end{equation*}
	      $f$ ist hier stetig und stetig differenzierbar, damit lokal lipschitzstetig, und genügt damit den Anforderungen. Separation der Variablen liefert hier die Lösung $\phi(t) = \frac{x_0}{1 - x_0 t}$. Diese ist insbesondere an der Stelle $\frac{1}{x_0}$ nicht definiert, wählen wir also beispielsweise $x_0 = 5$ so existiert $\phi$ nur auf dem Teilintervall $(-1, 1/5)$, nicht auf ganz $(-1,1)$.
	\item Die Aussage ist korrekt. Die Funktion $f(t,x) = \exp(15 t) \cos(x(t)^7)$ ist auf ganz $\mathds{R} \times \mathds{R}$ definiert (und dort stetig und bezüglich $x$ lokal lipschitzstetig). Wegen $|cos(x)| \leq 1$ ist $f$ außerdem linear beschränkt, d.h. $|f(t,x)| \leq \exp(15 t)$ (man wähle in Satz 3.9 $a(t)  = 0$ und $b(t) = \exp(15 t)$). Satz 3.9 zeigt, dass dann jede Lösung der DGL auf ganz $\mathds{R}$ existiert.
\end{enumerate}

\subsection{Autonome Lösungen}
\begin{enumerate}
	\item Die rechte Seite der DGL ist ein Polynom (in der Variablen $f$) und daher stetig und stetig differenzierbar, also lokal lipschitzstetig. Außerdem handelt es sich um eine autonome Differenzialgleichung. Weiterhin ist $\mathds{R}$ ein offenes Intervall und $\mathds{R} \subseteq \mathds{R}^1$ eine offene Teilmenge. Somit sind die Voraussetzungen des globalen Existenz- und Eindeutigkeitssatzes von Picard-Lindelöf erfüllt und für alle $(t_0, f_0) \in \mathds{R} \times \mathds{R}$, insbesondere also für alle $(0, f_0)$, existiert eine maximale eindeutige Lösung des geforderten Anfangswertproblems.
	\item Da es sich um eine autonome Differenzialgleichung handelt, dürfen sich die Trajektorien der (eindeutigen) Lösungen zu verschiedenen Anfangswerten nicht schneiden. Da die rechte Seite der Differenzialgleichungen Nullstellen bei $-1, 0, 1$ hat existieren weiterhin konstante Lösungen in diesen Ruhelagen, d.h. insbesondere ist $c_1(t) = 1$ eine Lösung der Differenzialgleichungen zum Anfangswert $f_0 = 1$. Eine Lösung $f$ zu einem Anfangswert $f_0 < 1$ darf daher $c_1$ nicht schneiden, insbesondere darf daher kein $t$ im Definitionsbereich von $f$ existieren mit $f(t) = 1$. Wegen $f(0) = f_0 < 1$ und dem Zwischenwertsatz schließt dies auch die Existenz eines $t$ im Definitionsbereich von $f$ mit $f(t) > 1$ aus.
	\item Wir zeigen zuerst, dass $f$ nach oben unbeschränkt ist. Falls der maximale Definitionsbereich von $f$ kleiner $\infty$ ist, so folgt dies direkt aus dem Satz über das Verhalten der Lösungen im Großen ($\mathds{R}^2$ hat keinen Rand).
	Ist $f$ auf ganz $\mathds{R}^2$ definiert, so können wir folgern: 
	$f > 1$ und daher $f' = f^3 - f > 0$ sowie $f'' = 3 f^2 - 1 > 2$, d.h. ist linksgekrümmt und streng monoton steigend, also $f \to \infty$ für $t \to \infty$.
	
	Sei nun $a > 1$ beliebig. Ist $a > f_0$, so folgt aus $f(0) = f_0$, der Unbeschränktheit von $f$ und dem Zwischenwertsatz die Existenz des gesuchten $t$ mit $f(t) = a$.

	Ist $1 < a < f_0$, so existiert eine eindeutige Lösung $g$ des Anfangswertproblems $f(0) = a$, die ebenfalls die obigen Eigenschaften erfüllt, d.h. es existiert ein $t_0$ im Definitionsbereich von $g$ mit $g(t_0) = f_0$. Dann ist aber aufgrund der Autonomie der Differenzialgleichung $f(t) = g(t - t_0)$ eine (d.h. die eindeutige) Lösung des Anfangswertproblems $f(0) = f_0$, diese Lösung ist also in $-t_0$ definiert und erfüllt dort $f(-t_0) = a$.

	Ist $a = f_0$, so ist nichts zu zeigen.

	Dies zeigt die Aussage für alle $a > 1$.
\end{enumerate}

\end{document}
















