\documentclass[a4paper]{article}

% --- LANGUAGE ---

\usepackage[german]{babel}	% language specific quotation marks etc.

% --- DATA ---

\def\lecture{Repetitorium zu Differenzialgleichungen}
\def\authors{Linus Mußmächer}
\def\sheetNumber{01}
\def\sumPoints{30} 

% --- PREAMBLE ---
% === USAGE ===

% when using this preamble, setup your environment variables like this beforehand:


% \title{Stochastik 2}  %Title of exercise 
% \def\lecture{Stochastik 2}
% \def\authors{Linus Mußmächer}
% \def\sheetNumber{02}
% \def\sumPoints{30}      % maximum number of points (leave undefined)

% then use one of these commands (german or english) to print the header:

% \makeexheaderger

% and finally use subsections for your subtasks - they will be numbered as <sheetNumber><task number> by themselves

% if you have an exercise as an external .pdf, use \includetask to include it and increase the task counter


% --- OTHER ---

\usepackage{booktabs}       % professional-quality tables
\usepackage[table]{xcolor}	% color
\usepackage{pdfpages}		% to include entire pdf pages in appendix etc.
\usepackage{enumitem}		% better custom enumerations
\setlist[enumerate, 1]{label=(\roman*)}
\usepackage{etoolbox}		% toolbox for command modification

% --- FONTS & TYPESETTING ---

\usepackage[utf8]{inputenc} % allow utf-8 input
\usepackage[T1]{fontenc}    % use 8-bit T1 fonts
\usepackage{dsfont}			% font with double lines for sets
\usepackage[german,ruled,vlined,linesnumbered,commentsnumbered,algoruled]
{algorithm2e} 				%pseudo code
\usepackage{listings}		%java code
\usepackage{csquotes}

% --- URLS ---

\usepackage[colorlinks=true, linkcolor=black, citecolor=blue, urlcolor=blue]{hyperref}   	% hyperlinks
\usepackage{url}            % simple URL typesetting

% --- MATH SYMBOLS ---

\usepackage{amsmath,amssymb}% more math symbols
\usepackage{amsfonts}       % blackboard math symbols
\usepackage{latexsym}		% more math symbols
\usepackage{chngcntr}		% more math symbols
\usepackage{mathrsfs}		% math-fonts
\usepackage{mathtools}		% more math symbols
\usepackage{nchairx}		% Waldmann package for general math symbols

% --- GRAPHICS & CAPTIONS ----

\usepackage{graphicx}		% including images
\graphicspath{ {./figs/} }
\usepackage{subcaption}		% custom caption formatting
\DeclareCaptionLabelFormat{custom}{ \textbf{#1 #2}}
\captionsetup{format=hang}
\captionsetup{width=0.9\textwidth,labelformat=custom}
\usepackage{pdfpages}		% to include entire pdf pages in appendix etc.

% --- FORMAT ---

\usepackage[a4paper]{geometry} % a4 paper
\usepackage{setspace}		% spacing
\usepackage{titlesec}
\allowdisplaybreaks			% allow page breaks within math environments

% --- CUSTOM COMMANDS ---
%Logic
\newcommand{\then}{\Rightarrow}
\newcommand{\since}{\Leftarrow}
\renewcommand{\iff}{\ensuremath{\Leftrightarrow}}

%pretty epsilon
\let\oldepsilon\epsilon
\let\epsilon\varepsilon
\let\varepsilon\oldepsilon
%pretty phi
\let\oldphi\phi
\let\phi\varphi
\let\varphi\oldphi

\newcommand{\includetask}[2][pages=-]{
    \includepdf[#1]{#2}
    \addtocounter{subsection}{1}
}

% set-up for exercise specific stuff
\ifdef{\sheetNumber}{
    \setcounter{section}{\sheetNumber}
}{}

\usepackage{titling}
\newcommand{\makeexheaderger}{
    \begin{doublespace}
        \begin{center}
            \textbf{\Large{Übungsblatt \sheetNumber}}\\
            \textbf{\Large\lecture}\\
            Abgabe von: \textbf{\authors}\\
            \today
        \end{center}
        \ifdef {\sumPoints}
        {
            \hfill  \large Punkte: $\boxed{\qquad  /\; \sumPoints}$\\
        }{}
    \end{doublespace}
}

\newcommand{\makeexheadereng}{
    \begin{doublespace}
        \begin{center}
            \textbf{\Large{Exercise Sheet \sheetNumber}}\\
            \textbf{\Large\lecture}\\
            Abgabe von: \textbf{\authors}\\
            \today
        \end{center}
        \ifdef {\sumPoints}
        {
            \hfill  \large Points: $\boxed{\qquad  /\; \sumPoints}$\\
        }{}
    \end{doublespace}
}

\begin{document}

\makeexheaderger

\subsection{Trennung der Variablen}

Es sind $g: (-\infty, 0) \to \mathds{R}, x \mapsto \frac{1}{x}$ sowie $h: (-\infty, 0) \to \mathds{R}, y \mapsto \frac{y^2+1}{2y}$ stetige Funktionen auf offenen, nicht-leeren reellen Intervallen. Damit handelt es sich bei
\begin{equation*}
	y'(x) = \frac{y^2 + 1}{2xy} = g(x) \cdot h(y) \qquad y(x_0) = y_0
\end{equation*}
um eine Differenzialgleichung mit getrennten Variablen, die demnach eine eindeutige Lösung besitzt. Um diese zu bestimmen, berechnen wir zuerst die Integrale
\begin{align*}
	\int_{y_0}^{y(x)} \frac{2s}{s^2 + 1} ds & = \left[ \ln(|s^2 + 1|) \right]_{y_0}^{y(x)}   & \text{da } \frac{d}{ds} s^2 + 1 = 2s \\
	                                        & = \ln(|y(x)^2 + 1|) - \ln(|y_0^2 + 1|)                                                \\
	                                        & = \ln\left(\frac{y(x)^2 + 1}{y_0^2 + 1}\right)
\end{align*}
sowie
\begin{align*}
	\int_{x_0}^{x} \frac{1}{\tau} d\tau & = \ln(|x|) - \ln(|x_0|) \ln\left(\frac{x}{x_0}\right)
\end{align*}
wobei wir die Beträge weglassen können, da $y(x)^2 + 1$ sowie $y_0^2 + 1$ sicher positiv und $x,x_0$ beide negativ sind. Gleichsetzen und Auflösen nach $y(x)$ liefert:
\begin{align*}
	       & \ln\left(\frac{y(x)^2 + 1}{y_0^2 + 1}\right) = \ln\left(\frac{x}{x_0}\right) \\
	\iff\  & \frac{y(x)^2 + 1}{y_0^2 + 1} = \frac{x}{x_0}                                 \\
	\iff\  & y(x) = - \sqrt{\frac{x}{x_0} (y_0^2 + 1) - 1}\text{,}
\end{align*}
wobei wir aufgrund von $y_0 < 0$ den negativen Ast wählen. Diese Lösung existiert, solange das Argument der Wurzel nicht-negativ bleibt, also falls
\begin{equation*}
	\frac{x}{x_0} (y_0^2 + 1) \geq 1 \iff x \leq \frac{x_0}{y_0^2 + 1}\text{.}
\end{equation*}
Das maximale Existenzintervall ist also $\left(-\infty, \frac{x_0}{y_0^2 + 1}\right]$ und
\begin{equation*}
	y: \left(-\infty, \frac{x_0}{y_0^2 + 1}\right] \to \mathds{R}, x \mapsto  -\sqrt{\frac{x}{x_0} (y_0^2 + 1) - 1}
\end{equation*}
ist die maximale eindeutige Lösung des Anfangswertproblems.

\subsection{Beweisen und Widerlegen}

\begin{enumerate}
	\item Diese Aussage ist falsch, nach Picard-Lindelöf existiert zwar eine (eindeutige) maximale Lösung, diese muss aber nicht notwendigerweise auf ganz $(-1, 1)$ definiert sein. Betrachte dazu das folgende Anfangswertproblem:
	      \begin{equation*}
		      x' = x^2 \qquad x(0) = x_0
	      \end{equation*}
	      $f$ ist hier stetig und stetig differenzierbar, damit lokal lipschitzstetig, und genügt damit den Anforderungen. Separation der Variablen liefert hier die Lösung $\phi(t) = \frac{x_0}{1 - x_0 t}$. Diese ist insbesondere an der Stelle $\frac{1}{x_0}$ nicht definiert, wählen wir also beispielsweise $x_0 = 5$ so existiert $\phi$ nur auf dem Teilintervall $(-1, 1/5)$, nicht auf ganz $(-1,1)$.
	\item Die Aussage ist korrekt. Die Funktion $f(t,x) = \exp(15 t) \cos(x(t)^7)$ ist auf ganz $\mathds{R} \times \mathds{R}$ definiert (und dort stetig und bezüglich $x$ lokal lipschitzstetig). Wegen $|cos(x)| \leq 1$ ist $f$ außerdem linear beschränkt, d.h. $|f(t,x)| \leq \exp(15 t)$ (man wähle in Satz 3.9 $a(t)  = 0$ und $b(t) = \exp(15 t)$). Satz 3.9 zeigt, dass dann jede Lösung der DGL auf ganz $\mathds{R}$ existiert.
\end{enumerate}

\end{document}
















