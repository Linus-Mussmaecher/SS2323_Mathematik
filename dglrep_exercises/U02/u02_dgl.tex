\documentclass[a4paper]{article}

% --- LANGUAGE ---

\usepackage[german]{babel}	% language specific quotation marks etc.

% --- DATA ---

\def\lecture{Repetitorium zu Differenzialgleichungen}
\def\authors{Linus Mußmächer}
\def\sheetNumber{02}
\def\sumPoints{30} 

% --- PREAMBLE ---
% === USAGE ===

% when using this preamble, setup your environment variables like this beforehand:


% \title{Stochastik 2}  %Title of exercise 
% \def\lecture{Stochastik 2}
% \def\authors{Linus Mußmächer}
% \def\sheetNumber{02}
% \def\sumPoints{30}      % maximum number of points (leave undefined)

% then use one of these commands (german or english) to print the header:

% \makeexheaderger

% and finally use subsections for your subtasks - they will be numbered as <sheetNumber><task number> by themselves

% if you have an exercise as an external .pdf, use \includetask to include it and increase the task counter


% --- OTHER ---

\usepackage{booktabs}       % professional-quality tables
\usepackage[table]{xcolor}	% color
\usepackage{pdfpages}		% to include entire pdf pages in appendix etc.
\usepackage{enumitem}		% better custom enumerations
\setlist[enumerate, 1]{label=(\roman*)}
\usepackage{etoolbox}		% toolbox for command modification

% --- FONTS & TYPESETTING ---

\usepackage[utf8]{inputenc} % allow utf-8 input
\usepackage[T1]{fontenc}    % use 8-bit T1 fonts
\usepackage{dsfont}			% font with double lines for sets
\usepackage[german,ruled,vlined,linesnumbered,commentsnumbered,algoruled]
{algorithm2e} 				%pseudo code
\usepackage{listings}		%java code
\usepackage{csquotes}

% --- URLS ---

\usepackage[colorlinks=true, linkcolor=black, citecolor=blue, urlcolor=blue]{hyperref}   	% hyperlinks
\usepackage{url}            % simple URL typesetting

% --- MATH SYMBOLS ---

\usepackage{amsmath,amssymb}% more math symbols
\usepackage{amsfonts}       % blackboard math symbols
\usepackage{latexsym}		% more math symbols
\usepackage{chngcntr}		% more math symbols
\usepackage{mathrsfs}		% math-fonts
\usepackage{mathtools}		% more math symbols
\usepackage{nchairx}		% Waldmann package for general math symbols

% --- GRAPHICS & CAPTIONS ----

\usepackage{graphicx}		% including images
\graphicspath{ {./figs/} }
\usepackage{subcaption}		% custom caption formatting
\DeclareCaptionLabelFormat{custom}{ \textbf{#1 #2}}
\captionsetup{format=hang}
\captionsetup{width=0.9\textwidth,labelformat=custom}
\usepackage{pdfpages}		% to include entire pdf pages in appendix etc.

% --- FORMAT ---

\usepackage[a4paper]{geometry} % a4 paper
\usepackage{setspace}		% spacing
\usepackage{titlesec}
\allowdisplaybreaks			% allow page breaks within math environments

% --- CUSTOM COMMANDS ---
%Logic
\newcommand{\then}{\Rightarrow}
\newcommand{\since}{\Leftarrow}
\renewcommand{\iff}{\ensuremath{\Leftrightarrow}}

%pretty epsilon
\let\oldepsilon\epsilon
\let\epsilon\varepsilon
\let\varepsilon\oldepsilon
%pretty phi
\let\oldphi\phi
\let\phi\varphi
\let\varphi\oldphi

\newcommand{\includetask}[2][pages=-]{
    \includepdf[#1]{#2}
    \addtocounter{subsection}{1}
}

% set-up for exercise specific stuff
\ifdef{\sheetNumber}{
    \setcounter{section}{\sheetNumber}
}{}

\usepackage{titling}
\newcommand{\makeexheaderger}{
    \begin{doublespace}
        \begin{center}
            \textbf{\Large{Übungsblatt \sheetNumber}}\\
            \textbf{\Large\lecture}\\
            Abgabe von: \textbf{\authors}\\
            \today
        \end{center}
        \ifdef {\sumPoints}
        {
            \hfill  \large Punkte: $\boxed{\qquad  /\; \sumPoints}$\\
        }{}
    \end{doublespace}
}

\newcommand{\makeexheadereng}{
    \begin{doublespace}
        \begin{center}
            \textbf{\Large{Exercise Sheet \sheetNumber}}\\
            \textbf{\Large\lecture}\\
            Abgabe von: \textbf{\authors}\\
            \today
        \end{center}
        \ifdef {\sumPoints}
        {
            \hfill  \large Points: $\boxed{\qquad  /\; \sumPoints}$\\
        }{}
    \end{doublespace}
}

\begin{document}

\makeexheader

\subsection{Erstes Integral (II.5)}

\begin{enumerate}[label=(\alph*)]
    \item Das äquivalente System hat die Form
    \begin{equation*}
        \qmatrix{x' \\ y'} = \qmatrix{y \\ -\cos(x)}
    \end{equation*}
    \item Ja, den $f(x,y)$ ist stetig differenzierbar und somit stetig und lokal lipschitzstetig. Der globale Existenz- und Eindeutigkeitssatz von Picard-Lindelöf liefert somit die Existenz einer maximalen, eindeutigen Lösung zu jedem Anfangswert.
    \item Ja, denn $f(x,y) = (y, -\cos(x))^T$ ist linear beschränkt mit $a(t) = 1$ und $b(t) = 1$:
    \begin{align*}
        \|f(x,y)\| &= \|(y, -\cos(x))^T\| \leq \| (y,0) \| + \|(0, \cos(x))\| = |y| + |\cos(x)|\\ &\leq 1 \cdot \sqrt{x^2 + y^2} + 1 = a(t) \cdot \|(x,y)^T\| + b(t)\text{.}
    \end{align*}
    Der Satz von der linear beschränkten rechten Seite zeigt nun, dass jede maximale Lösung global, also auf ganz $\mathds{R}$, existiert.
    \item Wir berechnen zuerst den Gradienten von $S$:
    \begin{equation*}
        \nabla S = \qmatrix{2 \cos (x) \\ 2 y} = 2 \cdot \qmatrix{\cos(x) \\ y}\text{.}
    \end{equation*}
    Dann folgt
    \begin{equation*}
        \langle \nabla S(x,y), f(x,y) \rangle = 2 \cdot (\cos(x) \cdot y + y \cdot (-\cos(x))) = 0
    \end{equation*}
    und da für jeden möglichen Anfangswert eine Lösung existiert handelt es sich somit um ein erstes Integral.
\end{enumerate}

\subsection{Mehr erste Integrale (II.7)}

\begin{enumerate}[label=(\alph*)]
    \item Es ist $E(x)$ nur von der Norm $\|x\|$ abhängig, d.h. es gilt $E(x) = \tilde{E}(\|x\|)$ für ein (differenzierbares) $\tilde{E}: \mathds{R^+_0} \to \mathds{R}$. Dann folgt für die $k$-te partielle Ableitung von $E$ am Punkt $x \neq 0$:
    \begin{align*}
        \frac{\partial}{\partial x_k} E (x) &= \frac{\partial}{\partial x_k} \tilde{E}(\|x\|) = \tilde{E}'(\|x\|) \frac{\partial}{\partial x_k} \|x\| = \tilde{E}'(\|x\|) \frac{\partial}{\partial x_k} \sqrt{\sum_{i = 0}^n x_i} \\
        &= \tilde{E}'(\|x\|) \frac{1}{2 \sqrt{\sum_{i = 0}^n x_i}}  \cdot 2 x_k =  \frac{\tilde{E}'(\|x\|)}{\|x\|} \cdot x_k
    \end{align*}
    und somit $\nabla E(x) = \frac{\tilde{E}'(\|x\|)}{\|x\|} x$. Dies zeigt 
    \begin{equation*}
        \langle \nabla E(x), \nu(x) \rangle = \frac{\tilde{E}'(\|x\|)}{\|x\|} \langle x, \nu(x) \rangle = 0
    \end{equation*}
    für alle $x \in \mathds{R}^n \setminus \{0\}$ nach Voraussetzung. Dies zeigt nach Satz 5.4, dass $E$ ein erstes Integral der DGL ist.
    \item Die Phasenkurven des Systems verlaufen innerhalb der Niveaumengen \textit{aller} Funktionen $E$, die die Bedingung aus (a) erfüllen, insbesondere den kreislinienförmigen Niveaumengen von $E(x) = \|x\|$. Man beachte, dass die Phasenkurven selbst nicht notwendigerweise Kreislinien sein müssen, nur Teilmengen solcher.
    Da die Niveaumengen jeder anderen Funktion $E$ mit Bedingungen aus (a) stets nur Kreisringe (also echte Obermengen der Kreislinien, die ja minimal dicke Kreisringe sind) sein können, ist keine genauere Bestimmung der Phasenkurven durch die Aussage (a) möglich.
\end{enumerate}

\subsection{Keine ersten Integrale (III.10)}

Wir betrachten zuerst die homogene Gleichung
\begin{equation*}
    y'(t) = \frac{6t}{1 + 3t^2} \cdot y \qquad y(0) = 2
\end{equation*}
Diese ist separabel, d.h. für eine Lösung $\phi: I \to \mathds{R}$ gilt
\begin{align*}
    & \int_{2}^{\phi(t)} \frac{1}{z} dz = \int_{0}^{t} \frac{6 \tau}{1 + 3 \tau^2} d\tau \\
    \iff \ & \ln|\phi(t)| - |\ln(2)| = \ln|1 + 3 t^2| - \ln|1 + 3 0^2| \\
    \iff \ & \phi(t) = 2 \cdot  (1 + 3 t^2)
\end{align*}
Da unser System skalar ist, bildet die Fundamentallösung $\phi(t) = 1 + 3 t^2$ bereits ein Fundamentalsystem (man wähle hierzu als Startwert nicht $2$, sondern ein beliebiges $\lambda \in \mathds{R}$ und erhält $\lambda (1 + 3t^2)$ als Lösungsraum).
Die Fundamentalmatrix ist damit eine $1 \times 1$-Matrix der Form $X(t) = (1 + 3t^2)$ und die Übergangsmatrix ist $X(t, \tau) = X(t) \cdot X(\tau)^{-1} = \left( \frac{1 + 3t^2}{1 + 3 \tau^2} \right)$.
Dies erlaubt uns eine Bestimmung der Lösung des inhomogenen Systems durch Variation der Konstanten:
\begin{align*}
    \phi(t, 0, 2) &= X(t) \left( X(0)^{-1} \cdot 2 + \int_{0}^{t} X(s)^{-1} \cdot 5 ds \right) \\
    &= (1 + 3t^2) \left( \frac{2}{1 + 30^2} + 5 \cdot \int_{0}^{t} \frac{1}{1 + 3s^2} ds \right)  & \text{Mit } (\frac{1}{1 + (\sqrt{3} s)^2}) \cdot \sqrt{3} = \arctan(\sqrt{3} t)'\\
    &= (1 + 3t^2) \left( 2 + 5 \cdot \int_{0}^{t} \frac{1}{1 + 3s^2} ds \right) \\
    &= (1 + 3t^2) \left( 2 + \frac{5}{\sqrt{3}} \cdot \left[\arctan(\sqrt{3} s)\right]_0^t\right) & \text{Mit } \arctan(0) = 0\\
    &= (1 + 3t^2) \left( 2 + \frac{5}{\sqrt{3}} \arctan(\sqrt{3}t) \right)
\end{align*}

\end{document}















