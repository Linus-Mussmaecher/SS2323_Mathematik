\documentclass[a4paper]{article}

% --- OTHER ---

\usepackage[german]{babel}	% german quotation marks etc.
\usepackage{booktabs}       % professional-quality tables
\usepackage[table]{xcolor}	% color
\usepackage{pdfpages}		% to include entire pdf pages in appendix etc.
\usepackage{enumitem}		% better custom enumerations
\setlist[enumerate, 1]{label=(\roman*)}
\usepackage{etoolbox}		% toolbox for command modification

% --- FONTS & TYPSETTING ---

\usepackage[utf8]{inputenc} % allow utf-8 input
\usepackage[T1]{fontenc}    % use 8-bit T1 fonts
\usepackage{dsfont}			% font with double lines for sets
\usepackage[german,ruled,vlined,linesnumbered,commentsnumbered,algoruled]
{algorithm2e} 				%Pseudocode
\usepackage{listings}		%JavaCode
\usepackage{csquotes}

% --- URLS ---

\usepackage[colorlinks=true, linkcolor=black, citecolor=blue, urlcolor=blue]{hyperref}   	% hyperlinks
\usepackage{url}            % simple URL typesetting

% --- MATH SYMBOLS ---

\usepackage{amsmath,amssymb}% more math symbols
\usepackage{amsfonts}       % blackboard math symbols
\usepackage{latexsym}		% more math symbols
\usepackage{chngcntr}		% more math symbols
\usepackage{mathrsfs}		% math-fonts
%\usepackage{marvosym}		% more math symbols (conflicts with nchairx)
\usepackage{mathktools}		% more math symbols
\usepackage{nchairx}		% Waldmann package for general math symbols and operators
% theorem formatting (contained in ncharx)
%\usepackage[amsmath,thmmarks,framed,thref]{ntheorem}

% --- GRAPHICS & CAPTIONS ----

\usepackage{graphicx}		% including images
\graphicspath{ {./figs/} }
\usepackage{subcaption}		% custom caption formating
\DeclareCaptionLabelFormat{custom}{ \textbf{#1 #2}}
\captionsetup{format=hang}
\captionsetup{width=0.9\textwidth,labelformat=custom}

% --- BIBLIOGRAPHY ---
%
%\usepackage[backend=biber, citestyle=numeric, bibstyle=numeric, sorting=none]{biblatex} 	% bibliography
%\bibliography{references}
%\defbibheading{myheading}[Literatur]{\section{#1}}

% --- TIKZ ---

\usepackage{tikz}			% basic tikz for custom images
\usetikzlibrary{cd}			% custom diagrams
\usetikzlibrary{external}	% extrnalize images for faster compilation
\tikzexternalize[prefix=figures/]
\AtBeginEnvironment{tikzcd}{\tikzexternaldisable} %fix cd/externalize
\AtEndEnvironment{tikzcd}{\tikzexternalenable}
\usepackage{pgfplots}		% custom plottin
\usepgfplotslibrary{colormaps}
\pgfplotsset{compat=newest}	
\usetikzlibrary{patterns}	% custom patterns

% --- FORMAT ---

\usepackage[a4paper]{geometry} % a4 paper
\usepackage{setspace}		% spacing
%\usepackage[nobottomtitles*]{titlesec} %prevent section titles from sometimes being on the bottom of a page
\usepackage{titlesec}
\allowdisplaybreaks			% allow pagebreaks within math environments
\doublespacing				% more spacing!
% pageskip every section
\let\oldsection\section
\renewcommand\section{\newpage\oldsection}

% --- DATA ---

\title{Exercises to Introduction to Operator Algebras}
\author{Alcides Buss\\Notes by: Linus Mußmächer\\2336440}
\date{Summer 2023}

% --- CUSTOM COMMANDS ---
%Logik
\newcommand{\then}{\Rightarrow}
\newcommand{\since}{\Leftarrow}
\renewcommand{\iff}{\ensuremath{\Leftrightarrow}}

%hübscheres epsilon
\let\oldepsilon\epsilon
\let\epsilon\varepsilon
\let\varepsilon\oldepsilon
%hübscheres phi
\let\oldphi\phi
\let\phi\varphi
\let\varphi\oldphi

%matrix
\newcommand{\qmatrix}[1]{\ensuremath{\left(\begin{matrix}#1\end{matrix}\right)}}

% --- DOCUMENT ---

\begin{document}

\maketitle


\tableofcontents

\newpage

\section{Exercise sheet 1}

\begin{exercise}
	(1)
\end{exercise}

\begin{proof}
	\textbf{Case 1}: If $b_1, b_2 \in A$, then $b_i = \alpha_i a$ for certain $\alpha_i \in \mathbb{C}$. Thus, $b_1 \cdot b_2 = \alpha_1 \alpha_2 a^2 = 0$. Thus, the multiplication is trivial. From this, it immediately follows that $\phi: \algebra{A} \to \algebra{M}, \lambda a \mapsto \qmatrix{0 & \lambda \\ 0 & 0}$ is an isomorphism.

	\textbf{Case 2}: $\lambda \neq 0$, and $a^2 = \lambda a$. Let $b = \frac{1}{\lambda} a$, then $b \cdot a = a = a \cdot b$. But then, for any $c = \mu a \in \algebra{A}$, we have $b c = \mu b a = \mu a = c = c b$, so the algebra is unital and isomorphic to $\mathbb{C}$.
\end{proof}

%Per sheet 3-4 tasks in the next week, two weeks => Bonus
%Nothing concerning the exam

\begin{exercise}[2]
	We consider pathological examples for $C_0(X)$.

	Let $X = \closure{\{x_0\}}$, e.g. $x_0 \in X$ with $\topology{O}(X) = \{\{x_0\} \cup Y \mid Y \subset X\} \cup \{\emptyset\}$. $X$ is highly non-Hausdorff unless we already have $X = \{x_0\}$.	In this space, the constant sequence $(x_0)$ converges to any $x \in X$.

	For a continouus function $f: X \to \mathds{C}$, this implies $f(x_0) \to f(x)$ for all $x \in X$, so every continouus function must already be constant. It follows that $C(X) \simeq \mathds{C}$.

	We now look at $C_0(X) = \{f \in C(X) \mid \forall_{\epsilon > 0} \{x \in X \mid  |f(x)| \geq \epsilon \} \text{ is compact.}\}$. But since all functions are constant, we can use $f(x_0)$ instead of $X$ and $\{x \in X \mid  |f(x)| \geq \epsilon \}$ is either empty or the whole space. $X$ is compact if and only if $X$ is finite. From here on, assume $X$ to be infinite. Then, only the finite subsets are compact. Thus, if we now have $f \not \equiv 0$, there exists an $|f(x_0)| > \epsilon > 0$ and thus $\{x \in X \mid  |f(x)| \geq \epsilon \} = X$ is not compact. This implies $C_0(X) = \{0\}$.

	To find a non-compact topological space that has non-zero unital $C_0(X)$, consider $X = X_0 \sqcup X_1$ with $X_0$ as befor and $X_1$ compact.
\end{exercise}

\begin{theorem}
	Let $\phi: \algebra{A} \to \algebra{B}$ be a $^*$-homomorphism betwenn $C^*$-algebras. Then we already have $\|\phi(a)\| \leq \|a\|$ for all $a \in \algebra{A}$.
\end{theorem}

\begin{exercise}[4 - Products]
	Let $(A_i)_{i \in I}$ be a family of $C^*$-algebras and define
	\begin{equation*}
		\prod_{i \in I} A_i = \{a = (a_i)_{i \in I} \mid a_i \in A_i \forall_{i \in I} \text{ and } \|a\| := \sup_{i \in I} \|a_i\| < \infty \}\text{.}
	\end{equation*}
	Addition, multiplication and involution are defined coordinatewise. We can prove that adding, multiplying and involving any bounded sequence yields another bounded sequence, so these are well-defined. We can also prove the $C^*$-axiom.
\end{exercise}

\begin{remark}[Differences between product and direct sum]~\\
	In addition to the product space, we define
	\begin{equation*}
		\bigoplus_{i \in I} A_i = \left\{(a_i) \in \prod_{i \in I} A_i \mid \forall_{\epsilon > 0} \exists_{\text{finite }F \subseteq I} \forall_{i \notin F} \|a_i\| < \epsilon\right\}\text{.}
	\end{equation*}
	This is a closed subspace of $\prod_{i \in I} A_i$ as the closure of $\bigoplus_{i \in I}^{alg} A_i$, where
	\begin{equation*}
		\bigoplus_{i \in I}^{alg} A_i = \left\{(a_i) \in \prod_{i \in I} A_i \mid \exists_{\text{finite }F \subseteq I} \forall_{i \notin F} \|a_i\| = 0\right\}\text{.}
	\end{equation*}
	For finite $I$, these are all equal. We see that any element in the direct sum can be approximated by a sequence of elements in the algebraic sum. This direct sum is a closed two-sided ideal in the product.

	The product has the following categorical universal property: We have \textbf{(surjective) $^*$-homomorphisms} $\pi_j:  \prod_{i \in I} A_i \to A_j$ for all $j \in I$. If $B$ is any $C^*$-algebra with $^*$-homomorphisms $\phi_j \to A_j$ for every $j \in I$, there is a unique $^*$-homomorphism $\phi: B \to \prod_{i \in I} A_i$ such that  $\pi_j \circ \phi = \phi_j$. This is equivalent to the commutativity of the following diagram:
	\begin{equation*}
		\begin{tikzcd}
			B \arrow{r}{\phi_j} \arrow{d}{\phi} & A_j\\
			A  \arrow[swap]{ur}{\pi_j}
		\end{tikzcd}
	\end{equation*}
\end{remark}

\begin{exercise}[5]
	$X$ is a locally compact Hausdorff space that can be written as $X = U \cup V$ with open and disjoint $U,V$ (so $U,V$ are clopen). We want to prove $C_0(X) \simeq C_0(U) \oplus C_0(V)$. To build this map, we map $f \mapsto (f|_U, f|_V)$. We check that this is well-defined and a $^*$-isomorphism.
\end{exercise}

\end{document}



























































