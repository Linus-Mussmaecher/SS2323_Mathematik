\documentclass[a4paper]{article}

% --- DATA ---

\def\lecture{Operator Algebras}
\def\authors{Linus Mußmächer, Minona Schäfer}
\def\sheetNumber{01}
%\def\sumPoints{30} 

% --- PREAMBLE ---

\usepackage[english]{babel}	% language specific quotation marks etc.
% === USAGE ===

% when using this preamble, setup your environment variables like this beforehand:


% \title{Stochastik 2}  %Title of exercise 
% \def\lecture{Stochastik 2}
% \def\authors{Linus Mußmächer}
% \def\sheetNumber{02}
% \def\sumPoints{30}      % maximum number of points (leave undefined)

% then use one of these commands (german or english) to print the header:

% \makeexheaderger

% and finally use subsections for your subtasks - they will be numbered as <sheetNumber><task number> by themselves

% if you have an exercise as an external .pdf, use \includetask to include it and increase the task counter


% --- OTHER ---

\usepackage{booktabs}       % professional-quality tables
\usepackage[table]{xcolor}	% color
\usepackage{pdfpages}		% to include entire pdf pages in appendix etc.
\usepackage{enumitem}		% better custom enumerations
\setlist[enumerate, 1]{label=(\roman*)}
\usepackage{etoolbox}		% toolbox for command modification

% --- FONTS & TYPESETTING ---

\usepackage[utf8]{inputenc} % allow utf-8 input
\usepackage[T1]{fontenc}    % use 8-bit T1 fonts
\usepackage{dsfont}			% font with double lines for sets
\usepackage[german,ruled,vlined,linesnumbered,commentsnumbered,algoruled]
{algorithm2e} 				%pseudo code
\usepackage{listings}		%java code
\usepackage{csquotes}

% --- URLS ---

\usepackage[colorlinks=true, linkcolor=black, citecolor=blue, urlcolor=blue]{hyperref}   	% hyperlinks
\usepackage{url}            % simple URL typesetting

% --- MATH SYMBOLS ---

\usepackage{amsmath,amssymb}% more math symbols
\usepackage{amsfonts}       % blackboard math symbols
\usepackage{latexsym}		% more math symbols
\usepackage{chngcntr}		% more math symbols
\usepackage{mathrsfs}		% math-fonts
\usepackage{mathtools}		% more math symbols
\usepackage{nchairx}		% Waldmann package for general math symbols

% --- GRAPHICS & CAPTIONS ----

\usepackage{graphicx}		% including images
\graphicspath{ {./figs/} }
\usepackage{subcaption}		% custom caption formatting
\DeclareCaptionLabelFormat{custom}{ \textbf{#1 #2}}
\captionsetup{format=hang}
\captionsetup{width=0.9\textwidth,labelformat=custom}
\usepackage{pdfpages}		% to include entire pdf pages in appendix etc.

% --- FORMAT ---

\usepackage[a4paper]{geometry} % a4 paper
\usepackage{setspace}		% spacing
\usepackage{titlesec}
\allowdisplaybreaks			% allow page breaks within math environments

% --- CUSTOM COMMANDS ---
%Logic
\newcommand{\then}{\Rightarrow}
\newcommand{\since}{\Leftarrow}
\renewcommand{\iff}{\ensuremath{\Leftrightarrow}}

%pretty epsilon
\let\oldepsilon\epsilon
\let\epsilon\varepsilon
\let\varepsilon\oldepsilon
%pretty phi
\let\oldphi\phi
\let\phi\varphi
\let\varphi\oldphi

\newcommand{\includetask}[2][pages=-]{
    \includepdf[#1]{#2}
    \addtocounter{subsection}{1}
}

% set-up for exercise specific stuff
\ifdef{\sheetNumber}{
    \setcounter{section}{\sheetNumber}
}{}

\usepackage{titling}
\newcommand{\makeexheaderger}{
    \begin{doublespace}
        \begin{center}
            \textbf{\Large{Übungsblatt \sheetNumber}}\\
            \textbf{\Large\lecture}\\
            Abgabe von: \textbf{\authors}\\
            \today
        \end{center}
        \ifdef {\sumPoints}
        {
            \hfill  \large Punkte: $\boxed{\qquad  /\; \sumPoints}$\\
        }{}
    \end{doublespace}
}

\newcommand{\makeexheadereng}{
    \begin{doublespace}
        \begin{center}
            \textbf{\Large{Exercise Sheet \sheetNumber}}\\
            \textbf{\Large\lecture}\\
            Abgabe von: \textbf{\authors}\\
            \today
        \end{center}
        \ifdef {\sumPoints}
        {
            \hfill  \large Points: $\boxed{\qquad  /\; \sumPoints}$\\
        }{}
    \end{doublespace}
}

% --- DOCUMENT ---

\begin{document}

\makeexheader

\begin{center}
    \includegraphics*[width=0.4\textwidth]{pi.png}
\end{center}

\setcounter{subsection}{4}
\subsection{}

\begin{itemize}
    \item First, we notice that if $p$ is idempotent we have $(1-p)^2 = 1 - 2p + p^2 = 1 - 2p + p = 1-p$, so $1-p$ is also idempotent. Now consider the following two functions:
    \begin{align*}
        \phi: &\algebra{A} \to p \algebra{A} \oplus (1-p) \algebra{A}, a \mapsto pa \oplus (1-p)a\\
        \psi: &p \algebra{A} \oplus (1-p) \algebra{A} \to \algebra{A}, pa \oplus (1-p)b \mapsto pa + b - pb
    \end{align*}
    Concatenation of these two functions yields
    \begin{align*}
        \psi(\phi(a)) &= \psi(pa \oplus (1-p)a) = pa + a - pa = a\\
        \phi(\psi(pa \oplus (1-p)b)) &= phi(pa + b - pb) = p (pa + b - pb) \oplus (1-p)(pa + b - pb)\\
        &= p^2 a + pb - p^2b \oplus pa - p^2a + b - pb - pb + p^2 b\\
        &= p a + pb - pb \oplus pa - pa + b - pb - pb + pb\\
        &= pa \oplus (1-p) b
    \end{align*}
    showing that $\phi$ is a bijective mapping. Furthermore, $\phi$ we have
    \begin{align*}
        \phi(a \cdot b) &= p a b \oplus (1-p) ab = p^2 ab \oplus (1-p)^2 ab = (pa)(pb) \oplus ((1-p)a)((1-p)b)\\
        &= (pa \oplus (1-p)a) (pb \oplus (1-p)b) = \phi(a) \phi(b)
    \end{align*}
    and $\phi$ is a homomorphism. Because $p \oplus (1-p)$ is the unit in $p \algebra{A} \oplus (1-p) \algebra{A}$ and $\phi(1) = p \oplus (1-p)$, $\phi$ is also compatible with the unit.
    \item Consider the functions $c_U$ and $c_V$, where $c_U|_U \equiv 1$ and $c_U|_V \equiv 0$ and likewise for $c_V$. These are continuous, idempotent and $c_U = 1 - c_V$ also holds. Note that these are in fact \textbf{not} elements of $C_0(X)$ as $U,V$ need not necessarily be compact. However, above we have not used $p \in \algebra{A}$ except for the fact that $p$ admits a well-defined multiplication with elements of $\algebra{A}$ yielding only elements of $\algebra{A}$. Since $c_U,c_V$ are still continuous the multiplication of $C(X)$ can be used here. By the argument above we then have $C_0(X) \simeq c_U C_0(X) \oplus c_V C_0(X) \simeq C_0(U) \oplus C_0(V)$ with 
    \begin{equation*}
        \phi: C_0(X) \to C_0(U) \oplus C_0(V), f \mapsto (f \cdot c_U)|_U \oplus (f \cdot c_V)|_V = f|_U \oplus f|_V
    \end{equation*}
    an isomorphism of unital algebras. To show that this is also an isomorphism of $C^*$-algebras, we also have to show that $\phi$ is continuous and commutes with $^*$. For the continuity, consider
    \begin{equation*}
        \| \phi(f) \| = \max\{ \| f|_U \|, \| f|_V \| \} = \max \{ \sup_{x \in U} | f(x) |, \sup_{x \in V} | f(x) | \} = \sup_{x \in X} |f(x)| = \|f\|
    \end{equation*}
    so $\phi$ is in fact even isometric (and thus bounded and continuous). Furthermore, we have
    \begin{equation*}
        \phi(f)^* = \overline{f|_U \oplus f|_V} = \overline{f|_U} \oplus \overline{f|_V} = \overline{f}|_U \oplus \overline{f}|_V =  \phi(\overline{f})
    \end{equation*}
    and $\phi$ is an isomorphism of $C^*$-algebras.
\end{itemize}



\addtocounter{subsection}{1}
\subsection{}

\begin{itemize}
    \item Assume $U$ is dense in $X$ and consider $a \in C_0(X)$ with $aI = 0$ (and thus $Ia = 0$ since $C_0(X)$ is commutative). Let $x_0 \in U$ be any point in $U$. We want to prove that $a(x_0) = 0$. Since $U$ is open, its complement $U^\complement$ is closed. Applying Uryson's Lemma to the compact set $\{x_0\}$ and the closed set $U^\complement$ (these sets are disjunct because of $x_0 \in U$) yields the existence of a function $f: X \to \mathds{C}$ with $f(x_0) = 1$ and $f|_{U^\complement} \equiv 0$. Since we have $f \in C_0(U) = I$ because of the latter condition and because ideals are strongly closed with respect to multiplication, we have $a f \in aI = 0$, so $a f$ is the zero function and in particular $a(x_0) f(x_0) = 0 \then a(x_0) = 0$. Repeating this chain of reasoning for any $x_0 \in U$ shows that $a|_U \equiv 0$ and since $U$ is dense in $X$ and $a$ is continuous, we have $a \equiv 0$. This shows that $I$ is an essential ideal.
    \item Proof by contraposition. Let $U$ be non-dense in $X$, i.e. there exists a point $x_0 \in X$ admitting an open neighborhood $V \subset X \setminus U$. We once again use Uryson's Lemma, this time for the two disjunct sets $\{x_0\}$ (compact) and $V^\complement$ (closed), proving the existence of a function $a: X \to \mathds{C}$ with $a|_{V^\complement} \equiv 0$ and $a(x_0) = 1$, which is therefore not equivalent to the zero function. However, for any $f \in I$ we have $a \cdot f \equiv 0$ since $f$ is zero on $U^\complement$ and $a$ is zero on $U \subseteq V^\complement$. Therefore, we have $aI = 0$ but $a \neq 0$ and $I$ cannot be an essential ideal of $C_0(X)$.
\end{itemize}

\subsection{}

We first show the prerequisites of the Stone-Weierstrass theorem.
\begin{enumerate}
    \item $I$ is a $C^*$-subalgebra of $C_0(U)$.
    $I$ is a subset of $C_0(U)$.
    If it were not, we would have an $f \in I, f \notin C_0(U)$ and there would exist a point $x_0 \notin U$ with $f(x_0) \neq 0$.
    But then $x_0$ would not be in $U^\complement$ by the definition of $U$.
    Furthermore, $I$ is an ideal, so it is closed with respect to addition and scalar multiplication (so it is a subspace) as well as multiplication (so it is a subalgebra).
    As $I$ is a closed subspace of $C_0(U)$, which is a closed subspace of the Banach space $C_0(X)$, $I$ is Banach.
    The involution and its property can be inherited from $C_0(X)$.

    It remains to show that $I$ is closed with respect to this involution.
    For this, we use the provided hint.
    It is $f \in I$, then note that $f^* \in C_0(X)$ and $f$ and $f^*$ are both bounded. We can write $f^* f_n$ as
    \begin{equation*}
        f^* f_n = f^* (f f^*)^{\frac{1}{n}} = f \cdot (\underbrace{f^{\frac{1}{n} - 1} (f^*)^{\frac{1}{n} + 1}}_{\coloneq g})
    \end{equation*}
    and set $g(x) := 0$ on the zeroes of $f$. Then we have $g \in C_0(X)$ and also
    \begin{equation*}
        |g(x)| = |f(x)^{\frac{1}{n} - 1} (f^*)(x)^{\frac{1}{n} + 1}| = |f(x)|^{\frac{1}{n} - 1} |\overline{f(x)}|^{\frac{1}{n} + 1} = |f(x)|^{\frac{2}{n}}
    \end{equation*}
    so $\{x \in X \mid |g(x)| \geq \epsilon\} = \{x \in X \mid |f(x)| \geq \epsilon^{\frac{n}{2}}\}$ is compact for every $\epsilon > 0$.
    This shows $g \in C_0(X)$ and therefore $f^* f_n = f g \in I$ (because of the ideal property) for every $n \in \mathds{N}$.
    The limit $\lim_{n \to \infty} f^* f_n$ converges in $C_0(X)$ to $f^*$, as $(f^* f)^{\frac{1}{n}} = |f|^{\frac{2}{n}}$ converges to the characteristic function of the support of $f$, i.e. to $0$ if $f(x) = 0$ and to $1$ otherwise.
    Since all elements $f^* f_n$ of the sequence are in $I$ and since $I$ is closed, the limit $f^*$ is also contained in $I$.
    This shows $I^* \subseteq I$ and therefore $I^* = I$, so $I$ is closed in respect to the involution.
    \item Given $x \in U$, there is $f \in I$ with $f(x) \neq 0$.
    Assume that such an $f$ did not exist, then for all $f \in I$ we have $f(x) = 0$.
    Per Definition of $U^\complement$, this implies $x \in U^\complement$, contradicting $x \in U$.
    \item $I$ separates points of $U$. Let $x,y$ be arbitrary points in $U$.
    As proven above, there exists a function $f \in I$ with $f(x) \neq 0$.
    As $X$ is Hausdorff, there also exists an open neighborhood $V$ of $x$ that does not contain $y$ and (without loss of generality) is a subset of $U$.
    Then Uryson's Lemma proves the existence of a function $g$ that is $1$ on the compact set $\{x\}$ and that is $0$ on the closed set $V^\complement \supset U^\complement$.
    The latter condition yields $g \in C_0(U)$, so the ideal property implies $fg \in I$.
    Additionally, we have $(fg)(x) = f(x) g(x) = f(x) \neq 0$ and $(fg)(y) = f(x) g(x) = f(x) \cdot 0 = 0$ (since $y \in V^\complement$). So $fg$ separates $x$ and $y$.
\end{enumerate}
So $I$ is a dense subspace of $C_0(U)$ by Stone-Weierstrass.
But since $I$ is closed, we have $I = \closure{I} = C_0(U)$.

Let $U \subset V$ be open sets in $X$.
Then we have $V^\complement \subset U^\complement$, so any function in $C_0(X)$ that is $0$ outside $U$ is also $0$ outside $V$, and we have $C_0(U) \subset C_0(V)$.
Conversely, let $U \nsubseteq V$ be open sets in $X$, so there exists a point $x \in U, x \notin V$.
Then Uryson's Lemma shows the existence of a function $f$ that is $1$ on the compact set $\{x\}$ and $0$ on the closed set $U^\complement$.
Since $f$ is $0$ outside $U$, we have $f \in C_0(U)$.
However, $f$ is non-zero on the point $x$ outside $V$, so $f$ cannot be in $C_0(V)$.
Therefore, we have $C_0(U) \nsubseteq C_0(V)$.
This shows $U \subseteq V \iff C_0(U) \subseteq C_0(V)$.

Lastly, let I be any maximal (and therefore closed) ideal in $C_0(X)$.
Then $I = C_0(U)$ for some $U \neq X$ (or $C_0(X)$ would be the whole space and thus not maximal) and $X \setminus U$ is a closed, non-empty set.
If $X \setminus U$ contains only a single element, our maximal ideal is of the form $C_0(X \setminus \{x\})$ for some $x \in X \setminus U$, and we are done.
If $X \setminus U$ contains more than one element, choose any fixed $x \in X \setminus U$.
Then, $X \setminus \{x\} \supset X \setminus U$ and thus $C_0(X\setminus\{x\}) \supset C_0(U)$.
Therefore, $C_0(U)$ cannot be a maximal ideal, as it has a super-ideal that is not yet the entire space.
So all maximal ideals of $C_0(X)$ must have form $C_0(X\setminus\{x\})$.




\end{document}