\documentclass[a4paper]{article}

% --- DATA ---

\def\lecture{Operator Algebras}
\def\authors{Valentin Hock, Linus Mußmächer, Minona Schäfer}
\def\sheetNumber{03}
%\def\sumPoints{30} 

% --- PREAMBLE ---

\usepackage[english]{babel}	% language specific quotation marks etc.
% === USAGE ===

% when using this preamble, setup your environment variables like this beforehand:


% \title{Stochastik 2}  %Title of exercise 
% \def\lecture{Stochastik 2}
% \def\authors{Linus Mußmächer}
% \def\sheetNumber{02}
% \def\sumPoints{30}      % maximum number of points (leave undefined)

% then use one of these commands (german or english) to print the header:

% \makeexheaderger

% and finally use subsections for your subtasks - they will be numbered as <sheetNumber><task number> by themselves

% if you have an exercise as an external .pdf, use \includetask to include it and increase the task counter


% --- OTHER ---

\usepackage{booktabs}       % professional-quality tables
\usepackage[table]{xcolor}	% color
\usepackage{pdfpages}		% to include entire pdf pages in appendix etc.
\usepackage{enumitem}		% better custom enumerations
\setlist[enumerate, 1]{label=(\roman*)}
\usepackage{etoolbox}		% toolbox for command modification

% --- FONTS & TYPESETTING ---

\usepackage[utf8]{inputenc} % allow utf-8 input
\usepackage[T1]{fontenc}    % use 8-bit T1 fonts
\usepackage{dsfont}			% font with double lines for sets
\usepackage[german,ruled,vlined,linesnumbered,commentsnumbered,algoruled]
{algorithm2e} 				%pseudo code
\usepackage{listings}		%java code
\usepackage{csquotes}

% --- URLS ---

\usepackage[colorlinks=true, linkcolor=black, citecolor=blue, urlcolor=blue]{hyperref}   	% hyperlinks
\usepackage{url}            % simple URL typesetting

% --- MATH SYMBOLS ---

\usepackage{amsmath,amssymb}% more math symbols
\usepackage{amsfonts}       % blackboard math symbols
\usepackage{latexsym}		% more math symbols
\usepackage{chngcntr}		% more math symbols
\usepackage{mathrsfs}		% math-fonts
\usepackage{mathtools}		% more math symbols
\usepackage{nchairx}		% Waldmann package for general math symbols

% --- GRAPHICS & CAPTIONS ----

\usepackage{graphicx}		% including images
\graphicspath{ {./figs/} }
\usepackage{subcaption}		% custom caption formatting
\DeclareCaptionLabelFormat{custom}{ \textbf{#1 #2}}
\captionsetup{format=hang}
\captionsetup{width=0.9\textwidth,labelformat=custom}
\usepackage{pdfpages}		% to include entire pdf pages in appendix etc.

% --- FORMAT ---

\usepackage[a4paper]{geometry} % a4 paper
\usepackage{setspace}		% spacing
\usepackage{titlesec}
\allowdisplaybreaks			% allow page breaks within math environments

% --- CUSTOM COMMANDS ---
%Logic
\newcommand{\then}{\Rightarrow}
\newcommand{\since}{\Leftarrow}
\renewcommand{\iff}{\ensuremath{\Leftrightarrow}}

%pretty epsilon
\let\oldepsilon\epsilon
\let\epsilon\varepsilon
\let\varepsilon\oldepsilon
%pretty phi
\let\oldphi\phi
\let\phi\varphi
\let\varphi\oldphi

\newcommand{\includetask}[2][pages=-]{
    \includepdf[#1]{#2}
    \addtocounter{subsection}{1}
}

% set-up for exercise specific stuff
\ifdef{\sheetNumber}{
    \setcounter{section}{\sheetNumber}
}{}

\usepackage{titling}
\newcommand{\makeexheaderger}{
    \begin{doublespace}
        \begin{center}
            \textbf{\Large{Übungsblatt \sheetNumber}}\\
            \textbf{\Large\lecture}\\
            Abgabe von: \textbf{\authors}\\
            \today
        \end{center}
        \ifdef {\sumPoints}
        {
            \hfill  \large Punkte: $\boxed{\qquad  /\; \sumPoints}$\\
        }{}
    \end{doublespace}
}

\newcommand{\makeexheadereng}{
    \begin{doublespace}
        \begin{center}
            \textbf{\Large{Exercise Sheet \sheetNumber}}\\
            \textbf{\Large\lecture}\\
            Abgabe von: \textbf{\authors}\\
            \today
        \end{center}
        \ifdef {\sumPoints}
        {
            \hfill  \large Points: $\boxed{\qquad  /\; \sumPoints}$\\
        }{}
    \end{doublespace}
}

% --- DOCUMENT ---

\begin{document}

\makeexheader

\begin{center}
    \includegraphics*[width=0.4\textwidth]{pi.png}
\end{center}

\setcounter{subsection}{1}

\subsection{}

The $C^*$-property shows $\|a^2\| = \|a^* a \| = \|a\|^2$, and by using this as well as the $C^*$ property again, we have for $n = 4$ that $\|a^4\| = \|a^* a^* a a\| = \|(a^2)^* (a^2)\| = \|a^2\|^2 = \|a^4\|$.
Inductively, we can likewise prove $\|a^{2^k}\| = \|a\|^{2^k}$ for all $k \in \mathds{N}$.

Now, for any $n \in \mathds{N}$ there exists $m \in \mathds{N}$ such that $n + m = 2^k$ for some $k \in \mathds{N}$. Then we have
\begin{equation*}
    \|a\|^{2^k} = \|a^{2^n}\| = \|a^n a^m\| \leq \|a^n\| \cdot \|a^m\| \leq \|a\|^n \cdot \|a^m\| \leq \|a\|^{n+m} = \|a\|^{2^k}
\end{equation*}
and because the first and last element are equal, we must have equality in every intermediate step.
This especially proves $\|a^n\| = \|a\|^n$.

Let now $a \in \algebra{A}$ be an arbitrary element.
Then $\|a^* a \dots a^* a\| = \|(a^* a)^\frac{n}{2}\| = \|a^* a\|^\frac{n}{2} = \|a\|^n$ as proven above, because $(a^* a)$ is self-adjunct.
For non-even $n$ (and thus even $n+1$) we can once again calculate
\begin{equation*}
    \|a\|^{n+1} = \|a^* a a^* \dots a^* a \| \leq \|a\| \cdot \|a a^* \dots a^*\| \leq \|a\| \cdot \|a\|^{n} = \|a\|^{n+1}
\end{equation*}
and therefore $\|a a^* \dots a^*\| = \|a\|^n$ by the same argument as above.

Now, for a normal $a \in \algebra{A}$ (that is, $a^* a = a a^*$) we have
\begin{align*}
    \|a^n\|^\frac{1}{n} = \left(\|a^n\|^2\right)^\frac{1}{2n} = \|(a^n)* a^n \|^\frac{1}{2n} = \|a a^* a \dots a^*\|^\frac{1}{2n} = \left(\|a\|^{2n}\right)^\frac{1}{2n} = \|a\|
\end{align*}
and therefore $r(a) = \lim_{n \to \infty} \|a^n\|^\frac{1}{n} = \|a\|$.

\setcounter{subsection}{5}

\subsection{}

First, to prove that $M(\algebra{A})$ fulfills the given property. We already know that $\algebra{A}$ is a closed, two-sided and essential ideal in $M(\algebra{A})$.
Consider the following morphism:
\begin{equation*}
    \phi: \algebra{B} \to M(\algebra{A}), b \mapsto (L_b, R_b) 
\end{equation*}
where
\begin{align*}
    L_b: \algebra{A} \to \algebra{A} &a \mapsto b \cdot a\\
    R_b: \algebra{A} \to \algebra{A} &a \mapsto a \cdot b
\end{align*}
defined via the multiplication in $\algebra{B}$.
Because $\algebra{A} \trianglelefteq \algebra{B}$, we actually have $a \cdot b, b \cdot a \in \algebra{A}$ for all $a,b$ and $L_b, R_b$ are well-defined and, as they are clearly linear, $\phi$ is also well-defined.
Because of $L_{ab} = L_{a} \circ L_{b}$ and $R_{ab} = R_b \circ R_a$, we have $\phi(ab) = \phi(a) \cdot \phi(b)$ with the multiplication as defined in the lecture.
Furthermore, $\phi(1) = (L_1, R_1) = (\id, \id)$ and $\phi$ is therefore a homomorphism.
Lastly we have $\phi(b^*) = (L_{b^*}, R_{b^*})$ and
\begin{align*}
    & L_{b^*}(a) = b^* a = (a^* b)^* = R_b(a^*)^* = (R_b)^*(a)\\
    & R_{b^*}(a) = a b^* = (b a^*)^* = L_b(a^*)^* = (L_b)^*(a) \\
    \then \ & \phi(b^*) = (R_b^*, L_b^*) = (L_b, R_b)^*
\end{align*}
so $\phi$ is indeed a $^*$-homomorphism.
Since $\phi|_\algebra{A}$ reduces to the normal left- and right-multiplication on $\algebra{A}$, it coincides with canonical inclusion map as defined in the lecture.
$\phi$ therefore fulfills all conditions as given.

To conclude that the universal property is indeed correct, we need to consider the case that $\algebra{A} \trianglelefteq \algebra{B}$ is an essential ideal. 
In this case, $b \algebra{A} = 0$ implies $b = 0$ for any $b \in \algebra{B}$.
Assume $\phi(b) = \phi(c)$ for any two $b, c \in \algebra{B}$. 
Then we have $(L_b, R_b) = (L_c, R_c)$ and thus $b a = c a$ and $a b = a c$ for all $a \in \algebra{A}$.
This is equivalent to $b \algebra{A} = c \algebra{A}$ and $\algebra{A} b = \algebra{A} c$ or, stated differently, $(b-c) \algebra{A} = 0$ and $\algebra{A} (b-c) = 0$.
As stated above, this implies $(b-c) = 0 \iff b = c$ and thus proves that $\phi$ is injective.

Next, we want to prove that any algebra $D \trianglerighteq \algebra{A}$ that fulfills the above property (and where $\algebra{A}$ is a closed, two-sided essential ideal in $D$) is already equal to $M(\algebra{A})$.

We already know that $\algebra{A}$ is an essential ideal in $M(\algebra{A})$, so if $D$ also fulfills the property above the therefore existent morphism $\phi_D: M(\algebra{A}) \to D$ must be injective.
We may thus treat $M(\algebra{A})$ as a subalgebra of $D$.
In parallel, since $\algebra{A}$ is also an essential ideal of $D$, the morphism $\phi_M: D \to M(\algebra{A})$ is also injective and we may consider $M(\algebra{A})$ as a subalgebra of $D$.
But then these two algebras are isomorphic to subalgebras of each other, so they must already be equal.

\subsection{}



\end{document}