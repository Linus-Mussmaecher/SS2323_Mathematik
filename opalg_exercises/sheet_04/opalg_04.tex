\documentclass[a4paper]{article}

% --- DATA ---

\def\lecture{Operator Algebras}
\def\authors{Linus Mußmächer, Minona Schäfer}
\def\sheetNumber{04}
%\def\sumPoints{30} 

% --- PREAMBLE ---

\usepackage[english]{babel}	% language specific quotation marks etc.
% === USAGE ===

% when using this preamble, setup your environment variables like this beforehand:


% \title{Stochastik 2}  %Title of exercise 
% \def\lecture{Stochastik 2}
% \def\authors{Linus Mußmächer}
% \def\sheetNumber{02}
% \def\sumPoints{30}      % maximum number of points (leave undefined)

% then use one of these commands (german or english) to print the header:

% \makeexheaderger

% and finally use subsections for your subtasks - they will be numbered as <sheetNumber><task number> by themselves

% if you have an exercise as an external .pdf, use \includetask to include it and increase the task counter


% --- OTHER ---

\usepackage{booktabs}       % professional-quality tables
\usepackage[table]{xcolor}	% color
\usepackage{pdfpages}		% to include entire pdf pages in appendix etc.
\usepackage{enumitem}		% better custom enumerations
\setlist[enumerate, 1]{label=(\roman*)}
\usepackage{etoolbox}		% toolbox for command modification

% --- FONTS & TYPESETTING ---

\usepackage[utf8]{inputenc} % allow utf-8 input
\usepackage[T1]{fontenc}    % use 8-bit T1 fonts
\usepackage{dsfont}			% font with double lines for sets
\usepackage[german,ruled,vlined,linesnumbered,commentsnumbered,algoruled]
{algorithm2e} 				%pseudo code
\usepackage{listings}		%java code
\usepackage{csquotes}

% --- URLS ---

\usepackage[colorlinks=true, linkcolor=black, citecolor=blue, urlcolor=blue]{hyperref}   	% hyperlinks
\usepackage{url}            % simple URL typesetting

% --- MATH SYMBOLS ---

\usepackage{amsmath,amssymb}% more math symbols
\usepackage{amsfonts}       % blackboard math symbols
\usepackage{latexsym}		% more math symbols
\usepackage{chngcntr}		% more math symbols
\usepackage{mathrsfs}		% math-fonts
\usepackage{mathtools}		% more math symbols
\usepackage{nchairx}		% Waldmann package for general math symbols

% --- GRAPHICS & CAPTIONS ----

\usepackage{graphicx}		% including images
\graphicspath{ {./figs/} }
\usepackage{subcaption}		% custom caption formatting
\DeclareCaptionLabelFormat{custom}{ \textbf{#1 #2}}
\captionsetup{format=hang}
\captionsetup{width=0.9\textwidth,labelformat=custom}
\usepackage{pdfpages}		% to include entire pdf pages in appendix etc.

% --- FORMAT ---

\usepackage[a4paper]{geometry} % a4 paper
\usepackage{setspace}		% spacing
\usepackage{titlesec}
\allowdisplaybreaks			% allow page breaks within math environments

% --- CUSTOM COMMANDS ---
%Logic
\newcommand{\then}{\Rightarrow}
\newcommand{\since}{\Leftarrow}
\renewcommand{\iff}{\ensuremath{\Leftrightarrow}}

%pretty epsilon
\let\oldepsilon\epsilon
\let\epsilon\varepsilon
\let\varepsilon\oldepsilon
%pretty phi
\let\oldphi\phi
\let\phi\varphi
\let\varphi\oldphi

\newcommand{\includetask}[2][pages=-]{
    \includepdf[#1]{#2}
    \addtocounter{subsection}{1}
}

% set-up for exercise specific stuff
\ifdef{\sheetNumber}{
    \setcounter{section}{\sheetNumber}
}{}

\usepackage{titling}
\newcommand{\makeexheaderger}{
    \begin{doublespace}
        \begin{center}
            \textbf{\Large{Übungsblatt \sheetNumber}}\\
            \textbf{\Large\lecture}\\
            Abgabe von: \textbf{\authors}\\
            \today
        \end{center}
        \ifdef {\sumPoints}
        {
            \hfill  \large Punkte: $\boxed{\qquad  /\; \sumPoints}$\\
        }{}
    \end{doublespace}
}

\newcommand{\makeexheadereng}{
    \begin{doublespace}
        \begin{center}
            \textbf{\Large{Exercise Sheet \sheetNumber}}\\
            \textbf{\Large\lecture}\\
            Abgabe von: \textbf{\authors}\\
            \today
        \end{center}
        \ifdef {\sumPoints}
        {
            \hfill  \large Points: $\boxed{\qquad  /\; \sumPoints}$\\
        }{}
    \end{doublespace}
}

% --- DOCUMENT ---

\begin{document}

\makeexheader

\begin{center}
    \includegraphics*[width=0.4\textwidth]{pi.png}
\end{center}

\subsection{}

The first statement follows immediately from the fact that the canonical inclusion $\algebra{B} \hookrightarrow \algebra{A}$ is an injective $^*$-homomorphism, so it is isometric as proven in the lecture.

If now $\algebra{B}$ is a dense proper $^*$-subalgebra of $\algebra{A}$, assuming it could be turned into a $C^*$-algebra, the norm on that $C^*$-algebra would already have to be the norm on $\algebra{A}$.
But then the canonical inclusion is isometric and injective, so it has closed range and $\algebra{B} \subseteq \algebra{A}$ is closed and dense in $\algebra{A}$.
Now, however, we have $\algebra{B} = \algebra{A}$, a contradiction.

\subsection{}

As hinted, the disk algebra $\algebra{A}(\mathds{D})$ is such an algebra.
As we have $\overline{\overline{z}} = z$ for all $z \in \mathds{D}$, the identity map is self-adjoint, but because of $i \in \id(\mathds{D})$, $\id - i \cdot 1$ is not invertible.
Thus, $i \in \sigma(\id) \subsetneq \mathds{R}$.

Analogously, we can also consider the character $\phi: \algebra{A}(\mathds{D}) \to \mathds{C}: f \mapsto f(i)$.
Here, we have $\phi(\id^*) = \phi(\id) = i \neq - i = \overline{\phi(\id)}$.

\begin{enumerate}
    \item $\then$ (ii): Take $a \in \algebra{A}$ be any self-adjoint element and $(a)$ the $^*$-subalgebra generated by $a$ in $\algebra{A}$.
    Then $\Omega((a)) \subseteq \Omega(\algebra{A})$ (as any character of $(a)$ can be extended to $\algebra{A}$), so any character in $\Omega((a))$ is also symmetric. As $a$ is self-adjoint, we have $\chi(a) = \overline{\chi(a^*)} = \overline{\chi(a)}$ and therefore $\chi(a) \in \mathds{R}$ for any $\chi \in \Omega((a))$.
    As $(a)$ is a commutative $^*$-Banach-algebra, we have
    \begin{equation*}
        \sigma_{(a)}(a) \subseteq \{\chi(a) \mid \chi \in \Omega((a)) \} \cup \{0\} \subseteq \mathds{R}\text{.}
    \end{equation*}
    Furthermore, if $a - \lambda \in (a)$ is invertible, it must also be invertible in $\algebra{A} \supseteq (a)$, so $\rho(a)_{(a)} \subseteq \rho_{\algebra{A}}(a)$ and therefore $\sigma_\algebra{A}(a) \subseteq \sigma_{(a)}(a) \subseteq \mathds{R}$.
    \item $\then$ (iii): Let $\Gamma: \algebra{A} \to C(\Omega(\algebra{A})), a \mapsto (\chi \mapsto \chi(a))$ be the Gelfand-transform of $\algebra{A}$.
    We want to prove $\Gamma(a) = \Gamma(a^*)^*$.
    By the Definition of the involution on $C(\Omega(\algebra{A}))$, this is equivalent to $\chi(a) = \overline{\chi(a^*)}$ for any $\chi$ the spectrum and $a \in \algebra{A}$.

    First, let $a \in \algebra{A}$ self-adjoint.
    Then $\overline{\chi(a^*)} = \overline{\chi(a)} = \chi(a)$ as $\chi(a) \in \sigma(a) \subseteq \mathds{R}$.

    If $a \in \algebra{A}$ is not self-adjoint, we can write $a = b + ic$ for self-adjoint elements $b = \frac{a + a^*}{2}$ and $c = \frac{a - a^*}{2i}$ and it follows that
    \begin{equation*}
        \chi(a) = \chi(b + ic) = \chi(b) + i \chi(c) = \overline{\chi(b^*)} + i \overline{\chi(c^*)} = \overline{\chi(b^*) - i \chi(c^*)} = \overline{b^* - i c^*} = \overline{\chi(a^*)}
    \end{equation*}
    and this shows (iii).
    \item $\then$ (i): If $\Gamma$ is a $^*$-homomorphism, then $\Gamma(a^*) = \Gamma(a)^*$ and by the definition of the involution as discussed above this already shows $\chi(a^*) = \overline{\chi(a)}$ for every character $\chi$.
\end{enumerate}

\subsection{}

Since the spectrum $\sigma(a)$ of $a$ in the non-unital algebra $\algebra{A}$ is defined as its spectrum in the unitization $\tilde{\algebra{A}}$, the spectra in $C(\sigma(a))$ and $C_0(\sigma(a))$ have the same meaning and are not merely notationally equivalent.

Let $\Phi: C^*(a,1) \to C(\sigma(a))$ be the isometric $^*$-isomorphism in the fundamental theorem of functional calculus applied to $\tilde{\algebra{A}}$ given by Gelfand.
If $0 \notin \sigma(a)$, $a$ is invertible so $1 \in C^*(a)$, and we therefore have both $C^*(a,1) = C^*(a)$ and $C_0(\sigma(a)) = C(\sigma(a))$.
Therefore, $\Phi^{-1}: C_0(\sigma(a)) \to C^*(a) \subseteq \algebra{A}$ is already the unique isometric $^*$-homomorphism we require, and its image is $C^*(a)$ as desired.

Now consider $0 \in \sigma(a)$ and the restriction $\Psi = \Phi|_{C^*(a)}$.
This restriction retains the properties of a $^*$-homomorphism, as well as the isometry and $a \mapsto \id \in C_0(\sigma(a))$.
It remains to show that $\Psi$ is still unique and $f(0) = 0$ for every element in the image of $\Psi$.
To see the last property, notice that $\Psi(a) = \id$ and $\id(0) = 0$ as well as $\Psi(C^*(a)) = C^*(\Psi(a)) = C^*(\id)$, so any element in the image of $\Psi$ is composed of sums and products of $\id$ and $\overline{\id}$ and therefore fulfills $f(0) = 0$.
So $\Psi^{-1}: C_0(\sigma(a)) \to C^*(a) \subseteq \algebra{A}$ is our $^*$-homomorphism.

To see the uniqueness, note that as $0 \in \sigma(a)$ we can write $\tilde{\algebra{A}} \simeq \algebra{A} \oplus \mathds{C}$ and $C(\sigma(a)) \simeq C_0(\sigma(a)) \oplus \mathds{C}$ (with a multiplication analogous to that of the unitization) and can therefore decompose $\Phi^{-1}$ into $\Phi = \Psi^{-1} \oplus \Phi^{-1}|_\mathds{C}$.
If there existed a second $^*$-homomorphism $\Psi_2: C_0(\sigma(a)) \to \algebra{A}$ (with $\id \mapsto a$), then $\Phi_2 = \Psi_2 \oplus \Phi^{-1}|_\mathds{C}$ were a second $^*$-homomorphism $C_0(\sigma(a)) \oplus \mathds{C} = C(\sigma(a)) \to \algebra{A} \oplus \mathds{C} = \tilde{\algebra{A}}$ with $\id \mapsto a$, a contradiction.

\end{document}