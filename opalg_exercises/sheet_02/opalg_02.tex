\documentclass[a4paper]{article}

% --- DATA ---

\def\lecture{Operator Algebras}
\def\authors{Valentin Hock, Linus Mußmächer, Minona Schäfer}
\def\sheetNumber{02}
%\def\sumPoints{30} 

% --- PREAMBLE ---

\usepackage[english]{babel}	% language specific quotation marks etc.
% === USAGE ===

% when using this preamble, setup your environment variables like this beforehand:


% \title{Stochastik 2}  %Title of exercise 
% \def\lecture{Stochastik 2}
% \def\authors{Linus Mußmächer}
% \def\sheetNumber{02}
% \def\sumPoints{30}      % maximum number of points (leave undefined)

% then use one of these commands (german or english) to print the header:

% \makeexheaderger

% and finally use subsections for your subtasks - they will be numbered as <sheetNumber><task number> by themselves

% if you have an exercise as an external .pdf, use \includetask to include it and increase the task counter


% --- OTHER ---

\usepackage{booktabs}       % professional-quality tables
\usepackage[table]{xcolor}	% color
\usepackage{pdfpages}		% to include entire pdf pages in appendix etc.
\usepackage{enumitem}		% better custom enumerations
\setlist[enumerate, 1]{label=(\roman*)}
\usepackage{etoolbox}		% toolbox for command modification

% --- FONTS & TYPESETTING ---

\usepackage[utf8]{inputenc} % allow utf-8 input
\usepackage[T1]{fontenc}    % use 8-bit T1 fonts
\usepackage{dsfont}			% font with double lines for sets
\usepackage[german,ruled,vlined,linesnumbered,commentsnumbered,algoruled]
{algorithm2e} 				%pseudo code
\usepackage{listings}		%java code
\usepackage{csquotes}

% --- URLS ---

\usepackage[colorlinks=true, linkcolor=black, citecolor=blue, urlcolor=blue]{hyperref}   	% hyperlinks
\usepackage{url}            % simple URL typesetting

% --- MATH SYMBOLS ---

\usepackage{amsmath,amssymb}% more math symbols
\usepackage{amsfonts}       % blackboard math symbols
\usepackage{latexsym}		% more math symbols
\usepackage{chngcntr}		% more math symbols
\usepackage{mathrsfs}		% math-fonts
\usepackage{mathtools}		% more math symbols
\usepackage{nchairx}		% Waldmann package for general math symbols

% --- GRAPHICS & CAPTIONS ----

\usepackage{graphicx}		% including images
\graphicspath{ {./figs/} }
\usepackage{subcaption}		% custom caption formatting
\DeclareCaptionLabelFormat{custom}{ \textbf{#1 #2}}
\captionsetup{format=hang}
\captionsetup{width=0.9\textwidth,labelformat=custom}
\usepackage{pdfpages}		% to include entire pdf pages in appendix etc.

% --- FORMAT ---

\usepackage[a4paper]{geometry} % a4 paper
\usepackage{setspace}		% spacing
\usepackage{titlesec}
\allowdisplaybreaks			% allow page breaks within math environments

% --- CUSTOM COMMANDS ---
%Logic
\newcommand{\then}{\Rightarrow}
\newcommand{\since}{\Leftarrow}
\renewcommand{\iff}{\ensuremath{\Leftrightarrow}}

%pretty epsilon
\let\oldepsilon\epsilon
\let\epsilon\varepsilon
\let\varepsilon\oldepsilon
%pretty phi
\let\oldphi\phi
\let\phi\varphi
\let\varphi\oldphi

\newcommand{\includetask}[2][pages=-]{
    \includepdf[#1]{#2}
    \addtocounter{subsection}{1}
}

% set-up for exercise specific stuff
\ifdef{\sheetNumber}{
    \setcounter{section}{\sheetNumber}
}{}

\usepackage{titling}
\newcommand{\makeexheaderger}{
    \begin{doublespace}
        \begin{center}
            \textbf{\Large{Übungsblatt \sheetNumber}}\\
            \textbf{\Large\lecture}\\
            Abgabe von: \textbf{\authors}\\
            \today
        \end{center}
        \ifdef {\sumPoints}
        {
            \hfill  \large Punkte: $\boxed{\qquad  /\; \sumPoints}$\\
        }{}
    \end{doublespace}
}

\newcommand{\makeexheadereng}{
    \begin{doublespace}
        \begin{center}
            \textbf{\Large{Exercise Sheet \sheetNumber}}\\
            \textbf{\Large\lecture}\\
            Abgabe von: \textbf{\authors}\\
            \today
        \end{center}
        \ifdef {\sumPoints}
        {
            \hfill  \large Points: $\boxed{\qquad  /\; \sumPoints}$\\
        }{}
    \end{doublespace}
}

% --- DOCUMENT ---

\begin{document}

\makeexheadereng

\begin{center}
    \includegraphics*[width=0.4\textwidth]{pi.png}
\end{center}


\setcounter{subsection}{8}
\subsection{Topological zero divisors}

We consider two cases:

\begin{itemize}
    \item First, let $X$ be non-compact.
    Then we have $\inf_{x \in X} |f(x)| = 0$, because if it were $\epsilon > 0$ we would have $X = \{ x \in X \mid |f(x)| \geq \epsilon \}$ non-compact and thus $f \notin C_0(X)$.
    Therefore, we need only prove $\zeta(f) = 0$.

    Choose any $\epsilon > 0$ define $K \coloneq \{ x \in X \mid |f(x)| \geq \epsilon \}$.
    Because $\inf_{x \in X} |f(x)| = 0$, there exists an $x_0 \in X$ for which $|f(x_0)| < \epsilon$ holds (and thus $x_0 \notin K$).
    Because $K$ is compact, it is closed and thus $X \setminus K$ is open.
    Choose an open, pre-compact neighborhood $U_0$ of $x_0$ in $X \setminus K$ and set $K' = X \setminus U_0$.
    The set $\{x_0\}$ is compact, and $K'$ is closed, so Uryson's Lemma yields the existence of a function $b: X \to \mathds{C}$ (with $\|b\| = 1$) (in $C_0(X)$) with $b(x_0) = 1$ and $b|_{K'} \equiv 0$.
    Then for $x \in K \subseteq K'$, we have $|(fb)(x)| = |f(x)| \cdot |b(x)| = |f(x) \cdot 0 < \epsilon$.
    For $x \in K^\complement$, it follows that $|(fb)(x)| = |f(x) \cdot |b(x)| < \epsilon \cdot 1 = \epsilon$ and thus $\|f b \| < \epsilon$. 
    This shows $\zeta(f) = \inf_{b \in C_0(X), \|b\| = 1} \| f b \| = 0$.

    So if $X$ is not compact, $\zeta(f) = \inf_{x \in X} |f(x)| = 0$ holds and every $f \in C_0(X)$ is a topological zero divisor.
    \item Now, let $X$ be a compact Hausdorff space and $f \in C_0(X)$.
    If $f$ is non-invertible, we have $0 \in f(X)$ and thus $\inf_{x \in X} |f(x)| = 0$. 
    In this case, we can argue as we did in the first point and thusly show $\zeta(f) = 0$ in much the same way.

    Consider now an invertible $f$ with $\inf_{x \in X} |f(x)| = k > 0$.
    We conclude 
    \begin{equation*}
        \left\| \frac{1}{f} \right\| = \sup_{x \in X} \frac{1}{|f(x)|} = \frac{1}{\inf_{x \in X} |f(x)|} = \frac{1}{k}\text{,}
    \end{equation*}
    so for any $b \in C_0(X)$ with $\|b\| = 1$ we have $\|f \cdot b\| \cdot \| \frac{1}{f}\| \geq \|f \cdot b \cdot \frac{1}{f} \| = \|b\| = 1$, so $\|f \cdot b \| \geq k$ and therefore $\zeta(f) \geq k$.

    Choose now any $\epsilon > 0$.
    Then $K \coloneq \{ x \in X \mid |f(x)| \geq k + \epsilon \}$ is compact and $K \neq X$ (or $k$ would not be the infimum of $|f(x)|$).
    Just like in the first bullet point, we can choose $x_0 \in X \setminus K$ and fitting neighborhoods to get the existence of a function $b$ fulfilling $\|b\| = 1$, $\|b f \| < k + \epsilon$ and $b \in C_0(X)$.
    Therefore, $\zeta(f) \leq k$ and thus $\zeta(f) = k > 0$.
    This also shows that the (invertible) element $f$ is not a topological zero divisor.
\end{itemize}
To summarize, we have proven $\zeta(f) = \inf_{x \in X} f(x)$ for any $f \in C_0(X)$, that $f$ is a topological zero divisor in a compact space always and in a non-compact space if and only if it is invertible.
It remains to show that in a commutative $C^*$-algebra $\algebra{A}$, $f \in \algebra{A}$ is a topological zero divisor if and only if $0 \in \sigma(f)$.
As $\algebra{A}$ is commutative, we can employ the Gelfand Representation (1.3.6) and conclude that $\algebra{A}$ can be embedded in the algebra $C_0(\Omega(\algebra{A}))$ by $\Gamma$, and $\sigma(f) = \image{\hat{f}}$ (because $\algebra{A}$ is unital, or $\sigma(f)$ would not be defined). 
Then $0 \in \sigma(f) \iff 0 \in \image{\hat{f}} \iff $ $f$ is non-invertible, and because $\Omega(\algebra{A})$ is compact (as $\algebra{A}$ is unital), this is equivalent to $f$ being a topological zero divisor.

\end{document}