\documentclass[a4paper]{article}

% --- PREAMBLE ---

\usepackage[english]{babel}	% language specific quotation marks etc.
% --- LANGUAGE ---

% Needs to be set individually!
%\usepackage[english]{babel}	% language specific quotation marks etc.

% --- OTHER ---

\usepackage{booktabs}       % professional-quality tables
\usepackage[table]{xcolor}	% color
\usepackage{pdfpages}		% to include entire pdf pages in appendix etc.
\usepackage{enumitem}		% better custom enumerations
\setlist[enumerate, 1]{label=(\roman*)}
\usepackage{etoolbox}		% toolbox for command modification

% --- FONTS & TYPESETTING ---

\usepackage[utf8]{inputenc} % allow utf-8 input
\usepackage[T1]{fontenc}    % use 8-bit T1 fonts
\usepackage{dsfont}			% font with double lines for sets
\usepackage[german,ruled,vlined,linesnumbered,commentsnumbered,algoruled]
{algorithm2e} 				%pseudo code
\usepackage{listings}		%java code
\usepackage{csquotes}

% --- URLS ---

\usepackage[colorlinks=true, linkcolor=black, citecolor=blue, urlcolor=blue]{hyperref}   	% hyperlinks
\usepackage{url}            % simple URL typesetting

% --- MATH SYMBOLS ---

\usepackage{amsmath,amssymb}% more math symbols
\usepackage{amsfonts}       % blackboard math symbols
\usepackage{latexsym}		% more math symbols
\usepackage{chngcntr}		% more math symbols
\usepackage{mathrsfs}		% math-fonts
%\usepackage{marvosym}		% more math symbols (conflicts with Waldmann)
\usepackage{mathtools}		% more math symbols
\usepackage{nchairx}		% Waldmann package for general math symbols and operators
% theorem formatting (contained in Waldmann)
%\usepackage[amsmath,thmmarks,framed,thref]{ntheorem}

% --- GRAPHICS & CAPTIONS ----

\usepackage{graphicx}		% including images
\graphicspath{ {./figs/} }
\usepackage{subcaption}		% custom caption formatting
\DeclareCaptionLabelFormat{custom}{ \textbf{#1 #2}}
\captionsetup{format=hang}
\captionsetup{width=0.9\textwidth,labelformat=custom}

% --- TIKZ ---

\usepackage{tikz}			% basic tikz for custom images
\usetikzlibrary{cd}			% custom diagrams
\usetikzlibrary{external}	% externalize images for faster compilation
\tikzexternalize[prefix=figures/]
\AtBeginEnvironment{tikzcd}{\tikzexternaldisable} %fix cd/externalize
\AtEndEnvironment{tikzcd}{\tikzexternalenable}
\usepackage{pgfplots}		% custom plotting
\usepgfplotslibrary{colormaps}
\pgfplotsset{compat=newest}	
\usetikzlibrary{patterns}	% custom patterns

% --- FORMAT ---

\usepackage[a4paper]{geometry} % a4 paper
\usepackage{setspace}		% spacing
\usepackage[nobottomtitles*]{titlesec} %prevent section titles from sometimes being on the bottom of a page
\usepackage{titlesec}
\allowdisplaybreaks			% allow page breaks within math environments

% --- CUSTOM COMMANDS ---
%Logic
\newcommand{\then}{\Rightarrow}
\newcommand{\since}{\Leftarrow}
\renewcommand{\iff}{\ensuremath{\Leftrightarrow}}

%pretty epsilon
\let\oldepsilon\epsilon
\let\epsilon\varepsilon
\let\varepsilon\oldepsilon
%pretty phi
\let\oldphi\phi
\let\phi\varphi
\let\varphi\oldphi

%matrix
\newcommand{\qmatrix}[1]{\ensuremath{\left(\begin{matrix}#1\end{matrix}\right)}}

% --- DATA ---

\title{Exercises to Introduction to Operator Algebras}
\author{Alcides Buss\\Notes by: Linus Mußmächer\\2336440}
\date{Summer 2023}
\let\varphi\oldphi

% --- DOCUMENT ---

\begin{document}

\maketitle


\tableofcontents

\newpage

\section{Topological Basics}

Let $X$ be a topological space, that is there exists a subset $\topology{O}(X) \in \mathds{P}(X)$. 

\begin{definition}
	$X$ is \textbf{Hausdorff} if for all $x,y \in X$ there exist open sets $U,V \in \topology{O}(X)$ such that $x \in U, y \in V$ and $U \cap V = \emptyset$.

	$X$ is \textbf{locally Hausdorff} if for all $x \in X$ there exists an open neighborhood $U \in \topology{O}(X)$ such that $U$ is Hausdorff with the relative topology from $X$.
\end{definition}

\begin{example}[Snake with two heads]
	We consider the space $[0,1] \cup \{1^+\}$ equipped with a topology such that both the subspace $[0,1]$ and $[0,1] \setminus \{1\} \cup \{1^+\}$ are isomorphic to $[0,1]$. Then $X$ is compact, locally Hausdorff but not Hausdorff.
\end{example}

\begin{definition}
	$X$ is compact if for every open cover $(U_i)_{i \in I}$ there exists a finite open subcover.

	$X$ is locally compact if for every $x \in X$ there exists a neighborhood basis of $x$ consisting of open relatively compact subsets of $X$, that is for every open neighborhood $U$ of $x$ there exists and open neighborhood $V$ of $x$ such that $\closure{V}$ is compact and $\closure{V} \subset U$.
\end{definition}

Observation: For a locally Hausdorff $X$, $X$ is locally compact if and only if for all $x \in X$ there exists an open neighborhood $U$  of $x$ such that $\closure{U}$ is compact.

\subsection{Results about locally compact Hausdorff spaces}

Let $X$ be Hausdorff and locally compact.

\begin{proposition}[Uryson's Lemma]
	For all closed $F \subset X$ and all compact $K \subseteq X$ with $F \cap K = \emptyset$, there is a continuous function $f: X \to [0,1]$ such that $f|_K \equiv 1$ and $f|_F \equiv 0$.
\end{proposition}

\begin{proposition}[Tietze's extension theorem]
	For all $K \subseteq X$ compact and $f: K \to \mathds{C}$ continuous, there exists and $\tilde f: X \to \mathds{C}$ continuous such that $\tilde f|_K \equiv f$.
\end{proposition}

\begin{proposition}[Alexandroff's compactification theorem]
	$\tilde X = X \cup \{\infty\}$ ($\infty \notin K$) is a compact Hausdorff space with $\topology{O}_{\tilde X} = \topology{O}_X \cup \{ K^\complement \cup \{\infty\} \mid K \subseteq X \text{compact} \}$.

	For example, compactifying $\mathds{R}$ yields the unit circle $\mathds{S}^1$.
\end{proposition}

\begin{proposition}
	Conversely, if $Y$ is a compact Hausdorff space, then for all $y_0 \in Y$ the space $X = Y \setminus \{ y_0 \}$ is a locally compact Hausdorff space.
\end{proposition}

\begin{proposition}
	More generally, if $Y$ is a locally compact Hausdorff space and $Z \subseteq Y$ is a difference of open and closed subsets of $Y$ (i.e. $Z = U \setminus F$ or $Z = F \setminus U$ where $U \subseteq Y$ is open and $F \subseteq Y$ is closed) then $Z$ is locally compact.
\end{proposition}

\begin{exercise}
	Let $X$ be a locally compact Hausdorff space. The following are equivalent:
	\begin{enumerate}[label=(\arabic*)]
		\item $X$ is compact.
		\item $C(X) = C_0(X)$ ($=C_b(X)$).
		\item $C_0(X)$ is unital.
		\item $1 \in C_0(X)$ where $1(x) = 1 \in \mathds{C}$ for all $x \in X$. 
	\end{enumerate}
\end{exercise}

\begin{proof}~
	\begin{itemize}
		\item (1) $\then$ (2): Recall:
		\begin{equation*}
			C_0(X) = \{f \in C(X) \mid \forall_{\epsilon > 0} \{x \in X \mid |f(x)| \geq \epsilon \} \text{ is compact} \}
		\end{equation*}
		If $X$ is compact, then every closed subset of $X$ is compact, so all sets of form $\{ x \in X \mid |f(x)| \geq \epsilon \}$ are compact, and we have $C(X) = C_0(X)$.
		\item (2) $\then$ (3): This is trivial because $C(X)$ is always unital.
		\item (3) $\then$ (4): Suppose $C_0(X)$ is unital and let $f \in C_0(X)$ be the unit.
		Then $f \cdot g = g$ for all $g \in C_0(X)$, that is $f(x) g(x) = 1$ for all $x \in X, g \in C_0(X)$.
		By Uryson's Lemma, given $x_0 \in X$, there exists a $g \in C_0(X)$ with $g(x_0) = 1$ (by looking at $K = \{x_0\}$, take any precompact open neighborhood $U$ of $x$ and look at $ F := U^\complement \subseteq X$).
		Then we have $f(x_0) = f(x_0) g(x_0) = g(x_0) = 1$.
		As this is possible for every $x_0 \in X$, we have $f \equiv 1$.
		\item (4) $\then$ (1): Suppose $f = 1 \in C_0(X)$.
		Then choosing $\epsilon = \frac{1}{2}$ shows that $X = \{ x \in X \mid |f(x)| \geq \frac{1}{2} \}$ is compact.
	\end{itemize}

\end{proof}

\begin{exercise}
	Let $X$ be a locally compact Hausdorff space. Prove that $C_0(X) \simeq \{f \in C(\tilde X) \mid f(\infty) = 0 \}$.
\end{exercise}


\section{Exercise sheet 1}

\begin{exercise}[1.1]
\end{exercise}

\begin{proof}
	\textbf{Case 1}: If $b_1, b_2 \in A$, then $b_i = \alpha_i a$ for certain $\alpha_i \in \mathbb{C}$. Thus, $b_1 \cdot b_2 = \alpha_1 \alpha_2 a^2 = 0$. Thus, the multiplication is trivial. From this, it immediately follows that $\phi: \algebra{A} \to \algebra{M}, \lambda a \mapsto \qmatrix{0 & \lambda \\ 0 & 0}$ is an isomorphism.

	\textbf{Case 2}: $\lambda \neq 0$, and $a^2 = \lambda a$. Let $b = \frac{1}{\lambda} a$, then $b \cdot a = a = a \cdot b$. But then, for any $c = \mu a \in \algebra{A}$, we have $b c = \mu b a = \mu a = c = c b$, so the algebra is unital and isomorphic to $\mathbb{C}$.
\end{proof}

%Per sheet 3-4 tasks in the next week, two weeks => Bonus
%Nothing concerning the exam

\begin{exercise}[1.2]
	We consider pathological examples for $C_0(X)$.

	Let $X = \closure{\{x_0\}}$, e.g. $x_0 \in X$ with $\topology{O}(X) = \{\{x_0\} \cup Y \mid Y \subset X\} \cup \{\emptyset\}$. $X$ is highly non-Hausdorff unless we already have $X = \{x_0\}$.	In this space, the constant sequence $(x_0)$ converges to any $x \in X$.

	For a continuous function $f: X \to \mathds{C}$, this implies $f(x_0) \to f(x)$ for all $x \in X$, so every continuous function must already be constant. It follows that $C(X) \simeq \mathds{C}$.

	We now look at $C_0(X) = \{f \in C(X) \mid \forall_{\epsilon > 0} \{x \in X \mid  |f(x)| \geq \epsilon \} \text{ is compact.}\}$. But since all functions are constant, we can use $f(x_0)$ instead of $X$ and $\{x \in X \mid  |f(x)| \geq \epsilon \}$ is either empty or the whole space. $X$ is compact if and only if $X$ is finite. From here on, assume $X$ to be infinite. Then, only the finite subsets are compact. Thus, if we now have $f \not \equiv 0$, there exists an $|f(x_0)| > \epsilon > 0$ and thus $\{x \in X \mid  |f(x)| \geq \epsilon \} = X$ is not compact. This implies $C_0(X) = \{0\}$.

	To find a non-compact topological space that has non-zero unital $C_0(X)$, consider $X = X_0 \sqcup X_1$ with $X_0$ as before and $X_1$ compact.
\end{exercise}

\begin{theorem}
	Let $\phi: \algebra{A} \to \algebra{B}$ be a $^*$-homomorphism between $C^*$-algebras. Then we already have $\|\phi(a)\| \leq \|a\|$ for all $a \in \algebra{A}$.
\end{theorem}

\begin{exercise}[1.4 - Products]
	Let $(A_i)_{i \in I}$ be a family of $C^*$-algebras and define
	\begin{equation*}
		\prod_{i \in I} A_i = \{a = (a_i)_{i \in I} \mid a_i \in A_i \forall_{i \in I} \text{ and } \|a\| := \sup_{i \in I} \|a_i\| < \infty \}\text{.}
	\end{equation*}
	Addition, multiplication and involution are defined coordinate-wise. We can prove that adding, multiplying and involving any bounded sequence yields another bounded sequence, so these are well-defined. We can also prove the $C^*$-axiom.
\end{exercise}

\begin{remark}[Differences between product and direct sum]~\\
	In addition to the product space, we define
	\begin{equation*}
		\bigoplus_{i \in I} A_i = \left\{(a_i) \in \prod_{i \in I} A_i \mid \forall_{\epsilon > 0} \exists_{\text{finite }F \subseteq I} \forall_{i \notin F} \|a_i\| < \epsilon\right\}\text{.}
	\end{equation*}
	This is a closed subspace of $\prod_{i \in I} A_i$ as the closure of $\bigoplus_{i \in I}^{alg} A_i$, where
	\begin{equation*}
		\bigoplus_{i \in I}^{alg} A_i = \left\{(a_i) \in \prod_{i \in I} A_i \mid \exists_{\text{finite }F \subseteq I} \forall_{i \notin F} \|a_i\| = 0\right\}\text{.}
	\end{equation*}
	For finite $I$, these are all equal. We see that any element in the direct sum can be approximated by a sequence of elements in the algebraic sum. This direct sum is a closed two-sided ideal in the product.

	The product has the following universal property:
	We have (surjective) $^*$-homomorphisms $\pi_j:  \prod_{i \in I} A_i \to A_j$ for all $j \in I$.
	If $B$ is any $C^*$-algebra with $^*$-homomorphisms $\phi_j \to A_j$ for every $j \in I$, there is a unique $^*$-homomorphism $\phi: B \to \prod_{i \in I} A_i$ such that  $\pi_j \circ \phi = \phi_j$.
	This is equivalent to the commutativity of the following diagram:
	\begin{equation*}
		\begin{tikzcd}
			B \arrow{r}{\phi_j} \arrow{d}{\phi} & A_j\\
			A  \arrow[swap]{ur}{\pi_j}
		\end{tikzcd}
	\end{equation*}
\end{remark}

\begin{exercise}[1.5]
	$X$ is a locally compact Hausdorff space that can be written as $X = U \cup V$ with open and disjoint $U,V$ (so $U,V$ are clopen). We want to prove $C_0(X) \simeq C_0(U) \oplus C_0(V)$. To build this map, we map $f \mapsto (f|_U, f|_V)$. We check that this is well-defined and a $^*$-isomorphism.
\end{exercise}

\begin{exercise}[2.6]
	Let $X$ be a locally compact Hausdorff space and $\widetilde{C_0(X)} \simeq C(\tilde{X})$ with $\tilde{X} \coloneq X \sqcup \{\infty\}$ with the topology $\topology{O}_{\tilde{X}} = \topology{O}_X \cup \{ \tilde{X} \setminus K \mid K \subseteq X \text{ kompakt}\}$.

	Observation: If $X$ is already compact, then $\infty$ is an isolated point of $\tilde X$ (i.e. $\{\infty\}$ is clopen).

	If $\algebra{A}$ is a $C^*$-algebra, then $\tilde{\algebra{A}}$ (this is not the same $\tilde{~}$ as on the $X$!) is a $C^*$-algebra with 
	\begin{equation*}
		\|a + \lambda 1\|_{C^*} \coloneq \sup_{b \in \algebra{A}, \|b\| \leq 1} \| ab + \lambda b\|_\algebra{A}
	\end{equation*}
	We check that $\tilde{\algebra{A}}$ is a $C^*$-algebra.
	\begin{itemize}
		\item $C^*$-axiom: $\|a + \lambda 1\|_{C^*}^2 != \|(a + \lambda 1)^* (a + \lambda 1)\|_{C^*}$. We have
		\begin{align*}
			\|a + \lambda \cdot 1\|_{C^*} &= \|(a^*a + \overline{\lambda} a + \lambda a^*) + |\lambda|^2 \cdot 1 \|_{C^*}\\
			&= \sup_{b \in \algebra{A}, \|b\| \leq 1} \| a^*a b + \overline{\lambda} a b + \lambda a^*b + |\lambda|^2 \cdot b\|_\algebra{A}
		\end{align*}
		On the other hand:
		\begin{align*}
			\|a + \lambda \cdot 1\|_{C^*}^2 &\coloneq \sup_{\|b\| \leq 1} \|ab + \lambda b\|_\algebra{A}^2\\
			&= \sup_{\|b\| \leq 1} \| (ab + \lambda b)^* (ab + \lambda b)\|_\algebra{A}\\
			&= \sup_{\|b\| \leq 1} \| b^* a^* a b + \overline{\lambda} b^* ab + \lambda b^* a^* b + |\lambda|^2 b^* b\|_\algebra{A}\\
			&\leq \sup_{\|b\| \leq 1} \|b^*\|_\algebra{A} \cdot \| a^* a b + \overline{\lambda} a b + \lambda  a^* b + |\lambda|^2 b\|_\algebra{A} \\
			&\leq \sup_{\|b\| \leq 1} \| a^* a b + \overline{\lambda} a b + \lambda a^* b + |\lambda|^2 b\|_\algebra{A}\\
			&= \|a + \lambda \cdot 1\|_{C^*}^2
		\end{align*}
		\item The other conditions are easy to check and are left for the student.		
	\end{itemize}
	We still want to prove $\phi: \widetilde{C_0(X)} \to C(\tilde X), f + \lambda \cdot 1 \mapsto f_\lambda$ with $f_\lambda(x) \coloneq \lambda$ for $x = \infty$ and $f_\lambda(x) = f(x) + \lambda$ otherwise.
	Nothe that once again these are not the same $\tilde{~}$.
	\begin{itemize}
		\item $f$ is well-defined:
		We have to check that $f_\lambda$ is continuous in $\tilde X$.
		Take any sequence $X \ni x_i \to \infty$ in $\tilde X$. We have to show $f_\lambda(x_i) \to f_\lambda(\infty) = \lambda$. Since $f_\lambda(x_i) = f(x_i) + \lambda$ this is equivalent to $f(x_i) \to 0$.
		But as $f \in C_0(X)$, we have that for every $\epsilon > 0$ the set $K_\epsilon(f) = \{ x \mid |f(x)| \geq \epsilon  \}$ is compact.
		Since $x_i$ will eventually leave this compact set (or it would not diverge to $\infty$), we know that $f(x_i)$ eventually becomes smaller than (any) $\epsilon$.
		So we have $f(x_i) \to 0$ and thus $f_\lambda(x_i) \to f_\lambda(\infty)$.
		So $f_\lambda$ is continuous in $\infty$. The continuity on every other point follows immediately from the continuity of $f$.
		\item $\phi$ is a $^*$-isomorphism:
		\begin{itemize}
			\item Linearity: $\phi$ is clearly linear as we can check component-wise:
			\begin{equation*}
				(f_1 + f_2)_\lambda = (f_1)_\lambda + (f_2)_\lambda
			\end{equation*}
			\item Homomorphism: For every $x \in X$ we have
			\begin{align*}
				\phi((f + \lambda \cdot 1) \cdot (g + \lambda' \cdot 1))(x) &= \phi((fg + \lambda' f + \lambda g) + \lambda \lambda' \cdot 1)(x)\\
				&= (fg + \lambda' f + \lambda g)(x) + \lambda \lambda' \\				
				&= (fg)(x) + \lambda' f(x) + \lambda g(x) + \lambda \lambda' \\
				&= (f(x) + \lambda) \cdot (g(x) + \lambda')\\
				&= (\phi(f + \lambda) \cdot \phi(g + \lambda'))(x) \text{.}
			\end{align*}
			In the case of $x = \infty$, this equality of course also holds. Thus we have $\phi((f + \lambda)(g + \lambda'))  = \phi(f + \lambda) \phi(g + \lambda')$.
			\item $^*$-homomorphism: 
			\begin{equation*}
				\phi(f + \lambda)^*(x) = \phi(f^* + \overline{\lambda} \cdot 1)(x)
			\end{equation*}
			For $x \in X$ this follows by $\overline{f(x)} + \overline{\lambda} =  f^*(x) + \overline{\lambda}$, for $x = \infty$ we have $\overline{\lambda} = \overline{\lambda}$. 
			\item Injective: $f_\lambda(0)$ leads to $f_\lambda(x) = 0$ for all $x \in \tilde X$, since if $x = \infty$ then $\lambda$ must be $0$ and $f(x) = 0$ for all $x \in X$. Thus $f = 0$ and $\lambda = 0$.
			\item Surjective: Take $g \in C(\tilde X)$  and choose $\lambda = g(\infty)$ and $f(x) \coloneq g(x) - \lambda$. and check $f \in C_0(X)$.
		\end{itemize}
		\item We can also prove that $\phi$ is isometric for the $C^*$-norm:
		\begin{equation*}
			\|f + \lambda \cdot 1\| \coloneq \sup_{g \in C_0(X), \|g\| \leq 1} \|fg + \lambda g \|_\infty
		\end{equation*}
		Look at
		\begin{align*}
			\| \phi(f + \lambda 1)\| &= \sup_{x \in \tilde{X}} | f_\lambda(x)| = \max \{ |\lambda|, \sup_{x \in X} |f(x) + \lambda| \}\\
			&\overset{(*)}{=} \sup_{x \in X} |f(x) + \lambda|
		\end{align*}
		and 
		\begin{align*}
			\| f + \lambda \cdot 1\|_{C^*} &= \sup_{\|g\| \leq 1} \|fg + \lambda g\|_{C_0(X)}\\
			&= \sup_{|g(x)| \leq 1 \forall_x} \sup_{x \in X} \| f(x) g(x) + \lambda g(x) \|\\
			&= \sup_{x \in X} |f(x) + \lambda|\\
			&\overset{(**)}{=} \sup_{x \in X} |f(x) + \lambda|
		\end{align*}
		This proof may need to be divided into two cases:
		\begin{itemize}
			\item $X$ is not compact: We can find a net $(x_i) \subseteq X$ with $f(x_i) \to 0$ and $(*)$ follows and use a $g(x) \approx 1$ for $(**)$.
			\item $X$ is compact: Choose $g \equiv 1$ for $(**)$ and think about $(*)$ later.
		\end{itemize}
	\end{itemize}
\end{exercise}

\begin{exercise}[1.8]
	It is difficult to prove $I^* = I$. The idea is to prove $I = C_0(U)$ where $C_0(U) = \{ f \in C_0(X) \mid f|_{U^\complement} \equiv 0 \}$. 

	One can also prove $C_0(X)/C_0(U) \simeq C_0(F)$ (as $C_0$ of the subspace) where $F = U^\complement$.
\end{exercise}

\begin{exercise}
	Prove that $\algebra{A}/\ring{I}$ is normed algebra, and
	\begin{enumerate}
		\item if $\algebra{A}$ is Banach and $I \trianglelefteq A$ is closed, then $\algebra{A} / \ring{I}$ is Banach.
		\item if $\algebra{A}$ is unital and Banach, then $\algebra{A} / \ring{I}$ is unital.
	\end{enumerate}
	
	unital if $\algebra{A}$ is, Banach if $\algebra{A}$ is and $\ring{I} \trianglelefteq \algebra{A}$ closed.
\end{exercise}

\begin{proof}
	Consider $\algebra{A}/\ring{I}$ with $(a + \ring{I})(b + \ring{I}) = ab + \ring{I}$.
	For the norm, use $\|a + \ring{I}\| = \dist(a, \ring{I}) = \inf_{x \in \ring{I}} \|a - x\|$.
	This is submultiplicative. For every $\epsilon > 0$, there exist $x,y \in \ring{I}$ for which we have
	\begin{equation*}
		(\epsilon + \|a + \ring{I}\|) \cdot (\epsilon + \|b + \ring{I}\|) \geq \|a + x\| \cdot \|b + y\| \geq \|(a+x)(b+y) \| \geq \|ab + \underbrace{ay + xb + xy}_{\in \ring{I}} \| \geq \|ab + \ring{I}\|
	\end{equation*}
	and taking the limit yields the desired result.

	Result (i) follows from functional analysis, that a space is Banach if and only if the convergence of $\sum_{k = 0}^{\infty} a_n$ is equivalent to the convergence of $\sum_{k =0}^{\infty} \|a_n\|$.

	Now let $\algebra{A}$ also be unital, then $\algebra{A} / \ring{I}$ is unital.
	If $\ring{I} = \algebra{A}$, the algebra is the zero-algebra.
	Thus, let $\ring{I}$ be a proper ideal.
	The fact that $1 = 1_\algebra{A} + \ring{I}$ is a unit is clear, but we need to prove $\| 1_\ring{A} + \ring{I}\| = 1$.
	Observe that, if $x \in \ring{I} \triangleleft \algebra{A}$ then $x \notin \inv(\algebra{A})$ and $\|1_\algebra{A} + x\| \geq 1$.
	Because otherwise, we have $\| 1_\algebra{A} + x \| < 1$ and then (because $\algebra{A}$ is Banach) $x = a - 1_\algebra{A} \in \inv(\algebra{A})$.
	Hence $\|1_\algebra{A}\| = \inf_{x \in \ring{I}} \|1_\algebra{A} + x \| \geq 1$.
	In addition, we have $1 \leq \|1_\algebra{A} + \ring{I}\| = \inf_{x \in \ring{I}} \|1_\algebra{A} - x \| \leq \|1_\algebra{A} + 0\| \leq 1$. This proves $\|1\| = \|1_\algebra{A} + \ring{I}\| = 1$.
\end{proof}

In the following, $\mathds{D}$ is the \textbf{closed} unit circle.

\begin{exercise}
	Consider $\chi \in \Omega(\algebra{A})$.
	We have proved $\|\chi\| \leq 1$. 
	It may happen that $\|\chi\| < 1$.
	We need a non-unital algebra for this, because we have $\|\chi\| = 1$ if $1 \in \algebra{A}$.

	Consider $S = (\mathds{N}, +)$ as an additive semigroup. Then
	\begin{equation*}
		\ell^1(S) = \{(a_n)_{n \in \mathds{N}} \mid \sum_{n = 0}^{\infty} |a_n| < \infty\}
	\end{equation*}
	is a unital Banach algebrea with $\delta_n \cdot \delta_m = \delta_{n + m}$ for all $n,m \in \mathds{N}$ where 
	\begin{equation*}
		\delta_n(m) = \left\{ \begin{matrix}
			1 & m = n \\ 0 & m \neq 0
		\end{matrix} \right.
	\end{equation*}
	Observe $\ell^1(S) = \closure{\mathrm{alg}}\{\delta_0, \delta_1\}$ because of $\delta_1^n = \delta_n$.
	The unit of the algebra is $\delta_0$.
	What are the characters of $\ell^1(S)$?

	We can write any $a \in \ell^1(S)$ as $	a = \sum_{n=0}^{\infty} a_n \delta_n$. So if $\chi \in \Omega(\ell^1(S))$ then
	\begin{equation*}
		\chi(a) = \sum_{n = 0}^{\infty} a_n \chi(\delta_n) \in \mathds{C}\text{.}
	\end{equation*}
	In particular, $\chi(1) = 1$ so $\chi(\delta_0) = 1$.
	This leads to $\chi(\delta_n) = \chi(\delta_1^n) = \chi(\delta_1)^n = \chi(\delta_1)^n$.
	So if we set $z \coloneq \chi(\delta_1) \in \mathds{C}$, we have $\chi(a) = \sum_{n = 0}^{\infty} a_n z^n$.
	Observe $|z| = |\chi(\delta_1)| \leq \|\delta_1\| = 1$ (because the Image of a character is a subset of the spectrum, which is bounded by the norm) so $z$ must be in $\mathds{D}$. By conventioning $z^0 = 1$ for every $z \in \mathds{C}$, we can even choose $z = 0$.

	Conversely, if $z \in \mathds{D}$, we define $\chi_z(a) \coloneq \sum_{n = 0}^{\infty} a_n z^n \in \mathds{C}$. Then $\chi_z(\delta_n) = z^n$ and
	\begin{equation*}
		\chi_z(\delta_n \cdot \delta_m) = \chi(\delta_{n + m}) = z^{n+m} = z^n \cdot z^m = \chi_z(\delta_n) \cdot \chi_z(\delta_m)
	\end{equation*}
	So we get a map $\mathds{D} \to \Omega(\ell^1(S)) \subseteq \ell^1(S)^*, z \mapsto \chi_z$ that is bijective and continuous.
	If $z_i \to z$ in $\mathds{D}$, we need to prove $\chi_{z_i} \to \chi_z$ in respect to the weak $^*$-topology.
	So we need to evaluate and prove $\chi_{z_i} (a) \to \chi_z(a)$, or $\sum_{n=0}^{\infty} a_n z_i^n \to \sum_{n = 0}^{\infty} a_m z^m$. Partial sums would obviously converge, so $\chi_{z_i}$ converges on a dense subspace of $\ell^1(S)$.
	The uniform boundedness principle (if a bounded set of operators converge on a dense subset $T_i \to T$, $\sup_i \| T_i \| < \infty$, they converge everywhere) shows that the infinite sums also converge.
	In general, showing that an operator converges on a dense set of an algebra always shows the convergence on any point of the algebra.

	Observe $\sigma(\delta_1) = \{\chi(\delta_1) \mid \chi \in \Omega(\ell^1(S)) \} = \mathds{D}$ and $\sigma(\delta_1) = \mathds{D}$ as well.

	Concerning the norm, we know that 
	\begin{equation*}
		|\chi_z(a)| = \left| \sum_{n = 0}^{\infty} a_n z^n \right| \leq \sum_{n=0}^{\infty} |a_n| |z|^n \leq \sum_{n=0}^{\infty} |a_n| = \|a\|
	\end{equation*}
	for all $a \in \ell^1(S)$, so $\|\chi_z\| \leq 1$.
	For $a = (a_0, 0, 0, \dots)$ we have $|\chi_z(a)| = |a_0| = \|a\|$, so $\|\chi_z\| = 1$ for any $z \in \mathds{C}$ (and thus for any $\chi = \chi_z \in \Omega(\ell^1(S)))$.
\end{exercise}

\begin{remark}[Gelfand-Representation]
	In general, we seek a mapping $\algebra{A} \to C_0(X), a \mapsto \hat{a}$, taking $X = \hat{\algebra{A}} = \Omega(\algebra{A})$ and $\hat{a}(\chi) = \chi(a)$.
	
	If we apply the Gelfand representation here, we have
	\begin{equation*}
		\ell^1(S) \to C(\mathds{D}), a \mapsto \hat{a} \text{ where } \hat{a}(z) = \chi_z(a) = \sum_{n=0}^{\infty} a_n z^n
	\end{equation*}
\end{remark}

\begin{example}[Norms $<1$]
	Consider 
	\begin{equation*}
		\ell^1_0(S) = \closure{\mathrm{alg}}(\delta_1) = \left\{ \sum_{n = 1}^{\infty} a_n \delta_n \mid a_n \in \mathds{C} \right\} \triangleleft \ell^1(S)
	\end{equation*}
	Observe $\widetilde{\ell^1_0(S)} \simeq \ell^1(S)$. Recall $\Omega(\tilde{\algebra{A}} = \Omega(\algebra{A}) \sqcup \{\chi_\infty\}$.
	So we are looking for our $\chi_\infty$, which is $\chi_\infty(a_0, a_1, \dots) = a_0$ -- that is $\chi_0$ and corresponds to $z=0$ in the unit circle.
	It follows $\Omega(\ell^1_0(S)) \simeq \mathds{D} \setminus \{ 0 \}$ and $\chi_0 \in \Omega(\ell^1(S)) \setminus \Omega(\ell^1_0(S))$. 

	We compute $\| \chi_z \| = \sup_{\|a\|_1 \leq 1} |\chi_z(a)|$. Consider:
	\begin{equation*}
		|\chi_z(a)| = \left| \sum_{n = 1}^\infty a_n z^n \right| = \left| z \left(\sum_{n=1}^{\infty} a_n z^{n-1} \right) \right| \leq |z| \cdot \| a \|_1
	\end{equation*}
	so because of $\chi_z(\delta_1) = z$, we have $\|\chi_z\| = |z|$, which can be smaller than $1$.
\end{example}

\begin{remark}
	Do we have $\ell^1(S) \hookrightarrow A(\mathds{D}), a \mapsto \hat{a}$ where $\hat{a}(z) \coloneq \sum_{n = 0}^\infty a_n z^n$?
\end{remark}

\begin{exercise}[02-03]
	Is $\algebra{A}(\mathds{D})$ a $C^*$-algebra?
	Consider $f(z) = \exp(iz)$, $f \in \algebra{A}$ and notice $z^* = z$.
	% Then $u = \exp(2a)$ is unitary, as $u^* = \exp(-2a)$ and thus $u^* u = 1 = u u^*$.
	% Thus $\|u\| = 1$.
	But we have $\|f^* f \|_\infty \neq \|f\|_\infty^2$, because $f^* f = 1$ and because $f(-i) = e$, we have $\|f\|_\infty \geq e$ and $\|f\|_\infty^2 \geq e^2 > 1 = \|f^* f\|_\infty$.
	Since the $^*$-property is not fulfilled.
\end{exercise}


\begin{remark}
	Talk about functoriality.
	If $X,Y$ are compact Hausdorff spaces and $f: X \to Y$ is continuous then 
	\begin{equation*}
		f_*: C(Y) \to C(X), g \mapsto g \circ f
	\end{equation*}
	You can check that $f_*$ is a unital $^*$-homomorphism. 
	So we receive a functor from the compact spaces	to the unital commutative $C^*$-algebras:
	\begin{align*}
		\text{Comp. Spaces} \to \text{unital abelian $C^*$}, X \mapsto C(X)\\
		Hom(X, Y) \to Hom(C(Y), C(X)),  f \mapsto f_*
	\end{align*}
	This is a contravariant function because for $f: X \to Y, g: Y \to Z$ we have $(g \circ f)_* = f_* \circ g_*$.
	It is also natural.
	If $\phi: C(Y) \to C(X)$ is a unital $^*$-homomorphism, we get a continuous map $f: X \to Y$ by duality.
\end{remark}

\section{More multiplier algebra}

We continue to look at the multiplier algebra. 
\begin{equation*}
	M(\algebra{A}) = \{ \mu = (L,R) \in L(\algebra{A}) \times L(\algebra{A}) \mid a L(b) = R(a) b, L(ab) = L(a)b, R(ab) = a R(b)\}
\end{equation*}
If $\algebra{A}$ is a $C^*$-algebra, $\algebra{A}$ embeds into $M(\algebra{A})$ as an essential ideal.
If $A$ embeds into a $C^*$-algebra $B$ as an essential ideal, then $B \to M(\algebra{A}), b \to \mu_b$ with $(a \mapsto ba, a \mapsto ab)$ is an isomorphism.

We also define the \textbf{strict topology} on $M(\algebra{A})$ as the smallest topology that makes the map $a \mapsto \mu \cdot a, a \mapsto \mu \cdot a$ norm-continuous on $\algebra{A} \to \algebra{A}$.
So if $(\mu_i) \subseteq M(\algebra{A})$ is a net, then $\mu_i \to \mu \in M(\algebra{A})$ if and only if $\mu_i a \to \mu a$ and $a \mu_i \to a \mu$ for all $a \in \algebra{A}$.

\begin{remark}
	Writing $0 \leq a \leq 1$ in a $C^*$-algbra means $a \geq 0$, so $\sigma(a) \subseteq [0, \infty)$ and $a \leq 1$ means $(1 - a)$ is positive in $\tilde{\algebra{A}}$ or $M(\algebra{A})$ which is equivalent to $\|a\| \leq 1$.

	All of this is equivalent to $\sigma(a) \subseteq [0,1]$.
\end{remark}

\underline{Relation with approximate units}: If $(e_i) \subseteq A_{+, 1}$ is an increasing net $(0 \leq e_i \leq 1)$ then $(e_i)$ is an approximate unit iff $e_i \to 1$ (strictly) in $M(\algebra{A})$.

By definition this means $e_i a \xrightarrow{\|\cdot\|} a$, $a \cdot e_i \xrightarrow{\|\cdot\|} a$.

\subsection{Non-degeneratre $^*$-homomorphisms}

\begin{definition}
	Let $\pi: \algebra{A} \to M(\algebra{B})$ a $^*$-homomorphism. We say that $\pi$ is \textbf{non-degenerate} if $\closure{\Span \pi(\algebra{A}) \cdot B} = B$.
\end{definition}

\begin{lemma}
	Let $\pi: \algebra{A} \to M(\algebra{B})$ be a $^*$-homomorphism. The following are equivalent:
	\begin{enumerate}
		\item $\pi$ is non-degenerate.
		\item $\pi(e_i) \to 1$ (strictly) in $M(B)$ if $(e_i)$ is some approximate unit in $\algebra{A}$.
		\item $\pi$ extends to a strictly continuous unital $^*$-homomorphism $\tilde{\pi}: M(\algebra{A}) \to M(\algebra{B})$.
	\end{enumerate}
\end{lemma}

\begin{proof}~
	\begin{enumerate}
		\item $\then$ (ii): Let $(e_i)$ be an approximate unit. 
		Prove $\pi(e_i) \to 1$ (strictly) in $M(\algebra{B})$, that is $\pi(e_i) b \to b$ for all $b \in \algebra{B}$.
		Since $(e_i)$ is bounded, it is enough to show that $\pi(e_i) b \to b$ for all $b \in \pi(A) B$ as this is dense in $B$.B
		ut if $b = \pi(a) c$ for $a \in \algebra{A}, c \in \algebra{B}$, then $\pi(e_i) b = \pi(e_i) \pi(a) c = \pi(e_i a) c \to \pi(a) c = b$ because $\pi$ is norm-continuous and $e_i a \to a$.  
		\item $\then$ (iii): We want to extend $\pi$ to $\tilde{\pi}: M(A) \to M(B)$.
		We need $\tilde{\pi}$ such that 
		\begin{equation*}
			\tilde{\pi}(\mu) \pi(a) = \tilde{\pi}(\mu) \tilde{\pi}(a)  \tilde{\pi}(\mu \cdot a) = \pi(\mu \cdot a)\text{.}
		\end{equation*}
		Therefore, define the multiplier $\tilde{\pi}(\mu)$ just on $\pi(\algebra{A}) \algebra{B}$ by the mappings
		\begin{align*}
			L(\pi(a) b) = \tilde{\pi}(\mu) \cdot (\pi(a) b) &= \pi(\mu a) \cdot b \in \algebra{B}\\
			R(\pi(a) b) = (\pi(a) b) \cdot \tilde{\pi}(\mu) &= b \pi(a \mu) \in \algebra{B} \text{.}
		\end{align*}
		These morphisms are certainly linear.
		By (ii), notice that $\closure{\pi(\algebra{A}) \algebra{B}} = \algebra{B}$.
		So the above defines morphism on all of $\algebra{B}$ by continuous extension.

		We need to prove: $L,R$ are well-defined and extend to $\algebra{B}$ and $\mu = (L,R)$ is a multiplier of $\algebra{B}$.

		Claim:
		\begin{equation*}
			\| \sum_{i=0}^n \pi(\mu \cdot a_i) \cdot b_i \| \leq \|\mu \| \cdot \| \sum_{i=0}^n \pi(a_i) b_i \|
		\end{equation*}
		for all sequences $(a_i) \subseteq \algebra{A}$ and $(b_i) \subseteq \algebra{B}$.
		% Consider
		% \begin{align*}
		% 	\left\| \sum_{i=0}^{n} \pi(\mu a_i) b_i  \right\|^2  &= \left\| \sum_{i,j = 0}^n b_i^* \pi(\mu a_i)^* \pi(\mu a_j) b_j \right\| \\
		% 	&= \left\| \sum_{i,j = 0}^{n} b_i^* \pi(a_i^* \mu^* \mu a_j) b_j \right\|
		% \end{align*}
		% Idea: $a^* \mu^* \mu a \leq \|\mu\|^* a^* a$ in the multiplier of $\algebra{A}$ because $\mu^* \mu \leq \|\mu\|^2 \cdot 1$ (think of them as functions).
		% But $a_i^* \mu^* \mu a_j \leq \|\mu\|^2 a_i^* a_j$ does not even make sense because $a_i^* a_j$ might not be self-adjoint.
		% Better way:
		To prove that $L$ is well-defined compute $\tilde{\pi} \pi(a) b = \pi(\mu a) b$.
		\begin{equation*}
			\pi(\mu a) b = \lim_i \pi(\mu e_i a) b = \lim_i \pi(\mu e_i) \pi(a) b
		\end{equation*}
		This proves well-definedness, because if the right sides are equal (multiple ways to write $\pi(a)b$), then the left side, $\tilde{\pi}(\mu) \pi(a) b$ must also be equal (for these multiple representations).

		So $L$ is well-defined. 
	\end{enumerate}
\end{proof}

\end{document}



























































