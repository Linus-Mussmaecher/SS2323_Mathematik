\documentclass[a4paper, twocolumn, 10pt]{article}

% --- PREAMBLE ---

\usepackage[english]{babel}	% language specific quotation marks etc.
% === USAGE ===

% when using this preamble, setup your environment variables like this beforehand:


% \title{Stochastik 2}  %Title of exercise 
% \def\lecture{Stochastik 2}
% \def\authors{Linus Mußmächer}
% \def\sheetNumber{02}
% \def\sumPoints{30}      % maximum number of points (leave undefined)

% then use one of these commands (german or english) to print the header:

% \makeexheaderger

% and finally use subsections for your subtasks - they will be numbered as <sheetNumber><task number> by themselves

% if you have an exercise as an external .pdf, use \includetask to include it and increase the task counter


% --- OTHER ---

\usepackage{booktabs}       % professional-quality tables
\usepackage[table]{xcolor}	% color
\usepackage{pdfpages}		% to include entire pdf pages in appendix etc.
\usepackage{enumitem}		% better custom enumerations
\setlist[enumerate, 1]{label=(\roman*)}
\usepackage{etoolbox}		% toolbox for command modification

% --- FONTS & TYPESETTING ---

\usepackage[utf8]{inputenc} % allow utf-8 input
\usepackage[T1]{fontenc}    % use 8-bit T1 fonts
\usepackage{dsfont}			% font with double lines for sets
\usepackage[german,ruled,vlined,linesnumbered,commentsnumbered,algoruled]
{algorithm2e} 				%pseudo code
\usepackage{listings}		%java code
\usepackage{csquotes}

% --- URLS ---

\usepackage[colorlinks=true, linkcolor=black, citecolor=blue, urlcolor=blue]{hyperref}   	% hyperlinks
\usepackage{url}            % simple URL typesetting

% --- MATH SYMBOLS ---

\usepackage{amsmath,amssymb}% more math symbols
\usepackage{amsfonts}       % blackboard math symbols
\usepackage{latexsym}		% more math symbols
\usepackage{chngcntr}		% more math symbols
\usepackage{mathrsfs}		% math-fonts
\usepackage{mathtools}		% more math symbols
\usepackage{nchairx}		% Waldmann package for general math symbols

% --- GRAPHICS & CAPTIONS ----

\usepackage{graphicx}		% including images
\graphicspath{ {./figs/} }
\usepackage{subcaption}		% custom caption formatting
\DeclareCaptionLabelFormat{custom}{ \textbf{#1 #2}}
\captionsetup{format=hang}
\captionsetup{width=0.9\textwidth,labelformat=custom}
\usepackage{pdfpages}		% to include entire pdf pages in appendix etc.

% --- FORMAT ---

\usepackage[a4paper]{geometry} % a4 paper
\usepackage{setspace}		% spacing
\usepackage{titlesec}
\allowdisplaybreaks			% allow page breaks within math environments

% --- CUSTOM COMMANDS ---
%Logic
\newcommand{\then}{\Rightarrow}
\newcommand{\since}{\Leftarrow}
\renewcommand{\iff}{\ensuremath{\Leftrightarrow}}

%pretty epsilon
\let\oldepsilon\epsilon
\let\epsilon\varepsilon
\let\varepsilon\oldepsilon
%pretty phi
\let\oldphi\phi
\let\phi\varphi
\let\varphi\oldphi

\newcommand{\includetask}[2][pages=-]{
    \includepdf[#1]{#2}
    \addtocounter{subsection}{1}
}

% set-up for exercise specific stuff
\ifdef{\sheetNumber}{
    \setcounter{section}{\sheetNumber}
}{}

\usepackage{titling}
\newcommand{\makeexheaderger}{
    \begin{doublespace}
        \begin{center}
            \textbf{\Large{Übungsblatt \sheetNumber}}\\
            \textbf{\Large\lecture}\\
            Abgabe von: \textbf{\authors}\\
            \today
        \end{center}
        \ifdef {\sumPoints}
        {
            \hfill  \large Punkte: $\boxed{\qquad  /\; \sumPoints}$\\
        }{}
    \end{doublespace}
}

\newcommand{\makeexheadereng}{
    \begin{doublespace}
        \begin{center}
            \textbf{\Large{Exercise Sheet \sheetNumber}}\\
            \textbf{\Large\lecture}\\
            Contributors: \textbf{\authors}\\
            \today
        \end{center}
        \ifdef {\sumPoints}
        {
            \hfill  \large Points: $\boxed{\qquad  /\; \sumPoints}$\\
        }{}
    \end{doublespace}
}

\newcommand{\qmatrix}[1]{\ensuremath{\left(\begin{matrix}#1\end{matrix}\right)}}
\geometry{left=1.0cm, top=1.0cm, right=1.0cm, bottom=1.0cm}


\begin{document}

\pagestyle{empty}

\section{Banach Algebras}

\begin{definition}
    Algebra, Subalgebra, Norm, Complete, Banach algebra, unital, homomorphisms    
\end{definition}

\begin{theorem}
    Closed subspace of Banach is Banach.
\end{theorem}

\begin{theorem}
    $I$ closed ideal $\then$ $A/I$ normed algebra with norm $\|a + I\| = \inf_{b \in I} \|a + b\|$.
\end{theorem}

\section{Spectrum and Spectral Radius}

Considering unital normed algebras.

\begin{definition}
    Invertible elements, spectrum, spectral radius.
\end{definition}

\begin{remark}
    $1 - ab$ invertible iff $1 - ba$ invertible. $\sigma(ba) \setminus 0 = \sigma(ba) \setminus 0$.
\end{remark}

\begin{theorem}
    $\sigma(a)$ non-empty and $p \in \mathds{C}[z]$ $\then$ $\sigma(p(a)) = p(\sigma(a))$.
\end{theorem}

\begin{theorem}
    $\|a\| < 1 \then 1 - a \in \inv(A), (1-a)^{-1} = \sum_{n=0}^\infty a^n$.
\end{theorem}

\begin{theorem}
    $\inv(A)$ open and $a \mapsto a^{-1}$ differentiable.
\end{theorem}

\begin{theorem}
    $\sigma(a)$ non-empty, closed and $\subseteq \overline{K_{\|a\|}(0)}$, $\mathds{C}\setminus \sigma(a) \to A, \lambda \mapsto (a - \lambda)^{-1}$ differentiable.
\end{theorem}

\begin{theorem}
    $A$ unital, Banach and $\inv(A) = A \setminus \{0\} \then A = \mathds{C}1$.
\end{theorem}

\begin{theorem}
    $r(a) = \lim_{n \to \infty} \|a^n\|^{1/n} = \inf_{n \geq 1} \|a^n\|^{1/n}$.
\end{theorem}

\begin{theorem}
    $1 \in B \leq A$ closed, $A$ Banach. Then $\inv(B) = B \cap \inv(A)$ closed. $\sigma_A(b) \subseteq \sigma_B(b)$, also for boundaries. Equality if $\sigma_A(b)$ has no holes or both are $C^*$.
\end{theorem}

\section{Gelfand Representation}

\begin{definition}
    Ideal, characters, character space.
\end{definition}

\begin{theorem}
    $A$ Banach.
    Proper ideals have proper closure.
    Maximal ideals are closed.
    If $A$ abelian, unital: Quotients of maximal ideals are fields.
\end{theorem}

\begin{theorem}
    $A$ Banach, abelian, unital. If $r \in \Omega(A) \then \|r\| = 1$. $\Omega(A)$ non-empty and $r \mapsto \ker(r)$ is a bijection between $\Omega(A)$ and the maximal ideals in $A$.
\end{theorem}

\begin{theorem}
    $A$ Banach, abelian. $A$ unital $\then \sigma(a) = \Omega(A)(a)$. $A$ non-unital $\then \sigma(a) = \Omega(A)(a) \cup \{0\}$. 
\end{theorem}

\begin{theorem}
    $A$ Banach, abelian $\then \Omega(A)$ locally compact Hausdorff space. $A$ unital $\then \Omega(A)$ compact.
\end{theorem}

\begin{theorem}
    $A$ Banach, abelian, $\Omega(A) \neq \emptyset$.
    \begin{equation*}
        \Phi: A \to C_0(\Omega(A)), a \mapsto (\hat{a}: \Omega(A) \to \mathds{C}, r \mapsto r(a))
    \end{equation*}
    norm-decreasing homomorphism and $r(a) = \|\hat{a}\|_\infty$.
    $A$ unital $\then \sigma(a) = \hat{a}(\Omega(A))$.
    $A$ non-unital $\then \sigma(a) = \hat{a}(\Omega(A)) \cup \{0\}$.

    $A$ Banach, $A = (1,a) \then A$ abelian and $\hat{a}$ homeomorphism.

    $A$ $C^* \then \Phi$ isometric isomorphism with weak-$^*$-topology.
\end{theorem}

\section{$C^*$-algebras}

\begin{definition}
    Involution, $^*$-algebra, $C^*$-algebra, self- adjoint, unital (isometry, co-isometry), normal, projection.
\end{definition}

\begin{theorem}
    $a =b + ic \frac{1}{2} (a + a^*) + i \frac{1}{2i} (a - a^*)$ with $b,c$ self-adjoint.
\end{theorem}

From now on: $C^*$-algebras, so $\|a a^*\| = \|a\|^2$ ($\geq$ enough).

\begin{theorem}
    If $A$ is self-adjoint then $\sigma(a) \subseteq \mathds{R}$ and $r(a) = \|a\|$. On every $^*$-algebra, there is at most one norm to make it $C^*$.
\end{theorem}

\begin{theorem}
    Multiplier-algebra of $C^*$: Largest unitization, $\|L\| = \|R\|$.
    Extension of norm of $C^*$ makes $\tilde{A}$ into $C^*$.
\end{theorem}

\begin{theorem}
    $^*$-hom between $^*$-alg and $C^*$ are norm-decreasing. $^*$-hom between $C^*$ are isometric if injective and the image is a $C^*$-subalgebra.
\end{theorem}

\begin{theorem}
    Characters on $C^*$ preserve adjoints.
\end{theorem}

\begin{theorem}
    $B$ $C^*$-subalgebra. $\sigma_B(b) = \sigma_A(a)$.
\end{theorem}

\begin{theorem}
    $a$ normal in unital $C^*$ $A$ $\then$ exists $\phi: C(\sigma(a)) \to C^*(1,a)$ unital isometric $^*$-iso with $\phi(\id) = a$.
    Write $f(a) \in A$ for $\phi(f)$.
\end{theorem}

\begin{theorem}
    $a$ normal, $f \in C(\sigma(a)) \then f(\sigma(a)) = \sigma(f(a))$. If $g \in C(\sigma(f(a))) \then (g \circ f)(a) = g(f(a))$.
\end{theorem}

\begin{theorem}
    $X$ compact Hausdorff. $X \simeq \Omega(C(X))$.
\end{theorem}

\section{Positive Elements in $C^*$}

\begin{definition}
    Positive elements (hermitsch und $\sigma(a) \subseteq \mathds{R}^+_0$), ordered elements
\end{definition}

\begin{theorem}
    $B^+ = A^+ \cap B$.
    $A^+ \subseteq A_{sa}$.
    $A^+ = \{a^* a \mid a \in A\}$.
    Conjugation self-adjoint elements keeps their order. 
    $a \leq b \then \|a\| \leq \|b\|$
    Inverting invertes order, square roots keep it (and square roots exist).
\end{theorem}

\section{Ideals in $C^*$}

\begin{definition}
    Approximate units (increasing net of positive elements ), essential ideals.
\end{definition}

\begin{theorem}
    $C^*$-algebras have approximate units (take $A^+$ with $\|a\| < 1$.)
\end{theorem}

\begin{theorem}
    Quotients and approximate units.
    Quotient of closed ideal is $C^*$-algebra.
    If $B$ is a $C^*$-subalgebra and $I$ a closed ideal, then $B+I$ is a $C^*$-subalgebra.
\end{theorem}

\begin{theorem}
    $I$ closed in $C^*$ $A \then \exists$ unique $^*$-extension $A \to M(I)$ of $I \to M(I)$, injective if $I$ essential.
\end{theorem}

\section{Positive linear functionals}
\begin{definition}
    Positive maps, positive linear functionals, states
\end{definition}

\begin{theorem}
    $^*$-homs are positive. $\phi(A_{sa}) \subseteq \phi(B_{sa})$ and $\phi|_{A_{sa}}$ is increasing.
\end{theorem}

\begin{theorem}
    PLFs are bounded and $r(a^*) = r(a)^-$ and $|r(a)|^2 \leq \|r\| r(a^* a)$.
    $\|r + r'\| = \|r\| + \|r'\|$.
    $r(a^* a) = 0 \iff r(ba) = 0$ for all $b \in A$.
    $r(b^* a^* ab) \leq \|a^*a\| r(b^*b)$.
\end{theorem}

\begin{theorem}
    For a bounded linear functional $r$, these are equivalent: $r$ is positive for each/some approx. unit we have $\|r\| = \lim_\lambda r(e_\lambda)$.
    If $A$ is unital, $r$ is positive iff $r(1) = \|r\|$ 
\end{theorem}

\begin{theorem}
    There exists a state $r$ of $A$ such that $\|a\| = |r(a)|$.
\end{theorem}

\begin{theorem}
    You can extend linear functionals on $C^*$-subalgebras to the whole algebra while keeping the norm.
\end{theorem}

\begin{theorem}
    Self-adjoint bounded linear functionals can be decomposed to positive linear functionals with $r = r_+ = r_-$ and $\|r\| = \|r_+\| + \|r_-\|$.
\end{theorem}

\section{Gelfand-Neymark-Representation}



\end{document}